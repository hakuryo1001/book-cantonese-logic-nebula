# 新香港九約竹枝詞十首

2023-06-24 预计阅读需要2分钟

**劉祖榮**

**宋皇台地鐵站**  
世事浮沉豈可猜,行朝輾轉盡悲哀。  
聖山石刻猶相貶,地鐵正名復帝台。

[![](https://cdn.hkwriters.ph4day.com/wp-content/uploads/2023/06/202306242208346.png)](https://cdn.hkwriters.ph4day.com/wp-content/uploads/2023/06/202306242208346.png)  
**楓香林**  
楓葉一紅動港城,大棠漫道沸騰聲。  
眾人只顧枝頭彩,誰理沙沙腳下鳴。

**石龍拱觀景台**  
石徑盤旋上峭峰,俯看青馬跨雲蹤。  
藍灣綺峽通維港,屯海高空鐵鳥縱。

**金山三疊泉  
**山行好坐水流旁,漱石清音解乏傷。  
閉目冥思塵世贅,靈光共與浪花揚。

**八仙嶺流水響**  
去年亦過羽林中,只見枯枝戳碧空。  
流水苟殘潭水淺,龍山橋下映孤瞳。

**美孚老榕壇**  
一樹成壇群卉鑲,青蔥尤勝粉紅揚。  
三千花色爭蜂蝶,眾鳥枝頭鬥唱昂。  
[![](https://cdn.hkwriters.ph4day.com/wp-content/uploads/2023/06/2023062422102235.jpg)](https://cdn.hkwriters.ph4day.com/wp-content/uploads/2023/06/2023062422102235.jpg)  
**南蓮園池**  
奇松異石簇崟崟,古色唐風漫步尋。  
舉目山門如咫尺,一街車馬隔千潯。

**土瓜灣  
**瓜中核落土中央,蘊藉繁華寓意長。  
簡寫灣情千載變,海心魚石憶滄桑。

**屯門菠蘿山峽谷**  
溝壑嶙峋壁角奇,岩金土褐炫虹霓。  
玲瓏峽谷盆中景,切片菠蘿且醉迷。

**青山杯渡岩**  
是非是是皆非是,無有無無自有無。  
不二法門何為法,半山碧樹半雲途。

(本文圖片為資料圖片)

**劉祖榮簡介:**香港人,祖籍福建南安市。作品曾獲得二〇二一年首屆全球「藝術與和平」詩詞大賽的三等獎,二〇二〇年第二屆義烏駱賓王國際兒童詩歌大賽的提名獎,二〇二一年第三屆中國徐霞客散文獎佳作獎,二〇一八年全球華文「曼麗雙輝」填詞大獎賽優異獎,二〇二一年首屆中國「華潯杯.我愛我家」詩歌優秀獎,二〇二一年第九屆「禾澤都林杯」城市、建築與文化散文大賽優秀獎等。