#文 

頭次見面用[久仰],好耐冇見講[久違]。
認人不清用[[[眼拙]]],向人表歉用[失敬]。  
請人批評講[指教],求人原諒用[包涵]。
請人幫忙講[勞駕],請畀方便用[借光]。  
麻煩人哋講[打擾],唔好意思用[冒昧]。

求人解答用[請問],請人指點講[賜教]。  
贊人見解用[高見],自己意見用[拙見]。

探望人哋用[拜訪],賓客到來用[光臨]。  
陪著朋友用[奉陪],中途先走用[失陪]。

等待客人用[恭候],迎接表歉用[失迎]。  
有人離開用[再見],請人別送講[留步]。

歡迎顧客稱[光顧],謝人問候用[托福]。  
問人年齡用[貴庚],老人年齡用[高壽]。

閱人文章用[拜讀],請人修改用[斧正]。  
對方字畫稱[墨寶],招待不周講[怠慢]。

請人收禮用[笑納],辭謝饋贈用[心領]。  
問人姓氏用[貴姓],答覆詢問用[免貴]。

表演技能用[獻醜],答謝讚揚講[過獎]。  
向人祝賀講[恭喜],覆人道賀用[多謝]。

請人擔職用[屈就],暫時充任講[承乏]。  
  
歷代有唔少謙讓嘅典故,係後人嘅典範。  
  
**

  

〔令〕字一族:  
用喺對方嘅親屬或有關係人物嘅用詞。  
如,令尊:尊稱對方嘅[父親];  
  令堂:尊稱對方嘅[母親];

  令郎:尊稱對方嘅[兒子];

  令愛:尊稱對方嘅[女兒];

  令嬡:同〔令愛〕;  
  令兄:尊稱對方嘅[兄長];

  令弟:尊稱對方嘅[弟弟];

  令侄:尊稱對方嘅[侄子]。

  
**  
  
〔賢〕字一族:  
用喺平輩或者晚輩身份嘅用詞。

如,賢弟:稱呼自己嘅[弟弟]或比自己年齡細嘅男性;  
  賢侄:稱呼自己嘅[侄子]或比自己輩份低嘅男性。  
  
**  
  
〔恭〕字一族:  
表示好恭敬咁對待對方嘅用詞。

 如,恭賀:恭敬向對方[祝賀];

  恭候:恭敬向對方[等候];

  恭請:恭敬向對方[邀請];

  恭迎:恭敬向對方[迎接];

  恭喜:祝賀對方有[喜事]。  
  
**  
  
〔拜〕字一族:

用喺對人事往來嘅用詞。  
如,拜讀:表示[閱讀]對方文章;

  拜辭:表示與對方[告辭];  
  拜訪:表示[訪問]對方;  
  拜服:表示[佩服]對方;  
  拜賀:表示[祝賀]對方;  
  拜識:表示[結識]對方;  
  拜託:表示[委託]對方辦事情;  
  拜望:表示[探望]對方。  
  
**  
  
〔奉〕字一族:  
用喺自己舉動涉及對方時嘅用詞。  
如,奉達:表示[告訴],[表達];(多數用喺書信)  
  奉覆:表示[回覆](多數用喺書信);  
  奉告:表示[告訴];  
  奉還:表示[歸還];  
  奉陪:表示[陪伴];  
  奉勸:表示[勸告];  
  奉送:表示[贈送];  
  奉贈:同〔奉送〕;  
  奉迎:表示[迎接];  
  奉托:表示[拜託]。  
  
**  
  
〔敬〕字一族:  
用喺自己行動涉及對人嘅用詞。  
如,敬告:表示[告訴];  
  敬賀:表示[祝賀];  
  敬候:表示[等候];  
  敬禮:表示[恭敬](用於書信結尾);  
  敬請:表示[邀請];  
  敬佩:表示[敬重佩服];  
  敬謝:敬謝不敏;表示[推辭]做某件事。  
  
**  
  
〔貴〕字一族:  
稱呼同對方有關事物嘅用詞。  
如,貴幹:問對方要做咩嘢;  
  貴庚:問對方[年齡];  
  貴姓:問對方[姓氏];  
  貴恙:稱對方[病況];  
  貴子:稱對方[兒子](含祝福之意);  
  貴國:稱對方[國家];  
  貴校:稱對方[學校]。  
  
**  
  
〔高〕字一族:  
稱許人哋有關事物嘅用詞。

如,高見:即係高明嘅[見解];  
  高就:指離開原來職位就任較高嘅職位;  
  高齡:稱老人家嘅[年齡](多數指六十歲以上);  
  高壽:用喺詢問老人家嘅年齡;  
  高足:稱呼人哋嘅[學生];  
  高論:稱許人哋嘅[議論]。  
  
**  
  
〔大〕字一族:  
尊稱對方或稱同對方有關事物嘅用詞。

如,大伯:除咗指伯父之外,亦可以尊稱年長嘅男人;  
  大哥:可以尊稱同自己年齡相仿嘅男人;  
  大姐:可以尊稱女性朋友或者熟人;  
  大媽:尊稱年長嘅婦女;  
  大娘:同〔大媽〕;  
  大爺:尊稱年長嘅男人;  
  大人:尊稱長輩(多數用喺書信);  
  大駕:尊稱對方;  
  大師:大師傅,尊稱和尚;  
  大名:尊稱對方嘅[名字];  
  大慶:尊稱老年人[壽辰];  
  大作:尊稱對方嘅[著作];  
  大札:尊稱對方嘅[書信]。  
  
**  
  
〔請〕字一族:  
用喺希望對方做某啲事嘅用詞。

如,請問:用喺請求對方回答問題;  
  請坐:請對方[坐低];  
  請進:請對方[入來]。  
  
**  
  
〔屈〕字一族:  
如,屈駕:委屈大駕(多數用喺邀請人);  
  屈就:委屈就任(多數用喺請人擔任職務);  
  屈居:委屈處於(較低嘅地位);  
  屈尊:委屈降低身份俯就。  
  
**  
  
〔光〕字一族:  
表示光榮,用喺對方來臨嘅用詞。

如,光顧:稱客人來到(多數用喺商家歡迎顧客);  
  光臨:稱賓客到來。  
  
**  
  
〔俯〕字一族:  
公文書信中用來稱對方對自己行動嘅用詞。

如,俯察:稱對方或上級對自己嘅理解;  
  俯就:用喺請對方同意擔任職務;  
  俯念:稱對方或上級體念;  
  俯允:稱對方或上級允許。  
  
**  
  
〔華〕字一族:  
稱對方有關事物嘅用詞。

如,華誕:稱對方嘅[生日];  
  華堂:稱對方嘅[房屋];  
  華翰:稱對方嘅[書信];  
  華宗:稱對方係[同姓氏]或[同宗族]。  
  
**  
  
〔叨〕字一族:  
如,叨光:沾光(受到好處,表示感謝);  
  叨教:領教(受到指教,表示感謝);  
  叨擾:打擾(受到款待,表示感謝)。  
  
**  
  
〔雅〕字一族:  
用喺稱對方情意或舉動嘅用詞。

如,雅教:稱對方嘅[指教];  
  雅意:稱對方嘅[情意]或[意見];  
  雅正:稱對方嘅[指正]同[批評](將自己嘅詩文書畫等送畀人時)。  
  
**  
  
〔玉〕字一族:  
用喺尊稱對方身體或行動嘅用詞。

如,玉體:尊稱對方嘅[身體];  
  玉音:尊稱對方嘅[書信](多數用喺書信)、言辭;  
  玉照:尊稱對方嘅[照片];  
  玉成:稱對方幫助[成全]好事。  
  
**  
  
〔芳〕字一族:  
用喺對方或同對方有關事物嘅用詞。

如,芳鄰:稱對方嘅[鄰居];  
  芳齡:稱對方嘅[年齡](多數用喺年輕女子);  
  芳名:稱對方嘅[名字](多數用喺年輕女子)。  
  
**  
  
〔垂〕字一族:  
用喺人哋對自己的行動嘅用詞(多數係長輩或上級)。

如,垂愛:稱對方對自己嘅[愛護](都係用喺書信);  
  垂青:稱別人對自己嘅[重視];  
  垂問:稱別人對自己嘅[詢問];  
  垂詢:同〔垂問〕;  
  垂念:稱別人對自己嘅[思念]。  
  
**  
  
〔惠〕字一族:  
用喺對方對待自己行為動作嘅用詞。

如,惠存:請對方保存(多數用喺送人相片、書籍等紀念品時所題嘅上款);  
  惠臨:指對方來到自己嘅地方;  
  惠顧:指來臨光顧(多數用喺鋪頭對顧客),又叩[幫襯];  
  惠允:指對方允許自己(做某事);  
  惠贈:指對方贈予(財物)。  
  
**  
  
謙 詞  
  
【愚】字一族:  
謙稱自己唔聰明嘅用詞。

如,愚兄:向比自己年輕嘅人稱呼自己;  
  愚見:謙稱自己嘅見解。  
  愚 :亦可以單獨用來謙稱自己。  
  
**  
  
【鄙】字一族:  
謙稱自己學識淺薄嘅用詞。

如,鄙人:謙稱自己;  
  鄙意:謙稱自己嘅意見;  
  鄙見:謙稱自己嘅見解。  
  
**  
  
【敝】字一族:  
謙稱自己或自己事物嘅用詞。

如,敝人:謙稱自己;  
  敝姓:謙稱自己嘅姓氏;  
  敝處:謙稱自己嘅屋企、處所;  
  敝校:謙稱自己所在嘅學校。  
  
**  
  
【卑】字一族:  
謙稱自己身份低微嘅用詞。  
如,卑人:謙稱自己;

  卑下:謙稱自己地位、品格等比對方低。  
  
**  
  
【竊】字一族:  
有私下、私自之意,有冒失、唐突嘅含義在內。  
如,竊念:私下想念(表示個人意見嘅謙辭);  
  竊惟:私下思惟;  
  竊比:私自比擬;  
  竊言:私下談論;  
  竊庇:私下包庇;  
  竊竊:暗中,偷偷地;  
  竊議:私下議論,私自評論。  
  
**  

  
【臣】字一族:  
謙稱自己不如對方嘅身份地位高。

如,臣僕:僕人嘅自稱;  
  臣妾:妻妾嘅自稱;  
  臣子:君主時代嘅官吏。  
  
**  

  
【僕】字一族:  
舊謙稱自己[我]嘅用詞。

如,僕:謙稱自己係對方嘅僕人,願意受差遣。而[為奴為僕]含有為對方效勞嘅意思。

  
**

  
【敢】字一族:  
表示冒昧請求人哋做事嘅用詞。

如,敢問:用喺詢問對方嘅問題;  
  敢請:用喺請求對方做某事;  
  敢煩:用喺麻煩對方做某事。  
  
**  
  
【拙】字一族:  
用喺對人哋稱呼自己事物嘅用詞。

如,拙筆:謙稱自己嘅[書畫];  
  拙着:謙稱自己嘅[文章];  
  拙作:同〔拙着〕;  
  拙見:謙稱自己嘅[見解];  
  拙荊:謙稱自己嘅[妻子],重有[賤內]、[內人]等。  
  
**  
  
【小】字一族:  
謙稱自己或同自己有關嘅人或事物嘅用詞。

如,小弟:男性喺朋友或熟人之間嘅謙稱自己;  
  小兒:謙稱自己嘅[兒子];  
  小女:謙稱自己嘅[女兒];  
  小人:地位低嘅人[自稱];  
  小子:子弟晚輩對父兄尊長嘅自稱;  
  小可:多見喺早期嘅白話,係有一定身份嘅人嘅自謙,  
     意思係自己好平凡、不足掛齒;  
  小店:謙稱自己嘅[鋪頭]。  
  小生:讀書人自謙詞,重有[晚生]、[晚學]等,表示自己係新學後輩。  
  
**  
  
【家】字一族:  
古人稱呼自己一方親屬朋友嘅常用謙詞。

家,係對人哋稱呼自己輩份高或年紀大嘅親屬時用嘅謙詞。  
如,家父:稱呼自己[父親];  
  家尊:同[家父];  
  家嚴:同[家父];  
  家君:同[家父];  
  家母:稱呼自己[母親];  
  家慈:同[家母];  
  家兄:稱呼自己[兄長];  
  家姐:稱呼自己[姐姐];  
  家叔:稱呼自己[叔叔]。  
  
**  
  
【舍】字一族:  
用來謙稱自己家或自己卑幼親屬嘅用詞。

如,寒舍:謙稱自己嘅家;  
  敝舍:客氣對朋友稱自己嘅家。  
如,舍弟:稱自己嘅[弟弟];  
  舍妹:稱自己嘅[妹妹];  
  舍侄:稱自己嘅[侄子];  
  舍親:稱自己嘅[親戚]。  
  
**  
  
【老】字一族:  
老人家自謙時用來謙稱自己或同自己有關事物嘅用詞。

如,老朽:謙稱自己做[老邁衰朽];  
  老夫:指年齡超過七十歲嘅男人謙稱自己;  
  老漢:年老嘅男人謙稱自己;  
  老拙:老年嘅男人謙稱自己,亦稱做〔老粗〕;  
  老粗:謙稱自己冇咩文化;  
  老臉:年老人指自己嘅面子;  
  老身:老年婦女謙稱自己;  
  老衲:老和尚謙稱自己;  
  老臣:老官員謙稱自己。  
  
**  
  
【貧】字一族:  
僧、道、尼姑自謙稱呼嘅用詞。

如,貧僧:和尚對自己謙稱;  
  貧道:道士對自己謙稱;  
  貧尼:尼姑對自己謙稱。  
  
**  
  