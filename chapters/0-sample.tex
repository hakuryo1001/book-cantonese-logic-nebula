

% \chapter{粵切字 Sample |  }

% 聲母

% \begin{table}[H]
%   \centering
%   \begin{tabular}{|>{\centering\arraybackslash}m{2cm}|>{\centering\arraybackslash}m{2cm}|>{\centering\arraybackslash}m{2cm}|>{\centering\arraybackslash}m{2cm}|}
%     \hline
%     \begin{tabular}[c]{@{}c@{}}b 比\\ ⿱\end{tabular}  & \begin{tabular}[c]{@{}c@{}}p 并\\ ⿰\end{tabular}  & \begin{tabular}[c]{@{}c@{}}m 文\\ ⿱\end{tabular} & \begin{tabular}[c]{@{}c@{}}f 夫\\ ⿰\end{tabular}                                             \\
%     \hline
%     \begin{tabular}[c]{@{}c@{}}d 大\\ ⿱\end{tabular}  & \begin{tabular}[c]{@{}c@{}}t 天\\ ⿱\end{tabular}  & \begin{tabular}[c]{@{}c@{}}n 乃\\ ⿰\end{tabular} & \begin{tabular}[c]{@{}c@{}}l 力\\ ⿰\end{tabular}                                             \\
%     \hline
%     \begin{tabular}[c]{@{}c@{}}z 止\\ ⿰\end{tabular}  & \begin{tabular}[c]{@{}c@{}}c 此\\ ⿱\end{tabular}  & \begin{tabular}[c]{@{}c@{}}s 厶\\ ⿱\end{tabular} & \begin{tabular}[c]{@{}c@{}}j 央\\ ⿱\end{tabular}                                             \\
%     \hline
%     \begin{tabular}[c]{@{}c@{}}g 丩\\ ⿰\end{tabular}  & \begin{tabular}[c]{@{}c@{}}k 臼\\ ⿱\end{tabular}  & \begin{tabular}[c]{@{}c@{}}h 亾\\ ⿰\end{tabular} & \begin{tabular}[c]{@{}c@{}}ng \scalebox{0.5}[1.0]{乂}\scalebox{0.5}[1.0]{乂}\\ ⿱\end{tabular} \\
%     \hline
%     \begin{tabular}[c]{@{}c@{}}gw 古\\ ⿰\end{tabular} & \begin{tabular}[c]{@{}c@{}}kw 夸\\ ⿰\end{tabular} & \begin{tabular}[c]{@{}c@{}}w 禾\\ ⿱\end{tabular} & \begin{tabular}[c]{@{}c@{}}m/ng 𫝀\\ \ \end{tabular}                                         \\
%     \hline
%   \end{tabular}
% \end{table}
% % 韻母

% 韻母

% \begin{table}[H]
%   \centering
%   \resizebox{\textwidth}{!}{ % Adjust the width to fit within the page
%     \begin{tblr}{
%       colspec={|X[c]|X[c]|X[c]|X[c]|X[c]|X[c]|X[c]|X[c]|X[c]|X[c]|}, % Column alignment
%       hlines, % Horizontal lines
%       vlines  % Vertical lines
%       }
%            & \empty          & -i               & -u               & -m               & -n               & -ng               & -p               & -t                            & -k               \\
%       /aa/ & aa \linebreak 乍 & aai \linebreak 介 & aau \linebreak 丂 & aam \linebreak 彡 & aan \linebreak 万 & aang \linebreak 生 & aap \linebreak 甲 & aat \linebreak 压              & aak \linebreak 百 \\
%       /a/  &                 & ai \linebreak 兮  & au \linebreak 久  & am \linebreak 今  & an \linebreak 云  & ang \linebreak 亙  & ap \linebreak 十  & at \linebreak 乜               & ak \linebreak 仄  \\
%       /e/  & e \linebreak 旡  & ei \linebreak 丌  & eu \linebreak 了  & em \linebreak 壬  & en \linebreak 円  & eng \linebreak 正  & ep \linebreak 夾  & et \linebreak 叐               & ek \linebreak 尺  \\
%       /i/  & i \linebreak 子  &                  & iu \linebreak 么  & im \linebreak 欠  & in \linebreak 千  & ing \linebreak 丁  & ip \linebreak 頁  & it \linebreak 必               & ik \linebreak 夕  \\
%       /o/  & o \linebreak 个  & oi \linebreak 丐  & ou \linebreak 冇  &                  & on \linebreak 干  & ong \linebreak 王  &                  & ot \linebreak 匃               & ok \linebreak 乇  \\
%       /u/  & u \linebreak 乎  & ui \linebreak 会  &                  &                  & un \linebreak 本  & ung \linebreak 工  &                  & ut \linebreak 末               & uk \linebreak 玉  \\
%       /oe/ & oe \linebreak 居 &                  &                  &                  &                  & oeng \linebreak 丈 &                  &                               & oek \linebreak 勺 \\
%       /eo/ &                 & eoi \linebreak 句 &                  &                  & eon \linebreak 卂 &                   &                  & eot \linebreak 𥘅$_{\text{朮}}$ &                  \\
%       /yu/ & yu \linebreak 仒 &                  &                  &                  & yun \linebreak 元 &                   &                  & yut \linebreak 乙              &                  \\
%     \end{tblr}
%   }
%   \caption{韻母}
% \end{table}

% 聲調

% \begin{table}[H]
%   \jcz{}
%   \centering
%   \begin{tblr}{
%     colspec={|X[c]|X[c]|X[c]|X[c]|X[c]|X[c]|},  % Equal-width columns and centered text
%     hlines,  % Draw horizontal lines
%     vlines   % Draw vertical lines
%     }
%     1   & 2 & 3 & 4   & 5 & 6 \\
%     󰘠、󰘦 & 󰘡 & 󰘢 & 󰘣、󰘧 & 󰘤 & 󰘥 \\
%     󰝰、󰝶 & 󰝱 & 󰝲 & 󰝳、󰝷 & 󰝴 & 󰝵 \\
%     分   & 粉 & 訓 & 墳   & 憤 & 份 \\
%   \end{tblr}
%   \caption{切字 聲調}
% \end{table}



% % \begin{table}[htbp]
% %   \jcz{}
% %   \centering
% %   \renewcommand{\arraystretch}{1.5} % Adjust row height
% %   \setlength{\tabcolsep}{4pt} % Adjust column padding
% %   \resizebox{\textwidth}{!}{
% %   \begin{tabularx}{\textwidth}{|X|X|X|X|}
% %   \hline
% %   % \rowcolor[HTML]{D0D0D0} 
% %   \textbf{坊間漢羅混用} & \textbf{漢字已整理版本} & \textbf{漢字粵切字混用(未組裝)} & \textbf{漢字粵切字混用(已組裝)} \\
% %   \hline
% %   咁都係果D嘢嘎啦,廿鯪蚊個餐又湯又剩唔通有得你食天九翅咩?求求其其有D肉有D菜蛋白質澱粉質撈撈埋埋打個白汁茄汁黑椒汁咁撐得你懵口懵面咪Lui返去返工返學返廠返寫字樓囉。唔係你估真係搵餐晏仔咁簡單啊。咁跟飯定跟意粉啊? 
% %   & 咁都係果啲嘢㗎啦,廿鯪蚊個餐又湯又剩唔通有得你食天九翅咩?求求其其有啲肉有啲菜蛋白質澱粉質撈撈埋埋打個白汁茄汁黑椒汁咁撐得你懵口懵面咪纍返去返工返學返廠返寫字樓囉。唔係你估真係搵餐晏仔咁簡單啊。咁跟飯定跟意粉啊? 
% %   & 丩今´都係丩个´大子¯野丩乍`力乍`,廿力正⁼蚊個餐又湯又剩𠄡通有得你食天九翅文旡¯?求々其々有大子¯肉有大子¯菜蛋白質澱粉質撈々埋々打個白汁茄汁黑椒汁丩今´止生゙得你懵口懵面文兮`力句¯返去返工返學返廠返寫字樓力个¯。𠄡係你估真係搵餐晏仔丩今`簡單⺍乍⁼。丩今´跟飯定跟意粉⺍乍`?
% %   & 󱜩都係󱟡󰦠野󱛒󰿒,廿󰻃蚊個餐又湯又剩𠄡通有得你食天九翅󰗘?求々其々有󰦠肉有󰦠菜蛋白質澱粉質撈々埋々打個白汁茄汁黑椒汁󱜩󰿽得你懵口懵面󰖚󰾠返去返工返學返廠返寫字樓󰼠。𠄡係你估真係搵餐晏仔󱜪簡單󰀓。󱜩跟飯定跟意粉󰀒? \\
% %   \hline
% %   \end{tabularx}
% %   }
% % \end{table}
