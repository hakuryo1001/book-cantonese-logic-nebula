#文 

華夏南北講嘢發音用辭都唔同。家下國語以北方話做底[1],主流文化寫粵語嘅傳統斷咗好耐,以致廿世紀末後生一代嘅廣府話用辭同上一代開始唔同,漸離古漢語同南話,趨近現代北方話。呢頁有粵京兩話用辭嘅對照表;有意見,歡迎去討論版傾。

維基文《粵語》入面都有張粵京用辭對照表,可以參考;其中有啲非傳統嘅詞。呢版同Wikipedia:粵語本字就儘量用傳統粵語同正字。如果用辭有考據資料或者俗寫,可以寫響「註」度。


  
名詞

|   |   |   |
|---|---|---|
|**粵**|**京**|**註**|
|[[咿唈:]]|動靜||
|輘輷|蹊蹺||
|冚唪唥|全部||
|色水|色澤||
|日頭|太陽||
|宵夜|夜宵||
|人工、糧|薪水、薪金||
|火水|煤油||
|樽|瓶||
|雪櫃|冰箱||
|窗(口/門)|窗(戶)||
|夾萬|保險箱||
|櫃桶|抽屜|「櫃桶」亦可以指水平敞開、無「抽」出結構嘅容器。|
|屋企|家|正寫係「屋下」[未記出處或冇根據]|
|荷包、銀包|錢包||
|皮篋|皮箱||
|林瀋嘢、家爛豆|勞什子||
|老襯|冤大頭(無對應詞,被愚弄的人)|唔同「老親」(親家)|
|番梘、番鹼|肥皂||
|掣|(按)鈕|讀zai3|
|挨憑|靠背|音aai1 peng1。「挨」係挨年近晚,挨晚嘅挨;「憑」原本讀 bang6音,例:憑喺幅牆度;俗讀 peng1。[2]|
|屋、鬥(竇)|房子||
|膥、春|卵|音ceon1|
|衫|衣(服)||
|鑊|(底較平的)鍋||
|煲|(壁較陡峭的)鍋||
|鐺|鐺(少用)、平底鍋||
|埞、埞方|地方||
|銀仔|硬幣|「銀」音ngan2|
|蚊、文|塊、圓、元|貨幣單位|
|毫、粒神|毛、角(正式)|貨幣單位|
|檯、枱|桌子||
|匙羹、匙|勺子||
|地氈|地毯||
|豉油|醬油||
|窿|孔、洞||
|罅|裂縫||
|炮仗|鞭炮、爆竹||
|遮|雨傘||
|坑渠|下水道||
|煙花|煙火、煙花||
|騸雞|閹雞||
|瓦罉|一種煲湯用的煲||
|冷氣(機)|空調(機)||
|貨櫃|集裝箱||
|雪條、雪批|冰棒、冰棍||
|雪糕|冰琪林||
|飯盒|盒飯||
|靚仔、令仔|帥哥|音leng3 zai2|
|鉸剪、較剪|剪刀||
|螺絲批|螺絲刀、改锥||
|餸|(隨同飯的)菜||
|鎖匙|鑰匙||
|古仔|故事||
|公仔|娃娃、玩具人物||
|箭嘴|箭頭||
|頸鏈、頸鍊|項鏈、項鍊||
|韆鞦|鞦韆||
|輘輷|秘密、玄機|讀「ging2 gwang2」,古漢語原意係「群車行走的隆隆巨響」|
|花灑|蓮蓬頭||
|膠擦、擦膠、擦紙膠|橡皮、橡皮擦|喺香港叫「擦膠」嘅比較多,而廣東就係「膠擦」比較常見。|
|鹹蛋黃|日落||
|射甩膽、士得膽|啟輝器|粵語乃英文「starter」嘅音譯。|
|凳|椅子||
|碟|碟子||
|光碟、CD、DVD|光盤/鐳射光碟||
|錔|鎖|讀「daap3」或「taap3」,廣州地區多數習慣讀成「taap3」。|
|燙斗|熨斗||
|椗|把兒、梗兒||
|煮飯仔|过家家|兒童遊戲|
|埞|位置||
|猜呈尋、包剪揼|剪刀石頭布|猜拳遊戲|

  

**人**[編輯]

|   |   |   |
|---|---|---|
|**粵**|**京**|**註**|
|人客|客人||
|老世、事頭、波士|老闆||
|先生、阿Sir、Miss|老師|「阿Sir」音「aa3 se4」|
|後生|後輩、年輕人||
|老豆(老竇)、阿爸、爹哋|爸、爹||
|老母、阿媽、媽咪、老媽子|媽、娘|喺「老母」前面或後面加字通常會變成粗口。|
|老爺、家公|公公||
|奶奶、家婆、阿婆|婆婆||
|外父(佬)|岳父||
|外母(乸)|岳母||
|爺爺、阿爺、阿公|爺爺、公公|「阿爺」喺粵語重有一種意思係「政府、公務」,譬如將「做公務員」稱為「打阿爺工」|
|嫲嫲、阿嫲、阿婆|奶奶||
|婆婆、阿婆|姥姥||
|公公、阿公|姥爺||
|細孥、細孥仔、細孥哥、細蚊仔|小孩||
|𡃁仔、𡃁妹|少年、少女||
|BB、BB仔、臊孲、臊孲子(蘇蝦仔)、孲𤘅、孲𤘅子|嬰兒|「孲」音aa1。「赤子」義。[3]|
|冚家|全家|「冚」音ham4,本字係「咸(haam4)」,粵語亦有用「全家」。|
|妗母|舅媽|「妗」為「舅母」之合音,讀kam5[4],南方諸語常用。|
|老公、老婆|丈夫/妻子/愛人(對外用)||
|大嫂|大媽||
|姑丈|姑夫||
|姑姐|姑姑||
|伯爺|伯父||
|伯娘|伯母||
|舅仔|小舅子、妻弟||
|師姑|尼姑||
|嬸嬸、嬸|姨姨/姨|對一啲中年女人嘅叫法,有時對一啲女工人都會噉叫。|
|鬼佬|白人/老外||
|嚤囉叉、摩囉差、阿叉|阿三|貶低嘅稱呼|
|乞兒|乞丐|音「乞衣」|
|塞豆窿|熊孩子|好拽嘅細路稱呼|
|花靚倞|小鮮肉|好後生啲男仔,出自2009年臺灣綜藝節目《康熙來了》——徐熙娣嘉賓嘅口頭禪。|
|仔、亞仔|兒子|男仔細路|
|女、亞女|女兒|女仔細路|
|影帝|戏精|加戲搏出位啲人。2017年網絡流行詞。|
|鬼佬|歪果仁|形容無國籍或非華人身份啲人。2016年中國大陸網上流行詞,出自歐美啲鬼佬講普通話嘅「外國人」一詞。|
|本地人、當地人|當地民眾、當地群眾||

  

**動植物**[編輯]

|   |   |   |
|---|---|---|
|**粵**|**京**|**註**|
|老鼠|老鼠、耗子||
|馬騮|猴子||
|曱甴、小強|蟑螂|「曱甴」係粵語最傳統嘅用法;而「小強」一詞出喺周星馳嘅電影《唐伯虎點秋香》中。|
|雀、雀仔|鳥、小鳥||
|生果|水果|喺廣東嘅粵語區重有人叫「水果」|
|菜|蔬菜||
|烏蠅|蒼蠅||
|黃犬、黃螾|蚯蚓|可以寫作「黃䘆」|
|蠄蟧|蜘蛛||
|崩紗|蝴蝶||
|塘尾|蜻蜓||
|粟米|玉米||
|薯仔|土豆、馬鈴薯||
|蕃茄|西紅柿、蕃茄||
|蕹菜、通菜|通心菜||
|蛤乸、蛤𧊅、田雞|蛙||
|蠄蟝|蟾蜍||
|芫茜|芫荽、香菜||
|勝瓜|絲瓜||
|涼瓜|苦瓜||

  

**身體部位**[編輯]

|   |   |   |   |   |   |   |   |   |   |   |
|---|---|---|---|---|---|---|---|---|---|---|
|**粵**|**京**|**註**||**粵**|**京**|**註**||**粵**|**京**|**註**|
|眼|眼睛|||鼻哥、鼻|鼻子|||頭皮|頭屑、頭皮屑|亦解做頭部皮膚,譬如「頭皮好痕」|
|耳仔|耳朵|||(一塊)面|(一張)臉|||樣|樣子、模樣|粵音陰上聲(joeng2)|
|膊頭|肩|||膶|肝(食用)|||翼|翅膀||
|脷|舌(頭)|||橫脷|胰(食用)|亦有講「脌貼」嘅||髀|腿||
|𢆡、姩、波|乳房|||手指公|大拇指|||手指屘、屘指|小指||
|膝頭哥|膝蓋|||啫啫、朘仔、賓舟|雞雞、陰莖、陽具|||頸|脖子||
|鬍鬚|鬍子|||胳肋底|腋窩、胳肢窩|||腰|腎(食用)||
|腎|胗(食用)|禽鳥嘅胃||背脊|背部|||屎窟|屁股||
|腳踭|腳跟|||手踭|手肘||||||

  

**外來譯音詞**[編輯]

下便呢啲係外語音譯詞,唔包括人名同地名(因為太多數唔嗮)。由於音譯詞大多數都係喺香港形成嘅,有部分喺內地唔一定會用。

|   |   |   |   |
|---|---|---|---|
|**粵**|**京**|**英**|**註**|
|巴士|公共汽車|bus|廣東近年都有人講「公交(車)」|
|的士|出租車、計程車|taxi|台灣叫「計程車」;星加坡叫「德士」|
|芝士|奶酪|cheese|台灣叫「起司」|
|多士|吐司|toast||
|安士、盎士|盎司|ounce||
|啫哩|果凍、凝膠|jelly|「啫哩」音ze1lei2|
|朱古力|巧克力|chocolate||
|士巴拿|扳手|spanner(美:wrench)||
|士多啤梨(港、粵)/草莓(粵)|草莓|strawberry|廣東多數人講「草莓」|
|布冧|李子|plum||
|𨋢()(港)/升降機、電梯(粵)|升降機、電梯|lift(美:elevator)|廣東主要講「升降機」同「電梯」,受香港影響亦都有講「𨋢」嘅|
|士擔、郵票|郵票|(postage) stamp||
|燕梳、保險|保險|insurance||
|車厘子|櫻桃|cherry|粵語亦有人講「櫻桃」|
|泊車、停車|停車|park||
|三文魚|鮭魚|salmon||
|菲林|膠卷|film||
|布菲|自助餐|buffet|粵語亦有人講「自助餐」|
|呔|輪胎|tyre/tire||
|呔|領帶|(neck)tie||
|梳化|沙發|sofa||
|沙律|沙拉|salad||
|三文治|三明治|sandwich||
|士碌架|斯諾克|snooker||
|笨豬跳|蹦極|bungee jumping||
|乒乓波|乒乓球|pingpong|台灣叫「台球」|
|摩打|馬達|motor||
|啤令|軸承|bearing|台灣叫「培林」|
|咪高峰、咪頭、咪|麥克風、麥|microphone|「咪」讀mai1|
|XX啤梨|X莓|*-berry|廣東比較少用。|
|結他|吉他|guitar||
|低音結他|貝斯、貝司|bass||
|士多|商店|store|通常指鋪頭比較細嗰種。|
|(發)毛|(發)霉|mold|「毛」讀 mou1|
|忌廉|奶油、乳脂|cream|「廉」讀 lim1|
|冷|毛線|法語:laine|讀 laang1。例:冷衫(毛衣)|

  

代詞同代詞詞尾[編輯]

|   |   |   |   |   |   |   |   |   |   |   |
|---|---|---|---|---|---|---|---|---|---|---|
|**粵**|**京**|**註**||**粵**|**京**|**註**||**粵**|**京**|**註**|
|佢|他、她、牠、祂(簡體字「牠、祂」統一為「它」)|「佢」,或作「渠」,本字「其」。「他人」義[5]。||邊|(疑問詞詞頭,相當於「哪」)|本字「焉」||點解|為什麼|「點」嘅本字可能係「怎」|
|(我、你、佢)哋|(我、你、他/她/它)們|粵語無「朋友哋」(「朋友們」)呢種用法。本字「等」||邊個、乜誰|誰,誰人|或作「物誰」||噉|這樣、那樣|「噉」讀 gam2,「咁」讀 gam3。本字「恁」|
|呢、爾|這|||邊道、邊處|哪兒,哪裡|||咁|這麼、那麼|
|嗰、箇|那|「嗰」喺其他地方嘅中國話相當於「嘅」||點、點樣|怎麼、如何|||你|你、您||
|人哋|人家、別人、他人||||||||||

  

動詞[編輯]

|   |   |   |   |   |   |   |   |   |   |   |
|---|---|---|---|---|---|---|---|---|---|---|
|**粵**|**京**|**註**||**粵**|**京**|**註**||**粵**|**京**|**註**|
|講明、講清楚|說清楚|||來往|往來、交往|||幫襯|光顧,惠顧|「幫襯」喺其他方言入便有「幫手」同「幫補」嘅意思|
|呷醋|吃醋|||趁墟|趕集|||應承|答應||
|鍾意|喜歡|||隻揪|單挑||||||
|得閒|有空|||蕩失(路)|迷路|||講畀……聽、話畀……聽|告訴……||
|屙尿|小便、排尿|||屙屎|大便、拉屎|||埋單|結賬、買單||
|拍拖|約會、談戀愛|||打乞嗤、打乞嚏|打噴嚏|||傾、傾偈|聊、聊天、談、談話||
|沖涼|洗澡|||抆屎|擦屁股|||講大話|說謊、撒謊||
|發吽豆、發怐瞀|發呆、發愣|「怐瞀」讀 ngau6 dau6||出世|出生|||起身|起来、起床||
|發夢|做夢|||打交|打架|||嘈交、嗌交|吵架||
|𠺘口/口|漱口|「𠺘」讀 long2||㿺㿭|(皮膚)破裂|||影相|照相||
|打尖/尖隊/櫼隊|插隊||||||||||

  

**單字動詞**[編輯]

|   |   |   |   |   |   |   |   |   |   |   |
|---|---|---|---|---|---|---|---|---|---|---|
|**粵**|**京**|**註**||**粵**|**京**|**註**||**粵**|**京**|**註**|
|係|是|普通話喺非常正式嘅場合會用「係」字代替「是」。||掟|擲,扔|||食,喫|吃,喫|喫音jaak3[6]|
|唱|傳播、散佈(消息)|例如「唱衰」||明|明白|||知|知道||
|行|走|音 haang4||企|站|本字可能係「徛」||撳(掣)|按(鈕)||
|𥄫|瞧|||睇|看|正視用「睼」斜視、隨意看眼用「睇」||拎|拿||
|攞、挪|取|本字可能係「挪」||搵、捃|找(東西)、掙(錢)|本字可能係「捃」||揈|甩|本字唔確定,睇fing(有人借寫作「捹」)|
|落|下|||𡁻|嚼|亦寫作「噍」||嘥|浪費|本字「㩄」|
|鎅|切割|||捽|搓擦|||𢫏、冚|蓋|本字可能係「衾」|
|厾|戳|有時寫成「篤」||揚|抖動|音 jeong2||孭|揹||
|燂|灼烤|||轆」或者「碌|滾動|音luk1||撚|玩弄,挑弄||
|蝕|虧(損)|「蝕」音 sit6。||耖[7]、摷|翻找|音caau3。可能寫成「抄」「找」||畀|給||
|踩|踏|||㷫|加熱(引申義:憤怒)|||嬲,發嬲|氣憤||
|坐、搭(車)|乘(車)|||嗌|喊|||嗑/噏|說|「說」嘅唔同意思分別對應「嗑」、「講」同「話」|
|飲|喝,飲(少用)|||撠|卡|||講|說,講|
|掹|拉扯、拔|||喊|哭|||話|說|
|諗、惗|想|本字「念」||𧨾/氹|哄騙|||估|猜,估計|「估」重可以解作「以為」:「你估我唔知咩?」|
|揸|拿|粵語有復合詞「揸拿」,另外「揸」重可以解做「駕駛」||着|穿,穿上(衣物等)|||除|脫,脫下(衣物等)||
|掀、揭|翻(書等)|||敨/唞|歇、休息|||瞓|睡|本字「睏」|
|剒|(無對應詞,突然用力拉扯)|||撞|碰(運氣)|||鬧|罵||
|斬,斲|砍,斬(少用)|斲音doek3[8]||撩|挑惹,撩(少用)|||呃、𧦠、gwan2、𧥺|欺騙|gwan2有時寫成「滾」,「𧥺」有時寫成「昆」或者「坤」|
|郁、㤢、逳|動|俗寫(郁).心㤢㤢, 㤢:《玉篇》心動也。逳:《玉篇》轉也,行也。《廣韻》步也。||閂|關|||噬|狠咬||
|侲|壞|||添|加|添飯||搽|塗、抹、搽(少用)|音 caa4|
|扠[9]|塗污|音 caa⁵,劃花噉解||搲||用手指捉住,音 we²||嘈|嚷,喧鬧,喧嘩|「嘈」可以做形容詞,同普通話意思一樣|
|拗(詏)|爭論,爭吵|「拗」音aau3||拗|彎折,折斷|「拗」音aau2||掂|觸碰,摸||
|撐|挺|解做支持、擁護||嚟、蒞|來|||撈|拌、混|「撈」音lou1|
|入|進|||掗|占|||孭|揹||
|惜|吻、疼愛|音「錫」||𢯎(痕)|搔(癢)|||慳|節省、省||
|黐|粘、黏|||驚|怕|||厭|膩||
|冧|塌|「冧」音lam3||擳|擠、胳肢|||返|回|(返去:回去;返嚟:回來)|
|匿(埋)|藏(起來)|||蝦|欺負|||詐|(假)裝||
|踎|蹲|||䟴|抖|䟴腳、腳䟴䟴||似|像||
|捉(棋)|下(棋)|||識|會、認識|||㓟|削||
|瀡|滑、溜|「瀡」音seoi5||揀|選|||搣|捏、撕||
|𢳂|舀|||扚|拉開、手掐、拔擢|扚起心肝、扚住件衫、扚佢起身[10]||褪|後退|「褪」音tan3,例如「褪後」|
|㧬|推擠|||湊(細路仔)|帶(小孩)|||瓊|凝結、凝固||
|濁|嗆|||抌|扔|||攬|(擁)抱、摟||

  

**助動詞**[編輯]

|   |   |   |   |   |   |   |   |   |   |   |
|---|---|---|---|---|---|---|---|---|---|---|
|**粵**|**京**|**註**||**粵**|**京**|**註**||**粵**|**京**|**註**|
|識、曉、能夠、可以|會、能、能夠、可以|||[[䎺]]、肯|願意、肯|||夠膽、敢|敢||
|得、掂(𠶧)|行、可以、成功|用喺動詞後邊。「掂」、「𠶧」有分工:唔好掂樖樹;搞𠶧晒。||唔使|不用、不必、不需要、不需|問問題會問「使唔使」=普通話「用不用」||抵|值得||

  

形容詞

|   |   |   |   |   |   |   |   |   |   |   |
|---|---|---|---|---|---|---|---|---|---|---|
|**粵**|**京**|**註**||**粵**|**京**|**註**||**粵**|**京**|**註**|
|[[虢礫𡃈嘞]]|林林總總|||燶|焦|||[[狼戾]]|固執無理||
|零星、濕碎|零星、零碎|||求祈、[[俹簁]]|馬虎|「俹簁」音laa² sai¹||斯文|斯文(書面語)、文雅||
|求祈|問天|錯寫:求其||攰|累|||[[姣]]|放蕩||
|事旦|明天才算|錯寫:是但、是旦、事但||冇所為|不為甚麽|錯寫:冇所謂||隨便|隨意及方便||
|戇居|呆|||後生|年輕|「後生」亦係名詞,表示「年輕人」。||熱鬧、墟撼|熱鬧||
|鶻突[11]、核突|惡心|形容啲嘢好難睇,或者形容一個人個樣好得人憎||腍|軟|||蠢|笨||
|肉酸、朒朘|難看|令人不舒服||骨痺|肉麻|||肉赤|心疼||
|凍|冷|||細|小|||耐|久||
|啱|對、合適、巧、正確|||犀利|厲害|||崖广、[[牙煙]]|危險|「崖广」同「牙煙」同音。「广」係一個康熙部首,唔係簡體嘅「廣」字。|
|鮓|差|||肥|胖|形容人,物,有時狀況。||晏|晚||
|曳、跳皮|調皮、頑皮、淘氣|||論盡|笨拙|||痕|癢||
|叻|聰明能幹、出色、有本事、很棒|||靚|好看、漂亮、美、帥、好|「靚」 音leng3||平|便宜|「平」音peng4[12]|
|||||勤力、努力|努力|||邋遢、污糟|髒、骯髒、齷齪||
|抵、著數/着數|合算、划算|||竄|拽|||燶|焦、糊||
|牙擦|自負|||惡|兇|||眼瞓|睏|「瞓」本字「睏」|
|唔啦更|不相關、不相干|||得意|可愛|||嬲|生氣||
|(好)似|(好)像|||麻麻地|馬馬虎虎、一般|||怕醜|害羞||
|[[菂薂]]|細小、小巧|||黃黚黚|淺黃帶黑|||[[腌臢]]|挑剔||
|蹺|湊巧|「蹺」音kiu2||孤寒|吝嗇|||[[酸醙]]|酸臭|「[[醙]]」音suk1|
|郇[13]、筍|正(貨)、好(東西)|例如「郇貨」、「郇盤」||盲舂舂|亂闖亂撞|||光掁掁、掁眼|明亮刺眼|   |
|好彩|幸運|||穿煲|露陷|||[[孬攪、撈絞]]、擇使|難處理||
|光|亮|||滐|稠|||水過鴨背|善忘、健忘||
|穩陣|靠譜|||闊|寬|||[[蠱惑]]|滑頭||
|[[娿哿]]|優柔寡斷|||山旯旮|偏僻|||得戚|得意||
|[[㪐㩿]]|斷斷續續|||標青|出眾|||[[捩咁𠾍]]|一塌糊塗||
|贔屭|一籌莫展||||||||||
|水皮|差勁|||||||硬頸|固執||
|眼冤|心煩|||肉緊|緊張|||心郁郁|心動||

  

副詞[編輯]

|              |             |                                        |     |             |                        |                         |     |           |          |                            |
| ------------ | ----------- | -------------------------------------- | --- | ----------- | ---------------------- | ----------------------- | --- | --------- | -------- | -------------------------- |
| **粵**        | **京**       | **註**                                  |     | **粵**       | **京**                  | **註**                   |     | **粵**     | **京**    | **註**                      |
| 明明、分明、擺明、明框  | 明明、分明       |                                        |     | 真箇、真個       | 確實                     | 「確實」做形容詞嘅時候,唔用「真箇」、「真個」 |     | 唔         | 不        |                            |
| 無、冇          | 沒、沒有、無(書面語) | 「無」係「冇」嘅本字,「沒(有)」重可以做動詞,粵語同樣係「冇」、「無」   |     | 仲、重         | 還                      | 「重」係「仲」嘅本字              |     | 又         | 又、也      | 「也」唔同嘅意思分別對應「又」、「亦」、「都」三個字 |
| 倒            | 約           | 位置唔同,例如普通話「約三斤重」對應粵語「三斤重倒」             |     | 約莫,大概,上下,左右 | 大約,大概,左右,約莫(少用),上下(少用) |                         |     | 都         | 都、也      |                            |
| 咪、唔好(合音mou2) | 別、不要        | 咪讀mai3,可能係「勿」嘅變音,另外咪重可以讀mai6,「唔係」嘅合音字。 |     | [話晒](話曬),點講 | 好歹(不管怎樣,無論如何)          |                         |     | 亦、亦都      | 也、亦(書面語) |                            |
| 幾……          | 多……         | 喺形容詞前便                                 |     | 幾多          | 多少                     |                         |     | 幾時        | 甚麼時候、幾時  |                            |
| 未            | 還沒有、尚未      |                                        |     | 先           | 才                      | 「先」成日擺喺句尾,用法比「才」多變      |     | 特登        | 故意       |                            |
| 梗係/更係        | 當然          |                                        |     | 好           | 很                      | 喺形容詞前便                  |     | 搏命        | 拚命       |                            |
| 千祈           | 千萬          | 一般話「千祈唔好……」,普通話係「千萬不要……」               |     | 夾硬 、監硬      | 強逼、強制                  |                         |     | 禁(用、著)    | 耐(用、穿)   | 「禁」音kam1                   |
| ……過龍、……過籠    | ……過頭        |                                        |     | 一齊          | 一起                     |                         |     | 第二啲、第啲、遞啲 | 其他       | 睇埋#時間入面「遞時」註解              |
| 實            | 一定          |                                        |     | 不留          | 從來、一向                  | 「留」讀lau1                |     | 成日        | 整天、總     | 「成」讀seng4                  |
| 好彩           | 好在、幸好、幸虧    |                                        |     | 淨係、剩係       | 只是、只有、單是、光是            |                         |     |           |          |                            |

  

量詞[編輯]

|   |   |   |   |   |   |   |   |   |   |   |
|---|---|---|---|---|---|---|---|---|---|---|
|**粵**|**京**|**註**||**粵**|**京**|**註**||**粵**|**京**|**註**|
|樖|棵|棵字廣東話讀fo2||嚿|塊/件|||啲|些|本字或為「尐」|
|坺|堆|||些小、丁咁多,丁doe1、一啲|少許、一點(兒)、些小(書面語)|||橛|段|「橛」原來指「小段」[14],不過喺而家嘅粵語裏面泛指「段」。|
|種、亭、隻|種|此情形亭字讀ting2或寫作「挺」||碌|條|||𠹻|股|味道|
|啖|口|||埲、幅|堵、道|牆||對|雙|筷子、鞋|
|羹|匙|鹽||兜|碗|糖水||隻|只|蛋|
|笪|處|污漬||棚|排|牙||浸|層|油|
|餐|頓|飯||篤|坨|糞便||拃|把|米|

  

介詞[編輯]

|   |   |   |   |   |   |   |   |   |   |   |
|---|---|---|---|---|---|---|---|---|---|---|
|**粵**|**京**|**註**||**粵**|**京**|**註**||**粵**|**京**|**註**|
|由|從、自、由|||向|往、朝|||喺、向(音「響」)|在||
|將|把|||令、令到、使、使到|使、使得、讓|||同|和、同(少用)、跟、給|「給」:我同你做嘢(我給你辦事)|
|俾、畀|被/讓/用||||||||||

  

連詞

|                 |        |         |     |          |       |       |     |                              |             |       |
| --------------- | ------ | ------- | --- | -------- | ----- | ----- | --- | ---------------------------- | ----------- | ----- |
| **粵**           | **京**  | **註**   |     | **粵**    | **京** | **註** |     | **粵**                        | **京**       | **註** |
| 同、同埋、連埋、以及、與及、與 | 和、及    | 「與」比較正式 |     | 或、或者     | 或     |       |     | 但係、之不過、不過                    | 但是、不過       |       |
| [就算](就算)、即使(正式) | 即使、即便  |         |     | 於是乎,於是   | 於是    |       |     | 卒之                           | 終於          |       |
| 即係              | 就是、意思是 |         |     | [唔通](唔通) | 難道    |       |     | 如果(有人讀成jyu4 gu2)、若果、若然、倘若、假如 | 如果、假如、倘若、假若 |       |
| 尤其係             | 尤是     |         |     | 費事       | 免得    |       |     | 定係                           | 還是          |       |
| 一係              | 要麼     |         |     |          |       |       |     |                              |             |       |

  

助詞[編輯]

|   |   |   |   |   |   |   |   |   |   |   |
|---|---|---|---|---|---|---|---|---|---|---|
|**粵**|**京**|**註**||**粵**|**京**|**註**||**粵**|**京**|**註**|
|嘅|的|||噉|地|||到|得||
|咗|了|本字「徂」||緊|着、正在 (用於動詞前邊)|「緊」同「住」都對應普通話嘅「着」,但係意思唔同||親(嚫)|(無對應詞,表示動作已經完成且對現在有影響)、到|例:整親、跌親|
|零[15]|來、把(用在數詞後表示約數)|「零」讀leng4||住|着|||||

  

**語氣助詞**[編輯]

|   |   |   |   |   |   |   |   |   |   |   |
|---|---|---|---|---|---|---|---|---|---|---|
|**粵**|**京**|**註**||**粵**|**京**|**註**||**粵**|**京**|**註**|
|咩|嗎|表示懷疑或疑問||啩|吧(表猜測)|||喎|啊(表強調或不滿)|音wo3|
|嘞/喇|吧(表催促)、了(表完成)|「嘞」讀laa3, 「喇」讀laa1||啝|類似「的啊」(表輕微訝異)|「啝」讀wo5||噃|(表提醒或勸告,無對應詞)|「噃」讀bo3|
|囉|唄(表勉強)、嘛(表明顯)|「囉」讀lo1||啫|而已、罷了|||㖭/𠻹|(無對應詞)|表遞進,普通話通常講「還……」|

  

方向[編輯]

|   |   |   |   |   |   |   |   |   |   |   |
|---|---|---|---|---|---|---|---|---|---|---|
|**粵**|**京**|**註**||**粵**|**京**|**註**||**粵**|**京**|**註**|
|前頭、前邊、前面、前便[16]|前面、前邊|||下頭、下邊、下面、下便|下面、下邊|||左手邊、左邊、左便|左邊、左面||
|右手邊、右邊、右便|右邊、右面|||出便、出面、外便、外面|外面|||入便、入面、裏面、裏便、裡便、裡面|裡面||
|隔籬|隔壁、旁邊|||側邊|旁邊||||||

  

時間[編輯]

|   |   |   |
|---|---|---|
|**粵**|**京**|**註**|
|家下、而家、依家、家陣|現在、眼下||
|一陣、陣間、一陣間|一會兒||
|一排、一牌|一陣子(通常多過一日)||
|傝正、搭正、踏正|整點||
|一個字|5分鐘|指鐘面嘅數目字相距嘅時間。|
|一個骨|一刻||
|十二點踏(搭/傝)四、十二點四個字、十二點四|12時20分|「踏四」即係踏正之後過咗四個字(廿分鐘)。呢個情況踏、搭、傝都讀低入聲(daap9)。|
|頭先、先頭、正話、啱啱、啱先|剛才||
|嗰陣|的時候||
|收尾、後尾|後來||
|今日|今天||
|聽日|明天|「聽」音ting1。|
|琴日、禽日、尋日|昨天||
|琴晚、禽晚、尋晚|昨晚||
|舊年|去年||
|出年|明年||
|今朝、今朝早|今早|「朝」音ziu1。|
|聽朝、聽朝早|明早|「聽」音ting1;「朝」音ziu1。|
|遞時、第時、第日、第史、第二時、第二史|換個時間/第二天/改天|本字係「遞」(《廣州語本字》卷二)。「史」字由「歷史」呢個詞借用過嚟,攞佢個時間意義。|

  

時份[編輯]

|   |   |   |   |   |   |   |   |   |   |   |
|---|---|---|---|---|---|---|---|---|---|---|
|**粵**|**京**|**註**||**粵**|**京**|**註**||**粵**|**京**|**註**|
|挨晚|傍晚|||晏晝、下晝|下午|||上晝|上午||
|晚黑、晚頭黑、夜晚|晚上、夜晚|||挨晚|黃昏|||日頭|白天||
|朝(早)、朝頭早|早(上)|「朝」音ziu1|||||||||

  

日常用語[編輯]

|   |   |   |
|---|---|---|
|**粵**|**京**|**註**|
|好話、過獎、言重、見笑|謝謝、誇獎了|受人稱讚|
|唔該|謝謝|受行為之恩|
|多謝|謝謝|受物件之恩|
|早晨|早上好、早安||
|早敨、早唞|晚安|音zou2 tau2,有時第二個字打唔出,就打成「早透」。|
|再會、請請、後會有期、拜拜|再見||
|慢慢行|慢走||
|唔送咯喎|不送了||
|對唔住、唔好意思|對不起、不好意思||
|唔緊要、冇所謂、冇相干|沒關係||
|唔使客氣、乜說話呀、唔使唔該、舉手之勞啫|不用客氣||
|借過、借歪、借借|讓一下|歪音me2[17]。通常喺前邊加個「唔該」。|

  

數詞[編輯]

|   |   |   |
|---|---|---|
|**粵**|**京**|**註**|
|廿-|二十-|音jaa6。「二十幾」都有人用,但係嚴格嚟講,最正宗嘅係「廿幾」[18]。|
|卅-|三十-|音saa1,連讀時音調可能會有變化。「三十幾」都有人用,但係嚴格嚟講,最正宗嘅係「卅幾」[18]。|
|卌-|四十-|音sei3,連讀時可能會變成類似「se-a」嘅發音。「四十幾」都有人用,但係嚴格嚟講,最正宗嘅係「卌幾」[18]。|

  

網絡用詞[編輯]

從2008年開始,由於中國北京同臺灣高雄先後舉辦過奧運同世運,令網絡資訊越嚟越發達,加上一堆詞都出自北話嘅媒體,令未翻譯嘅新詞入侵粵語,導致相當一部分2000年代同2010年代出世啲細路唔知點用粵語表達。以下係內地同臺灣用嘅部分網上用詞啲粵語表達方式[19]:

|   |   |   |
|---|---|---|
|**粵**|**普通話同國語嘅熱門網絡詞**|**註**|
|咦咦唈唈|打情罵俏||
|着草啦,契弟!|奔跑吧,兄弟!|出自浙江衛視《奔跑吧兄弟》同《奔跑吧》系列節目|
|搵笨/坤水|中國大陸:坑爹<br><br>臺灣:耍花樣||
|好㜺<br><br>zaan2<br><br>鬼|萌萌噠<br><br>美美哒||
|有冇|有木有|2011年中國大陸網絡流行詞|
|暈得一陣陣/頭都大埋|也是醉了|   |
|O嘴/O嗮嘴|驚呆了|2012年中國大陸網絡流行詞|
|玩嘢/五條煙|(中國北方)忽悠<br><br>(臺灣)耍花樣||
|唔衰攞嚟衰|不作死就不会死<br><br>no zuo no die||
|細路仔唔識世界|图样图森破|「too young」同「too simple」嘅音譯,係江澤民鬧香港記者時講嘅|
|無啦啦,多噠瘌。|躺着也中枪||
|大花洒/耕濫嗮|有钱(就是)任性<br><br>有钱就是大爷|2014年4月初喺浙江寧波發生嘅45萬人民幣嘅銀行轉賬詐騙案|
|搏嗮(老)命|也是蛮拼的||
|心悒/贔屭|心塞|2014年中國大陸熱門網絡詞|
|擁躉、Fans、番薯|粉丝|出自2005年湖南衛視嘅《超級女聲》節目|
|鍾意/中意|么么哒|2014年中國大陸熱門網絡詞|
|定檔|稳稳的|2015年中國大陸熱門網絡詞|
|撐場/奸爸爹|打call|日本啲演唱會嘅直播文化嘅「コール」口號|
|裝彈弓/裝彈氹|套路|競技網遊術語,2016年中國大陸熱門網絡詞。|
|心照|心里没点逼数|山東話膠東方言片|
|玩嗮/玩澌、噉都得|还有这种操作<br><br>还带这种操作|2017年中國大陸熱門網絡詞|
|老屎忽|老司机|競技網遊術語,源自雲南民歌《老司機帶帶我》嗰首。|
|擔櫈(仔)霸頭位/七加一|吃瓜群众|2016年網絡流行詞,中國大陸網絡或線上論壇覆帖常用語之一。|
|個心揦<br><br>laa2<br><br>住揦住|蓝瘦香菇|臭青南普方言,出自「難受想哭」嘅諧音。2016年10月9號由廣西南寧人——韋勇喺失戀之後發佈嘅直播短片而流行[20]。|
|舐腳趾、舐鞋底、舐春袋、舐袋底|跪舔||
|眼冤/冇眼睇|辣眼睛|2016年4月23號劉梓晨喺美拍上傳嘅短片入面。|
|苦過Dee Dee|苦逼||
|廣州:廢柴<br><br>香港:毒撚/毒L/宅男|屌丝|2011年10月由雷霆三巨頭吧啲人整出嚟,源自百度貼吧嘅李毅吧同雷霆三巨頭吧鬧人戰鬥,後嚟喺2012年初逐漸流行成中國大陸熱門網絡詞。|
|動L|duang/DUANG|出自bilibili網站網友——绯色toy喺2015年2月20號發佈嘅「【成龙】我的洗发液」惡搞短片[21],後嚟俾同站網友「泪腺战士」喺新浪微博轉發之後就猛咁喺網站流行。|
|好犀利|猴賽雷|出自《2016年央視春晚》廣州演區嘅部分北佬節目表演團隊啲細路口音|
|批個頭落嚟畀你當凳坐|你们尽管XX了算我输||
|閒人:你好衰㗎<br><br>大頭:打鑊金|小拳(拳)锤你胸口||
|重有冇良心㗎/冇心肝|你的良心不会痛吗||
|串/窒|怼/diss|2017年網絡流行詞,出自英文單詞disrespect嘅翻譯。|
|果然有錢就大嗮<br><br>有錢想點就點<br><br>有錢真係為所欲為[22]|贫穷限制了我的想象力|2017年網絡流行語句|
|雞同鴨講|尬聊|出自2016年5月嘅中國大陸兒童題材劇集——《舞法天女》,後嚟通過網絡傳播而大幅度流行[23]。|
|(形容好崖广)走鬼啊<br><br>(形容好嘢)犀利/冇得彈|666666...|2017年中國大陸網絡流行詞。出自騰訊屬下網遊——《英雄聯盟》嘅網友對話——普通話「溜」嘅諧音字。|
|No. 1/第一|中國大陸:吃鸡<br><br>臺灣:吃烤雞|源自2017年電影《鬥智21點》嘅台詞——「WINNER WINNER, CHICKEN DINNER!」嘅中文翻譯[24][25]。後嚟喺遊戲《食雞》入面出現咗「WINNER WINNER, CHICKEN DINNER!」嘅詞而開始猛咁流行。|
|犀飛利|厉害了|2018年3月嘅CCTV-2電影作品《厉害了,我的国》。「利」好多時發音「啤李」個「李」。|
|笑餐懵/笑餐飽/笑爆嘴/笑爆肚/笑到鬼|23333...|中國大陸網絡流行詞。出自貓撲網論壇表情包嘅第233張相。|
|叼/仆(街)|次奥|中國大陸嘅北話粗口,喺中國大陸嘅網上常用。|
|(形容愛情)錫啖<br><br>(形容其它嘢)——|么么哒/摸摸大/摸摸打|2010年中國大陸網絡流行詞。源自2010年湖南衛視嘅《快樂男聲》節目季軍歌手——武藝嘅其一口頭禪。|
|懵咗|懵圈(儿)|北京話嘅口頭禪|
|密實姑娘假正經[26]|綠茶婊|中國大陸網語,泛指外貌清純,實質生活糜爛,思想拜金,扮到楚楚可憐,但善於心計,靠出賣肉體上位嘅妙齡女子。該詞出自2013年海南三亞舉辦「海天盛筵」,大批o靚模參加,陪睡3日可得60萬元人民幣報酬,網友因此發明「綠茶婊」一詞加以譏諷,其使用範圍亦已不局限於海天盛筵事件。|
|衰咗|凉凉/凉了凉了|2017年11月中國大陸新出網絡流行詞。最初出現喺直播間,後嚟好快喺貼吧、朋友圈同微博入面猛咁出現。|
|熱天落冚大雨|下开水|2017年夏天中國大陸新出網絡流行詞。最先喺廣東省氣象廳發佈嘅高溫同暴雨警報信號。|
|搏嗮命/瞓嗮身/盡嗮力[27]|洪荒之力|2016年夏季奧運會期間,傅園慧第一次接受央視採訪時出現嘅口頭禪。|
|玩嗮啦[27]|无FUCK(可)说|源自「無話可說」的方言諧音,個詞出自新浪微博博主——十一月末君喺2017年2月13號喺新浪微博發佈嘅所謂「自帶英文餸名」[28]。|
|攬埋一齊死[27]/大家一齊死|来啊,互相伤害啊!|2017年秋天新出嘅網絡流行词。最先啲網友用做黃玲喺2007年發佈嘅《痒》呢首歌嘅「來啊 快活啊 反正有大把時間」嘅歌詞嚟猛咁惡搞,改成「來啊 XXX啊」呢啲類似語句品種[29]。|
|冇嗮符[27]|我能怎么办,我也很绝望啊。<br><br>我能怎么样,我真的很绝望。|源自TVB8播出嘅劇集——《》嘅台詞。|
|哨牙佬/哨牙仔|龅牙哥|2011年3月喺網絡拍客出現嘅相,後嚟經過PS處理之後出現好多惡搞版。|
|花靚倞|小奶狗|形容啲好靚仔兼好細個、好得意嘅男仔。個詞出自新浪微博博主——莫里___(福建泉州晉江人)喺2017年12月17號發佈嘅微博文章。|
|好嘢/好波|中國大陸:skr<br><br>臺灣:絲哥兒|中國大陸2018年網絡流行詞,出自2018年7月16號《中國新說唱》節目入面,吳亦凡嘅口頭禪[30][31]。|
|淆底|怂|2018年嘅網上流行詞|
|毒撚|钢铁直男||
|好緊要,所以要講n次。|重要的事情说三遍||
|係愛呀哈利|真愛啊/好有愛|出自2018年嘅電影小說——《哈利波特魔法石》入面,喺哈利波特打死怪獸之後出現嘅對話。|
|好自私/唔生性|巨婴||
|埋嚟睇、埋嚟揀|這邊有東西出售,請過來看,挑選合心意的。|舊史街邊攤檔,小販、檔主喊叫呢個詞令到路人注意佢嘅商品,體現第一代粵式「應用市場學」,較為硬銷,發展到而家,越嚟越少街邊喊叫推廣,視覺視效軟性手法成熟亦較為大眾所接受。|