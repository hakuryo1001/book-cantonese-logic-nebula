戾

  



粵語

**發音**

粵拼:**lai⁶**

**寫法**

- 簡體/繁體:**戾**

**含義**

- (本義)彎曲
- (引申)罪過、到、至

出處

- 《說文解字》:曲也。從犬出戶下。戾者,身曲戾也。
- 《孟子·滕文公上》:樂歲粒米戾也。
- 趙岐註說:狼戾,猶狼藉也