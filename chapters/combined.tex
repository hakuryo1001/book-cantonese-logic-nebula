% \chapter{Combined Materials}

% This chapter contains all material files combined into one document.

% \newpage

% \section{Cantonese Tense}

% https://www.reddit.com/r/CantoneseScriptReform/comments/18p1j7i/cantonese_linguistics_re_aspect_and_the_lack_of/ 
% Tense

% Cantonese has no tense system. Verbs have no inflection, that means they do not conjugate.

% [Tense and Time are two different concepts](http://www.beniculturali.unipd.it/extra/dispense%26materiale/dispense_materialepubblico/Lingua%20inglese/2013-14%20-%20Lezione%203%20ottobre%202013%20-%20Time%2C%20tense%2C%20and%20aspect.pdf). Time is basically our real-world perception of _when something happens_, but tense is the grammaticalized expression of location in time. Compare the following sentences:

% > Yesterday I **ate** cartnoodles.  
% > 尋日我**食**車仔麵。  
% > _Cam4jat6 ngo5_ _**sik6**_ _ce1zai2min6._

% > Today I **eat** rice.  
% > 今日我**食**飯。  
% > _Gam1jat6 ngo5_ _**sik6**_ _faan6._

% "Yesterday”, “today”, “尋日”, “今日” are _adverbs of time_; they are **lexical**. They show _time_.

% “Eat” and “ate” are different **grammatical** forms of the verb “to eat”. They show _tense_.

% “食” in both sentences is not inflected.

% Note the example sentences referring to the past do not require the use of aspect markers that has past meaning e.g. _zo2_ and _gwo3_.

% Aspect

% Aspect is the speaker's notion that enables the same situation to be viewed in different ways, i.e. _how does the speaker look at a situation?_ In English, verbs have three aspects: _simple_, _progressive_ ("-ing") and _perfect_ (=_retrospective_). Aspect should not be confused with _tense_.

% Although Cantonese has no tense system, it does have an aspect system, which puts affixes around the verb. Cantonese has eight aspects.

% **NB** Aspect markers are not to be confused with verbal particles. Verbal particles indicate ideas such as result (effect of an object) and phase of action (beginning, continuing or ending). In some websites, some verb particles are wrongly listed as aspect markers, e.g. 晒 _saai3_ (quantifying particle "all, completely").

% ---

% Perfective 咗 zo2

% (Implies past time reference) Indicates a completed event. It is used to report an event, seen as a whole or as completed.  
% There are three typical uses of the perfective _zo2_, each of them comparable to three different English tenses:

% (1) The resultative meaning. Translates to the English perfect, where the event has a result.

% ```
% 佢      炒咗        部     車。
% Keoi5   caau2zo2    bou6   ce1  
% He      crash-PFV  CL     car
% "He's crashed the car." (The car is a wreck now.)
% ```

% (2) Reporting past events without any such result. Translates to the English simple past:

% ```
% 公司      舊年       賺咗       唔少         錢。
% Gung1si1  gau6nin4  zaan6-zo2  m4siu2       cin2
% company   last-year earn-PFV   not-little   money
% "The company made a good deal of money last year."
% (A statement on the company's performance)
% ```

% (3) Expressing a period of time up to and including the present. Translates to the English perfect progressive:

% ```
% 我    部    車    揸咗       兩      年    幾。
% Ngo5  bou6  ce1  zaa1-zo2   loeng5  nin4  gei2
% I     CL    car  drive-PFV  two     year
% "I've been driving the car for over two years."
% ```

% Replacing _zo2_ with _gwo3_ (experiential aspect) implies that the state of affairs no longer holds.

% The perfective can be combined with adverbs of the past, or adverbs of the recent past such as 啱啱 _ngaam1-ngaam1_ and 頭先 _tau4sin1_.

% ```
% 佢      頭先        炒咗        部     車。
% Keoi5   tau4sin1   caau2-zo2   bou6   ce1. 
% He      just-now   crash-PFV   CL     car
% "Just now he's crashed the car."
% ```

% **NB** The perfective aspect is not to be confused with the "perfect" aspect (nonexistent in Cantonese), which denotes an event with relevance to the point of speech.

% **NB** _Zo2_ can be used in imperative sentences (an order or command) or complement clauses referring to the present or future, and should not be treated as a past tense marker.

% ```
% 食咗      佢     先。
% Sik6-zo2  keoi5 sin1
% Eat-PFV   it    first
% Eat it up. (Imperative)

% 我   想      賣咗         部    車。
% Ngo5 soeng2  maai6-zo2   bou6  ce1
% I    wish    sell-PFV    CL    car
% I want to sell the car. (Complement clauses referring to the present or future)
% ```

% **NB** In rapid speech, _zo2_ may be realised as a tone change.

% **NB** _zo2_ does not occur in negative sentences. The negative existential 冇 _mou5_ is used before the verb.

% ```
% 佢      冇炒         部     車。
% Keoi5   mou5-caau2   bou6   ce1  
% He      not-crash    CL     car
% "He didn't crash the car."
% ```

% ---

% Experiential 過 gwo3

% (Implies past time reference) Similar to present perfect in English. It implies that the situation took place prior to the point of speech or reference. It suggests experience, or something having occurred "at least once before".

% ```
% 我   學過        紮鐵。  
% Ngo5 hok6-gwo3   zaat3tit3.  
% I    learn-EXP  steel fixing.
% "I have learnt steel fixing before."  
% ```

% Another use of the experiential aspect is the "indefinite past", which is common for non-human subjects. Similar to the English perfect.

% ```
% 部    電腦      壞過         幾   次
% Bou6  din6nou5  waai6-gwo3  gei2  ci3
% CL    computer  break-EXP  few  time
% "The computer has crashed a few times before."
% ```

% Another use of _gwo_ is "inferential perfect", where the speaker infers from the evidence available that something has happened:

% ```
% 好似      落過       雨    喎。
% Hou2ci5  lok6-gwo3  jyu5  wo3.
% "It seems to have been raining."
% ```

% The experiential can be combined with adverbs of the distant past such as 以前 _ji5cin4_ and 曾經 _cang4ging1_.

% ```
% 我   曾經         學過       紮鐵。  
% Ngo5 cang4ging1  hok6-gwo3  zaat3tit3.  
% I    previously  learn-EXP  steel fixing.
% "I have learnt steel fixing before."  
% ```

% **NB** The difference between _zo2_ and _gwo3_ is usually whether a result of the event holds at the time of speaking (perfective) or not (experiential).

% **NB** _Gwo3_ also has other uses than an aspect marker.

% ---

% Progressive 緊 gan2 / 喺度 hai2dou6

% For dynamic, ongoing actions **only**; it implies change over time, i.e. the action is not timeless. By default, _gan2_ applies to the present unless indicated otherwise. Similar to Progressive "-ing" in English.

% ```
% 啲  學生        上緊網。  
% Di1 hok6saang1  soeng5-gan2-mong5  
% CL  student     get on-PROG-Internet
% "The students are surfing the web."  

% 黃     生      嘆緊         杯    鴛鴦。  
% Wong4  saang1  taan3-gan2   bui1  jin1joeng1
% Wong   Mr      enjoy-PROV   CL    milk coffee
% "Mr Wong is enjoying a cup of milk coffee."  

% 噖晚        黃      生      嘆緊        杯    鴛鴦。  
% Kam4-maan5  Wong4  saang1  taan3-gan2  bui1  jin1joeng1.
% Last-night  Wong   Mr      enjoy-PROV  CL    milk coffee
% "Last night Mr Wong was enjoying a cup of milk coffee."  
% ```

% _Hai2dou6_ (lit. _be here_) has the same function, but it precedes the verb:

% ```
% 啲  學生        喺度       上網。  
% Di1 hok6saang1  hai2dou6  soeng5-mong5 
% CL  student     be-here   get on Internet
% "The students are surfing the web."  
% ```

% _Hai2dou6_ can be used together with _gan2_ to reinforce the progressive meaning:

% ```
% 啲  學生        喺度       上緊網。  
% Di1 hok6saang1  hai2dou6  soeng5-gan2-mong5  
% CL  student     be-here   get on-PROG-Internet
% "The students are surfing the web."  
% ```

% ---

% Continuous 住 zyu6

% Denotes a continuous activity or state without change, typically present or timeless. No strict equivalent in English.

% ```
% 啲   雲     遮住         個   太陽 
% Di1  wan4   ze1-zyu6     go3  taai3joeng4.
% CL   cloud  block-CONT   CL  sunlight.  
% "The clouds are blocking out the sunlight."  
% ```

% **NB** Some verbs are frequently used with _zyu6_, such as 對住 _deoi3-zyu6_ ("face"), 趕住 _gon2-zyu6_ ("in a hurry"), 掛住 _gwaa3-zyu6_ ("miss"), 趕住 _gon2-zyu6_ ("in a hurry"), 揸住 _zaa1-zyu6_ ("keep hold of"), 阻住 _zo2-zyu6_ ("block, obstruct"), 望住 _mong6-zyu6_ ("stare at").

% **NB** In a few cases, some [verb _zyu6_] combination has slightly different meaning from the simple verb, e.g. 諗 _lam2_ ("think") vs 諗住 _lam2-zyu6_ ("intend").

% **NB** _zyu6_ is also used in serial verb constructions to denote simultaneous activities. _zyu6_ can be used as a particle at the end of a clause to mean "yet". It can also be used together with 先 _sìn_ to mean "for the time being".

% ---

% Delimitative 吓 haa5

% Has the meaning "do ... for a while" or "have a ...". It is typically used with verbs denoting activities, with or without an object.

% ```
% 行吓 
% hang4-haa5
% walk-DEL  
% "take a walk"  

% 飲吓茶 
% jam2-haa2 caa4 
% drink-DEL-tea
% "have some tea"  
% ```

% Another equivalent construction is "_... jat1 ..._", which is often contracted to become a tone change. A similar construction "_... leong5 ..._" has no contracted form.

% ```
% 行一行 
% hang4-jat1-hang4

% 行行 
% hang4→2-hang4

% 行兩行 
% hang4-leong5-hang4
% "take a walk"  
% ```

% Reduplication of verb followed by _haa5_ indicates that the action is prolonged or repeated. It translates to the present perfect progressive in English.

% ```
% 我   諗諗吓,          都   係   買     多     打   好。
% ngo5 lam2-lam2-haa5   dou1 hai6 maai5  do1   daa1  hou2
% I    think-think-DEL  also is   buy    extra dozen good.
% "I've been thinking, it's best to buy a dozen more."
% ```

% In serial verb constructions that express two simultaneous actions, the reduplicated verb denotes an interrupted action.

% **NB** An idiomatic use of the delimitative can be found in the [verb _haa5_ ... _sìn_] construction, which means "to do something temporarily", especially for killing time.

% ```
% 你   望吓        食乜      先,  轉頭       同    你   落單。
% nei5 mong6-haa5 sik6-mat1 sin1  zyun2tau4 tung4 nei5 lok6-daan1
% You  look-DEL   eat-what  first turn-head with  you  place-order
% "Have a look at what to eat first, (I'll) be back soon to place the order for you." 
% ```

% **NB** The idiom 眨吓眼 _zaam2-haa6-ngaan5_ is an adverbial phrase meaning "in the blink of an eyelid".

% ---

% Inchoative 起上嚟 hei2-soeng5-lai4

% _Seong5-lai4_ is a directional component meaning "come-up", but as an aspect marker, _hei2-seong5-lai4_ means "begin". There is no equivalent aspect in English that has the same meaning.

% ```
% 棟    樓           突然       燒   起上嚟。  
% dung6 lau4        dat6jin4  siu1 hei2-soeng5-lai4  
% CL    building    sudden    burn rise-up-come
% "The building suddenly went up in flames."  
% ```

% _Seong5_ is sometimes omitted.

% ---

% Continuative 落去 lok6-heoi3

% It is a directional complement that means "go down", but by extension it is an aspect marker meaning "continue".

% ```
% 咁    樣     燒    落去,      棟    樓         遲早       冧。  
% gam3  joeng6 siu1 lok6-heoi3  dung6 lau4      ci4zou2    lam3  
% such  way    burn continue    CL    building  late-early collapse
% "Sooner or later, the building is going to collapse if it keeps on burning."  
% ```

% ---

% Habitual 開 hoi1 / 慣 gwaan3

% Denotes habitual, customary activity. Unlike the English _used to_ construction, it is not restricted to past time. In fact, it is typically used of the present. It can be used together with an adverb of time.

% ```
% 佢    睇開       中       醫      嘅  
% keoi5 tai2-hoi1 zung1    ji1     ge3
% S/he  see-HAB   Chinese  doctor  SFP
% "S/he usually goes to a Chinese doctor."  

% 佢    以前      睇開       中       醫      嘅  
% keoi5 ji5cin4  tai2-hoi1 zung1    ji1     ge3
% S/he  before   see-HAB   Chinese  doctor  SFP
% "S/he used to go to a Chinese doctor."
% ```

% _Gwaan3_ can also be used to indicate habitual aspect, with the more specific meaning "accustomed to":

% ```
% 佢    食  慣        貴          嘢
% keoi5 sik6 gwaan3   gwai3      je5
% S/he  eat used      expensive  thing 
% "S/he is used to eating expensive food."
% ```

% ---

% Source: Matthews, S., & Yip, V. (2011). _Cantonese : A comprehensive grammar_ (2nd ed., Routledge comprehensive grammars). London: Routledge.

% ---

% For learners, I recommend the following book to learn more about Cantonese grammar:

% Yip, V., & Matthews, S. (2017). _Basic Cantonese: A Grammar and Workbook_: Taylor & Francis.

% Edit: formatting Edit 2:. Fixed jyutping errors

% \newpage

% \section{Godel's Incompleteness theorem - Cantonese}

% #metaphysics #logic
% 嗱,首先呢,當我話「證明咗」嗰陣呢,我嘅意思係「喺冚數學嘅幫助下證明咗」。噉你知啦:兩加二就係四。而且,當然啦,係可以證明倒兩加二係四嘅。(證明倒姐係話呢,孖我啱先所講,姐係喺冚數學嘅幫助下,做嘅話就會遲早證明咗,雖然喺兩加二嘅呢個畸士下,當然我哋唔使要曬冚數學去證明佢係四啦)。兼且呢,講明埋啦,因為可能唔係零舍明顯,就係呢,係可以證明倒埋係可以證明倒兩加二係四嘅。兼且呢,係而且可以倒埋,係可以證明倒,係可以證明倒,二加二係四。如此類推。而事實上呢,如果一句野講呢係證明倒嘅話,噉就證明倒呢句野講係證明倒架嘞。而呢個亦都係證明倒嘅。嗱,家陣呢,兩加二唔係五啦。又且證明倒兩加二唔係五。再兼且證明倒,可以證明二加二唔係五,依樣咁一直證明倒落去到。所以呢:係證明倒兩加二唔係五嘅。嗱,咁可唔可以證明倒兩加二係五呢下?嗱,如果可以嘅話呢,呢家野就係對數學嚟講係冚家富貴。如果可以證明倒二加二係五,噉就證明倒五唔係五,噉就唔會有話證明唔到嘅嘢講倒,咁數學就會堅係一大堆嘅膠噏。咁,我哋而家想問嘞,係咪可以證明倒證明唔倒二加二係五?嗱,呢吓就係銀呀個位嘞:嗱,係唔可以嘅。或者話啦,要對沖帶下頭盔啦:如果可以證明倒證明唔倒兩加二係五,噉就可以證明倒兩加二係五,咁數學就會堅係一大堆嘅膠噏。而再講深一層呢,如果數學唔係堅係一大堆嘅膠噏嘅話呢,噉就冇「嘢講 乜乜乜 證明唔到」呢個形式嘅嘢講係證明倒架喇。所以呢,如果數學唔係堅係一大堆嘅膠噏嘅話呢,咁,雖然證明唔到二加二係五,但亦係都證明唔到兩加二係五嘅。嗱,順便講下啦,萬一你想知嘅話:係,係可以證明倒,如果係證明倒係唔可以證明倒二加二係五,噉就可以證明倒二加二係五㗎嘞。


% 嗱,首先呢,當我話「講得[通](通)」嗰陣呢,我嘅意思係「喺[冚](冚)數學嘅幫助下講得通」。噉你知啦:兩加二就係四。而且,當然啦,兩加二係四係講得通嘅。(講得通姐係話呢,[孖](孖)我啱先所講,姐係喺冚數學嘅幫助下,做嘅話就會遲早講得通,雖然喺兩加二嘅呢個畸士下,當然我哋唔使要曬冚數學去講通佢係四啦)。兼且呢,講明埋啦,因為可能唔係零舍明顯,就係呢,係可以講得通倒埋係可以講得通倒兩加二係四嘅。兼且呢,係而且講得通倒埋,係可以講得通倒,係可以講得通倒,二加二係四。如此類推。[而事實上呢](),如果一句野講呢係講得通倒嘅話,噉就講得通呢句野講係講得通倒架嘞。而呢個亦都係講得通倒嘅。嗱,家陣呢,兩加二唔係五啦。又且講得通倒兩加二唔係五。再兼且講得通倒,可以講得通二加二唔係五,依樣咁一直講得通倒落去到。所以呢:係講得通倒兩加二唔係五嘅。嗱,咁可唔可以講得通倒兩加二係五呢下?嗱,如果可以嘅話呢,呢家野就係對數學嚟講係冚家富貴。如果可以證明倒二加二係五,噉就講得通倒五唔係五,噉就唔會有話講唔通到嘅嘢講得出,咁數學就會[堅](堅)係一大堆嘅膠噏。咁,我哋而家想問嘞,係咪可以講得通倒講通唔倒二加二係五?嗱,呢吓就係銀呀個位嘞:嗱,係唔可以嘅。或者話啦,要對沖帶下頭盔啦:如果可以講得通倒講通唔倒兩加二係五,噉就可以講得通倒兩加二係五,咁數學就會堅係一大堆嘅膠噏。而再講深一層呢,如果數學唔係堅係一大堆嘅膠噏嘅話呢,噉就冇「嘢講 乜乜乜 證明唔到」呢個形式嘅嘢講係證明倒架喇。所以呢,如果數學唔係堅係一大堆嘅膠噏嘅話呢,咁,雖然講得通唔到二加二係五,但亦係都講得通唔到兩加二係五嘅。嗱,順便講下啦,萬一你想知嘅話:係,係可以講得通倒,如果係講得通倒係唔可以講得通倒二加二係五,噉就可以講得通倒二加二係五㗎嘞。

% \newpage

% \section{In fact}

% #logic #metaphysics

% \newpage

% \section{Untitled 2}

% (Empty file)

% \newpage

% \section{Untitled}

% (Empty file)

% \newpage

% \section{for}

% #介詞連詞
% [話曬](話曬)

% \newpage

% \section{idea}

% 
% [計仔](計仔)

% \newpage

% \section{in}

% (Empty file)

% \newpage

% \section{in_fact}

% (Empty file)

% \newpage

% \section{support}

% [挨依](挨依)

% \newpage

% \section{《俗話傾談》反映的 19 世紀中粵方言特徵}

% #老粵語 #文

% 本文件是呈送編輯的定稿。全文已刊於:戴忠沛(2009)「《俗話傾談》反映的 19 世紀中粵方言特徵」,載錢志安、郭必之、李寶倫、鄒嘉彥編《粵語跨學科研究:第十三屆國際粵方言研討會論文集》,香港:香港城市大學語言資訊科學研究中心,第 245-259 頁。

% # 《俗話傾談》反映的 **19** 世紀中粵方言特徵

% 戴忠沛

% 香港大學

% 《俗話傾談》是目前可考最早的「三及第」小說。學界很早就注意到《俗話傾談》,但一般認為「三及第」著作裏粵語與文言相混,難以用來研究清代粵語。其實《俗話傾談》的性質與後來的「三及第」略有分別,它本來是講生的話本,其中旁白敘述部份雖然文白相混,但人物對話卻幾乎全屬口語,例如以下對話:

% 「做乜叫人刮痧刮得咁淒涼呀?」

% 「刮刮刮刮你個條命。分明係被籐鞭所打,重話我刮痧。」(60)

% 因此,只要將書中的人物對話抽出,即可作為研究當時粵語的材料。這些對話與同時期翻譯得來的粵語材料或作為教材的粵語文句相比,更加通俗生動;而與「粵謳」一類的粵語曲詞相比,則更接近實際口語。本文將整理《俗話傾談》裏的人物對話,研究其中反映的字音、詞彙、語用及語法特徵,期望為清代粵語的研究提供一份新材料。

% **1.**     背景

% 《俗話傾談》輯選者邵彬儒,字紀棠,廣東四會人,生卒生不詳,以說書為業,活躍於同治、光緒年間,是當時廣州、佛山、南海一帶著名的講生,擅長以通俗手法宣講人倫禮教(魯金 1990、黃仲鳴 2002:53、耿淑艷 2007),而《俗話傾談》很可能是在他的話本基礎上輯選而成的。該書共兩集四卷,其中初集收 11 則故事,分別是「橫紋柴」、「七畝肥田」、「邱瓊山」、「種福兒郎」、「閃山風」、「九魔托世」、「饑荒詩」、「瓜棚遇鬼」、「鬼怕孝心人」、「張閻王」、「修整爛命」,二集收七則故事,分別是「骨肉試真情」、「潑婦」、「生魂遊地獄」、「借火食烟」、「好秀才」、「砒霜砵」、「茅寮訓子」。這些故事部份可考以民間傳說或其他小說為藍本,如「橫紋柴」故事原型為《聊齋誌異》的「珊瑚」;「積福兒郎」故事初見於《德育古鑒》等。《俗話傾談》曾出現多種版本。目前所見最早的版本,初集有中國國家圖書館藏同治九年(1870)秋鐫「粵東省城十七甫五經樓」本,及北京大學圖書館藏同治九年(1870)「粵東立經樓」本,而二集最早的是同治九年(1870)春鐫「羊城十七甫五經樓」本。按說初集應於二集前出版,因此同治九年秋出版的初集很可能不是最早的版本。葉春生(1996:318)指《俗話傾談》初版於同治三年(1864)(#_ftn2),我們也不妨推測該書初版於 1860 年代中至末期出版。

% 本文所據的版本,初集為同治九年(1870)秋「粵東省城十七甫五經樓」本,二集為同治九年(1870)春「羊城十七甫五經樓」本,皆已由上海古籍出版社(1994)影印出版。二集「羊城十七甫五經樓」本缺最後一頁,以出版地點不明、同治十年(1871)春鐫版的《俗話傾談二集》配補(中華書局 1990),這版本除了封面和少數內文的旁注直音略有分別,其他皆與「羊城十七甫五經樓」本完全相同。

% **2.**     語音

% **2.1** 直音

% 《俗話傾談》在兩方面提供了當時粵語的語音訊息。首先,該書作為講生的話本,為了應用方便,在部份漢字旁邊注有直音。以下所列各組,前一次為原字,後一字為直音:

% ![](file:////Users/hongjan/Library/Group%20Containers/UBF8T346G9.Office/TemporaryItems/msohtmlclip/clip_image001.jpg)![](file:////Users/hongjan/Library/Group%20Containers/UBF8T346G9.Office/TemporaryItems/msohtmlclip/clip_image001.jpg)填-田(2187/9)、鹵-老(38)、撞-狀(38)、折-節(46)、![](file:////Users/hongjan/Library/Group%20Containers/UBF8T346G9.Office/TemporaryItems/msohtmlclip/clip_image002.jpg)-篤(50)、筶-教(69)、懲-呈(126)、趁-親(225)、螺-羅(251)、褸-慮(254)、倘-湯上聲(268)、遞-弟(270)、瓶-平(277)、馗-葵(281)、羲-希(282)、斥-勅(282)、脊-炙(288)、唆-踈(289)、腔-康(290)、概-蓋(291)、烘-紅(296)、樁-莊(359)、段-斷(359)、畀-彼(370)、撼-坎(383)、鑄-註(388)、鋼-降(388)、膠-交(401)、嘯-笑(401)、桅-為(402)、贏-形(411)、扯-者上聲(412)、筋-根(413)、拼-并(414)、窄-責(415)、蹈-道(418)、駁-博(422)、衙-牙(440)、炊-吹(444)、膥-春(334、

% 445)

% 另有三例直音與原字音明顯不符:

% 攪-叔(2191/13)(#_ftn5)、箸-目(37)、隊-在(280)

% 以下各字,旁邊雖注有直音,但在影印本內直音字模糊不清,姑且存目備考:

% 畀(2218/40、99,從上同例估計直音為「彼」)、釉(37)、顱(44)、擅(52)、膥(84,從上同例估計直音為「春」)、濬(109)、臬(159)、嗽(175)、僱(179)、護(181)、儼(187)、拼(193,從上同例估計直音為「并」)、碓(292)、春

% (292)、嘶(295)、壓(320)

% **2.2** 同音別字

% 其次,透過書中的別字,也可以得知當時部份漢字的讀音。以下各組前一字為原本的別字,後一字為正字,括號內附別字所出的詞語句子及相應頁數:

% 引-癮(烟引 4、酒引 278)、順-遜(出言不順 38)、式-適(看過合式 58)、事-肆(大事辦過衣裝 63)、準-准(太爺不準 142)、便-邊/面(城之東便大起瘟疫 179)、證-症(時證大行 179)、到-道(知到 182、202、230、238、 306、363、417-418)、止-只(你止曉得日日醉 224)、什-拾(收什 239、411)、資-貲(積此資財 232)、办-扮(打办 254、410)、剝-薄(刻剝百姓 285)、氏-侍(妾氏 304、325、328)、稟-品(稟性難易 309、醜稟 431)、桩-裝(桩定身勢 378)、捫-毛(腰下束一條捫巾 313)、跌-鐵(跌鍊鎖住頸上 417)

%             以上直音和別字,雖然未足以讓我們分析當時的粵語音系,但也足見若干字音的演變。例如「螺」今讀 lo2,但《粵音韻彙》讀音只有 lo4,與《俗話傾談》直音為「羅」相合。「贏」據《粵音韻彙》有 jeng4 及 jing4 二讀,今多讀 jeng4,從《俗話傾談》直音為「形」可知當時讀作 jing4。

% **3.** 詞匯

% 在研究清末粵語的材料當中,絕大部份是典籍翻譯,或例句教材。這些材料的用詞,一般都比較文雅。相比之下,《俗話傾談》裏依粵語口語寫成的人物對話,用詞則比較通俗,其中包括了不少當今粵語[[6]](#_ftn6)已不常見或意思已經改變的詞匯,能補其他同期粵語材料之不足,以下略舉數例:

% **3.1** [[色水]]

% 現今一般指貴價金屬或玉器的色澤。在《俗話傾談》裏「色水」似乎有兩個意思,較常用的意思是指女性的外表樣貌,例如:

% (1)    我當初做新婦時,重好色水過你十倍,唔估今日老得個樣醜態,減去三分。(2)

% (2)    唔通六七十歲老大婆重整作咁好色水麼?(410)

% 另一個意思指贓害別人的把戲,或類近今日粵語「畀啲顏色佢睇」的「顏色」,今粵語唯一相關的用法只有「整色整水」:

% (3)    遲數日兩個去探過佢,若係恭恭敬敬,有的禮貌便了,若仍然冷淡,要整佢色水開井水過人食。(236)

% **3.2** [[周致/周至]]

% 此詞現今粵語未見,大槪意思是「仔細」、「妥當」:

% (4)    你估同我地後生,慢慢梳光頭、搽了粉、戴好花,又要扎周致雙脚麼?(21)

% (5)    誰知蚊帳、被裖,樣樣虔潔光鮮,方知珊瑚每日整理周至。(83)

% (6)    父親臨病之時,見我服事得佢周至,話我孝心,父在牀頭,親筆寫云,七畝餘田,交與亞定永遠耕管。(94)

% **3.3** [[抽身抽勢]]

% 現今形容人坐立不安。在《俗話傾談》大槪指人意氣風發、大搖大擺:

% (7)    臧姑凸起眼精曰,我就咒你,你點樣惡法呀?我唔怕惡,共你打清。然後食飯都唔做得,話完即捲起衫袖、扎緊包頭帶,抽身抽勢,裝模作樣,好似猛虎下山想人肉食。(21)

% (8)    二成抽身抽勢,向兜肚內擒出一渣袋,約一百之多。(58)

% (9)    馮氏曰:打都唔怕你。話完即抽身抽勢,扎緊隻髻。(382)

% 與「抽身抽勢」相近的還有「整定身勢」、「桩(裝)定身勢」:

% (10)  各人整定身勢,今日去攞人命呀(東莞叫做食腊鴨飯)。(333)

% (11)  誰不知繼業桩定身勢,扎起髻氏的,繼功亦抽高褲脚,捲實衫袖。(378-379)

% **3.4** [遇時](遇時)

% 此詞今粵語未見,大槪意思同「經常」:

% (12)  二成生得兩個仔,臧姑遇時自己贊好命。(39-40)

% (13)  我前日買定一張單刀,放在床頭,遇時預備要用佢,若真來尋打,就先下手為強,免至受虧一著。(370-371)

% **3.5** [[兇性]]/[[兇勢]]/[[勢兇]]/[[兇橫]]

% 「兇」在今日粵語多獨用作形容詞,在《俗話傾談》經常與其他字組合成詞,意思除了本意「兇惡」外,還可指人「狠心」,略同今粵語詞「[[狼死]]」:

% (14)  適值旁邊有一個婦人,見他如此兇性,即用力擒住他手,盡勢推開,大喝一聲,乜你咁勢兇呀?(10)

% (15)  各人見他咁兇勢,咁撒賴,難以用手相爭。(265)

% (16)  有咁樣惡法,我个新婦既死,巳經傷心不了,重來毁我房屋,散我家私,將我老婆咁樣凌辱,有咁太過兇橫。佢恃拳頭在近,官府在遠麼?(341)

% **3.6** 真正

% 即「真的」、「真是」,同今粵語「真係」,此詞在《俗話傾談》裏出現頻率頗高:

% (17)  暇暇,你個橫紋柴,真正好笑咯,你個仔既寫分書,就如路人,那一個重係你新婦呀?(12)

% (18)  難得咯、難得咯,真正第一好新婦咯。(34)

% (19)  亞哥你真正冇本心,盡將銅銀分過我,你自已要了好銀,我被人捉住搽黑面辦做烏龜,毒打一身,真正唔抵咯。(62)

% (20)  姚氏聽到此話,知係真情,个陣口軟聲低,細聲問曰:亞叔真正嗎?(306)

% 「真正」在今粵語已罕用,但在 60 年代粵語電影還可聽到,可知此詞消失是近年的事。在《俗話傾談》裏「真正」不時與「係」連用,似乎是「真正」過渡至今日「真係」的中間階段:

% (21)  唔通都係銅銀?伯爺真正係唔好人咯,佢听用之銀,聞得俱是好的。我所用係假的,分明欺你愚蠢。(60-61)

% (22)  話起亦有理,今晚我飲酒,食了一砵仔鹹蘿蔔,唔通真正係心躁發夢?(75)

% (23)  睇你唔出做如人咁佮俐呢,你個把嘴真正係審死官咯。(76)

% **3.7** 撞板/梗板/硬板/個一板豆腐

% 在《俗話傾談》裏有一組與「板」有關的詞彙。其中「撞板」雖然在今日粵語依然常用,但意思略有不同。今日「撞板」指人犯錯。在《俗話傾談》裏「撞板」除了「犯錯」的意思,還能形容一件事很糟糕、倒霉:

% (24)  無奈咁撞板,想孝心,老母就死,天不從人願,整定要該衰咯。(90)

% (25)  點算呀?撞板咯,嚇死我兩個仔咯。(100)

% (26)  女呀,你唔好去,個的唔係別樣病,係叫做冇牙老虎。你偏回去,若撞板起來,連你都死乾淨咯。(180)

% 除了「撞板」,在《俗話傾談》還有「硬板」和「梗板」,這兩個詞語在今粵語已甚少出現:

% (27)  父母家財,亦唔係定局,佢話要多的,我作父母剩少的,假如生多幾個兄弟,唔通硬板要翻咁多麼?(37)

% (28)  暇暇,數日之間,又是一場變卦,方信閻王簿上有添有改,都無梗板寫法。(175)

% 另外,《俗話傾談》還有熟語「個一板豆腐」,大槪意同「不是好東西」:

% (29)  內心大驚,[料必](料必)又係個一板豆腐咯。(68)

% (30)  你兩個真正好舉荐,好發財門路,製个板豆腐,打得我死過翻生,真唔抵咯。(247)

% **3.8** 光棍

% 今粵語「光棍」多指單身漢,但在《俗話傾談》卻指騙子,詞意發生了變化:

% (31)  唔係做賊,人家話我做光棍,用假銀買真貨,白白受打一場。(60)

% (32)  向以世上好多周身八寶,計多過米,曉做光棍,曉謀害人,曰撈曰縮,到底依然貧困也。(165)

% **3.9** 心事

% 在《俗話傾談》裏出現與「心事」有關的詞語包括「假心事」、「有心事」、「好心事」,今日粵語「心事」只能用作名詞,一般指負面的想法,但在《俗話傾談》可組成形容詞短語,意思也不一定負面:

% (33)  你勿整成個的假心事來戲弄我。(假心事都勝過有心事)。我知你底子,不是個樣人,不知你聽誰人所教。(26)

% (34)  真正好心事,唔話得咯,算第一個婦人。(61)

% (35)  睇佢心事,好似思疑你做亞哥,瞞騙于佢。(220)

% (36)  到是真咯,唔請你唔嫁,就係你死,我都唔娶。不憂無老婆,難得你唔好心事呀。(250)

% **3.10** 光輝/周身輝

% 《俗話傾談》裏以「輝」形容一個人衣著打扮得體,如「光輝」、「周身輝」,這種用法今日未見:

% (37)  細佬哥,个陣拋了個隻砵,買的好衣裳,裝得周身輝,去歸買屋,娶老婆做財主,都係哩条門路咯。(239)

% (38)  你估我用個的錢文真正冇想象麼?狗醜主人羞,唔打办吓光輝,人話齊思賢老婆,衣衫襤褸,失禮到你呀。(253)

% **3.11** [想像](想像)

% 今「想像」只作動詞,但在《俗話傾談》可作名詞,即「計劃」、「打算」之意:

% (39)  你估我用個的錢文真正冇想象麼?狗醜主人羞,唔打办吓光輝,人話齊思賢老婆,衣衫襤褸,失禮到你呀。所以遇時拜神拜佛,無非見自己命鄙,歸到你門兩年,未有所出,都係想菩薩庇佑,早日生個花仔,得到三十七八時,娶個新婦。(學翻你咁好)你做家公,我做家婆,有仔有孫,慢慢享福(不可先折福),人家同話你好命咯。唔通等到五六十歲生仔,扒向棺材頭麼?你做男人曉得發財,唔慌有个的想像吓咯。(253-254)

% (40)  有子有孫,亦人生之想像也。(176)

% 以下詞語,意思尚待考証:

% **3.12** [[穿崩閗湊]]

% 此詞在《俗話傾談》只有一例,從文句看此詞意思似乎與「粉飾浮跨」類同:

% (41)  板障花窗,可以粉飾浮誇,穿崩閗湊。獨至四條大柱,須用堅石,須用實木,自頭到脚,都要咁堅,都要咁實。(17)

%  在書中近似的詞語有「穿崩爛破」或「穿崩破爛」,然而這兩個詞語似乎與「穿崩閗湊」正好相反:

% (42)  好似扯得穿崩爛破。(68)

% (43)  有的去打爛水缸,有的去打穿米塔,有的去打崩飯鑊,有的拈斧頭砍破大門,有的執竹篙掃屋瓦,打得穿崩破爛,好處無存。(331)

% **3.13** 阿瓜/亞瓜

% 此詞在《俗話傾談》中有二例,只知是一種人物性的負面比喻,具體來源和含義不明:

% (44)  二成夫妻暗偷歡喜,可以無拘無束,自作自為。置一張鬼子枱,油了金漆,兩張竹椅,可以伸腰。象牙快箸,磁器碗碟,白釉茶壺,描花局盅等頂頂件件俱全,鮮明雅潔,居然要鬧做阿瓜,老婆好似十萬身家都冇咁鬧罵。餐餐要飲有色酒。(37-38)

% (45)  朋友相交,未嘗不設飲食,亦唔係專以飲食為題,當飲食時,講得了不得咁知心,唔通冇飲食就水咁淡,觀佢形容,整聲色,講惡氣,如敗水亞瓜,新出匪類。(231)

% **3.14** [[泒]]/沠

% 此字從語料看似乎有三種意思,第一種意思類同「批評」、「抱怨」:

% (46)  兩公婆只怨老母不仁、沠老母不是。(39)

% (47)  若只曉得泒翁姑不是,叔伯不是,做男子就唔著聽咯。(432)

%        而第二種意思類同系詞,但只用於排名:

% (48)  論起層次,長子亞孝泒第一,亞忠泒第三,亞信泒第四,此三個仔,俱係正妻所生,亞悌泒第二,亞仁泒第五,亞義泒第六,此三個仔俱係妾氏所生。(325)

% (49)  長子繼業泒第一,繼德泒第三,此兩个係結髮所生。繼功泒第二,繼績泒第四,此兩个係妾所生。繼祖泒第五,此一个係婢所生。(368)

%        而最後一個意思,類同「分送」:

% (50)  誰不知你行前人指後,話你等豬兄狗弟,實在都唔係人,今鬧起官司要將我大仔沠與乞兒,問你於心何忍?(103)

%             據《異體字字典》(教育部國語推行委員會 2002),此字可以是「派」或「流」的異體,但不論視作「派」還是「流」,都不能圓滿解釋以上所有意思。如果視作「派」字,則能符合最後一個意思,也可在第二個意思裏理解為「排」的近音別字,卻解釋不了第一種意思。如果將該字視作「流」,在第一種意思裏也許能視作「鬧」的近音別字,但未能解釋第二和第三種意思。

% **4.** 熟語、詈語

% **4.1** 熟語

% 《俗話傾談》作為講生的故事話本,為了吸引聽眾,行文用句十分通俗,使用了不少熟語或比喻手法,部份在今日粵語依然能夠找到:

% (51)[[傷風夾膩]]:誰知肚內尚有風痰,未能疏發得透,食了豬肉,謂之傷風夾膩。(7)

% (52)[[心頭跌落脚筋踭]]:你肯聽我教,我就心頭跌落脚筋踭咯。(20)

% (53)[[稟神咁樣稟]]:橫紋柴有時落得水多、落得水少,其飯煮得太軟太硬,臧姑就沈吟密咒,好似稟神咁樣稟(21)

% (54)圩咁嘈、蝦咁跳:去到死者之家,如雀鳥歸巢,鵝鴨到埠,圩咁嘈,蝦咁跳,話逼死佢个女,逼死佢个妹,逼死佢亞姨,詐哭得嗚嗚。(333)

% (55)好喉頸、日子長、好尾運:勸雞頸與珊瑚曰,你一生好喉頸。勸雞腸與珊瑚曰,你後來日子長。勸雞尾與珊瑚曰,你將來好尾運。又勸珊瑚飲雞酒,話後生飲過好兆頭。(33-34)

% (56)龍肉:你不用問我,我與你分開食,你唔管得我。個的就是龍肉,與你無干。(38)

% (57)亞崩養狗:兩公婆,十年唔叫一句老母,十年唔叫一句家婆,為何今早如此恭敬,好似亞崩養狗,轉了性,都唔定咯。(82)

% (58)實過鐵釘:我講沙虫變蚊仔,人人共見。道理至愛真實,最忌虛浮,我句對文,重實過鐵釘,落水都唔浸得爛,重話唔好過佢麼?(116-117)

% (59)冇牙老虎:女呀,你唔好去,個的唔係別樣病,係叫做冇牙老虎。你偏回去,若撞板起來,連你都死乾淨咯。(180)

% (60)大花筒:捉狗仔、切魚生、彈琵琶,吹鴉片,嫖賭飲蕩,練得周身引,好似大花筒。(223)

% (61)捉虱上頭壳養:本心之講,事關人命,連累非輕,非比同狗肉魚生就帮吓手,都贃得的食呀,個死佬見過都衰,有乜咁蠢才,捉虱上頭壳養呢?(235)

% (62)鹹豆都唔食得一粒:我亦為此之故,所以即刻推辞,佢尚唔知利害,實在佢有條人命案,在我兩個手來。我兩個若容忍他,佢便有碗安樂飯食,若係唔顧舊相與,我要佢鹹豆都唔食得一粒。(235-236)

% (63)風吹臘鴨:其妻使人走往女處,誰知吊在親家門上,好似風吹臘鴨,搖搖擺擺咯。(269)

% (64)易過執豆:个件事重易過執豆,執豆尚要顧低頭。(278)

% (65)一隻牛唔好攪壞一欄:亞孝自高自傲,以亞悌亞仁亞義,係庶母所生,不以骨肉相待,作佢為低一格,而卑賤之。結埋亞忠亞信,作為一党,話我三兄弟係大婆仔,佢三個係妾氏仔,就欺佢打佢,都唔奈得我乜何。(果然好亞哥好帶头好倡率所謂一隻牛唔好攪壞一欄)。(328)

% (66)橫吞欖核:細仔呀,我知你屈氣咯,个的龜蛋唔中用,我來教佢,佢一句頂住我喉嚨,好似橫吞欖核(生鵞喉都唔定),話佢唔聽,打佢唔贏,鬱抑憂愁,何處可寬懷一二。(376)

% (67)吊燒豬:你唔願打,要用吊法。二成未曾見人吊過,以為吊好過打。二成曰,我願吊罷咯。巡丁將他吊起,名為吊燒豬。(66)

% (68)食腊鴨飯:各人整定身勢,今日去攞人命呀(東莞叫做食腊鴨飯)。(333)

% **4.2** 詈語

% 在《俗話傾談》的故事裏,經常會出現一些負面的人物角色,或描寫家人爭執不和,藉此突出不修善行、不重人和的惡果,最終達至宣揚人倫的目的。因此,《俗話傾談》內不難找到一些當時的罵人話,這些詞語,在其他的清末粵語材料裏少有記錄:眼中釘(11)、心頭火(12)、壞鬼女(15)、惡家婆(15)、昏婆(15)、冇天裝(20)、[[霸巷雞乸]](20,指經常發惡的婦女)、老龜婆(21,用來罵家姑)、老狗乸(21,用來罵家姑)、衰家狗(21,用來罵媳婦)、乞食骨(23,用來罵兒子)、[[盲虫頭]](23,用來罵兒子)、發戇(35)、賤婦人(43)、蠢婆(102)、蠢才(122)、廢物(122)、負心人(149)、掃把星(157)、敗家精(157)、丟駕(217)、龜蛋(220)、潑婦(268)、[[砧板蟻]](330,指人廢物)、[[溝渠鴨]](330,指人廢物)、臘豬頭(330,指人廢物)、烏龍尾(330,指人廢物)、攪屎棍(333,指人無事生非)、風炉扇(333,指人無事生非)、[[食尿甕雞]](404,指人言語討厭)、食死顛狗(404,指人言語討厭)、[[砒霜砵]](407,指狠毒的婦女)、蠢才

% (430)、陰陰濕濕(430)、老婆奴(430)

% **5.** 句法

%             語法句式的歷時演變,一向是清末粵語材料的研究重點。《俗話傾談》裏粵語句子的句式,也顯露出當時語法的一般特點,以下略舉數例,並與 1893 年[[7]](#_ftn7)寫成的《麥仕治廣州俗話書經解義》(余靄芹 2000)作比較:

% **5.1** 正反問句

% 《俗話傾談》裏的正反問句,主要以「VP-neg-V」形式出現,共有 7 例:

% 從我唔從(239)、應在天堂唔應(280)、應落地獄唔應(281)、服佢唔服(365)、甘心唔甘(386)、係我欺負你唔係(411)、食著唔食(425)

% 另有 2 例是「VP-neg-VP」形式:

% 記得唔記得(25)、得做唔得做(306)

% 「V-neg-V」形式一例:

% 你起唔起(418)

% 「VP-neg」型,只見 4 例,其中 3 例同屬一個句子:

% 做得唔呢(278、303、423)、你話妙到極唔呢(113)

% 至於今日最常見的「V-neg-VP」結構,則沒有任何例子。

% 余藹芹(Yue-Hashimoto 1993)認為「VP-neg」型的正反問句是粵語裏最基本的句式,「V-neg-V」及「VP-neg-V」後來才產生。在《書經解義》裏,最常見的是「VP-neg」類型,但《俗話傾談》裏反而是「VP-neg-V」比較常見。似乎《俗話傾談》成書雖在《書經解義》之先,但卻呈現了較後期的語法狀態,這情況大抵說明粵語正反問句的演變,在各地方並不同步。

% **5.2** 比較句

%        在《俗話傾談》裏,所有的比較句都以「過」作標記:

% 重好色水過你十倍(2)、你要好過佢為是(20)、新婦惡過家婆(21)、假心事勝過冇心事(26)、為何你重要多過亞哥呢(36)、雞肉重高過鼻哥(83)、佢好得過我個比(112)、重唔勝過佢(116)、重實過鐵釘(117)、重話唔好過佢麼(117)、哭得眼胞腫起,大過雞膥(156)、計多過米(165)、但有荔枝,勝過冇荔枝(199)、有時話朋友好過兄弟(220)、大約勝過他人(230)、我兩個重打得多過你(247)、我亦話夫妻親過父母(251)、你丈夫親過我(259)、个件事重易過執豆(278)、重關係過做賊(285)、斷唔輸得過佢(329)、佢重醜過我十分(364)、大約我重先做過佢(365)、你慌駛輸過佢麼(370)、徒弟惡過師傅咯(372)、飽死好過餓死(416)、你到來惡得過我(419)、重關係過砒霜砵。(423)、天眼明過鏡(429)

%  《俗話傾談》裏沒有找到以「比」作標記的比較句,情況與同時期的其他粵語材料相同。有趣的是,《俗話傾談》作為三及第小說,文言、白話與粵語夾雜,但全書即使是非粵語句式,都沒有發現以「比」作標記的比較句。這一點與其他清末的粵語材料情況一致。

% **5.3** 雙賓結構

% 這裏所指的雙賓結構包括同時有直接和間接賓語的句式及它的轉換形式。余藹芹(2000)為了比較間接賓語前用「過」或「俾」的歷時變化,在討論《書經解義》雙賓結構時將狹義的雙賓句(如「賜佢個官銜」)、謂補結構(如「封賜福樂俾人」)和處置式(「將呢一件事先稟告過上帝知」)都列入討論範圍,本文也採用相同的標準。《俗話傾談》裏的雙賓結構在間接賓語之前絕大多數都要加上標誌「過」、「與」,只有少數例子在間接賓語前沒有標誌。

% 5.3.1       沒有標誌的句子

%        在沒有標誌的句子裏,有 4 例是處置式將直接賓語提前,例如:

% (69)   誰肯將女嫁佢個仔呢?(19)

% (70)   我今發你還陽,將此事轉傳於人。(421)

%        而非處置式的有 3 例,其中兩例屬「V+Oi+V+Od」結構:

% (71)   兩公婆……四時八節唔叫老母食一餐飯,唔請亞哥飲一杯酒。(39)

% (72)   你三兄弟唔請我食一餐,留一宿。(358)

% 屬最典型雙賓句的「V+Od+Oi」只有一例:

% (73)   著咯、著咯,唔駛畀情面佢,佢叫我做亞哥,都唔好應佢。(370)

% 5.3.2       間接賓語前以「過」作標誌

%             在以「過」為標誌的句子裏,大約有一半(5 例)是直接賓語與間接賓語兼備的「V+Od+ 過+Oi+(V)」結構,例如:

% (74)   就係繡条大紅裙,聯件花衫袖過你著,你都無愧咯。(230-231)

% (75)   你慌我冇飯過你食,冇屋過你住麼?因你父唔知,於理不合。(374)

% 至於另外一半的例子(7 例)是只有間接賓語的「V+過+Oi+(V)」結構,省略掉的直接賓語可據上文下理推敲出來,例如:

% (76)   我窮然後賣女,賣過你使喚,唔係賣過你打死呀。(43)

% (77)   太爺一一解過我知咯。(99)

% (78)   乜你先時唔話過我知呀?(100)

%             還有兩例以處置式「將+Od+V+過+Oi」結構出現,由於處置式能將直接賓語提前,因此直接賓語不能省略:

% (79)   亞哥你真正冇本心,盡將銅銀分過我。(62)

% (80)   將我個仔來分過乞食佬。(99)

% 5.3.3       間接賓語前以「與」作標誌

%  以「與」作標誌的句子,多數(8 例)是同時有直接和間接賓語的「V+Od+與+Oi+(V)」結構,例如:

% (81)   勸雞頸與珊瑚曰,你一生好喉頸。勸雞腸與珊瑚曰,你後來日子長。勸雞尾與珊瑚曰,你將來好尾運。(33-34)

% (82)   所以天有眼,賜福賜祿與佢。(57)

% (83)   公公,我煲粥與你共大眾食。(183)

% (84)   亞孝等唔肯分田地與我。(361)

% 有三例是省略直接賓語或直接賓語在句子前段已有出現的「V+與+Oi+(V)」結構:

% (85)   你要換就換與你。(62)

% (86)   世上有一等人,買魚買肉多讓與仔食。(77)

% (87)   擇好雞肉,勸與老母。(83)

% 另外有三例屬處置式的「將+Od+V+與+Oi+(V)」結構:

% (88)   今鬧起官司要將我大仔沠與乞兒。(103)

% (89)   命差役將亞明次子亞定長子,押去養濟院,交與乞食頭做親男。(97)

% (90)   你將臘鴨送與亞姨,送與契女。(409) 《俗話傾談》雙賓結構的一大特點,就是間接賓語前經常以「與」或「過」作為標誌,這在《書經解義》及其他清末粵語材料似乎不多見。相反,在其他材料常見的「俾」和「俾過」字句,於《俗話傾談》只能找到一例。余藹芹(2000)曾指出「過」是早期用法,後來逐漸被「俾」字取代。《俗話傾談》的情況比較特殊,最常見的不是「俾」或「過」,而是「與」。我們很難判定這裏的「與」在當時口語是否真的存在,還是受文言文影響的結果。不過即使將所有以「與」作標誌的例子排除,《俗話傾談》裏以「過」

% 作標誌的例子還是遠比以「俾」作標誌的多,反映了比較早期的面貌。

%                                             **6.**    《俗話傾談》的共時差異

% 在考察《俗話傾談》裏的粵語時,除了要注意與今日粵語的歷時差異,還應該注意方音之間的共時差異。《俗話傾談》作者邵彬儒是四會人,四會話在《中國語言地圖集》

% (中國社會科學院、澳大利亞人文科學院 1987)歸入勾漏片,李新魁(1994:26-27)劃為羅廣片,詹伯慧(2002:168)則將這一帶的粵語次方言稱為「西江流域粵語」。不論如何劃分命名,各家都認同四會話與廣州話分屬不同的次方言。在《俗話傾談》裏,我們也從個別詞匯和語音窺見四會話的痕跡。

% **6.1** 詞匯

% 以下三個詞匯很可能反映了四會話的影響:

% (91)   狗牯:呢个乞食仔,你話失了亞叔,个隻狗牯,就係你亞叔呢?(245)

% 粵語裏稱「公狗」為「狗牯」於廣州未見,但於粵西、粵北都比較普遍(詹伯慧、張日昇 1994:520;1998:524),其中包括了《俗話傾談》作者邵彬儒家鄉的四會話。

% (92)   快箸:象牙快箸,磁器碗碟,白釉茶壺,描花局盅等頂件件俱全。(37-38)

% 粵西四會、廣寧一帶皆稱「筷子」為「箸」(詹伯慧、張日昇 1998:550),此說法於廣州話未見。

% (93)   個處:由是摩頭摩耳,眼望天、脚拍地,磨吓墨,又拈吓筆,走去小便個處,企住想一回,行理(按:應為「埋」之誤)書位,坐住椅,扭完手指,伏低枱頭,都唔想得出。(112-113)

% 今日粵語裏可用「處」或「度」指代一個地方或地點,其中廣州主要使用「度」,如「呢度」、「嗰度」,而四會主要使用「處」,如「呢處」、「嗰處」(詹伯慧、張日昇 1998:664)。在《俗話傾談》裏,指代地方或地點的用詞只見「處」而沒有「度」,更接近四會話的用法。

% 不過,在四會、廣寧一帶,人稱代詞眾數標記可說成「呢」,如「我呢」、「你呢」(詹伯慧、張日昇 1998:661),但《俗話傾談》裏卻只有廣州一帶通用的「我地」、「你地」。似乎作者遇到四會話與廣州話差異較明顯的地方,都儘量改用廣州話的說法,但在不為人注意的地方,則保留了四會話的痕跡。

% **6.2**    語音

% 正如 2.2 節所述,《俗話傾談》裏有部份漢字注有直音,其中「扯-者上聲」、「跌-鐵」這兩組例子,似乎都涉及送氣和不送氣聲母的交替。而在《俗話傾談》的作者家鄉的四會話裏,古全濁塞擦音聲母平聲字幾乎都不送氣(詹伯慧、張日昇 1998:46),這是四會話與廣州話語音的主要區別。「扯-者上聲」、「跌-鐵」這兩組例子雖然都不涉及古全濁塞擦音的平聲字,但作者卻依然將送氣與不送氣的字音互注。似乎他也意識到自己家鄉話裏很多不送氣的字音在廣州、佛山一帶都唸成送氣,使他矯枉過正將一些本應不送氣的字也誤為送氣,導致誤讀。

% 其實,如果檢視一下書中注有直音的例子,可發現部份被注的都不是甚麼難字罕用字,如「瓶」、「折」、「拼」等,使人不禁懷疑作者注直音的動機,並非因為他不懂那些字,而是用來提醒自己這些字在廣州音的讀法。

% **6.3**    總結

% 《俗話傾談》作為講生話本,性質與後來的「三及第」不同,其中的人物對話部份,白話程度甚高,是研究 19 世紀中葉粵語的好材料。《俗話傾談》與同期的粵語材料相比,十分通俗入世,記錄了大量當時的俗語、熟語,以至詈語,正好補同期資料的不足。尤其值得注意的是,《俗話傾談》作者邵彬儒本身是四會人,他長期在廣州佛山說書,要學會廣州一帶的口語並不困難,但他原有語言習慣在《俗話傾談》裏依然會留下痕跡。除了上述詞匯和語音的線索外,第 5 節所述的部份《俗話傾談》語法特徵與《書經解義》等同期材料不盡一致,很可能同樣是地域間的共時差異,未來值得進一步探究。

% 引用文獻

% [清]博陵紀棠氏評輯. 1990.《俗話傾談》、《俗話傾談二集》。上海:上海古籍出版社《古本小說集成》影印《俗話傾談》同治九年(1870)粵東省城十七甫五經樓本、《俗話傾談二集》同治九年(1870)羊城十七甫五經樓本。

% [清]博陵紀棠氏評輯. 1994.《俗話傾談》、《俗話傾談二集》。北京:中華書局《古本小說叢刊》影印《俗話傾談》年份不詳省城學院前華玉堂本、《俗話傾談二集》同治十年(1871)出版地不詳本。

% 教育部國語推行委員會. 2002.《異體字字典》。台北:教育部。(光碟版)李新魁. 1994.《廣東的方言》。廣州:廣東人民出版社。

% 魯金. 1990.〈用「三及第」文體寫成的俗話傾談〉。香港:《明報》1990 年 8 月 3 日。

% 耿淑艷. 2007.〈嶺南晚清小說家邵彬儒生平著作考〉,《文獻》2007.3:93-98。

% 孔仲南. 1992.《廣東俗語考》。上海:上海文藝出版社影印 1933 年南方扶輪社版本。

% 黃仲鳴. 2002.《香港三及第文體流變史》。香港:香港作家協會。葉春生. 1996.《嶺南俗文學簡史》。廣州:廣東高等教育出版社。

% 余藹芹. 2000.〈粵語方言的歷史研究──讀《麥仕治廣州俗話〈書經〉解義》〉,《中國語文》

% 2000.6:497-507。

% 詹伯慧、張日昇主編. 1994.《粵北十縣市粵方言調查報告》。廣州:暨南大學出版社。詹伯慧、張日昇主編. 1998.《粵西十縣市粵方言調查報告》。廣州:暨南大學出版社。

% 詹伯慧主編. 2002.《廣東粵方言槪要》。廣州:暨南大學出版社。

% 中國社會科學院、澳大利亞人文科學院編. 1987.《中國語言地圖集》。香港:朗文出版社。

% Yue-Hashimoto, Anna. 1993. “The lexicon in syntactic change: lexical diffusion in Chinese syntax”. _Journal of Chinese Linguistics_ 21.2:213-254.

  

% ---

% (#_ftnref1) 括號內數字表示該例子所在的頁碼,除非另外說明,所有頁碼均為上海古籍出版社影印本(1994)的新編頁碼。

% (#_ftnref2) 葉春生未有提供資料出處,也許他於廣東省的圖書館內見過該書初版。

% (#_ftnref3) 此字只在上海古籍出版社影印本(1994)注有直音,括號內前一號為上海古籍出版社影印本(1994)的新編頁碼,斜線後是對應中華書局影印本(1990)的新編頁碼。下同。

% (#_ftnref4) 原句為「你去巷亞美叔借一張熟鐵鋤頭。」(50)。「」據《廣東俗語考》上卷(孔仲南 1992[1933]:40)音篤,「尾後」之意,此處當指「末端、盡頭」。承黃得森賜教,特此鳴謝。

% (#_ftnref5) 原句為「為何不回母家而在此攪擾姨婆。」估計誤將「攪擾」作「騷擾」再以「叔」為「騷」直音。

% (#_ftnref6) 本文以香港粵語作為當代粵語的參照。

% (#_ftnref7) 《麥仕治廣州俗話書經解義》據余藹芹(2000)乃癸巳年出版,合 1833 年或 1893 年。承郭必之指出該書載有麥仕治照片,而攝影技術於 19 世紀中晚期才傳入中國,故《書經解義》出版年期當以 1893 年為合。

% \newpage

% \section{䎺}

% 󱀚:「我唔䎺」

% \newpage

% \section{之}

% 又叫[豬],所謂嘅[豬粒]即係[一粒花]  
%      以前話食咗人隻[豬],即系攞咗人哋第一次,有[破處]嘅意思

% \newpage

% \section{事懶}

% 事懶

% \newpage

% \section{休要}

% 休要

% \newpage

% \section{休阻滯}

% (Empty file)

% \newpage

% \section{估}

% #logic

% \newpage

% \section{估話}

% Potential for counterfactuals 

% 估話

% \newpage

% \section{假}

% #metaphysics #logic

% \newpage

% \section{傷風夾膩}

% #老粵語

% \newpage

% \section{兇勢}

% 3.5** [[兇性]]/[[兇勢]]/[[勢兇]]/[[兇橫]]

% 「兇」在今日粵語多獨用作形容詞,在《俗話傾談》經常與其他字組合成詞,意思除了本意「兇惡」外,還可指人「狠心」,略同今粵語詞「[[狼死]]」:

% (14)  適值旁邊有一個婦人,見他如此兇性,即用力擒住他手,盡勢推開,大喝一聲,乜你咁勢兇呀?(10)

% (15)  各人見他咁兇勢,咁撒賴,難以用手相爭。(265)

% (16)  有咁樣惡法,我个新婦既死,巳經傷心不了,重來毁我房屋,散我家私,將我老婆咁樣凌辱,有咁太過兇橫。佢恃拳頭在近,官府在遠麼?(341)

% \newpage

% \section{兇性}

% 3.5** [[兇性]]/[[兇勢]]/[[勢兇]]/[[兇橫]]

% 「兇」在今日粵語多獨用作形容詞,在《俗話傾談》經常與其他字組合成詞,意思除了本意「兇惡」外,還可指人「狠心」,略同今粵語詞「[[狼死]]」:

% (14)  適值旁邊有一個婦人,見他如此兇性,即用力擒住他手,盡勢推開,大喝一聲,乜你咁勢兇呀?(10)

% (15)  各人見他咁兇勢,咁撒賴,難以用手相爭。(265)

% (16)  有咁樣惡法,我个新婦既死,巳經傷心不了,重來毁我房屋,散我家私,將我老婆咁樣凌辱,有咁太過兇橫。佢恃拳頭在近,官府在遠麼?(341)

% \newpage

% \section{兇橫}

% 3.5** [[兇性]]/[[兇勢]]/[[勢兇]]/[[兇橫]]

% 「兇」在今日粵語多獨用作形容詞,在《俗話傾談》經常與其他字組合成詞,意思除了本意「兇惡」外,還可指人「狠心」,略同今粵語詞「[[狼死]]」:

% (14)  適值旁邊有一個婦人,見他如此兇性,即用力擒住他手,盡勢推開,大喝一聲,乜你咁勢兇呀?(10)

% (15)  各人見他咁兇勢,咁撒賴,難以用手相爭。(265)

% (16)  有咁樣惡法,我个新婦既死,巳經傷心不了,重來毁我房屋,散我家私,將我老婆咁樣凌辱,有咁太過兇橫。佢恃拳頭在近,官府在遠麼?(341)

% \newpage

% \section{免使}

% 免使
% 慳番
% 唔使

% \newpage

% \section{共}

% 哪怕一朝只得你共我
% 俗塵渺渺,天意茫茫,將你共我分開
% [水共油撈](水共油撈)
% [同](同)
% [孖](孖)

% \newpage

% \section{冚}

% #百越底層詞

% \newpage

% \section{冧}

% #百越底層詞

% \newpage

% \section{勢兇}

% 3.5** [[兇性]]/[[兇勢]]/[[勢兇]]/[[兇橫]]

% 「兇」在今日粵語多獨用作形容詞,在《俗話傾談》經常與其他字組合成詞,意思除了本意「兇惡」外,還可指人「狠心」,略同今粵語詞「[[狼死]]」:

% (14)  適值旁邊有一個婦人,見他如此兇性,即用力擒住他手,盡勢推開,大喝一聲,乜你咁勢兇呀?(10)

% (15)  各人見他咁兇勢,咁撒賴,難以用手相爭。(265)

% (16)  有咁樣惡法,我个新婦既死,巳經傷心不了,重來毁我房屋,散我家私,將我老婆咁樣凌辱,有咁太過兇橫。佢恃拳頭在近,官府在遠麼?(341)

% \newpage

% \section{千一個}

% 后来发展出"[千日](千日)"和"[千个](千個)",如: (2) 千日唔带得妹上街'千日都仍 日像累。(新粤讴解心·大葵扇, 2-4) (3) 你千日重人地护花'都唔伶、上算。(新粤讴解心·牡丹花, 2-1) (4) 千个听人地手指掉开埋'都几份、啃惨。(新粤讴解心·[咪估話同你啃好](咪估話同你啃好), 2-2)

% 與[千祈](千祈)或同源。

% #logic #時間 #老粵語 


%  [成日](成日)

% \newpage

% \section{千個}

% #logic #時間 #老粵語

% \newpage

% \section{千日}

% [好日](好日)

% #logic #時間 #老粵語 


% [千一个](千一個)唔你住埋,千一个唔得到底。(粵謳 留客)

% \newpage

% \section{千祈}

% #logic #時間 #老粵語

% \newpage

% \section{可能}

% #logic

% \newpage

% \section{吉屎}

% #英源粵詞 # 
% 吉屎  勇氣〔Guts.英文翻譯〕

% \newpage

% \section{呃}

% #百越底層詞

% \newpage

% \section{咪估話同你啃好}

% #粵謳 

% 咪[估話](估話)
% [估](估),[斷](斷)

% \newpage

% \section{咿唈:}

% (Empty file)

% \newpage

% \section{囈}

% #百越底層詞

% \newpage

% \section{堅}

% #metaphysics
% 係[[[真]]]嘅意思

% \newpage

% \section{失落鄉關}

% 失落[鄉關](鄉關)

% \newpage

% \section{奀嫋鬼命}

% 奀嫋鬼命

  

% 粵語

  

% - 香港音:

% - 粵拼:**ngan1** **tiu1** **gwai2** **meng6**

% 打開

% - 廣東話 (粵拼):ngan1 tiu1 gwai2 meng6

% **解乜**

% 1. 【成語】形容一個人身體孱弱,體型瘦削。例子:佢細個冇飯食,營養不良,搞到而家廿歲人都仲係**奀挑鬼命**。keoi5 sai3 go3 mou5 faan6 sik6, jing4 joeng5 bat1 loeng4, gaau2 dou3 ji4 gaa1 jaa6 seoi3 jan4 dou1 zung6 hai6 **ngan1 tiu1 gwai2 meng6**.  
      
%     英譯:He had little to eat and was malnourished during childhood, and consequently is still **a bag of bones** well in his twenties.

% \newpage

% \section{好日}

% 好耐 一段長日子
% 你好日都唔黎見我

% #時間 #logic  #老粵語

% \newpage

% \section{孖}

% #介詞連詞

% \newpage

% \section{孬攪、撈絞}

% (Empty file)

% \newpage

% \section{廣東話常用字}

% #文 

% |原字(本字)|[拼音](https://zh.wikipedia.org/wiki/yue:Jyutping "w:yue:Jyutping")|[五筆](https://zh.wikipedia.org/wiki/yue:%E4%BA%94%E7%AD%86%E8%BC%B8%E5%85%A5%E6%B3%95 "w:yue:五筆輸入法")|[倉頡](https://zh.wikipedia.org/wiki/yue:%E5%80%89%E9%A0%A1%E8%BC%B8%E5%85%A5%E6%B3%95 "w:yue:倉頡輸入法")|意思|粵文例句-拼音-普通话译文|
% |---|---|---|---|---|---|
% |[嗌](https://zh.wiktionary.org/wiki/%E5%97%8C "嗌")|aai3|kuwl|rtct|叫喊|大嗌 ― 大叫|
% |[呃](https://zh.wiktionary.org/wiki/%E5%91%83 "呃")|aak1|kdbn|rmsu|欺騙|[呃](https://zh.wiktionary.org/wiki/%E5%91%83#%E6%BC%A2%E8%AA%9E "呃")[神](https://zh.wiktionary.org/wiki/%E7%A5%9E#%E6%BC%A2%E8%AA%9E "神")[騙](https://zh.wiktionary.org/wiki/%E9%A8%99#%E6%BC%A2%E8%AA%9E "騙")[鬼](https://zh.wiktionary.org/wiki/%E9%AC%BC#%E6%BC%A2%E8%AA%9E "鬼") / [呃](https://zh.wiktionary.org/wiki/%E5%91%83#%E6%BC%A2%E8%AA%9E "呃")[神](https://zh.wiktionary.org/wiki/%E7%A5%9E#%E6%BC%A2%E8%AA%9E "神")[骗](https://zh.wiktionary.org/wiki/%E9%AA%97#%E6%BC%A2%E8%AA%9E "骗")[鬼](https://zh.wiktionary.org/wiki/%E9%AC%BC#%E6%BC%A2%E8%AA%9E "鬼") [[粵語](https://zh.wikipedia.org/wiki/%E7%B2%B5%E8%AA%9E "w:粵語")]  ―  _aak1 san4 pin3 gwai2_ [[粵拼](https://zh.wikipedia.org/wiki/%E9%A6%99%E6%B8%AF%E8%AA%9E%E8%A8%80%E5%AD%B8%E5%AD%B8%E6%9C%83%E7%B2%B5%E8%AA%9E%E6%8B%BC%E9%9F%B3%E6%96%B9%E6%A1%88 "w:香港語言學學會粵語拼音方案")]  ―  四处骗人|
% |[拗](https://zh.wiktionary.org/wiki/%E6%8B%97 "拗")|aau2|rxet|qvis|折彎|[拗斷](https://zh.wiktionary.org/wiki/%E6%8B%97%E6%96%AD "拗断")[條](https://zh.wiktionary.org/wiki/%E6%A2%9D#%E6%BC%A2%E8%AA%9E "條")[樹枝](https://zh.wiktionary.org/wiki/%E6%A8%B9%E6%9E%9D#%E6%BC%A2%E8%AA%9E "樹枝") / [拗断](https://zh.wiktionary.org/wiki/%E6%8B%97%E6%96%AD#%E6%BC%A2%E8%AA%9E "拗断")[条](https://zh.wiktionary.org/wiki/%E6%9D%A1#%E6%BC%A2%E8%AA%9E "条")[树枝](https://zh.wiktionary.org/wiki/%E6%A0%91%E6%9E%9D#%E6%BC%A2%E8%AA%9E "树枝") [[粵語](https://zh.wikipedia.org/wiki/%E7%B2%B5%E8%AA%9E "w:粵語")]  ―  _aau2 dyun3 tiu4 syu6 zi1_ [[粵拼](https://zh.wikipedia.org/wiki/%E9%A6%99%E6%B8%AF%E8%AA%9E%E8%A8%80%E5%AD%B8%E5%AD%B8%E6%9C%83%E7%B2%B5%E8%AA%9E%E6%8B%BC%E9%9F%B3%E6%96%B9%E6%A1%88 "w:香港語言學學會粵語拼音方案")]  ―  把这个树枝折断|
% |[詏](https://zh.wiktionary.org/wiki/%E8%A9%8F "詏")|aau3|rxln|yrvis|矛盾|[詏交](https://zh.wiktionary.org/w/index.php?title=%E8%A9%8F%E4%BA%A4&action=edit&redlink=1 "詏交(页面不存在)") / [𬣦交](https://zh.wiktionary.org/w/index.php?title=%F0%AC%A3%A6%E4%BA%A4&action=edit&redlink=1 "𬣦交(页面不存在)") [[粵語](https://zh.wikipedia.org/wiki/%E7%B2%B5%E8%AA%9E "w:粵語")]  ―  _aau2 gaau1_ [[粵拼](https://zh.wikipedia.org/wiki/%E9%A6%99%E6%B8%AF%E8%AA%9E%E8%A8%80%E5%AD%B8%E5%AD%B8%E6%9C%83%E7%B2%B5%E8%AA%9E%E6%8B%BC%E9%9F%B3%E6%96%B9%E6%A1%88 "w:香港語言學學會粵語拼音方案")]  ―  吵架|
% |[畀](https://zh.wiktionary.org/wiki/%E7%95%80 "畀")|bei2|lgjj|wml|給予|[畀](https://zh.wiktionary.org/wiki/%E7%95%80#%E6%BC%A2%E8%AA%9E "畀")[我](https://zh.wiktionary.org/wiki/%E6%88%91#%E6%BC%A2%E8%AA%9E "我")[啦](https://zh.wiktionary.org/wiki/%E5%95%A6#%E6%BC%A2%E8%AA%9E "啦") [[粵語](https://zh.wikipedia.org/wiki/%E7%B2%B5%E8%AA%9E "w:粵語")]  ―  _bei2 ngo5 laa1_ [[粵拼](https://zh.wikipedia.org/wiki/%E9%A6%99%E6%B8%AF%E8%AA%9E%E8%A8%80%E5%AD%B8%E5%AD%B8%E6%9C%83%E7%B2%B5%E8%AA%9E%E6%8B%BC%E9%9F%B3%E6%96%B9%E6%A1%88 "w:香港語言學學會粵語拼音方案")]  ―  你给我吧~|
% |[抦](https://zh.wiktionary.org/wiki/%E6%8A%A6 "抦")|bing2|rgmw|qmob|毆打|[我哋](https://zh.wiktionary.org/wiki/%E6%88%91%E5%93%8B#%E6%BC%A2%E8%AA%9E "我哋")[去](https://zh.wiktionary.org/wiki/%E5%8E%BB#%E6%BC%A2%E8%AA%9E "去")[抦](https://zh.wiktionary.org/wiki/%E6%8A%A6#%E6%BC%A2%E8%AA%9E "抦")[嗰](https://zh.wiktionary.org/wiki/%E5%97%B0#%E6%BC%A2%E8%AA%9E "嗰")[條友](https://zh.wiktionary.org/w/index.php?title=%E6%A2%9D%E5%8F%8B&action=edit&redlink=1 "條友(页面不存在)") [[粵語](https://zh.wikipedia.org/wiki/%E7%B2%B5%E8%AA%9E "w:粵語"),[繁體](https://zh.wiktionary.org/wiki/%E7%B9%81%E9%AB%94%E4%B8%AD%E6%96%87 "繁體中文")]  <br>[我哋](https://zh.wiktionary.org/wiki/%E6%88%91%E5%93%8B#%E6%BC%A2%E8%AA%9E "我哋")[去](https://zh.wiktionary.org/wiki/%E5%8E%BB#%E6%BC%A2%E8%AA%9E "去")[抦](https://zh.wiktionary.org/wiki/%E6%8A%A6#%E6%BC%A2%E8%AA%9E "抦")[嗰](https://zh.wiktionary.org/wiki/%E5%97%B0#%E6%BC%A2%E8%AA%9E "嗰")[条友](https://zh.wiktionary.org/w/index.php?title=%E6%9D%A1%E5%8F%8B&action=edit&redlink=1 "条友(页面不存在)") [[粵語](https://zh.wikipedia.org/wiki/%E7%B2%B5%E8%AA%9E "w:粵語"),[簡體](https://zh.wiktionary.org/wiki/%E7%B0%A1%E9%AB%94%E4%B8%AD%E6%96%87 "簡體中文")]<br><br>_ngo5 dei6 heoi3 bing2 go2 tiu4 jau5_ [[粵拼](https://zh.wikipedia.org/wiki/%E9%A6%99%E6%B8%AF%E8%AA%9E%E8%A8%80%E5%AD%B8%E5%AD%B8%E6%9C%83%E7%B2%B5%E8%AA%9E%E6%8B%BC%E9%9F%B3%E6%96%B9%E6%A1%88 "w:香港語言學學會粵語拼音方案")]<br><br>我们去揍那个家伙|
% |[噃](https://zh.wiktionary.org/wiki/%E5%99%83 "噃")|bo3|ktol|rhdw|語氣助詞,帶有提醒或勸告嘅意味|[記住](https://zh.wiktionary.org/wiki/%E8%AE%B0%E4%BD%8F "记住")[陣間](https://zh.wiktionary.org/wiki/%E9%99%A3%E9%96%93#%E6%BC%A2%E8%AA%9E "陣間")[準時](https://zh.wiktionary.org/wiki/%E5%87%86%E6%97%B6 "准时")[到](https://zh.wiktionary.org/wiki/%E5%88%B0#%E6%BC%A2%E8%AA%9E "到")[噃](https://zh.wiktionary.org/wiki/%E5%99%83#%E6%BC%A2%E8%AA%9E "噃") [[粵語](https://zh.wikipedia.org/wiki/%E7%B2%B5%E8%AA%9E "w:粵語"),[繁體](https://zh.wiktionary.org/wiki/%E7%B9%81%E9%AB%94%E4%B8%AD%E6%96%87 "繁體中文")]  <br>[记住](https://zh.wiktionary.org/wiki/%E8%AE%B0%E4%BD%8F#%E6%BC%A2%E8%AA%9E "记住")[阵间](https://zh.wiktionary.org/wiki/%E9%99%A3%E9%96%93 "陣間")[准时](https://zh.wiktionary.org/wiki/%E5%87%86%E6%97%B6#%E6%BC%A2%E8%AA%9E "准时")[到](https://zh.wiktionary.org/wiki/%E5%88%B0#%E6%BC%A2%E8%AA%9E "到")[噃](https://zh.wiktionary.org/wiki/%E5%99%83#%E6%BC%A2%E8%AA%9E "噃") [[粵語](https://zh.wikipedia.org/wiki/%E7%B2%B5%E8%AA%9E "w:粵語"),[簡體](https://zh.wiktionary.org/wiki/%E7%B0%A1%E9%AB%94%E4%B8%AD%E6%96%87 "簡體中文")]<br><br>_gei3 zyu6 zan6 gaan1 zeon2 si4 dou3 bo3_ [[粵拼](https://zh.wikipedia.org/wiki/%E9%A6%99%E6%B8%AF%E8%AA%9E%E8%A8%80%E5%AD%B8%E5%AD%B8%E6%9C%83%E7%B2%B5%E8%AA%9E%E6%8B%BC%E9%9F%B3%E6%96%B9%E6%A1%88 "w:香港語言學學會粵語拼音方案")]|
% |[𨳍](https://zh.wiktionary.org/wiki/%F0%A8%B3%8D "𨳍")|cat6|uav|anp|粗口,男性外生殖器||
% |[唓](https://zh.wiktionary.org/wiki/%E5%94%93 "唓")|ce1|klh|rjwj|語氣助詞,表示鄙夷|[唓](https://zh.wiktionary.org/wiki/%E5%94%93#%E6%BC%A2%E8%AA%9E "唓")! / [𪠳](https://zh.wiktionary.org/wiki/%F0%AA%A0%B3#%E6%BC%A2%E8%AA%9E "𪠳")! [[粵語](https://zh.wikipedia.org/wiki/%E7%B2%B5%E8%AA%9E "w:粵語")]  ―  _ce1!_ [[粵拼](https://zh.wikipedia.org/wiki/%E9%A6%99%E6%B8%AF%E8%AA%9E%E8%A8%80%E5%AD%B8%E5%AD%B8%E6%9C%83%E7%B2%B5%E8%AA%9E%E6%8B%BC%E9%9F%B3%E6%96%B9%E6%A1%88 "w:香港語言學學會粵語拼音方案")]  ―  切!|
% |[嘈](https://zh.wiktionary.org/wiki/%E5%98%88 "嘈")|cou4|kgmj|rtwa|嘈吵|[嘈](https://zh.wiktionary.org/wiki/%E5%98%88#%E6%BC%A2%E8%AA%9E "嘈")[乜](https://zh.wiktionary.org/wiki/%E4%B9%9C#%E6%BC%A2%E8%AA%9E "乜")[嘢](https://zh.wiktionary.org/wiki/%E5%98%A2#%E6%BC%A2%E8%AA%9E "嘢") [[粵語](https://zh.wikipedia.org/wiki/%E7%B2%B5%E8%AA%9E "w:粵語")]  ―  _cou4 mat1 je5_ [[粵拼](https://zh.wikipedia.org/wiki/%E9%A6%99%E6%B8%AF%E8%AA%9E%E8%A8%80%E5%AD%B8%E5%AD%B8%E6%9C%83%E7%B2%B5%E8%AA%9E%E6%8B%BC%E9%9F%B3%E6%96%B9%E6%A1%88 "w:香港語言學學會粵語拼音方案")]  ―  吵什么东西|
% |[啖](https://zh.wiktionary.org/wiki/%E5%95%96 "啖")|daam6|kooy|rff|口(量詞)|[咬](https://zh.wiktionary.org/wiki/%E5%92%AC#%E6%BC%A2%E8%AA%9E "咬")[啖](https://zh.wiktionary.org/wiki/%E5%95%96#%E6%BC%A2%E8%AA%9E "啖") [[粵語](https://zh.wikipedia.org/wiki/%E7%B2%B5%E8%AA%9E "w:粵語")]  ―  _ngaau5 daam6_ [[粵拼](https://zh.wikipedia.org/wiki/%E9%A6%99%E6%B8%AF%E8%AA%9E%E8%A8%80%E5%AD%B8%E5%AD%B8%E6%9C%83%E7%B2%B5%E8%AA%9E%E6%8B%BC%E9%9F%B3%E6%96%B9%E6%A1%88 "w:香港語言學學會粵語拼音方案")]  ―  咬一口|
% |[哋](https://zh.wiktionary.org/wiki/%E5%93%8B "哋")|dei6|kfbn|rgpd|複數人稱後綴|[我哋](https://zh.wiktionary.org/wiki/%E6%88%91%E5%93%8B#%E6%BC%A2%E8%AA%9E "我哋") [[粵語](https://zh.wikipedia.org/wiki/%E7%B2%B5%E8%AA%9E "w:粵語")]  ―  _ngo5 dei6_ [[粵拼](https://zh.wikipedia.org/wiki/%E9%A6%99%E6%B8%AF%E8%AA%9E%E8%A8%80%E5%AD%B8%E5%AD%B8%E6%9C%83%E7%B2%B5%E8%AA%9E%E6%8B%BC%E9%9F%B3%E6%96%B9%E6%A1%88 "w:香港語言學學會粵語拼音方案")]  ―  我们|
% |[掟](https://zh.wiktionary.org/wiki/%E6%8E%9F "掟")|deng3|rpgh|qjmo|擲|[掟](https://zh.wiktionary.org/wiki/%E6%8E%9F#%E6%BC%A2%E8%AA%9E "掟")[出去](https://zh.wiktionary.org/wiki/%E5%87%BA%E5%8E%BB#%E6%BC%A2%E8%AA%9E "出去") [[粵語](https://zh.wikipedia.org/wiki/%E7%B2%B5%E8%AA%9E "w:粵語")]  ―  _deng3 ceot1 heoi3_ [[粵拼](https://zh.wikipedia.org/wiki/%E9%A6%99%E6%B8%AF%E8%AA%9E%E8%A8%80%E5%AD%B8%E5%AD%B8%E6%9C%83%E7%B2%B5%E8%AA%9E%E6%8B%BC%E9%9F%B3%E6%96%B9%E6%A1%88 "w:香港語言學學會粵語拼音方案")]  ―  扔出去|
% |[埞](https://zh.wiktionary.org/wiki/%E5%9F%9E "埞")|deng6|fpgh|gjmo|地方|有冇埞去? ― 有没有地方去?|
% |[啲](https://zh.wiktionary.org/wiki/%E5%95%B2 "啲")|di1|krqy|rhai|少許|[畀](https://zh.wiktionary.org/wiki/%E7%95%80#%E6%BC%A2%E8%AA%9E "畀")[啲](https://zh.wiktionary.org/wiki/%E5%95%B2#%E6%BC%A2%E8%AA%9E "啲")[錢](https://zh.wiktionary.org/wiki/%E9%8C%A2#%E6%BC%A2%E8%AA%9E "錢")[佢](https://zh.wiktionary.org/wiki/%E4%BD%A2#%E6%BC%A2%E8%AA%9E "佢") / [畀](https://zh.wiktionary.org/wiki/%E7%95%80#%E6%BC%A2%E8%AA%9E "畀")[啲](https://zh.wiktionary.org/wiki/%E5%95%B2#%E6%BC%A2%E8%AA%9E "啲")[钱](https://zh.wiktionary.org/wiki/%E9%92%B1#%E6%BC%A2%E8%AA%9E "钱")[佢](https://zh.wiktionary.org/wiki/%E4%BD%A2#%E6%BC%A2%E8%AA%9E "佢") [[粵語](https://zh.wikipedia.org/wiki/%E7%B2%B5%E8%AA%9E "w:粵語")]  ―  _bei2 di1 cin2 keoi5_ [[粵拼](https://zh.wikipedia.org/wiki/%E9%A6%99%E6%B8%AF%E8%AA%9E%E8%A8%80%E5%AD%B8%E5%AD%B8%E6%9C%83%E7%B2%B5%E8%AA%9E%E6%8B%BC%E9%9F%B3%E6%96%B9%E6%A1%88 "w:香港語言學學會粵語拼音方案")]  ―  给他一点钱|
% |[點](https://zh.wiktionary.org/wiki/%E9%BB%9E "點")|dim2|lfok|wfyr|如何|點樣? ― 怎么样?|
% |[掂](https://zh.wiktionary.org/wiki/%E6%8E%82 "掂")|dim6|ryhk|qiyr|完結或狀态佳之意|[搞掂](https://zh.wiktionary.org/wiki/%E6%90%9E%E6%8E%82#%E6%BC%A2%E8%AA%9E "搞掂") [[粵語](https://zh.wikipedia.org/wiki/%E7%B2%B5%E8%AA%9E "w:粵語")]  ―  _gaau2 dim6_ [[粵拼](https://zh.wikipedia.org/wiki/%E9%A6%99%E6%B8%AF%E8%AA%9E%E8%A8%80%E5%AD%B8%E5%AD%B8%E6%9C%83%E7%B2%B5%E8%AA%9E%E6%8B%BC%E9%9F%B3%E6%96%B9%E6%A1%88 "w:香港語言學學會粵語拼音方案")]  ―  办妥|
% |[𨳒](https://zh.wiktionary.org/wiki/%F0%A8%B3%92 "𨳒")([屌](https://zh.wiktionary.org/wiki/%E5%B1%8C "屌"))|diu2|nkmh|srlb|粗口,性交動作||
% |[番](https://zh.wiktionary.org/wiki/%E7%95%AA "番")|faan1|tolf|hdw|外國|番茄 ― 西红柿|
% |[翻](https://zh.wiktionary.org/wiki/%E7%BF%BB "翻")|faan1|toln|hwsmm|重新|翻兜 ― 重新再来|
% |[返](https://zh.wiktionary.org/wiki/%E8%BF%94 "返")|faan1|rcpi|yhe|返回|返屋企|
% |[瞓](https://zh.wiktionary.org/wiki/%E7%9E%93 "瞓")|fan3|hykh|buyrl|睡眠|[瞓覺](https://zh.wiktionary.org/wiki/%E7%9E%93%E8%A6%BA#%E6%BC%A2%E8%AA%9E "瞓覺") / [𰥛觉](https://zh.wiktionary.org/w/index.php?title=%F0%B0%A5%9B%E8%A7%89&action=edit&redlink=1 "𰥛觉(页面不存在)") [[粵語](https://zh.wikipedia.org/wiki/%E7%B2%B5%E8%AA%9E "w:粵語")]  ―  _fan3 gaau3_ [[粵拼](https://zh.wikipedia.org/wiki/%E9%A6%99%E6%B8%AF%E8%AA%9E%E8%A8%80%E5%AD%B8%E5%AD%B8%E6%9C%83%E7%B2%B5%E8%AA%9E%E6%8B%BC%E9%9F%B3%E6%96%B9%E6%A1%88 "w:香港語言學學會粵語拼音方案")]  ―  睡觉|
% |[㗎](https://zh.wiktionary.org/wiki/%E3%97%8E "㗎")|gaa3|klks|rkrd|語氣助詞|[唔係](https://zh.wiktionary.org/wiki/%E5%94%94%E4%BF%82#%E6%BC%A2%E8%AA%9E "唔係")[噉](https://zh.wiktionary.org/wiki/%E5%99%89#%E6%BC%A2%E8%AA%9E "噉")[㗎](https://zh.wiktionary.org/wiki/%E3%97%8E#%E6%BC%A2%E8%AA%9E "㗎") / [唔系](https://zh.wiktionary.org/wiki/%E5%94%94%E7%B3%BB#%E6%BC%A2%E8%AA%9E "唔系")[噉](https://zh.wiktionary.org/wiki/%E5%99%89#%E6%BC%A2%E8%AA%9E "噉")[㗎](https://zh.wiktionary.org/wiki/%E3%97%8E#%E6%BC%A2%E8%AA%9E "㗎") [[粵語](https://zh.wikipedia.org/wiki/%E7%B2%B5%E8%AA%9E "w:粵語")]  ―  _m4 hai6 gam2 gaa3_ [[粵拼](https://zh.wikipedia.org/wiki/%E9%A6%99%E6%B8%AF%E8%AA%9E%E8%A8%80%E5%AD%B8%E5%AD%B8%E6%9C%83%E7%B2%B5%E8%AA%9E%E6%8B%BC%E9%9F%B3%E6%96%B9%E6%A1%88 "w:香港語言學學會粵語拼音方案")]  ―  不是这样的|
% |[曱甴](https://zh.wiktionary.org/wiki/%E6%9B%B1%E7%94%B4 "曱甴")|gaat6 zaat2|mfk mhfd|wmll lwlm|蟑螂|[有](https://zh.wiktionary.org/wiki/%E6%9C%89#%E6%BC%A2%E8%AA%9E "有")[隻](https://zh.wiktionary.org/wiki/%E9%9A%BB#%E6%BC%A2%E8%AA%9E "隻")[曱甴](https://zh.wiktionary.org/wiki/%E6%9B%B1%E7%94%B4#%E6%BC%A2%E8%AA%9E "曱甴") / [有](https://zh.wiktionary.org/wiki/%E6%9C%89#%E6%BC%A2%E8%AA%9E "有")[只](https://zh.wiktionary.org/wiki/%E5%8F%AA#%E6%BC%A2%E8%AA%9E "只")[曱甴](https://zh.wiktionary.org/wiki/%E6%9B%B1%E7%94%B4#%E6%BC%A2%E8%AA%9E "曱甴") [[粵語](https://zh.wikipedia.org/wiki/%E7%B2%B5%E8%AA%9E "w:粵語")]  ―  _jau5 zek3 gaak6 zaat2_ [[粵拼](https://zh.wikipedia.org/wiki/%E9%A6%99%E6%B8%AF%E8%AA%9E%E8%A8%80%E5%AD%B8%E5%AD%B8%E6%9C%83%E7%B2%B5%E8%AA%9E%E6%8B%BC%E9%9F%B3%E6%96%B9%E6%A1%88 "w:香港語言學學會粵語拼音方案")]  ―  有一只蟑螂|
% |[噉](https://zh.wiktionary.org/wiki/%E5%99%89 "噉")|gam2|kafg|rmjk|如此|[噉](https://zh.wiktionary.org/wiki/%E5%99%89#%E6%BC%A2%E8%AA%9E "噉")[啊](https://zh.wiktionary.org/wiki/%E5%95%8A#%E6%BC%A2%E8%AA%9E "啊")? [[粵語](https://zh.wikipedia.org/wiki/%E7%B2%B5%E8%AA%9E "w:粵語")]  ―  _gam2 aa1?_ [[粵拼](https://zh.wikipedia.org/wiki/%E9%A6%99%E6%B8%AF%E8%AA%9E%E8%A8%80%E5%AD%B8%E5%AD%B8%E6%9C%83%E7%B2%B5%E8%AA%9E%E6%8B%BC%E9%9F%B3%E6%96%B9%E6%A1%88 "w:香港語言學學會粵語拼音方案")]  ―  这样啊?|
% |[咁](https://zh.wiktionary.org/wiki/%E5%92%81 "咁")|gam3|kafg|rtm|這麼(程度)|[咁](https://zh.wiktionary.org/wiki/%E5%92%81#%E6%BC%A2%E8%AA%9E "咁")[犀利](https://zh.wiktionary.org/wiki/%E7%8A%80%E5%88%A9#%E6%BC%A2%E8%AA%9E "犀利")[嘅](https://zh.wiktionary.org/wiki/%E5%98%85#%E6%BC%A2%E8%AA%9E "嘅")? [[粵語](https://zh.wikipedia.org/wiki/%E7%B2%B5%E8%AA%9E "w:粵語")]  ―  _gam3 sai1 lei6 ge3?_ [[粵拼](https://zh.wikipedia.org/wiki/%E9%A6%99%E6%B8%AF%E8%AA%9E%E8%A8%80%E5%AD%B8%E5%AD%B8%E6%9C%83%E7%B2%B5%E8%AA%9E%E6%8B%BC%E9%9F%B3%E6%96%B9%E6%A1%88 "w:香港語言學學會粵語拼音方案")]  ―  这么厉害的呀!|
% |[撳](https://zh.wiktionary.org/wiki/%E6%92%B3 "撳")|gam6|rqqw|qcno|按壓|撳電掣 ― 按电钮|
% |[緊](https://zh.wiktionary.org/wiki/%E7%B7%8A "緊")|gan2|ahni|sevif|正在|睇緊電視 ― 正在看电视|
% |[梗](https://zh.wiktionary.org/wiki/%E6%A2%97 "梗")|gang2|sgjq|dmlk|當然|[梗係](https://zh.wiktionary.org/wiki/%E6%A2%97%E4%BF%82#%E6%BC%A2%E8%AA%9E "梗係") / [梗系](https://zh.wiktionary.org/w/index.php?title=%E6%A2%97%E7%B3%BB&action=edit&redlink=1 "梗系(页面不存在)") [[粵語](https://zh.wikipedia.org/wiki/%E7%B2%B5%E8%AA%9E "w:粵語")]  ―  _gang2 hai6_ [[粵拼](https://zh.wikipedia.org/wiki/%E9%A6%99%E6%B8%AF%E8%AA%9E%E8%A8%80%E5%AD%B8%E5%AD%B8%E6%9C%83%E7%B2%B5%E8%AA%9E%E6%8B%BC%E9%9F%B3%E6%96%B9%E6%A1%88 "w:香港語言學學會粵語拼音方案")]  ―  当然是|
% |[𥄫](https://zh.wiktionary.org/wiki/%F0%A5%84%AB "𥄫")|gap6|heyy|bunhe|保持注視|𥄫實佢哋 ― 盯着他们|
% |[𨳊](https://zh.wiktionary.org/wiki/%F0%A8%B3%8A "𨳊")([㞗](https://zh.wiktionary.org/wiki/%E3%9E%97 "㞗"))|gau1|nfiy|sije|粗口,男性外生殖器||
% |[嚿](https://zh.wiktionary.org/wiki/%E5%9A%BF "嚿")|gau6|kawv|rtox|量詞|一嚿石 ― 一块石头|
% |[嘅](https://zh.wiktionary.org/wiki/%E5%98%85 "嘅")|ge3|kvcq|raiu|從屬關係|[你](https://zh.wiktionary.org/wiki/%E4%BD%A0#%E6%BC%A2%E8%AA%9E "你")[嘅](https://zh.wiktionary.org/wiki/%E5%98%85#%E6%BC%A2%E8%AA%9E "嘅")[諗法](https://zh.wiktionary.org/wiki/%E8%AB%97%E6%B3%95#%E6%BC%A2%E8%AA%9E "諗法")[呢](https://zh.wiktionary.org/wiki/%E5%91%A2#%E6%BC%A2%E8%AA%9E "呢")? / [你](https://zh.wiktionary.org/wiki/%E4%BD%A0#%E6%BC%A2%E8%AA%9E "你")[嘅](https://zh.wiktionary.org/wiki/%E5%98%85#%E6%BC%A2%E8%AA%9E "嘅")[谂法](https://zh.wiktionary.org/wiki/%E8%AB%97%E6%B3%95 "諗法")[呢](https://zh.wiktionary.org/wiki/%E5%91%A2#%E6%BC%A2%E8%AA%9E "呢")? [[粵語](https://zh.wikipedia.org/wiki/%E7%B2%B5%E8%AA%9E "w:粵語")]  ―  _nei5 ge3 nam2 faat3 ne1?_ [[粵拼](https://zh.wikipedia.org/wiki/%E9%A6%99%E6%B8%AF%E8%AA%9E%E8%A8%80%E5%AD%B8%E5%AD%B8%E6%9C%83%E7%B2%B5%E8%AA%9E%E6%8B%BC%E9%9F%B3%E6%96%B9%E6%A1%88 "w:香港語言學學會粵語拼音方案")]  ―  你的想法呢?|
% |[嗰](https://zh.wiktionary.org/wiki/%E5%97%B0 "嗰")|go2|kwld|rowr|彼|[嗰啲](https://zh.wiktionary.org/wiki/%E5%97%B0%E5%95%B2#%E6%BC%A2%E8%AA%9E "嗰啲") [[粵語](https://zh.wikipedia.org/wiki/%E7%B2%B5%E8%AA%9E "w:粵語")]  ―  _go2 di1_ [[粵拼](https://zh.wikipedia.org/wiki/%E9%A6%99%E6%B8%AF%E8%AA%9E%E8%A8%80%E5%AD%B8%E5%AD%B8%E6%9C%83%E7%B2%B5%E8%AA%9E%E6%8B%BC%E9%9F%B3%E6%96%B9%E6%A1%88 "w:香港語言學學會粵語拼音方案")]  ―  那些|
% |[閪](https://zh.wiktionary.org/wiki/%E9%96%AA "閪")([屄](https://zh.wiktionary.org/wiki/%E5%B1%84 "屄"))|hai1|usd|anmcw|粗口,女性外生殖器||
% |[喺](https://zh.wiktionary.org/wiki/%E5%96%BA "喺")|hai2|kwti|rohf|位於|[喺](https://zh.wiktionary.org/wiki/%E5%96%BA#%E6%BC%A2%E8%AA%9E "喺")[邊度](https://zh.wiktionary.org/wiki/%E9%82%8A%E5%BA%A6#%E6%BC%A2%E8%AA%9E "邊度")[呢](https://zh.wiktionary.org/wiki/%E5%91%A2#%E6%BC%A2%E8%AA%9E "呢")? / [喺](https://zh.wiktionary.org/wiki/%E5%96%BA#%E6%BC%A2%E8%AA%9E "喺")[边度](https://zh.wiktionary.org/wiki/%E9%82%8A%E5%BA%A6 "邊度")[呢](https://zh.wiktionary.org/wiki/%E5%91%A2#%E6%BC%A2%E8%AA%9E "呢")? [[粵語](https://zh.wikipedia.org/wiki/%E7%B2%B5%E8%AA%9E "w:粵語")]  ―  _hai2 bin1 dou6 ne1?_ [[粵拼](https://zh.wikipedia.org/wiki/%E9%A6%99%E6%B8%AF%E8%AA%9E%E8%A8%80%E5%AD%B8%E5%AD%B8%E6%9C%83%E7%B2%B5%E8%AA%9E%E6%8B%BC%E9%9F%B3%E6%96%B9%E6%A1%88 "w:香港語言學學會粵語拼音方案")]  ―  在什么地方呢?|
% |[繫](https://zh.wiktionary.org/wiki/%E7%B9%AB "繫")|hai6|lbmi|jevif|連接|[聯繫](https://zh.wiktionary.org/wiki/%E8%81%AF%E7%B9%AB#%E6%BC%A2%E8%AA%9E "聯繫")、[維繫](https://zh.wiktionary.org/wiki/%E7%BB%B4%E7%B3%BB "维系") / [联系](https://zh.wiktionary.org/wiki/%E8%81%94%E7%B3%BB#%E6%BC%A2%E8%AA%9E "联系")、[维系](https://zh.wiktionary.org/wiki/%E7%BB%B4%E7%B3%BB#%E6%BC%A2%E8%AA%9E "维系") [[粵語](https://zh.wikipedia.org/wiki/%E7%B2%B5%E8%AA%9E "w:粵語")]  ―  _lyun4 hai6, wai4 hai6_ [[粵拼](https://zh.wikipedia.org/wiki/%E9%A6%99%E6%B8%AF%E8%AA%9E%E8%A8%80%E5%AD%B8%E5%AD%B8%E6%9C%83%E7%B2%B5%E8%AA%9E%E6%8B%BC%E9%9F%B3%E6%96%B9%E6%A1%88 "w:香港語言學學會粵語拼音方案")]  ―|
% |[係](https://zh.wiktionary.org/wiki/%E4%BF%82 "係")|hai6|wtxi|ohvf|表肯定、答應及事物關係|[係](https://zh.wiktionary.org/wiki/%E4%BF%82#%E6%BC%A2%E8%AA%9E "係")[乜嘢](https://zh.wiktionary.org/wiki/%E4%B9%9C%E5%98%A2#%E6%BC%A2%E8%AA%9E "乜嘢")[嚟㗎](https://zh.wiktionary.org/w/index.php?title=%E5%9A%9F%E3%97%8E&action=edit&redlink=1 "嚟㗎(页面不存在)")? / [系](https://zh.wiktionary.org/wiki/%E7%B3%BB#%E6%BC%A2%E8%AA%9E "系")[乜嘢](https://zh.wiktionary.org/wiki/%E4%B9%9C%E5%98%A2#%E6%BC%A2%E8%AA%9E "乜嘢")[嚟㗎](https://zh.wiktionary.org/w/index.php?title=%E5%9A%9F%E3%97%8E&action=edit&redlink=1 "嚟㗎(页面不存在)")? [[粵語](https://zh.wikipedia.org/wiki/%E7%B2%B5%E8%AA%9E "w:粵語")]  ―  _hai6 mat1 je5 lai4 gaa3?_ [[粵拼](https://zh.wikipedia.org/wiki/%E9%A6%99%E6%B8%AF%E8%AA%9E%E8%A8%80%E5%AD%B8%E5%AD%B8%E6%9C%83%E7%B2%B5%E8%AA%9E%E6%8B%BC%E9%9F%B3%E6%96%B9%E6%A1%88 "w:香港語言學學會粵語拼音方案")]  ―  是什么来的?|
% |[系](https://zh.wiktionary.org/wiki/%E7%B3%BB "系")|hai6|txiu|hvif|聯屬關係;學科單位|[系統](https://zh.wiktionary.org/wiki/%E7%B3%BB%E7%B5%B1#%E6%BC%A2%E8%AA%9E "系統")、[化學系](https://zh.wiktionary.org/wiki/%E5%8C%96%E5%AD%B8%E7%B3%BB#%E6%BC%A2%E8%AA%9E "化學系") / [系统](https://zh.wiktionary.org/wiki/%E7%B3%BB%E7%BB%9F#%E6%BC%A2%E8%AA%9E "系统")、[化学系](https://zh.wiktionary.org/wiki/%E5%8C%96%E5%AD%B8%E7%B3%BB "化學系") [[粵語](https://zh.wikipedia.org/wiki/%E7%B2%B5%E8%AA%9E "w:粵語")]  ―  _hai6 tung2, faa3 hok6 hai6_ [[粵拼](https://zh.wikipedia.org/wiki/%E9%A6%99%E6%B8%AF%E8%AA%9E%E8%A8%80%E5%AD%B8%E5%AD%B8%E6%9C%83%E7%B2%B5%E8%AA%9E%E6%8B%BC%E9%9F%B3%E6%96%B9%E6%A1%88 "w:香港語言學學會粵語拼音方案")]  ―|
% |[冚](https://zh.wiktionary.org/wiki/%E5%86%9A "冚")|ham6|pmj|bu|全部|冚𠾴唥(ham6 baang6 laang6) ― 所有东西|
% |[廿](https://zh.wiktionary.org/wiki/%E5%BB%BF "廿")|jaa6|aghg|t|二十|廿幾歲 ― 二十几岁|
% |[嘢](https://zh.wiktionary.org/wiki/%E5%98%A2 "嘢")|je5|kjfb|rwgn|事物|[有](https://zh.wiktionary.org/wiki/%E6%9C%89#%E6%BC%A2%E8%AA%9E "有")[嘢](https://zh.wiktionary.org/wiki/%E5%98%A2#%E6%BC%A2%E8%AA%9E "嘢")[睇](https://zh.wiktionary.org/wiki/%E7%9D%87#%E6%BC%A2%E8%AA%9E "睇") [[粵語](https://zh.wikipedia.org/wiki/%E7%B2%B5%E8%AA%9E "w:粵語")]  ―  _jau5 je5 tai2_ [[粵拼](https://zh.wikipedia.org/wiki/%E9%A6%99%E6%B8%AF%E8%AA%9E%E8%A8%80%E5%AD%B8%E5%AD%B8%E6%9C%83%E7%B2%B5%E8%AA%9E%E6%8B%BC%E9%9F%B3%E6%96%B9%E6%A1%88 "w:香港語言學學會粵語拼音方案")]  ―  有东西看|
% |[咦](https://zh.wiktionary.org/wiki/%E5%92%A6 "咦")|ji2|kgxw|rkn|語氣助詞(表奇怪)|[咦](https://zh.wiktionary.org/wiki/%E5%92%A6#%E6%BC%A2%E8%AA%9E "咦")?[噉樣](https://zh.wiktionary.org/wiki/%E5%99%89%E6%A8%A3#%E6%BC%A2%E8%AA%9E "噉樣")[嘅](https://zh.wiktionary.org/wiki/%E5%98%85#%E6%BC%A2%E8%AA%9E "嘅")? / [咦](https://zh.wiktionary.org/wiki/%E5%92%A6#%E6%BC%A2%E8%AA%9E "咦")?[噉样](https://zh.wiktionary.org/wiki/%E5%99%89%E6%A8%A3 "噉樣")[嘅](https://zh.wiktionary.org/wiki/%E5%98%85#%E6%BC%A2%E8%AA%9E "嘅")? [[粵語](https://zh.wikipedia.org/wiki/%E7%B2%B5%E8%AA%9E "w:粵語")]  ―  _ji2? gam2 joeng6 ge3?_ [[粵拼](https://zh.wikipedia.org/wiki/%E9%A6%99%E6%B8%AF%E8%AA%9E%E8%A8%80%E5%AD%B8%E5%AD%B8%E6%9C%83%E7%B2%B5%E8%AA%9E%E6%8B%BC%E9%9F%B3%E6%96%B9%E6%A1%88 "w:香港語言學學會粵語拼音方案")]  ―|
% |[𢫏](https://zh.wiktionary.org/wiki/%F0%A2%AB%8F "𢫏")|kam2|pmj|qbu|覆蓋|[𢫏](https://zh.wiktionary.org/wiki/%F0%A2%AB%8F#%E6%BC%A2%E8%AA%9E "𢫏")[住](https://zh.wiktionary.org/wiki/%E4%BD%8F#%E6%BC%A2%E8%AA%9E "住") [[粵語](https://zh.wikipedia.org/wiki/%E7%B2%B5%E8%AA%9E "w:粵語")]  ―  _kam2 zyu6_ [[粵拼](https://zh.wikipedia.org/wiki/%E9%A6%99%E6%B8%AF%E8%AA%9E%E8%A8%80%E5%AD%B8%E5%AD%B8%E6%9C%83%E7%B2%B5%E8%AA%9E%E6%8B%BC%E9%9F%B3%E6%96%B9%E6%A1%88 "w:香港語言學學會粵語拼音方案")]  ―  盖着|
% |[企](https://zh.wiktionary.org/wiki/%E4%BC%81 "企")|kei5|whf|oylm|立|企喺度 ― 站在这里|
% |[佢](https://zh.wiktionary.org/wiki/%E4%BD%A2 "佢")|keoi5|wan|oss|第三人稱單數|[佢](https://zh.wiktionary.org/wiki/%E4%BD%A2#%E6%BC%A2%E8%AA%9E "佢")[係](https://zh.wiktionary.org/wiki/%E4%BF%82#%E6%BC%A2%E8%AA%9E "係")[邊個](https://zh.wiktionary.org/wiki/%E9%82%8A%E5%80%8B#%E6%BC%A2%E8%AA%9E "邊個")? / [佢](https://zh.wiktionary.org/wiki/%E4%BD%A2#%E6%BC%A2%E8%AA%9E "佢")[系](https://zh.wiktionary.org/wiki/%E7%B3%BB#%E6%BC%A2%E8%AA%9E "系")[边个](https://zh.wiktionary.org/wiki/%E9%82%8A%E5%80%8B "邊個")? [[粵語](https://zh.wikipedia.org/wiki/%E7%B2%B5%E8%AA%9E "w:粵語")]  ―  _keoi5 hai6 bin1 go3?_ [[粵拼](https://zh.wikipedia.org/wiki/%E9%A6%99%E6%B8%AF%E8%AA%9E%E8%A8%80%E5%AD%B8%E5%AD%B8%E6%9C%83%E7%B2%B5%E8%AA%9E%E6%8B%BC%E9%9F%B3%E6%96%B9%E6%A1%88 "w:香港語言學學會粵語拼音方案")]  ―  他是谁?|
% |[嘞](https://zh.wiktionary.org/wiki/%E5%98%9E "嘞")|laak3|kafl|rtjs|語氣助詞,表示確定完成|[做](https://zh.wiktionary.org/wiki/%E5%81%9A#%E6%BC%A2%E8%AA%9E "做")[晒](https://zh.wiktionary.org/wiki/%E6%99%92#%E6%BC%A2%E8%AA%9E "晒")[啲](https://zh.wiktionary.org/wiki/%E5%95%B2#%E6%BC%A2%E8%AA%9E "啲")[功課](https://zh.wiktionary.org/wiki/%E5%8A%9F%E8%AA%B2#%E6%BC%A2%E8%AA%9E "功課")[嘞](https://zh.wiktionary.org/wiki/%E5%98%9E#%E6%BC%A2%E8%AA%9E "嘞") [[粵語](https://zh.wikipedia.org/wiki/%E7%B2%B5%E8%AA%9E "w:粵語"),[繁體](https://zh.wiktionary.org/wiki/%E7%B9%81%E9%AB%94%E4%B8%AD%E6%96%87 "繁體中文")]  <br>[做](https://zh.wiktionary.org/wiki/%E5%81%9A#%E6%BC%A2%E8%AA%9E "做")[晒](https://zh.wiktionary.org/wiki/%E6%99%92#%E6%BC%A2%E8%AA%9E "晒")[啲](https://zh.wiktionary.org/wiki/%E5%95%B2#%E6%BC%A2%E8%AA%9E "啲")[功课](https://zh.wiktionary.org/wiki/%E5%8A%9F%E8%AF%BE#%E6%BC%A2%E8%AA%9E "功课")[嘞](https://zh.wiktionary.org/wiki/%E5%98%9E#%E6%BC%A2%E8%AA%9E "嘞") [[粵語](https://zh.wikipedia.org/wiki/%E7%B2%B5%E8%AA%9E "w:粵語"),[簡體](https://zh.wiktionary.org/wiki/%E7%B0%A1%E9%AB%94%E4%B8%AD%E6%96%87 "簡體中文")]<br><br>_zou6 saai3 di1 gung1 fo3 laak3_ [[粵拼](https://zh.wikipedia.org/wiki/%E9%A6%99%E6%B8%AF%E8%AA%9E%E8%A8%80%E5%AD%B8%E5%AD%B8%E6%9C%83%E7%B2%B5%E8%AA%9E%E6%8B%BC%E9%9F%B3%E6%96%B9%E6%A1%88 "w:香港語言學學會粵語拼音方案")]|
% |[嚟](https://zh.wiktionary.org/wiki/%E5%9A%9F "嚟")([來](https://zh.wiktionary.org/wiki/%E4%BE%86 "來"))|lai4|ktqi|rhhe|來|[入嚟](https://zh.wiktionary.org/wiki/%E5%85%A5%E5%9A%9F#%E6%BC%A2%E8%AA%9E "入嚟") [[粵語](https://zh.wikipedia.org/wiki/%E7%B2%B5%E8%AA%9E "w:粵語")]  ―  _jap6 lai4_ [[粵拼](https://zh.wikipedia.org/wiki/%E9%A6%99%E6%B8%AF%E8%AA%9E%E8%A8%80%E5%AD%B8%E5%AD%B8%E6%9C%83%E7%B2%B5%E8%AA%9E%E6%8B%BC%E9%9F%B3%E6%96%B9%E6%A1%88 "w:香港語言學學會粵語拼音方案")]  ―  进来|
% |[冧](https://zh.wiktionary.org/wiki/%E5%86%A7 "冧")|lam1|pssu|bdd|陶醉於|[佢](https://zh.wiktionary.org/wiki/%E4%BD%A2#%E6%BC%A2%E8%AA%9E "佢")[冧](https://zh.wiktionary.org/wiki/%E5%86%A7#%E6%BC%A2%E8%AA%9E "冧")[我](https://zh.wiktionary.org/wiki/%E6%88%91#%E6%BC%A2%E8%AA%9E "我") [[粵語](https://zh.wikipedia.org/wiki/%E7%B2%B5%E8%AA%9E "w:粵語")]  ―  _keoi5 lam1 ngo5_ [[粵拼](https://zh.wikipedia.org/wiki/%E9%A6%99%E6%B8%AF%E8%AA%9E%E8%A8%80%E5%AD%B8%E5%AD%B8%E6%9C%83%E7%B2%B5%E8%AA%9E%E6%8B%BC%E9%9F%B3%E6%96%B9%E6%A1%88 "w:香港語言學學會粵語拼音方案")]  ―  她喜欢我|
% |[𨶙](https://zh.wiktionary.org/wiki/%F0%A8%B6%99 "𨶙")([𡳞](https://zh.wiktionary.org/wiki/%F0%A1%B3%9E "𡳞"))|lan2|ucex|anibp|粗口,男性外生殖器||
% |[哩](https://zh.wiktionary.org/wiki/%E5%93%A9 "哩")|lei5|kjfg|rwg|英美製長度單位|一哩等於5280英尺,合1609米|
% |[叻](https://zh.wiktionary.org/wiki/%E5%8F%BB "叻")|lek1|kln|rks|聰明|[佢](https://zh.wiktionary.org/wiki/%E4%BD%A2#%E6%BC%A2%E8%AA%9E "佢")[讀書](https://zh.wiktionary.org/wiki/%E8%AE%80%E6%9B%B8#%E6%BC%A2%E8%AA%9E "讀書")[好](https://zh.wiktionary.org/wiki/%E5%A5%BD#%E6%BC%A2%E8%AA%9E "好")[叻](https://zh.wiktionary.org/wiki/%E5%8F%BB#%E6%BC%A2%E8%AA%9E "叻")[㗎](https://zh.wiktionary.org/wiki/%E3%97%8E#%E6%BC%A2%E8%AA%9E "㗎") [[粵語](https://zh.wikipedia.org/wiki/%E7%B2%B5%E8%AA%9E "w:粵語"),[繁體](https://zh.wiktionary.org/wiki/%E7%B9%81%E9%AB%94%E4%B8%AD%E6%96%87 "繁體中文")]  <br>[佢](https://zh.wiktionary.org/wiki/%E4%BD%A2#%E6%BC%A2%E8%AA%9E "佢")[读书](https://zh.wiktionary.org/wiki/%E8%AF%BB%E4%B9%A6#%E6%BC%A2%E8%AA%9E "读书")[好](https://zh.wiktionary.org/wiki/%E5%A5%BD#%E6%BC%A2%E8%AA%9E "好")[叻](https://zh.wiktionary.org/wiki/%E5%8F%BB#%E6%BC%A2%E8%AA%9E "叻")[㗎](https://zh.wiktionary.org/wiki/%E3%97%8E#%E6%BC%A2%E8%AA%9E "㗎") [[粵語](https://zh.wikipedia.org/wiki/%E7%B2%B5%E8%AA%9E "w:粵語"),[簡體](https://zh.wiktionary.org/wiki/%E7%B0%A1%E9%AB%94%E4%B8%AD%E6%96%87 "簡體中文")]<br><br>_keoi5 duk6 syu1 hou2 lek1 gaa3_ [[粵拼](https://zh.wikipedia.org/wiki/%E9%A6%99%E6%B8%AF%E8%AA%9E%E8%A8%80%E5%AD%B8%E5%AD%B8%E6%9C%83%E7%B2%B5%E8%AA%9E%E6%8B%BC%E9%9F%B3%E6%96%B9%E6%A1%88 "w:香港語言學學會粵語拼音方案")]<br><br>他读书很棒的呀|
% |[靚](https://zh.wiktionary.org/wiki/%E9%9D%9A "靚")|leng3|gemq|qbbuu|美麗|[靚仔](https://zh.wiktionary.org/wiki/%E9%9D%9A%E4%BB%94#%E6%BC%A2%E8%AA%9E "靚仔") / [靓仔](https://zh.wiktionary.org/wiki/%E9%9D%93%E4%BB%94#%E6%BC%A2%E8%AA%9E "靓仔") [[粵語](https://zh.wikipedia.org/wiki/%E7%B2%B5%E8%AA%9E "w:粵語")]  ―  _leng3 zai2_ [[粵拼](https://zh.wikipedia.org/wiki/%E9%A6%99%E6%B8%AF%E8%AA%9E%E8%A8%80%E5%AD%B8%E5%AD%B8%E6%9C%83%E7%B2%B5%E8%AA%9E%E6%8B%BC%E9%9F%B3%E6%96%B9%E6%A1%88 "w:香港語言學學會粵語拼音方案")]  ―  帅哥|
% |[拎](https://zh.wiktionary.org/wiki/%E6%8B%8E "拎")|ling1|rwyc|qoii/qoni|提(取)起|[拎](https://zh.wiktionary.org/wiki/%E6%8B%8E#%E6%BC%A2%E8%AA%9E "拎")[走](https://zh.wiktionary.org/wiki/%E8%B5%B0#%E6%BC%A2%E8%AA%9E "走") [[粵語](https://zh.wikipedia.org/wiki/%E7%B2%B5%E8%AA%9E "w:粵語")]  ―  _ling1 zau2_ [[粵拼](https://zh.wikipedia.org/wiki/%E9%A6%99%E6%B8%AF%E8%AA%9E%E8%A8%80%E5%AD%B8%E5%AD%B8%E6%9C%83%E7%B2%B5%E8%AA%9E%E6%8B%BC%E9%9F%B3%E6%96%B9%E6%A1%88 "w:香港語言學學會粵語拼音方案")]  ―|
% |[撩](https://zh.wiktionary.org/wiki/%E6%92%A9 "撩")|liu4|rdui|qkcf|挑弄|[撩是鬥非](https://zh.wiktionary.org/w/index.php?title=%E6%92%A9%E6%98%AF%E9%AC%A5%E9%9D%9E&action=edit&redlink=1 "撩是鬥非(页面不存在)") / [撩是斗非](https://zh.wiktionary.org/w/index.php?title=%E6%92%A9%E6%98%AF%E6%96%97%E9%9D%9E&action=edit&redlink=1 "撩是斗非(页面不存在)") [[粵語](https://zh.wikipedia.org/wiki/%E7%B2%B5%E8%AA%9E "w:粵語")]  ―  _liu4 si6 dau3 fei1_ [[粵拼](https://zh.wikipedia.org/wiki/%E9%A6%99%E6%B8%AF%E8%AA%9E%E8%A8%80%E5%AD%B8%E5%AD%B8%E6%9C%83%E7%B2%B5%E8%AA%9E%E6%8B%BC%E9%9F%B3%E6%96%B9%E6%A1%88 "w:香港語言學學會粵語拼音方案")]  ―  惹是生非|
% |[囉](https://zh.wiktionary.org/wiki/%E5%9B%89 "囉")|lo1|klqy|rwlg|語氣助詞,表示應當得此結果|[咪](https://zh.wiktionary.org/wiki/%E5%92%AA#%E6%BC%A2%E8%AA%9E "咪")[就](https://zh.wiktionary.org/wiki/%E5%B0%B1#%E6%BC%A2%E8%AA%9E "就")[係](https://zh.wiktionary.org/wiki/%E4%BF%82#%E6%BC%A2%E8%AA%9E "係")[噉](https://zh.wiktionary.org/wiki/%E5%99%89#%E6%BC%A2%E8%AA%9E "噉")[囉](https://zh.wiktionary.org/wiki/%E5%9B%89#%E6%BC%A2%E8%AA%9E "囉") / [咪](https://zh.wiktionary.org/wiki/%E5%92%AA#%E6%BC%A2%E8%AA%9E "咪")[就](https://zh.wiktionary.org/wiki/%E5%B0%B1#%E6%BC%A2%E8%AA%9E "就")[系](https://zh.wiktionary.org/wiki/%E7%B3%BB#%E6%BC%A2%E8%AA%9E "系")[噉](https://zh.wiktionary.org/wiki/%E5%99%89#%E6%BC%A2%E8%AA%9E "噉")[啰](https://zh.wiktionary.org/wiki/%E5%95%B0#%E6%BC%A2%E8%AA%9E "啰") [[粵語](https://zh.wikipedia.org/wiki/%E7%B2%B5%E8%AA%9E "w:粵語")]  ―  _mai1 zau6 hai6 gam2 lo1_ [[粵拼](https://zh.wikipedia.org/wiki/%E9%A6%99%E6%B8%AF%E8%AA%9E%E8%A8%80%E5%AD%B8%E5%AD%B8%E6%9C%83%E7%B2%B5%E8%AA%9E%E6%8B%BC%E9%9F%B3%E6%96%B9%E6%A1%88 "w:香港語言學學會粵語拼音方案")]  ―  不就是这样嘛|
% |[攞](https://zh.wiktionary.org/wiki/%E6%94%9E "攞")|lo2|rlxy|qwlg|拿取|[攞](https://zh.wiktionary.org/wiki/%E6%94%9E#%E6%BC%A2%E8%AA%9E "攞")[嘢](https://zh.wiktionary.org/wiki/%E5%98%A2#%E6%BC%A2%E8%AA%9E "嘢") / [𫽋](https://zh.wiktionary.org/wiki/%F0%AB%BD%8B#%E6%BC%A2%E8%AA%9E "𫽋")[嘢](https://zh.wiktionary.org/wiki/%E5%98%A2#%E6%BC%A2%E8%AA%9E "嘢") [[粵語](https://zh.wikipedia.org/wiki/%E7%B2%B5%E8%AA%9E "w:粵語")]  ―  _lo2 je5_ [[粵拼](https://zh.wikipedia.org/wiki/%E9%A6%99%E6%B8%AF%E8%AA%9E%E8%A8%80%E5%AD%B8%E5%AD%B8%E6%9C%83%E7%B2%B5%E8%AA%9E%E6%8B%BC%E9%9F%B3%E6%96%B9%E6%A1%88 "w:香港語言學學會粵語拼音方案")]  ―  拿东西|
% |[咯](https://zh.wiktionary.org/wiki/%E5%92%AF "咯")|lo3|rher|lo|語氣助詞表示特定條件下理應如此|[噉](https://zh.wiktionary.org/wiki/%E5%99%89#%E6%BC%A2%E8%AA%9E "噉")[就](https://zh.wiktionary.org/wiki/%E5%B0%B1#%E6%BC%A2%E8%AA%9E "就")[真係](https://zh.wiktionary.org/wiki/%E7%9C%9F%E4%BF%82#%E6%BC%A2%E8%AA%9E "真係")[該煨](https://zh.wiktionary.org/w/index.php?title=%E8%A9%B2%E7%85%A8&action=edit&redlink=1 "該煨(页面不存在)")[咯](https://zh.wiktionary.org/wiki/%E5%92%AF#%E6%BC%A2%E8%AA%9E "咯") [[粵語](https://zh.wikipedia.org/wiki/%E7%B2%B5%E8%AA%9E "w:粵語"),[繁體](https://zh.wiktionary.org/wiki/%E7%B9%81%E9%AB%94%E4%B8%AD%E6%96%87 "繁體中文")]  <br>[噉](https://zh.wiktionary.org/wiki/%E5%99%89#%E6%BC%A2%E8%AA%9E "噉")[就](https://zh.wiktionary.org/wiki/%E5%B0%B1#%E6%BC%A2%E8%AA%9E "就")[真系](https://zh.wiktionary.org/wiki/%E7%9C%9F%E7%B3%BB#%E6%BC%A2%E8%AA%9E "真系")[该煨](https://zh.wiktionary.org/w/index.php?title=%E8%AF%A5%E7%85%A8&action=edit&redlink=1 "该煨(页面不存在)")[咯](https://zh.wiktionary.org/wiki/%E5%92%AF#%E6%BC%A2%E8%AA%9E "咯") [[粵語](https://zh.wikipedia.org/wiki/%E7%B2%B5%E8%AA%9E "w:粵語"),[簡體](https://zh.wiktionary.org/wiki/%E7%B0%A1%E9%AB%94%E4%B8%AD%E6%96%87 "簡體中文")]<br><br>_gam2 zau6 zan1 hai6 goi1 wui1 lo3_ [[粵拼](https://zh.wikipedia.org/wiki/%E9%A6%99%E6%B8%AF%E8%AA%9E%E8%A8%80%E5%AD%B8%E5%AD%B8%E6%9C%83%E7%B2%B5%E8%AA%9E%E6%8B%BC%E9%9F%B3%E6%96%B9%E6%A1%88 "w:香港語言學學會粵語拼音方案")]|
% |[咯](https://zh.wiktionary.org/wiki/%E5%92%AF "咯")|lok3|rher|lo|語氣助詞表示反詰、申辯|[佢](https://zh.wiktionary.org/wiki/%E4%BD%A2#%E6%BC%A2%E8%AA%9E "佢")[夠係](https://zh.wiktionary.org/w/index.php?title=%E5%A4%A0%E4%BF%82&action=edit&redlink=1 "夠係(页面不存在)")[咯](https://zh.wiktionary.org/wiki/%E5%92%AF#%E6%BC%A2%E8%AA%9E "咯")! / [佢](https://zh.wiktionary.org/wiki/%E4%BD%A2#%E6%BC%A2%E8%AA%9E "佢")[够系](https://zh.wiktionary.org/w/index.php?title=%E5%A4%9F%E7%B3%BB&action=edit&redlink=1 "够系(页面不存在)")[咯](https://zh.wiktionary.org/wiki/%E5%92%AF#%E6%BC%A2%E8%AA%9E "咯")! [[粵語](https://zh.wikipedia.org/wiki/%E7%B2%B5%E8%AA%9E "w:粵語")]  ―  _keoi5 gau3 hai6 lok3!_ [[粵拼](https://zh.wikipedia.org/wiki/%E9%A6%99%E6%B8%AF%E8%AA%9E%E8%A8%80%E5%AD%B8%E5%AD%B8%E6%9C%83%E7%B2%B5%E8%AA%9E%E6%8B%BC%E9%9F%B3%E6%96%B9%E6%A1%88 "w:香港語言學學會粵語拼音方案")]  ―  他不也是嘛|
% |[唔](https://zh.wiktionary.org/wiki/%E5%94%94 "唔")|m4|kgkg|rmmr|否定|[唔係](https://zh.wiktionary.org/wiki/%E5%94%94%E4%BF%82#%E6%BC%A2%E8%AA%9E "唔係") / [唔系](https://zh.wiktionary.org/wiki/%E5%94%94%E7%B3%BB#%E6%BC%A2%E8%AA%9E "唔系") [[粵語](https://zh.wikipedia.org/wiki/%E7%B2%B5%E8%AA%9E "w:粵語")]  ―  _m4 hai6_ [[粵拼](https://zh.wikipedia.org/wiki/%E9%A6%99%E6%B8%AF%E8%AA%9E%E8%A8%80%E5%AD%B8%E5%AD%B8%E6%9C%83%E7%B2%B5%E8%AA%9E%E6%8B%BC%E9%9F%B3%E6%96%B9%E6%A1%88 "w:香港語言學學會粵語拼音方案")]  ―  不是|
% |[咪](https://zh.wiktionary.org/wiki/%E5%92%AA "咪")|mai5|koy|rfd|勿|[咪](https://zh.wiktionary.org/wiki/%E5%92%AA#%E6%BC%A2%E8%AA%9E "咪")[走](https://zh.wiktionary.org/wiki/%E8%B5%B0#%E6%BC%A2%E8%AA%9E "走") [[粵語](https://zh.wikipedia.org/wiki/%E7%B2%B5%E8%AA%9E "w:粵語")]  ―  _mai5 zau2_ [[粵拼](https://zh.wikipedia.org/wiki/%E9%A6%99%E6%B8%AF%E8%AA%9E%E8%A8%80%E5%AD%B8%E5%AD%B8%E6%9C%83%E7%B2%B5%E8%AA%9E%E6%8B%BC%E9%9F%B3%E6%96%B9%E6%A1%88 "w:香港語言學學會粵語拼音方案")]  ―  不要走~|
% |[乜](https://zh.wiktionary.org/wiki/%E4%B9%9C "乜")|mat1|nnv|pn/ps|何|[你](https://zh.wiktionary.org/wiki/%E4%BD%A0#%E6%BC%A2%E8%AA%9E "你")[有](https://zh.wiktionary.org/wiki/%E6%9C%89#%E6%BC%A2%E8%AA%9E "有")[乜](https://zh.wiktionary.org/wiki/%E4%B9%9C#%E6%BC%A2%E8%AA%9E "乜")? [[粵語](https://zh.wikipedia.org/wiki/%E7%B2%B5%E8%AA%9E "w:粵語")]  ―  _nei5 jau5 mat1?_ [[粵拼](https://zh.wikipedia.org/wiki/%E9%A6%99%E6%B8%AF%E8%AA%9E%E8%A8%80%E5%AD%B8%E5%AD%B8%E6%9C%83%E7%B2%B5%E8%AA%9E%E6%8B%BC%E9%9F%B3%E6%96%B9%E6%A1%88 "w:香港語言學學會粵語拼音方案")]  ―  你有什么?|
% |[咩](https://zh.wiktionary.org/wiki/%E5%92%A9 "咩")|me1|kudh|rtq|語氣助詞,表疑問|[佢](https://zh.wiktionary.org/wiki/%E4%BD%A2#%E6%BC%A2%E8%AA%9E "佢")[瞓](https://zh.wiktionary.org/wiki/%E7%9E%93#%E6%BC%A2%E8%AA%9E "瞓")[咗](https://zh.wiktionary.org/wiki/%E5%92%97#%E6%BC%A2%E8%AA%9E "咗")[喇](https://zh.wiktionary.org/wiki/%E5%96%87#%E6%BC%A2%E8%AA%9E "喇")[咩](https://zh.wiktionary.org/wiki/%E5%92%A9#%E6%BC%A2%E8%AA%9E "咩")? / [佢](https://zh.wiktionary.org/wiki/%E4%BD%A2#%E6%BC%A2%E8%AA%9E "佢")[𰥛](https://zh.wiktionary.org/wiki/%F0%B0%A5%9B#%E6%BC%A2%E8%AA%9E "𰥛")[咗](https://zh.wiktionary.org/wiki/%E5%92%97#%E6%BC%A2%E8%AA%9E "咗")[喇](https://zh.wiktionary.org/wiki/%E5%96%87#%E6%BC%A2%E8%AA%9E "喇")[咩](https://zh.wiktionary.org/wiki/%E5%92%A9#%E6%BC%A2%E8%AA%9E "咩")? [[粵語](https://zh.wikipedia.org/wiki/%E7%B2%B5%E8%AA%9E "w:粵語")]  ―  _keoi5 fan3 zo2 laa3 me1?_ [[粵拼](https://zh.wikipedia.org/wiki/%E9%A6%99%E6%B8%AF%E8%AA%9E%E8%A8%80%E5%AD%B8%E5%AD%B8%E6%9C%83%E7%B2%B5%E8%AA%9E%E6%8B%BC%E9%9F%B3%E6%96%B9%E6%A1%88 "w:香港語言學學會粵語拼音方案")]  ―  他睡了吗?|
% |[冇](https://zh.wiktionary.org/wiki/%E5%86%87 "冇")([無](https://zh.wiktionary.org/wiki/%E7%84%A1 "無"))|mou5|dmb|kb|無|[有](https://zh.wiktionary.org/wiki/%E6%9C%89#%E6%BC%A2%E8%AA%9E "有")[冇](https://zh.wiktionary.org/wiki/%E5%86%87#%E6%BC%A2%E8%AA%9E "冇")[錢](https://zh.wiktionary.org/wiki/%E9%8C%A2#%E6%BC%A2%E8%AA%9E "錢")? / [有](https://zh.wiktionary.org/wiki/%E6%9C%89#%E6%BC%A2%E8%AA%9E "有")[冇](https://zh.wiktionary.org/wiki/%E5%86%87#%E6%BC%A2%E8%AA%9E "冇")[钱](https://zh.wiktionary.org/wiki/%E9%92%B1#%E6%BC%A2%E8%AA%9E "钱")? [[粵語](https://zh.wikipedia.org/wiki/%E7%B2%B5%E8%AA%9E "w:粵語")]  ―  _jau5 mou5 cin2?_ [[粵拼](https://zh.wikipedia.org/wiki/%E9%A6%99%E6%B8%AF%E8%AA%9E%E8%A8%80%E5%AD%B8%E5%AD%B8%E6%9C%83%E7%B2%B5%E8%AA%9E%E6%8B%BC%E9%9F%B3%E6%96%B9%E6%A1%88 "w:香港語言學學會粵語拼音方案")]  ―  有没有钱?|
% |[乸](https://zh.wiktionary.org/wiki/%E4%B9%B8 "乸")|naa2|bxgu|pdwyi|雌性|[雞乸](https://zh.wiktionary.org/wiki/%E9%9B%9E%E4%B9%B8#%E6%BC%A2%E8%AA%9E "雞乸") / [鸡乸](https://zh.wiktionary.org/wiki/%E9%9B%9E%E4%B9%B8 "雞乸") [[粵語](https://zh.wikipedia.org/wiki/%E7%B2%B5%E8%AA%9E "w:粵語")]  ―  _gai1 naa2_ [[粵拼](https://zh.wikipedia.org/wiki/%E9%A6%99%E6%B8%AF%E8%AA%9E%E8%A8%80%E5%AD%B8%E5%AD%B8%E6%9C%83%E7%B2%B5%E8%AA%9E%E6%8B%BC%E9%9F%B3%E6%96%B9%E6%A1%88 "w:香港語言學學會粵語拼音方案")]  ―  母鸡|
% |[諗](https://zh.wiktionary.org/wiki/%E8%AB%97 "諗")|nam2|ywyn|yroip|思考|[我](https://zh.wiktionary.org/wiki/%E6%88%91#%E6%BC%A2%E8%AA%9E "我")[諗](https://zh.wiktionary.org/wiki/%E8%AB%97#%E6%BC%A2%E8%AA%9E "諗")[噉](https://zh.wiktionary.org/wiki/%E5%99%89#%E6%BC%A2%E8%AA%9E "噉")[做](https://zh.wiktionary.org/wiki/%E5%81%9A#%E6%BC%A2%E8%AA%9E "做")[唔係](https://zh.wiktionary.org/wiki/%E5%94%94%E4%BF%82#%E6%BC%A2%E8%AA%9E "唔係")[咁](https://zh.wiktionary.org/wiki/%E5%92%81#%E6%BC%A2%E8%AA%9E "咁")[好](https://zh.wiktionary.org/wiki/%E5%A5%BD#%E6%BC%A2%E8%AA%9E "好")[啩](https://zh.wiktionary.org/wiki/%E5%95%A9#%E6%BC%A2%E8%AA%9E "啩")? [[粵語](https://zh.wikipedia.org/wiki/%E7%B2%B5%E8%AA%9E "w:粵語"),[繁體](https://zh.wiktionary.org/wiki/%E7%B9%81%E9%AB%94%E4%B8%AD%E6%96%87 "繁體中文")]  <br>[我](https://zh.wiktionary.org/wiki/%E6%88%91#%E6%BC%A2%E8%AA%9E "我")[谂](https://zh.wiktionary.org/wiki/%E8%B0%82#%E6%BC%A2%E8%AA%9E "谂")[噉](https://zh.wiktionary.org/wiki/%E5%99%89#%E6%BC%A2%E8%AA%9E "噉")[做](https://zh.wiktionary.org/wiki/%E5%81%9A#%E6%BC%A2%E8%AA%9E "做")[唔系](https://zh.wiktionary.org/wiki/%E5%94%94%E7%B3%BB#%E6%BC%A2%E8%AA%9E "唔系")[咁](https://zh.wiktionary.org/wiki/%E5%92%81#%E6%BC%A2%E8%AA%9E "咁")[好](https://zh.wiktionary.org/wiki/%E5%A5%BD#%E6%BC%A2%E8%AA%9E "好")[啩](https://zh.wiktionary.org/wiki/%E5%95%A9#%E6%BC%A2%E8%AA%9E "啩")? [[粵語](https://zh.wikipedia.org/wiki/%E7%B2%B5%E8%AA%9E "w:粵語"),[簡體](https://zh.wiktionary.org/wiki/%E7%B0%A1%E9%AB%94%E4%B8%AD%E6%96%87 "簡體中文")]<br><br>_ngo5 nam2 gam2 zou6 m4 hai6 gam3 hou2 gwaa3?_ [[粵拼](https://zh.wikipedia.org/wiki/%E9%A6%99%E6%B8%AF%E8%AA%9E%E8%A8%80%E5%AD%B8%E5%AD%B8%E6%9C%83%E7%B2%B5%E8%AA%9E%E6%8B%BC%E9%9F%B3%E6%96%B9%E6%A1%88 "w:香港語言學學會粵語拼音方案")]<br><br>我想这样做不是太好吧?|
% |[嬲](https://zh.wiktionary.org/wiki/%E5%AC%B2 "嬲")|nau1|llvl|wsvws|嬲怒|[咪](https://zh.wiktionary.org/wiki/%E5%92%AA#%E6%BC%A2%E8%AA%9E "咪")[激](https://zh.wiktionary.org/wiki/%E6%BF%80#%E6%BC%A2%E8%AA%9E "激")[嬲](https://zh.wiktionary.org/wiki/%E5%AC%B2#%E6%BC%A2%E8%AA%9E "嬲")[佢](https://zh.wiktionary.org/wiki/%E4%BD%A2#%E6%BC%A2%E8%AA%9E "佢")[啦](https://zh.wiktionary.org/wiki/%E5%95%A6#%E6%BC%A2%E8%AA%9E "啦") [[粵語](https://zh.wikipedia.org/wiki/%E7%B2%B5%E8%AA%9E "w:粵語")]  ―  _mai1 gik1 nau1 keoi5 laa1_ [[粵拼](https://zh.wikipedia.org/wiki/%E9%A6%99%E6%B8%AF%E8%AA%9E%E8%A8%80%E5%AD%B8%E5%AD%B8%E6%9C%83%E7%B2%B5%E8%AA%9E%E6%8B%BC%E9%9F%B3%E6%96%B9%E6%A1%88 "w:香港語言學學會粵語拼音方案")]  ―  不要惹他生气啦|
% |[啱](https://zh.wiktionary.org/wiki/%E5%95%B1 "啱")|ngaam1|kmdg|rumr|合适、恰巧|[啱](https://zh.wiktionary.org/wiki/%E5%95%B1#%E6%BC%A2%E8%AA%9E "啱")[心水](https://zh.wiktionary.org/wiki/%E5%BF%83%E6%B0%B4#%E6%BC%A2%E8%AA%9E "心水") [[粵語](https://zh.wikipedia.org/wiki/%E7%B2%B5%E8%AA%9E "w:粵語")]  ―  _ngaam1 sam1 seoi2_ [[粵拼](https://zh.wikipedia.org/wiki/%E9%A6%99%E6%B8%AF%E8%AA%9E%E8%A8%80%E5%AD%B8%E5%AD%B8%E6%9C%83%E7%B2%B5%E8%AA%9E%E6%8B%BC%E9%9F%B3%E6%96%B9%E6%A1%88 "w:香港語言學學會粵語拼音方案")]  ―  合心意|
% |[噏](https://zh.wiktionary.org/wiki/%E5%99%8F "噏")|ngap1|kwgn|rory|唸唸有詞噉講|噏三噏四|
% |[悒](https://zh.wiktionary.org/wiki/%E6%82%92 "悒")|ngap1|nkcn|prau|內心憂鬱|心悒|
% |[呢](https://zh.wiktionary.org/wiki/%E5%91%A2 "呢")|ni1|knx|rsp|此|[呢啲](https://zh.wiktionary.org/wiki/%E5%91%A2%E5%95%B2#%E6%BC%A2%E8%AA%9E "呢啲")[事](https://zh.wiktionary.org/wiki/%E4%BA%8B#%E6%BC%A2%E8%AA%9E "事") [[粵語](https://zh.wikipedia.org/wiki/%E7%B2%B5%E8%AA%9E "w:粵語")]  ―  _ne1 di1 si6_ [[粵拼](https://zh.wikipedia.org/wiki/%E9%A6%99%E6%B8%AF%E8%AA%9E%E8%A8%80%E5%AD%B8%E5%AD%B8%E6%9C%83%E7%B2%B5%E8%AA%9E%E6%8B%BC%E9%9F%B3%E6%96%B9%E6%A1%88 "w:香港語言學學會粵語拼音方案")]  ―  这些事|
% |[屙](https://zh.wiktionary.org/wiki/%E5%B1%99 "屙")|o1|nbsk|snlr|排泄|[屙尿](https://zh.wiktionary.org/wiki/%E5%B1%99%E5%B0%BF#%E6%BC%A2%E8%AA%9E "屙尿") [[粵語](https://zh.wikipedia.org/wiki/%E7%B2%B5%E8%AA%9E "w:粵語")]  ―  _o1 niu6_ [[粵拼](https://zh.wikipedia.org/wiki/%E9%A6%99%E6%B8%AF%E8%AA%9E%E8%A8%80%E5%AD%B8%E5%AD%B8%E6%9C%83%E7%B2%B5%E8%AA%9E%E6%8B%BC%E9%9F%B3%E6%96%B9%E6%A1%88 "w:香港語言學學會粵語拼音方案")]  ―  拉尿|
% |[撇](https://zh.wiktionary.org/wiki/%E6%92%87 "撇")|pit3|iumt|qfbk|迅速離開|[快](https://zh.wiktionary.org/wiki/%E5%BF%AB#%E6%BC%A2%E8%AA%9E "快")[啲](https://zh.wiktionary.org/wiki/%E5%95%B2#%E6%BC%A2%E8%AA%9E "啲")[撇](https://zh.wiktionary.org/wiki/%E6%92%87#%E6%BC%A2%E8%AA%9E "撇")[啦](https://zh.wiktionary.org/wiki/%E5%95%A6#%E6%BC%A2%E8%AA%9E "啦") [[粵語](https://zh.wikipedia.org/wiki/%E7%B2%B5%E8%AA%9E "w:粵語")]  ―  _faai3 di1 pit3 laa1_ [[粵拼](https://zh.wikipedia.org/wiki/%E9%A6%99%E6%B8%AF%E8%AA%9E%E8%A8%80%E5%AD%B8%E5%AD%B8%E6%9C%83%E7%B2%B5%E8%AA%9E%E6%8B%BC%E9%9F%B3%E6%96%B9%E6%A1%88 "w:香港語言學學會粵語拼音方案")]  ―|
% |[仆街](https://zh.wiktionary.org/wiki/%E4%BB%86%E8%A1%97 "仆街")|puk1 gaai1|why tffs|oy hoggn|不雅用語,表示不妙或咒罵|今次仆街嘞! ― 这下糟糕了!|
% |[卅](https://zh.wiktionary.org/wiki/%E5%8D%85 "卅")|saa1|gkk|tj|三十|卅而立 ― 三十而立|
% |[嘥](https://zh.wiktionary.org/wiki/%E5%98%A5 "嘥")|saai1|kthh|rhoo|浪費|[嘥](https://zh.wiktionary.org/wiki/%E5%98%A5#%E6%BC%A2%E8%AA%9E "嘥")[晒](https://zh.wiktionary.org/wiki/%E6%99%92#%E6%BC%A2%E8%AA%9E "晒")[喇](https://zh.wiktionary.org/wiki/%E5%96%87#%E6%BC%A2%E8%AA%9E "喇") [[粵語](https://zh.wikipedia.org/wiki/%E7%B2%B5%E8%AA%9E "w:粵語")]  ―  _saai1 saai3 laa3_ [[粵拼](https://zh.wikipedia.org/wiki/%E9%A6%99%E6%B8%AF%E8%AA%9E%E8%A8%80%E5%AD%B8%E5%AD%B8%E6%9C%83%E7%B2%B5%E8%AA%9E%E6%8B%BC%E9%9F%B3%E6%96%B9%E6%A1%88 "w:香港語言學學會粵語拼音方案")]  ―  全浪费掉了|
% |[晒](https://zh.wiktionary.org/wiki/%E6%99%92 "晒")|saai3|jsg|amcw|語氣助詞,表全部|[嘥](https://zh.wiktionary.org/wiki/%E5%98%A5#%E6%BC%A2%E8%AA%9E "嘥")[晒](https://zh.wiktionary.org/wiki/%E6%99%92#%E6%BC%A2%E8%AA%9E "晒")[喇](https://zh.wiktionary.org/wiki/%E5%96%87#%E6%BC%A2%E8%AA%9E "喇") [[粵語](https://zh.wikipedia.org/wiki/%E7%B2%B5%E8%AA%9E "w:粵語")]  ―  _saai1 saai3 laa3_ [[粵拼](https://zh.wikipedia.org/wiki/%E9%A6%99%E6%B8%AF%E8%AA%9E%E8%A8%80%E5%AD%B8%E5%AD%B8%E6%9C%83%E7%B2%B5%E8%AA%9E%E6%8B%BC%E9%9F%B3%E6%96%B9%E6%A1%88 "w:香港語言學學會粵語拼音方案")]  ―  全浪费掉了|
% |[曬](https://zh.wiktionary.org/wiki/%E6%9B%AC "曬")|saai3|jgmx|ammp|炫耀、展示|[曬](https://zh.wiktionary.org/wiki/%E6%9B%AC#%E6%BC%A2%E8%AA%9E "曬")[到](https://zh.wiktionary.org/wiki/%E5%88%B0#%E6%BC%A2%E8%AA%9E "到")[黑](https://zh.wiktionary.org/wiki/%E9%BB%91#%E6%BC%A2%E8%AA%9E "黑")[晒](https://zh.wiktionary.org/wiki/%E6%99%92#%E6%BC%A2%E8%AA%9E "晒") / [晒](https://zh.wiktionary.org/wiki/%E6%99%92#%E6%BC%A2%E8%AA%9E "晒")[到](https://zh.wiktionary.org/wiki/%E5%88%B0#%E6%BC%A2%E8%AA%9E "到")[黑](https://zh.wiktionary.org/wiki/%E9%BB%91#%E6%BC%A2%E8%AA%9E "黑")[晒](https://zh.wiktionary.org/wiki/%E6%99%92#%E6%BC%A2%E8%AA%9E "晒") [[粵語](https://zh.wikipedia.org/wiki/%E7%B2%B5%E8%AA%9E "w:粵語")]  ―  _saai3 dou3 haak1 saai3_ [[粵拼](https://zh.wikipedia.org/wiki/%E9%A6%99%E6%B8%AF%E8%AA%9E%E8%A8%80%E5%AD%B8%E5%AD%B8%E6%9C%83%E7%B2%B5%E8%AA%9E%E6%8B%BC%E9%9F%B3%E6%96%B9%E6%A1%88 "w:香港語言學學會粵語拼音方案")]  ―  全晒黑了|
% |[閂](https://zh.wiktionary.org/wiki/%E9%96%82 "閂")|saan1|ugd|anm|關閉|閂咗道門 ― 把门关上|
% |[卌](https://zh.wiktionary.org/wiki/%E5%8D%8C "卌")|se36|glk|tt|四十|卌幾個人 ― 四十几个人|
% |[睇](https://zh.wiktionary.org/wiki/%E7%9D%87 "睇")|tai2|huxt|bucnh|觀看、閱讀|[睇](https://zh.wiktionary.org/wiki/%E7%9D%87#%E6%BC%A2%E8%AA%9E "睇")[電視](https://zh.wiktionary.org/wiki/%E9%9B%BB%E8%A6%96#%E6%BC%A2%E8%AA%9E "電視") / [睇](https://zh.wiktionary.org/wiki/%E7%9D%87#%E6%BC%A2%E8%AA%9E "睇")[电视](https://zh.wiktionary.org/wiki/%E7%94%B5%E8%A7%86#%E6%BC%A2%E8%AA%9E "电视") [[粵語](https://zh.wikipedia.org/wiki/%E7%B2%B5%E8%AA%9E "w:粵語")]  ―  _tai2 din6 si6_ [[粵拼](https://zh.wikipedia.org/wiki/%E9%A6%99%E6%B8%AF%E8%AA%9E%E8%A8%80%E5%AD%B8%E5%AD%B8%E6%9C%83%E7%B2%B5%E8%AA%9E%E6%8B%BC%E9%9F%B3%E6%96%B9%E6%A1%88 "w:香港語言學學會粵語拼音方案")]  ―  看电视|
% |[𧨾](https://zh.wiktionary.org/wiki/%F0%A7%A8%BE "𧨾")|tam3|ypws|yrbcd|討好|[𧨾](https://zh.wiktionary.org/wiki/%F0%A7%A8%BE#%E6%BC%A2%E8%AA%9E "𧨾")[你](https://zh.wiktionary.org/wiki/%E4%BD%A0#%E6%BC%A2%E8%AA%9E "你")[開心](https://zh.wiktionary.org/wiki/%E5%BC%80%E5%BF%83 "开心") / [𬤂](https://zh.wiktionary.org/wiki/%F0%AC%A4%82#%E6%BC%A2%E8%AA%9E "𬤂")[你](https://zh.wiktionary.org/wiki/%E4%BD%A0#%E6%BC%A2%E8%AA%9E "你")[开心](https://zh.wiktionary.org/wiki/%E5%BC%80%E5%BF%83#%E6%BC%A2%E8%AA%9E "开心") [[粵語](https://zh.wikipedia.org/wiki/%E7%B2%B5%E8%AA%9E "w:粵語")]  ―  _tam3 nei5 hoi1 sam1_ [[粵拼](https://zh.wikipedia.org/wiki/%E9%A6%99%E6%B8%AF%E8%AA%9E%E8%A8%80%E5%AD%B8%E5%AD%B8%E6%9C%83%E7%B2%B5%E8%AA%9E%E6%8B%BC%E9%9F%B3%E6%96%B9%E6%A1%88 "w:香港語言學學會粵語拼音方案")]  ―  哄你开心|
% |[揾](https://zh.wiktionary.org/wiki/%E6%8F%BE "揾")|wan2|rjlg|qwot|尋找|[揾食](https://zh.wiktionary.org/wiki/%E6%90%B5%E9%A3%9F "搵食") [[粵語](https://zh.wikipedia.org/wiki/%E7%B2%B5%E8%AA%9E "w:粵語")]  ―  _wan2 sik6_ [[粵拼](https://zh.wikipedia.org/wiki/%E9%A6%99%E6%B8%AF%E8%AA%9E%E8%A8%80%E5%AD%B8%E5%AD%B8%E6%9C%83%E7%B2%B5%E8%AA%9E%E6%8B%BC%E9%9F%B3%E6%96%B9%E6%A1%88 "w:香港語言學學會粵語拼音方案")]  ―  谋生|
% |[喎](https://zh.wiktionary.org/wiki/%E5%96%8E "喎")|wo3|kkmw|rbbr|語氣助詞|[唔係](https://zh.wiktionary.org/wiki/%E5%94%94%E4%BF%82#%E6%BC%A2%E8%AA%9E "唔係")[噉](https://zh.wiktionary.org/wiki/%E5%99%89#%E6%BC%A2%E8%AA%9E "噉")[喎](https://zh.wiktionary.org/wiki/%E5%96%8E#%E6%BC%A2%E8%AA%9E "喎") / [唔系](https://zh.wiktionary.org/wiki/%E5%94%94%E7%B3%BB#%E6%BC%A2%E8%AA%9E "唔系")[噉](https://zh.wiktionary.org/wiki/%E5%99%89#%E6%BC%A2%E8%AA%9E "噉")[㖞](https://zh.wiktionary.org/wiki/%E3%96%9E#%E6%BC%A2%E8%AA%9E "㖞") [[粵語](https://zh.wikipedia.org/wiki/%E7%B2%B5%E8%AA%9E "w:粵語")]  ―  _m4 hai6 gam2 wo3_ [[粵拼](https://zh.wikipedia.org/wiki/%E9%A6%99%E6%B8%AF%E8%AA%9E%E8%A8%80%E5%AD%B8%E5%AD%B8%E6%9C%83%E7%B2%B5%E8%AA%9E%E6%8B%BC%E9%9F%B3%E6%96%B9%E6%A1%88 "w:香港語言學學會粵語拼音方案")]  ―|
% |[揸](https://zh.wiktionary.org/wiki/%E6%8F%B8 "揸")|zaa1|rsjg|qdbm|手持|揸住個門柄 ― 握住门的把手|
% |[咋](https://zh.wiktionary.org/wiki/%E5%92%8B "咋")|zaa3|kthf|ros|語氣助詞,表示有限|係得咁多咋 ― 就剩下这么少了|
% |[嗻](https://zh.wiktionary.org/wiki/%E5%97%BB "嗻")|ze1|kyao|ritf|語氣助詞,表示不至於|[我](https://zh.wiktionary.org/wiki/%E6%88%91#%E6%BC%A2%E8%AA%9E "我")[先至](https://zh.wiktionary.org/wiki/%E5%85%88%E8%87%B3#%E6%BC%A2%E8%AA%9E "先至")[廿三](https://zh.wiktionary.org/wiki/%E5%BB%BF%E4%B8%89#%E6%BC%A2%E8%AA%9E "廿三")[嗻](https://zh.wiktionary.org/wiki/%E5%97%BB#%E6%BC%A2%E8%AA%9E "嗻") [[粵語](https://zh.wikipedia.org/wiki/%E7%B2%B5%E8%AA%9E "w:粵語"),[繁體](https://zh.wiktionary.org/wiki/%E7%B9%81%E9%AB%94%E4%B8%AD%E6%96%87 "繁體中文")和[簡體](https://zh.wiktionary.org/wiki/%E7%B0%A1%E9%AB%94%E4%B8%AD%E6%96%87 "簡體中文")]<br><br>_ngo5 sin1 zi3 jaa6 saam1 ze1_ [[粵拼](https://zh.wikipedia.org/wiki/%E9%A6%99%E6%B8%AF%E8%AA%9E%E8%A8%80%E5%AD%B8%E5%AD%B8%E6%9C%83%E7%B2%B5%E8%AA%9E%E6%8B%BC%E9%9F%B3%E6%96%B9%E6%A1%88 "w:香港語言學學會粵語拼音方案")]<br><br>我才二十三啊|
% |[樽](https://zh.wiktionary.org/wiki/%E6%A8%BD "樽")|zeon1|susf|dtwi|液體容器|玻璃樽|
% |[即](https://zh.wiktionary.org/wiki/%E5%8D%B3 "即")|zik1|vcbh|aisl|就係、當刻|即係要佢即刻出發 ― 就是要他马上出发|
% |[咗](https://zh.wiktionary.org/wiki/%E5%92%97 "咗")|zo2|kdag|rkm|動作完成結束|[食](https://zh.wiktionary.org/wiki/%E9%A3%9F#%E6%BC%A2%E8%AA%9E "食")[咗](https://zh.wiktionary.org/wiki/%E5%92%97#%E6%BC%A2%E8%AA%9E "咗")[飯](https://zh.wiktionary.org/wiki/%E9%A3%AF#%E6%BC%A2%E8%AA%9E "飯")[未](https://zh.wiktionary.org/wiki/%E6%9C%AA#%E6%BC%A2%E8%AA%9E "未")? / [食](https://zh.wiktionary.org/wiki/%E9%A3%9F#%E6%BC%A2%E8%AA%9E "食")[咗](https://zh.wiktionary.org/wiki/%E5%92%97#%E6%BC%A2%E8%AA%9E "咗")[饭](https://zh.wiktionary.org/wiki/%E9%A5%AD#%E6%BC%A2%E8%AA%9E "饭")[未](https://zh.wiktionary.org/wiki/%E6%9C%AA#%E6%BC%A2%E8%AA%9E "未")? [[粵語](https://zh.wikipedia.org/wiki/%E7%B2%B5%E8%AA%9E "w:粵語")]  ―  _sik6 zo2 faan6 mei6?_ [[粵拼](https://zh.wikipedia.org/wiki/%E9%A6%99%E6%B8%AF%E8%AA%9E%E8%A8%80%E5%AD%B8%E5%AD%B8%E6%9C%83%E7%B2%B5%E8%AA%9E%E6%8B%BC%E9%9F%B3%E6%96%B9%E6%A1%88 "w:香港語言學學會粵語拼音方案")]  ―  吃了饭没有?|
% |[仲](https://zh.wiktionary.org/wiki/%E4%BB%B2 "仲")|zung6|tgjf|hjwg|尚、仍然|仲未搞掂 ― 还没有完成|
% |[啜](https://zh.wiktionary.org/wiki/%E5%95%9C "啜")|zyut3|kccc|reee|吻、吸吮|用飲管吸 ― 用吸管吸|

% \newpage

% \section{影隻形單}

% (Empty file)

% \newpage

% \section{從文字認識香港:地名與地形的關係}

% #百越底層詞 #geography
% ## 坑

% **解釋:** 坑指的是一個小而深的窪地或溝渠,通常由自然力量如雨水沖刷而成。例如:蕉坑。

  

% ## 井

% **解釋:** 井通常指的是挖掘在地面上的深洞,用於取水或其他用途。例如:井欄樹。

  

% ## 埗

% **解釋:** 埗指的是一個小型的港口或碼頭,用於船隻停泊和貨物裝卸。例如:深水埗。  

% ## 灣

% **解釋:** 灣指的是一個彎曲的海岸線,通常形成一個天然的避風港。例如:銅鑼灣。  

% ## 口

% **解釋:** 口指的是進入某地的入口或出口,通常是河流或海灣的入海口。例如:大涌口。  

% ## 河

% **解釋:** 河指的是流動的水體,通常是河流或溪流。例如:沙田河。  

% ## 潭

% **解釋:** 潭指的是一個深水池,通常位於河流或瀑布的底部。例如:新娘潭。  

% ## 笏

% **解釋:** 笏指的是一種平坦的石板或石塊,通常用於地名中表示該地區的石質地形。例如:笏石。  

% ## 濠

% **解釋:** 濠指的是一個人工或自然形成的護城河或水道。例如:濠江。  

% ## 港

% **解釋:** 港指的是一個天然或人工的港口,用於船隻停泊和貨物裝卸。例如:香港。  

% ## 灘

% **解釋:** 灘指的是一個平坦的沙灘或泥灘,通常位於海岸線附近。例如:紅磡灘。  

% ## 洲

% **解釋:** 洲指的是一片由沉積物形成的陸地,通常位於河流或海洋中。例如:長洲。  

% ## 谷

% **解釋:** 谷指的是兩山之間的低地或河谷,通常形成一個狹長的地形。例如:大埔谷。  

% ## 地

% **解釋:** 地指的是廣泛的土地或地區,通常用於表示某地的地形特徵。例如:濕地。  

% ## 坪

% **解釋:** 坪指的是一片平坦的土地,通常用於耕地或建築用地。例如:秀茂坪。  

% ## 屻

% **解釋:** 屻指的是一個小山丘或高地,通常用於描述該地區的地形特徵。例如:大刀屻。  

% ## 門

% **解釋:** 門指的是一個進入某地的入口,通常用於表示該地區的地理位置。例如:鯉魚門。  

% ## 窩

% **解釋:** 窩指的是一個小而隱蔽的地方,通常位於山谷或山坡上。例如:水浪窩。

  

% ## 頭

% **解釋:** 頭指的是一個地區的最高點或最突出的部分。例如:青龍頭。

  

% ## 埔

% **解釋:** 埔指的是一片平坦的土地,通常用於耕地或建築用地。例如:大埔。

  

% ## 尖

% **解釋:** 尖指的是一個尖銳的地形特徵,通常是山峰或突出部分。例如:蚺蛇尖。

  

% ## 滘

% **解釋:** 滘指的是一個狹窄的水道或河流,通常位於兩岸之間。例如:大埔滘。

  

% ## 㘭

% **解釋:** 㘭指的是一個山坡或斜坡,通常用於描述該地區的地形特徵。例如:大風㘭。

  

% ## 寮

% **解釋:** 寮指的是一個簡陋的房屋或棚屋,通常用於描述農村或偏遠地區。例如:雞寮。

  

% ## 壩

% **解釋:** 壩指的是用於攔截水流的堤壩或水壩,通常用於灌溉或防洪。例如:船灣淡水湖壩。

  

% ## 圍

% **解釋:** 圍指的是一個用圍牆或柵欄包圍的地區,通常是村莊或農田。例如:屏山圍。

  

% ## 岩

% **解釋:** 岩指的是堅硬的岩石或山岩,通常用於描述地區的岩石地形。例如:老虎岩。


% 涌
% - 鰂魚涌
% - 東涌

% 澳
% - 將軍澳 
% - 大澳
% - 澳門
% - 

% 這些地名字不僅是地形特徵的描述,也反映了香港豐富的自然景觀和多樣的文化歷史。理解這些地名和地形的關係,有助於更深入地認識香港的地理環境和歷史文化。

% \newpage

% \section{心辭}

% 心辭

% \newpage

% \section{思思縮縮}

% (Empty file)

% \newpage

% \section{想}

% (Empty file)

% \newpage

% \section{想像}

% **3.11** 想像

% 今「[想]()像」只作動詞,但在《俗話傾談》可作名詞,即「計劃」、「打算」之意:

% (39)  你估我用個的錢文真正冇想象麼?狗醜主人羞,唔打办吓光輝,人話齊思賢老婆,衣衫襤褸,失禮到你呀。所以遇時拜神拜佛,無非見自己命鄙,歸到你門兩年,未有所出,都係想菩薩庇佑,早日生個花仔,得到三十七八時,娶個新婦。(學翻你咁好)你做家公,我做家婆,有仔有孫,慢慢享福(不可先折福),人家同話你好命咯。唔通等到五六十歲生仔,扒向棺材頭麼?你做男人曉得發財,唔慌有个的想像吓咯。(253-254)

% (40)  有子有孫,亦人生之想像也。(176)

% \newpage

% \section{想必}

% []() 想必 = [梗係](梗係)

% \newpage

% \section{應該}

% 應該
% [應份](應份)
% [應當](應當)

% \newpage

% \section{成日}

% 
% 成日
% 成世
% 成間公司

% [遇時](遇時)

% \newpage

% \section{拏耕}

% #metaphysics #百越底層詞
% [[拏褦]]

% 一個話題同另外一個話題係相關同埋有邏輯關連,或者有話題共鳴性嘅,就叫做拏耕。
% 一樣[[野]]同另外一樣野有關係,就係有拏褦

% \newpage

% \section{拏褦}

% #百越底層詞 #metaphysics 
% [[拏耕]]

% \newpage

% \section{挨依}

% [support](support)

% \newpage

% \section{捩橫折曲}

% [[捩:戾]]橫折曲

% 粵語

  
% - 香港音:

% - 粵拼:**lai2** **waang4** **zit3** **kuk1**

% 打開

% - 廣東話 (粵拼):lai2 waang4 zit3 kuk1

% **解乜**

% 1. 【成語】歪曲事實、顛倒是非

% \newpage

% \section{捩:戾}

% 戾

  



% 粵語

% **發音**

% 粵拼:**lai⁶**

% **寫法**

% - 簡體/繁體:**戾**

% **含義**

% - (本義)彎曲
% - (引申)罪過、到、至

% 出處

% - 《說文解字》:曲也。從犬出戶下。戾者,身曲戾也。
% - 《孟子·滕文公上》:樂歲粒米戾也。
% - 趙岐註說:狼戾,猶狼藉也

% \newpage

% \section{搣}

% #百越底層詞

% \newpage

% \section{料必}

% 料必
% 勢必
% [是必](是必)
% [想必](想必)

% \newpage

% \section{斟}

% - 你睇睇有冇得點斟?呢個問題都幾難斟。
% - 你 睇々 有冇  斟?󰳞󱟡 問題 󰧶 幾難 斟。

% - 斟酌
% - 斟法

% \newpage

% \section{新香港九約竹枝詞九首}

% # 新香港九約竹枝詞九首



% **劉祖榮**

% **前言:**《香港九約竹枝詞》又稱《圍名歌》,或《新界竹枝詞》。據文獻記載,由清代晚期兩名秀才(私塾老師),居於沙田石古壟的許永慶和火炭九肚村的羅文祥共同創作。書中收集一百五十五首四行七言絕句,描述香港當時景物、名勝、風俗及民情,並且嵌入了新界各村圍的地名,猶如一個世紀前的香港地理誌。二○一四年,該書列入「首份香港非物質文化遺產清單」。

% 鄙人亦好遊山玩水,登高遠足。多年來,造訪了香港許多名勝古跡。有意仿前人用竹枝詞,把瀏覽這些地方的所見所聞所想抒寫出來,傳承下去。以饗讀者。  
% [![](https://cdn.hkwriters.ph4day.com/wp-content/uploads/2024/06/2024062517353860.jpg)](https://cdn.hkwriters.ph4day.com/wp-content/uploads/2024/06/2024062517353860.jpg)

% **一,大帽山**

% 百山環繞勢軒昂  
% 大霧籠頭似帽妝  
% 疊翠雲間通訊塔  
% 仿如佇立一金剛  
% [![](https://cdn.hkwriters.ph4day.com/wp-content/uploads/2024/06/2024062517354976.jpg)](https://cdn.hkwriters.ph4day.com/wp-content/uploads/2024/06/2024062517354976.jpg)**二,重遊鹽田梓**

% 簇翠山頭天主堂  
% 朝輝共與普祥光  
% 紅林掩映滄桑史  
% 玉帶橫穿碧海央

% **三,米埔**

% 米埔河灘匯海潮  
% 魚蝦跳躍嬉波濤  
% 時聞百鳥盤空至  
% 逐水爭撈競翅翺

% **四,荃灣西方寺**

% 九層高塔倚雄峰  
% 八角銜雲紫氣籠  
% 聳峙西天迎萬佛  
% 釋迦彌勒共融融

% **五,汀九泳灘**

% 三塔橋橫汲水門  
% 巍然屹立聳瑤軒  
% 長灘泳會如鱗次  
% 碧海灣前舴艋喧

% **六,昂坪棧道**

% 彌勒山中木棧旋  
% 凌空壑谷響溪泉  
% 時聞笑語纜車過  
% 俯拾祥雲上寶蓮

% [![](https://cdn.hkwriters.ph4day.com/wp-content/uploads/2024/06/2024062517353942.jpg)](https://cdn.hkwriters.ph4day.com/wp-content/uploads/2024/06/2024062517353942.jpg)  
% **七,城門水塘**

% 一湖碧水盈春色  
% 八面青山拱日輝  
% 守得心門風月韻  
% 城池內外盡芳菲  
% [![](https://cdn.hkwriters.ph4day.com/wp-content/uploads/2024/06/2024062517354427.jpg)](https://cdn.hkwriters.ph4day.com/wp-content/uploads/2024/06/2024062517354427.jpg)

% **八,山頂盧吉道**

% 金鐘踏步覽秋歡  
% 環繞爐峰賞壯觀  
% 廣廈高樓逶兩岸  
% 白雲滄海翠屏巒

% **九,清水灣泳灘**

% 波光翡翠逐斑斕  
% 足底銀沙趾甲彈  
% 左右群峰迎綠島  
% 輕遊潛泳戲魚歡

% (本文圖片由作者提供)

% **劉祖榮簡介:**香港人,祖籍福建南安市。作品曾獲得二○二一年首屆全球「藝術與和平」詩詞大賽的三等獎,二○二○年第二屆義烏駱賓王國際兒童詩歌大賽的提名獎,二○二一年第三屆中國徐霞客散文獎佳作獎,二○一八年全球華文「曼麗雙輝」填詞大獎賽優異獎,二○二一年首屆中國「華潯杯.我愛我家」詩歌優秀獎,二○二一年第九屆「禾澤都林杯」城市、建築與文化散文大賽優秀獎,二○二二年第三屆猴王杯詩歌大賽佳作獎等。

% \newpage

% \section{新香港九約竹枝詞十首}

% # 新香港九約竹枝詞十首

% 2023-06-24 预计阅读需要2分钟

% **劉祖榮**

% **宋皇台地鐵站**  
% 世事浮沉豈可猜,行朝輾轉盡悲哀。  
% 聖山石刻猶相貶,地鐵正名復帝台。

% [![](https://cdn.hkwriters.ph4day.com/wp-content/uploads/2023/06/202306242208346.png)](https://cdn.hkwriters.ph4day.com/wp-content/uploads/2023/06/202306242208346.png)  
% **楓香林**  
% 楓葉一紅動港城,大棠漫道沸騰聲。  
% 眾人只顧枝頭彩,誰理沙沙腳下鳴。

% **石龍拱觀景台**  
% 石徑盤旋上峭峰,俯看青馬跨雲蹤。  
% 藍灣綺峽通維港,屯海高空鐵鳥縱。

% **金山三疊泉  
% **山行好坐水流旁,漱石清音解乏傷。  
% 閉目冥思塵世贅,靈光共與浪花揚。

% **八仙嶺流水響**  
% 去年亦過羽林中,只見枯枝戳碧空。  
% 流水苟殘潭水淺,龍山橋下映孤瞳。

% **美孚老榕壇**  
% 一樹成壇群卉鑲,青蔥尤勝粉紅揚。  
% 三千花色爭蜂蝶,眾鳥枝頭鬥唱昂。  
% [![](https://cdn.hkwriters.ph4day.com/wp-content/uploads/2023/06/2023062422102235.jpg)](https://cdn.hkwriters.ph4day.com/wp-content/uploads/2023/06/2023062422102235.jpg)  
% **南蓮園池**  
% 奇松異石簇崟崟,古色唐風漫步尋。  
% 舉目山門如咫尺,一街車馬隔千潯。

% **土瓜灣  
% **瓜中核落土中央,蘊藉繁華寓意長。  
% 簡寫灣情千載變,海心魚石憶滄桑。

% **屯門菠蘿山峽谷**  
% 溝壑嶙峋壁角奇,岩金土褐炫虹霓。  
% 玲瓏峽谷盆中景,切片菠蘿且醉迷。

% **青山杯渡岩**  
% 是非是是皆非是,無有無無自有無。  
% 不二法門何為法,半山碧樹半雲途。

% (本文圖片為資料圖片)

% **劉祖榮簡介:**香港人,祖籍福建南安市。作品曾獲得二〇二一年首屆全球「藝術與和平」詩詞大賽的三等獎,二〇二〇年第二屆義烏駱賓王國際兒童詩歌大賽的提名獎,二〇二一年第三屆中國徐霞客散文獎佳作獎,二〇一八年全球華文「曼麗雙輝」填詞大獎賽優異獎,二〇二一年首屆中國「華潯杯.我愛我家」詩歌優秀獎,二〇二一年第九屆「禾澤都林杯」城市、建築與文化散文大賽優秀獎等。

% \newpage

% \section{斷}

% #logic 
% 我斷唔孤負你個一點情長。
% assert 
% [估](估)
% 斷估

% \newpage

% \section{早期粵語聖經}

% ## 一. 語料簡介

% 早期粵語聖經主要分為漢字版、羅馬字版,中西字(粵語漢字加英文)版,以及點字版。本項目語料庫選取的文本是六個版本的粵語《路加福音》,分別為1867、1872、1883、1906、1924,及1927版。

% 1867版是最早的羅馬字版,採用標準羅馬字方案,由禮賢會傳教士W. Louis(呂威廉)翻譯、E. Faber(花之安)音譯。為方便讀者閱讀,本語料庫已將羅馬字轉寫為漢字)。1872-1927版均為粵語漢字聯合版。1872年版由美北長老會牧師C. F. Preston(丕思業)、英國循道會牧師G. Piercy(俾士),和德國禮賢會牧師公孫惠(A. Krolczyk)三人共同翻譯,一般稱為Union Version(聯合版),其後,其他傳教士參與修訂工作,1883-1924版本均以1872版為基礎。1919年《官話和合本》出版,1927年粵語聖經受《官話和合本》的影響較大,用語修改較多。

% 除此之外,本語料庫亦採用了1863版節選自傳教士C. F. Preston(丕思業)翻譯的《耶穌言行撮要》。《耶穌言行撮要》共100章,由四福音書及《使徒行傳》的精選章節組成,當中《路加福音》佔30章,本語料庫使用了該書中《路加福音》的全部章節。

% ## 二. 本語料庫使用方法小提示:

% 方法一:您可選擇一種或多種早期粵語《路加福音》版本閱讀。

% 方法二:您可選擇語料庫中《路加福音》的個別章節來比較。

% 方法三:您可使用本語料庫的檢索功能來檢查某些詞在當時的用法。

% 十九世紀的粵語與現時的廣東話存在差異。相較十九世紀的粵語,150年後的現代粵語已變化頗多,用語有所不同。大家如有興趣,可透過本語料庫了解當時粵語的面貌。您可使用本語料庫的檢索功能,查閱十九世紀中至二十世紀初的粵語聖經翻譯。

% 例如,您可查閱到不期同時的表達相同意義的詞語用法,例子如下表:

% |十九世紀中|二十一世紀|
% |---|---|
% |乜誰|邊個|
% |嘵|咗|

% ## 三. 備註

% 為方便讀者查閱,本語料庫特設如下備註。

% **註1:**以下為本語料庫原文中不常用字詞或異體字,與當前常用字詞或字義對照表:

% **字體對照表**

% 以下為本語料庫原文中不常用字詞或異體字,與當前常用字詞或字義對照表:

% |  |  |  |  |
% | ---- | ---- | ---- | ---- |
% | 聲母 | 本語料庫中某些年份聖經原文用字 | 本語料庫中用字 | 現時常用字或常用字義 |
% | ㅇ | 𠮩 | 吖 | 吖 aa1(語氣助詞。) |
% |  | 唉 | 唉 aai4(語氣助詞。) |  |
% | b | 偪 | 偪 | 逼 bik1 |
% |  | 𡃓 | 噃 | 噃 bo3(語氣助詞。) |
% |  | 𡂈 | 蹼 | 蹼 buk6(意為「伏下」。) |
% |  | [口焙] | [口焙] | (暗)[口焙] bui6(意為「昏暗」。) |
% | d | 荅、㗳、撘 | 答、搭 | 答 daap3 (意為「回答」。);  <br>搭 daap3 (意為「粘上連接」。) |
% |  | [口笪] | 笪 | 笪 daat6 |
% |  | [口躭](高) | 擔(高) | 擔(高)daam1 |
% |  | 矴 | 掟 | 掟 deng3 (意為 「扔。」) |
% |  | (個 / 呢)的 | (個 / 呢)的 | 啲 di1 |
% |  | 叚 | 段 | 段 dyun6 |
% | f | 畨 | 番 | 番 faan1 |
% | 𠵽唎㘔人 | 𠵽唎㘔人 | 法利賽人 faat3 lei6 coi3 jan4 |  |
% | g | 𡃉 | 𡃉 | 㗎 gaa3 |
% | (豆)[口莢] | (豆)莢 | (豆)莢 gaap3 |  |
% | (失)𡄎 | (失)𡄎 | (失)𡄎 gam2(動詞後綴。) |  |
% | 㓤 | 㓤 | 拮;㓤 gat1 |  |
% | 呌 | 叫 | 叫 giu3 |  |
% | h | [口广叚] | [口广叚] | 哈haa4(感歎詞。) |
% | hēʹ | (唎)嘻 | (唎le3)嘻he2 (語氣助詞。) |  |
% | j | (光)燄 | (光)燄 | (光)焰 jim6 |
% | (應)騐 | (應)驗 | (應)驗 jim6 |  |
% | [執/火] | 熱 | 熱 jit6 |  |
% | 喓 | 喓 | 喓 jiu1(意為「呼叫」。) |  |
% | k | 𢬿 | 𢬿 | 𢬿 kaai5,(意為:將……(動詞);拿取。) |
% | 𠹸 | 搇 | 搇(意為:蓋上。) |  |
% | (屋)[ ⻊企] | (屋)企 | (屋)企 kei2 |  |
% | l | [口扌力] | [口手力] | [口扌力] lak3(語氣助詞。) |
% | 㨆、[口冧] | 冧 | 冧 lam3 (意為「倒塌」。) |  |
% | 甪 | 甩 | 甩 lat1 |  |
% | 唎 | 唎 | 咧 le3 |  |
% | m | (細)伩(仔) | (細)蚊(仔) | (細)蚊(仔) man1 |
% | 擝 | 擝 | 掹 mang1 |  |
% | 𧴯 | 孭 | 孭 me1 |  |
% | (約)嗼 | (約)嗼 | (約)莫 mok2 |  |
% | n |  | [口月貳] | 呢 ne1 / ni1 (語氣助詞。) |
% |  | 𢆡 | 𢆡 nin1(意為「乳房」。) |  |
% | [口佞] (轉) | [口佞](轉) | 擰(轉) ning6 |  |
% | ng | 𠼮 | 𠼮 | 𠼮;𠱓 ngai1,(意為「懇求」。) |
% | p | (白)[口抱] | (白)泡 | (白)泡 pou5 |
% | s | [口撒]吐𡀲人 | 撒吐該人 | 撒吐該人 saat3 tou3 goi1 jan4  <br>(和合本:撒都該人。) |
% | [竹/目/大] | 算 | 算 syun3 |  |
% | w | 𢫕 | 𢫕 | 𢫕 wing1 (同「扔」。) |
% | 噁 | 噁 | 嗚 wu1(意為「伏下」。 ) |  |
% | 囘 | 回 | 回 wui4 |  |
% | 噲 | 噲 | 會 wui5 |  |
% | z | 𠴥 | 浸 | 浸 zam6 |
% | 贃 | 贃 | 賺 zan6 |  |
% | 呪 | 咒 | 咒 zau3 |  |
% | 呮 | 呮 | 喞 zek1 (語氣助詞。) |  |
% | 翦 | 剪 | 剪 zin2 |  |
% | (污)[口糟] | (污)糟 | (污)糟 zou1 |  |

% **註2:**1867版本原文為羅馬字,本語料庫1867版所用漢字系依據羅馬字音轉譯而成。

% 例:

% |1867 羅馬字原文|本語料庫對應用字|
% |---|---|
% |![](data:image/png;base64,iVBORw0KGgoAAAANSUhEUgAAAEIAAAAjCAYAAAAg/NwXAAAAAXNSR0IArs4c6QAAAHhlWElmTU0AKgAAAAgABAEaAAUAAAABAAAAPgEbAAUAAAABAAAARgEoAAMAAAABAAIAAIdpAAQAAAABAAAATgAAAAAAAACWAAAAAQAAAJYAAAABAAOgAQADAAAAAQABAACgAgAEAAAAAQAAAEKgAwAEAAAAAQAAACMAAAAA/hzwNAAAAAlwSFlzAAAXEgAAFxIBZ5/SUgAABoJJREFUaAXtWVlIVV0UXqYNNidlVhI0gM3ZZEUl1EOTEDQ8CA0vJuhDhJHgQyMVRSFRRHP5kGkJNlFh9WBKElgJQZGZ2WhFs5Vk4/rXt/7/HM659557z7WyfrsLrmfvtdfaZ+2113gMYwEKAbUI6eBfDYQU8Z8lhBQRUoQ9KPywRXz69Inevn1r3/V/OItwI/OHDx8oKyuLRo4cSZMnT6Zv374p27t372j9+vU0f/58mjt3rrkVcP3796cxY8aQNSl9//6d2rVrRz169KAWLX74Dsz3/ZQB0mcgkINzWVkZL168mNu0acORkZH6FMXwtWvX+MuXL7YtDh8+zL169VI60Iqg+hs2bBjv37+fv379aqP/Eya4saDAOBSeBQUFjrzPnz9nsQhu2bIlr127lnfs2MFPnjxhsQpHnt+5ELQiTpw4Yd7wsmXLWGKET/nPnz/P4gJ8/Phx23pDQwO/fPmS8TQs48WLFzq3ETbxJGhFVFRUcKdOnVQZYWFhnJ+f7yVyXl4ez5kzh0+dOuW1VlJSopaSkpKiFgKCSZMmcXFxsRdtUyKCVgSEO3jwoJq8BDzevXu3Td4NGzYw8AcOHLDhjQnizcePH9UCDDfBHPjfCY1SxKFDh7hVq1ZqFTNnzuT6+no9Q21tLXfu3JknTJjAr169cjwXDg2eq1ev8v379x3pmnKhUYqAj3fs2FEV0bt3b3706JGaOdzhyJEj/Pr1a8cz3Lt3jxctWsTz5s1TfriZU5xx3OQXLDRKEZBj27ZtehBkj+HDh/PYsWO5sLDQr4i4/fbt23OHDh347t27mpKlBuG0tDS/fE2x2GhF4NBQAgImnm5g1apVSnvr1i2TPDs7m4cMGWLO3Q7OnTvHsbGx6op4vxR0jqywUlgr3Ba/mzdvetG6O4EXG3N8fLxpEf6EsLL27duXJ06cyFKSKxrPxMRE3rdvn5Us4BiZB65VWVnJDx484M2bN2vM8mSsqqri9PR0XrBgAWP87NkzzszM5NmzZ2vAttI3ShHw865du5qKQLXoD5AdUlNTlb5fv3788OFDvnDhAku5rbinT5/6Y/dagzVeunRJ8Z8/f2a4FzKVFRCQt2zZolZjxUu7oG5848YNK5pd9RpieiagX8jJySEJiIpD79CzZ09z3ddAMgSJ8CTKo7i4OFq3bh0VFRXRqFGjKCkpiWJiYpRNFELl5eUUHh5OUmzRrFmzCO+TEp6kVDe3liqVpFwnyU4klSuNGDGC0PdYQapeWr16NR09etSKNseSss2xDmxqcTGRF/K4ceNMa1i5cmVALvQjSLdGgQULAc4TpFkz9xXhtIc5efIkL1y40Ea6Z88ejoiIYDyNPa0Ejx8/5m7dumkQ9uyDkLalcWSkfSsE7RroIeQmVeDo6GhGeRwI0GfAdG/fvm0jhUKsgHWkU+x/8eJFXULNcuXKFSuZuhYyFZTVunVrllu3re/du1fXjh07ZsNj8ubNGx4/fjwPGjTItha0IjZt2mTeGgonN/D+/XtesmSJjbSmpoaXL19uBk7bosMEFahRvCH9Tpkyhdu2bcviGrZ9kJ3Q9RpxxLrd5cuXVf7t27db0Ry0IlAD4Cbw27Vrl20zp0ldXR13796dp02bxmjarl+/rm06AiiCnRuAieOAEmNMcvCifklISGC4LACKGjp0KHfp0sWn+0lMUtnv3Llj7oOBK0VACAlupksYisBt4FahZX+ALhPVJOJEcnIyr1mzhrdu3Wrerj9eYw3VK947Y8YMA6VPpGPELEMRQKLP6dOnj40OE1gILhLp1rO3cZU1zp49q1Ee0Vm0D+WJTCKVPCVmUG5uLokwivP8g6iPT3nSjJG07SRxRb9SgdfYx5PH1xyZBD9kHmQIZAtxL0KmWbFiBckBTbbBgwfruLq6Wr+UYVJaWkpSVFFGRgZJLWHSmgMR5pcCmi8Ev6ioKK098OUK/QmCIuKNW0A/gkJIBOeBAwdq1BfFaEbx5V4bN27k6dOnq/XhPWj1EbSdwJVrODG7wSMzoAlDo4YvVEuXLuWpU6cyChsj8LnZBzRwUaTr0aNHa2xAZWl83PG1h+GSCKqeMcGTPgwI0zz+4sEf9in5590E7nfnzp10+vRpd5t6mkhzmaO1Fw1otxnoTHDfZmsR0uprlhkwYEBAizhz5kzz/W84/skkbb9j02XVjpTz9NcHS/lOof+Ra7auYb1xf2P5YkVS9YYsAtkF1e8/pp4Ptu7TDoAAAAAASUVORK5CYII=)|耶穌|
% |![](data:image/png;base64,iVBORw0KGgoAAAANSUhEUgAAAFkAAAAfCAYAAACMCmHeAAAAAXNSR0IArs4c6QAAAHhlWElmTU0AKgAAAAgABAEaAAUAAAABAAAAPgEbAAUAAAABAAAARgEoAAMAAAABAAIAAIdpAAQAAAABAAAATgAAAAAAAACWAAAAAQAAAJYAAAABAAOgAQADAAAAAQABAACgAgAEAAAAAQAAAFmgAwAEAAAAAQAAAB8AAAAAOBYo1AAAAAlwSFlzAAAXEgAAFxIBZ5/SUgAAB3hJREFUaAXtWWlsTU8Un1ZF7dQuEltLxNIg9ogPYomQoioqtcQXW4UvSkjFEkEQCWLXikjol2qFiNiqtPYQS9Da1R6178txfieZyb333de+1/9779/IO8l9d+bMmTPnnpk55zfzIohJhSloHohgigya9rBi44Gwk40rglcIOzl4vjWaw042rgheIezk4PnWaA472bgieIUynQx0d/HiRdW3b1/16tUr9fz5c7Hk9evX6sWLF7bn2bNnXq38/v27a9ulS5fUrFmz1J8/f1zb/xVmRFk4+d27d2ry5MmqSpUq6uvXr+r06dOqa9eu8oYDmjZtqrp06SK+gCMXLFigTpw4oc6fPy990AD+rVu31KNHj1TNmjVFFj+Q6d27t9Rv3LihOnbsaNoqS2Hnzp1q3bp1qlGjRorhrmrZsqXKzMz0yzzgZAUn+0rJyck4uMhTv3594klw7dq5c2cjV716dWrbti19+fLFJrt//34jc/v2bVtbZavExcWJrTdv3vTbNMxIlD/T8vPnTyOOFV63bl1T14W9e/fKyq1du7basGGDGjRokIqJiVHR0dFaRN49e/ZUWVlZshv4I2xtla2CcBYfHy8ruUK2+To1vOWpRo0aMqNRUVFUVFRk6/r48WNauXIltWjRgrKzs4kNs7WHorJjxw7CbistLQ3YcBwmqVmzZjRy5Ej68eOH33plUnztlZaWZrZ3jx49bN327dtHCB+JiYlUUlJiawtV5enTp9S8eXNZCJygAzbswYMHqVatWlRYWFghnXCyz+Fi/fr1ZqcAEYA+ffqk5s+frzp16qQuXLigYmNjjYy3wq9fvxTvBG/NFeZD57dv39S0adMkBFVYkaUje1WdOXNG8e5Uffr0sbT4WfRlegYMGGBWMSME2rVrFzG0o+7duwv/7t275ao5cOAAjR49mlJSUgiJEdswULRt2zbimCm2DBw4kDgvEOcPoz4jI4MmTpxIjHAMz1ooLi6mO3fuWFlSfvPmDVWtWpWsO+P379+0ZcsWmjp1Kp06dcqjj5Mh0+FkOusMr4hRiHEy4lO3bt1MnZXQ9OnTnd1s9c2bN1OvXr2Ed+TIEemLGB4owoTBjnbt2hFDRqMWDlm8eLE4FzYPGzbMtOnC2bNnpS+c5qSjR49Km56cBw8eiAgWC8bDd5VHPjkZswaF+uGwIHpnzJhheKmpqV7Hevv2LU2ZMoW0gRMmTKDWrVsT+IGiuXPnEpJxfn6+TWVBQQGdPHlSeKtWrRJ7r1+/bpNhfC4Q08bkyv3794kREg0dOpQ4xNGVK1fo+PHjNl2+QDqfnDxp0iTjTCQ2TZhlbCU4v169euQL1kXygDzDP8IqK4uAwbFiMGZ6ejotXLhQntWrV3t0QygAFueTqEcbGEAFY8aMEadZQ9uaNWvEntmzZ3ugoZycHGnjXERXr14lhqMeMq6DOZjlOhkxqUmTJjIYnJObm2tUAKLNnDnTtK1du9a0uRUAqxCLGS9LPHeTsfKAUjA2njZt2hCftmQsbHsrPXz4UGSwmr2Rdtj48eNtIsuXLxc0cvnyZRsfIWfEiBGyOxBO+vfvT4j1nFhtcr5UynXyoUOHKDIyUj4O24rRhE0vEgwfuaW9Tp06tjZn5dy5cyK3ceNGZ5PXOj7K+mHYCc6EiW0NiAVbvRFCFBYJJkTTx48fJYbzvYxmmffLly+pQYMGhMkBIdRhDMBEf6lMJ+PjrAkO5Q8fPtjGQNbVIQMfMXz4cFu7tYLJaNWqlbCg29cVbdXhVp43b544ULc5bQQfti1dulSLyJtPm8LHAQr05MkTunfvnpRx8EIfHQJx8EIdiAqEXYyzgS/E/binF0KyQrN+sLXcyBqzMfsw0I2w0rGiEIISEhJo7NixHmIIKe/fv/cr9i1atEhshLIlS5ZI/HYqRlIEqtGEFaoRk07AgKM6cQJmIuFpQuiAH/jWUFh5eXlmTC2DN3YZ/KHhIyaD+3k6GYNa4zBE8MCBwKLXrl0TZXwAIZz8kPS0DN4NGzZ0xY+DBw82cgg9SHzWozfGxbG9ffv2HiHB+iHOsl7JWJmwG28nwS4N34CHAR8xITil7tmzRxLqsmXLpBvCGiZgyJAhRg2+CTpWrFhB27dvl7KWN0Jc2LRpE82ZM8d81+fPnyHLPQNIQAPI4t7o2LFjYqhbO5IqzMHtnD/E99wS1jA5WGFuBMd16NBBEIjGtlh1SUlJhMMVX7kaxwCqQZfzbgbhplq1atSvXz/B3s5xdIjBMVwTwmLAnayV+/sGDtVIBZdL/hJWjDVBuvUHYnBetwLaASpadxT6OuXAw87DboOtbjRu3DjiW0mCLVaqNE6GYTitOSGW1djKXMZuwuXU1q1bPcyEk8v8+0lmIQQ/hw8fVhxP1e7du0MwWuCH4MOK4oTt/WLKw/UhZuBuBP+eIDHprBxiE/7TcEBDPG3UuHFjDz24mMKU/u8rmYG/wvXnqFGjzP+CgV9rwdPI0FTxYUXhO9jLZiBGMIpPoVIv849U0yPIBU4oiuFUkEcJvXr8Ec3fxYDQ6v7Q2/HPj8iYO+Iv8mm8LwPTAuQAAAAASUVORK5CYII=)|基督|

% **註3:**1872-1927版原文採用頓號及句號,為方便查閱,本語料庫採用逗號及句號。

% |原文標點|本語料庫標點|
% |---|---|
% |、(頓號)|,(逗號)|

% \newpage

% \section{是必}

% (Empty file)

% \newpage

% \section{時文}

% #logic #老粵語 

% 時文: si4man4
% (pos:名詞)
% (label:舊式)
% yue:詞彙,講嘅嘢
% eng:words
% <eg>
% yue:X
% eng:X

% \newpage

% \section{未來}

% [遞時](遞時)

% \newpage

% \section{梗係}

% 梗

% X[硬](硬)

% \newpage

% \section{泒}

% 泒/沠

% 此字從語料看似乎有三種意思,第一種意思類同「批評」、「抱怨」:

% (46)  兩公婆只怨老母不仁、沠老母不是。(39)

% (47)  若只曉得泒翁姑不是,叔伯不是,做男子就唔著聽咯。(432)

%        而第二種意思類同系詞,但只用於排名:

% (48)  論起層次,長子亞孝泒第一,亞忠泒第三,亞信泒第四,此三個仔,俱係正妻所生,亞悌泒第二,亞仁泒第五,亞義泒第六,此三個仔俱係妾氏所生。(325)

% (49)  長子繼業泒第一,繼德泒第三,此兩个係結髮所生。繼功泒第二,繼績泒第四,此兩个係妾所生。繼祖泒第五,此一个係婢所生。(368)

%        而最後一個意思,類同「分送」:

% (50)  誰不知你行前人指後,話你等豬兄狗弟,實在都唔係人,今鬧起官司要將我大仔沠與乞兒,問你於心何忍?(103)

%             據《異體字字典》(教育部國語推行委員會 2002),此字可以是「派」或「流」的異體,但不論視作「派」還是「流」,都不能圓滿解釋以上所有意思。如果視作「派」字,則能符合最後一個意思,也可在第二個意思裏理解為「排」的近音別字,卻解釋不了第一種意思。如果將該字視作「流」,在第一種意思裏也許能視作「鬧」的近音別字,但未能解釋第二和第三種意思。

% \newpage

% \section{流}

% #metaphysics 係[[[假]]]嘅意思

% \newpage

% \section{淵}

% 󱛆

% 淵/淵痛

% 酸痛  
% e.g. 太耐冇打波,打完一場周身~/~~。

% \newpage

% \section{無侶可共}

% 無侶可[共](共)

% \newpage

% \section{理由}

% [因由](因由)

% \newpage

% \section{畀}

% 畀
% #介詞連詞

  
% [𢬿](𢬿)

% \newpage

% \section{疳疔}

% [魚口疳疔](魚口疳疔)

% \newpage

% \section{痕}

% #百越底層詞

% \newpage

% \section{白鮓}

% 白鮓  交通[[警察]]  

% #粵語生物詞彙

% \newpage

% \section{百越底層詞}

% #文 


% It is generally accepted that Cantonese is a Sinitic language with heavy influences from ancient Baiyue languages in terms of phonology, grammatical structure and vocabulary. Even though our Baak-jyut/Baiyue ancestors were colonised and eventually Sinicised, they had nevertheless left in our language an obvious Baiyue substratum, including substrate words which are mostly of Tai-Kradai, Austroasiatic and Hmong-Mien origins. These substrate words, though relatively small in number compared to Sinitic words, are among the commonest and most frequently used words in daily Cantonese conversations.

% Here is a list of substrate words I can think of, with possible cognates inside the brackets.

% Disclaimer: I am neither a scholar nor an expert in relevant fields. I've made this list purely based on my memories, a few essays of “dialect” (🙄) studies in Chinese and wiktionary. Regarding some words I've listed below, whether they are substrate words or not is still unconfirmed or disputed.

% _verbs_

% [[搣]] mit1: to tear up  
% 掹 mang1: to pull  
% 篤 duk1: to poke  
% [[冧]] lam3: to fall [Thai ล้ม (lom), Zhuang "laemx"]  
% [[冚]] kam2: to cover [Thai ห่ม (hom)]  
% 諗 nam2: to think  
% [[𨂽]] dam6: to stamp one's foot [Zhuang "daemh"]  
% 撳 gam6: to press down, to click [Thai ข่ม (khom), Zhuang "gaemh"]  
% 嘥 saai1: to waste [Zhuang "sai", Thai เสีย (sia)]  
% 踎 mau1: to squat  
% 郁 juk1: to move  
% 躝 laan1: to crawl, “get out!” [Thai คลาน (khlaan)]  
% 𨂾/𨈇 naam3/laam3: to cross [Thai ข้าม (khaam)]  
% [[𨅝]] jaang3: to tread on [Thai ย่าง (yaang)]  
% [[呃]] ngak1/ngaak1: to lie to, to deceive [Zhuang "ngaek"]  
% [[囈]] ngai1: to beg  
% 扱 kap1: to lid, to cover  
% 孭 me1: to carry on the back  
% 棟/戙 dung6: to erect, to stand straight [Thai ตั้ง (dtang)]  
% 𠺘 long2: to rinse [Thai ล้าง (laang)]  
% hap1 or 蝦 haa1: to bully, to pick on.  
% 甩 lat1: to slip off, to drop [Vietnamese "lột"]  
% 批 pai1: to peel  
% 鎅/𠝹 gaai3: to cut  
% 擁 ung2: to push  
% 焫 naat3: to sear [Zhuang "ndat", Thai เดือด (dueat)]  
% 淥 luk6: (liquid) to scald [Thai ลวก (luak)]  
% 㓤/拮 gat1: to pierce, to prick  
% 揈 fing6: to swing, to sway [Bouyei "veengh"]  
% 𢯎 ngaau1: to scratch [Thai เกา (gao)]

% _adjectives_

% [[痕]] han4: itchy [Zhuang "haenz"]  
% lak1 kak1 (couldn't find characters): road bumping or person stuttering  
% 孖 maa1: twin, double [Thai **ฝา**แฝด (**faa**faet)]  
% 乸 naa5: female, original meaning "female who fills the role of mother" as in 後底乸 [Thai น้า (naa)]  
% 腍 nam4: soft, tender [Thai นุ่ม (num)]  
% 啱 ngaam1: correct, suitable [Thai งาม (ngaam), Vietnamese "ngám"]  
% 凹 nap1: concave  
% 嬲 nau1: angry [Zhuang "naeuq"]  
% 奄尖 jim1 zim1: picky  
% 曳 jai5/jai4: ill-behaved, naughty, of poor quality [Zhuang "yaez"]  
% 𢛴憎 mang2 zang2: irritable  
% 蚊 man1 as in 细蚊仔, 蚊皮 ("naughty" in my dialect): naughty [Zhuang "manz"]  
% 杰 git6: viscous [Zhuang "gwd", Bouyei "geg"]  
% 孻 laai1 as in 孻仔, 拉尾/孻尾: last [Zhuang "byai", Thai ปลาย (bplaai)]

% _nouns_

% 甩 lat1 as in 麻甩: sparrow [Zhuang "laej"]  
% 㯷 buk6: pomelo [Zhuang "mak**bug**" ("mak" meaning "fruit")]

% > The fruit is called 碌柚 in Canton/Hong Kong Cantonese, but 孤㯷 is actually the common name for pomelo in my dialect. "Buk" is the original Tai-Kradai word for pomelo. Many other Cantonese dialects, such as Sanwui dialect or Taishanese also call it 布碌/波碌/etc. meaning "little pomelo" (buk/bu/bo=pomelo, luk=child/classifier for fruit)

% 蠄蟧 kam4 lou2: spider  
% 馬騮 maa5 lau1: monkey  
% 蛤 gap3/gaap3 as in 蛤乸: frog [Thai กบ (kop)]  
% 項 hong2 as in 雞項: young hen  
% [[拏褦]] naa1 nang3: connection, link, relation [Zhuang "nanaengq"]  
% 𧕴 naan3: inflammation on the skin resulting from an insect bite or an illness [Zhuang "nwnj"]  
% 蝻 naam4 as in 蝻蛇 or 大蝻蛇: python [Zhuang "nuem", Thai เหลือม (lueam)]  
% 陸 luk6 as in 豬陸: pen, sty [Thai คอก (khaok)]  
% 馬蹄 maa5 tai4: water chestnut [Zhuang "makdaez"]  
% 椗 ding3 as in 慈姑椗: stem of plants  
% 佬 lou2: (colloquial) man [related to the historical Raew/Rau people (僚人) and Laos (寮國/老撾)]

% > 廣州謂平人曰佬,亦曰獠,賤稱也。— 《廣東新語》  
% > People in Canton call common people “lau” or “leu”, both derogatory. -- _Guangdong Xinyu_, 1678

% 碌 luk1 as in 碌柚: child [Thai ลูก (luuk)]  
% 细路仔 sai3 lou6 zai2: child, boy [路 is the same word with 碌 above. 路仔 is obviously a cognate with Thai ลูกชาย (luukchaai) or Zhuang "lwgsai" meaning "son"]

% _others_

% 呢 nei1/ni1: this [Thai นี้ (ni), Zhuang "neix", Khmer នេះ (nih), Vietnamese "nầy", Malay "ini"]  
% 咪 mai5: don't [possibly related to Thai มิ (mi) or ไม่ (mai)]  
% 啲 di1/dit1: a bit, some [Zhuang "di", uncertain]

% _words of unknown origin_

% 嘢 je5: thing

% \newpage

% \section{皇氣}

% 有[[警察]] ,顯然指香港[[殖民時代]][[皇家警察]]

% #香港城邦

% \newpage

% \section{眧}

% 讀[超];[[睇]];眼鏡;以前好興講一句[[[眼眧眧]],[[唔順眧]]]

% \newpage

% \section{眼拙}

% #禮貌用語

% [[眼殘]]

% \newpage

% \section{禮貌用語}

% #文 

% 頭次見面用[久仰],好耐冇見講[久違]。
% 認人不清用[[[眼拙]]],向人表歉用[失敬]。  
% 請人批評講[指教],求人原諒用[包涵]。
% 請人幫忙講[勞駕],請畀方便用[借光]。  
% 麻煩人哋講[打擾],唔好意思用[冒昧]。

% 求人解答用[請問],請人指點講[賜教]。  
% 贊人見解用[高見],自己意見用[拙見]。

% 探望人哋用[拜訪],賓客到來用[光臨]。  
% 陪著朋友用[奉陪],中途先走用[失陪]。

% 等待客人用[恭候],迎接表歉用[失迎]。  
% 有人離開用[再見],請人別送講[留步]。

% 歡迎顧客稱[光顧],謝人問候用[托福]。  
% 問人年齡用[貴庚],老人年齡用[高壽]。

% 閱人文章用[拜讀],請人修改用[斧正]。  
% 對方字畫稱[墨寶],招待不周講[怠慢]。

% 請人收禮用[笑納],辭謝饋贈用[心領]。  
% 問人姓氏用[貴姓],答覆詢問用[免貴]。

% 表演技能用[獻醜],答謝讚揚講[過獎]。  
% 向人祝賀講[恭喜],覆人道賀用[多謝]。

% 請人擔職用[屈就],暫時充任講[承乏]。  
  
% 歷代有唔少謙讓嘅典故,係後人嘅典範。  
  
% **

  

% 〔令〕字一族:  
% 用喺對方嘅親屬或有關係人物嘅用詞。  
% 如,令尊:尊稱對方嘅[父親];  
%   令堂:尊稱對方嘅[母親];

%   令郎:尊稱對方嘅[兒子];

%   令愛:尊稱對方嘅[女兒];

%   令嬡:同〔令愛〕;  
%   令兄:尊稱對方嘅[兄長];

%   令弟:尊稱對方嘅[弟弟];

%   令侄:尊稱對方嘅[侄子]。

  
% **  
  
% 〔賢〕字一族:  
% 用喺平輩或者晚輩身份嘅用詞。

% 如,賢弟:稱呼自己嘅[弟弟]或比自己年齡細嘅男性;  
%   賢侄:稱呼自己嘅[侄子]或比自己輩份低嘅男性。  
  
% **  
  
% 〔恭〕字一族:  
% 表示好恭敬咁對待對方嘅用詞。

%  如,恭賀:恭敬向對方[祝賀];

%   恭候:恭敬向對方[等候];

%   恭請:恭敬向對方[邀請];

%   恭迎:恭敬向對方[迎接];

%   恭喜:祝賀對方有[喜事]。  
  
% **  
  
% 〔拜〕字一族:

% 用喺對人事往來嘅用詞。  
% 如,拜讀:表示[閱讀]對方文章;

%   拜辭:表示與對方[告辭];  
%   拜訪:表示[訪問]對方;  
%   拜服:表示[佩服]對方;  
%   拜賀:表示[祝賀]對方;  
%   拜識:表示[結識]對方;  
%   拜託:表示[委託]對方辦事情;  
%   拜望:表示[探望]對方。  
  
% **  
  
% 〔奉〕字一族:  
% 用喺自己舉動涉及對方時嘅用詞。  
% 如,奉達:表示[告訴],[表達];(多數用喺書信)  
%   奉覆:表示[回覆](多數用喺書信);  
%   奉告:表示[告訴];  
%   奉還:表示[歸還];  
%   奉陪:表示[陪伴];  
%   奉勸:表示[勸告];  
%   奉送:表示[贈送];  
%   奉贈:同〔奉送〕;  
%   奉迎:表示[迎接];  
%   奉托:表示[拜託]。  
  
% **  
  
% 〔敬〕字一族:  
% 用喺自己行動涉及對人嘅用詞。  
% 如,敬告:表示[告訴];  
%   敬賀:表示[祝賀];  
%   敬候:表示[等候];  
%   敬禮:表示[恭敬](用於書信結尾);  
%   敬請:表示[邀請];  
%   敬佩:表示[敬重佩服];  
%   敬謝:敬謝不敏;表示[推辭]做某件事。  
  
% **  
  
% 〔貴〕字一族:  
% 稱呼同對方有關事物嘅用詞。  
% 如,貴幹:問對方要做咩嘢;  
%   貴庚:問對方[年齡];  
%   貴姓:問對方[姓氏];  
%   貴恙:稱對方[病況];  
%   貴子:稱對方[兒子](含祝福之意);  
%   貴國:稱對方[國家];  
%   貴校:稱對方[學校]。  
  
% **  
  
% 〔高〕字一族:  
% 稱許人哋有關事物嘅用詞。

% 如,高見:即係高明嘅[見解];  
%   高就:指離開原來職位就任較高嘅職位;  
%   高齡:稱老人家嘅[年齡](多數指六十歲以上);  
%   高壽:用喺詢問老人家嘅年齡;  
%   高足:稱呼人哋嘅[學生];  
%   高論:稱許人哋嘅[議論]。  
  
% **  
  
% 〔大〕字一族:  
% 尊稱對方或稱同對方有關事物嘅用詞。

% 如,大伯:除咗指伯父之外,亦可以尊稱年長嘅男人;  
%   大哥:可以尊稱同自己年齡相仿嘅男人;  
%   大姐:可以尊稱女性朋友或者熟人;  
%   大媽:尊稱年長嘅婦女;  
%   大娘:同〔大媽〕;  
%   大爺:尊稱年長嘅男人;  
%   大人:尊稱長輩(多數用喺書信);  
%   大駕:尊稱對方;  
%   大師:大師傅,尊稱和尚;  
%   大名:尊稱對方嘅[名字];  
%   大慶:尊稱老年人[壽辰];  
%   大作:尊稱對方嘅[著作];  
%   大札:尊稱對方嘅[書信]。  
  
% **  
  
% 〔請〕字一族:  
% 用喺希望對方做某啲事嘅用詞。

% 如,請問:用喺請求對方回答問題;  
%   請坐:請對方[坐低];  
%   請進:請對方[入來]。  
  
% **  
  
% 〔屈〕字一族:  
% 如,屈駕:委屈大駕(多數用喺邀請人);  
%   屈就:委屈就任(多數用喺請人擔任職務);  
%   屈居:委屈處於(較低嘅地位);  
%   屈尊:委屈降低身份俯就。  
  
% **  
  
% 〔光〕字一族:  
% 表示光榮,用喺對方來臨嘅用詞。

% 如,光顧:稱客人來到(多數用喺商家歡迎顧客);  
%   光臨:稱賓客到來。  
  
% **  
  
% 〔俯〕字一族:  
% 公文書信中用來稱對方對自己行動嘅用詞。

% 如,俯察:稱對方或上級對自己嘅理解;  
%   俯就:用喺請對方同意擔任職務;  
%   俯念:稱對方或上級體念;  
%   俯允:稱對方或上級允許。  
  
% **  
  
% 〔華〕字一族:  
% 稱對方有關事物嘅用詞。

% 如,華誕:稱對方嘅[生日];  
%   華堂:稱對方嘅[房屋];  
%   華翰:稱對方嘅[書信];  
%   華宗:稱對方係[同姓氏]或[同宗族]。  
  
% **  
  
% 〔叨〕字一族:  
% 如,叨光:沾光(受到好處,表示感謝);  
%   叨教:領教(受到指教,表示感謝);  
%   叨擾:打擾(受到款待,表示感謝)。  
  
% **  
  
% 〔雅〕字一族:  
% 用喺稱對方情意或舉動嘅用詞。

% 如,雅教:稱對方嘅[指教];  
%   雅意:稱對方嘅[情意]或[意見];  
%   雅正:稱對方嘅[指正]同[批評](將自己嘅詩文書畫等送畀人時)。  
  
% **  
  
% 〔玉〕字一族:  
% 用喺尊稱對方身體或行動嘅用詞。

% 如,玉體:尊稱對方嘅[身體];  
%   玉音:尊稱對方嘅[書信](多數用喺書信)、言辭;  
%   玉照:尊稱對方嘅[照片];  
%   玉成:稱對方幫助[成全]好事。  
  
% **  
  
% 〔芳〕字一族:  
% 用喺對方或同對方有關事物嘅用詞。

% 如,芳鄰:稱對方嘅[鄰居];  
%   芳齡:稱對方嘅[年齡](多數用喺年輕女子);  
%   芳名:稱對方嘅[名字](多數用喺年輕女子)。  
  
% **  
  
% 〔垂〕字一族:  
% 用喺人哋對自己的行動嘅用詞(多數係長輩或上級)。

% 如,垂愛:稱對方對自己嘅[愛護](都係用喺書信);  
%   垂青:稱別人對自己嘅[重視];  
%   垂問:稱別人對自己嘅[詢問];  
%   垂詢:同〔垂問〕;  
%   垂念:稱別人對自己嘅[思念]。  
  
% **  
  
% 〔惠〕字一族:  
% 用喺對方對待自己行為動作嘅用詞。

% 如,惠存:請對方保存(多數用喺送人相片、書籍等紀念品時所題嘅上款);  
%   惠臨:指對方來到自己嘅地方;  
%   惠顧:指來臨光顧(多數用喺鋪頭對顧客),又叩[幫襯];  
%   惠允:指對方允許自己(做某事);  
%   惠贈:指對方贈予(財物)。  
  
% **  
  
% 謙 詞  
  
% 【愚】字一族:  
% 謙稱自己唔聰明嘅用詞。

% 如,愚兄:向比自己年輕嘅人稱呼自己;  
%   愚見:謙稱自己嘅見解。  
%   愚 :亦可以單獨用來謙稱自己。  
  
% **  
  
% 【鄙】字一族:  
% 謙稱自己學識淺薄嘅用詞。

% 如,鄙人:謙稱自己;  
%   鄙意:謙稱自己嘅意見;  
%   鄙見:謙稱自己嘅見解。  
  
% **  
  
% 【敝】字一族:  
% 謙稱自己或自己事物嘅用詞。

% 如,敝人:謙稱自己;  
%   敝姓:謙稱自己嘅姓氏;  
%   敝處:謙稱自己嘅屋企、處所;  
%   敝校:謙稱自己所在嘅學校。  
  
% **  
  
% 【卑】字一族:  
% 謙稱自己身份低微嘅用詞。  
% 如,卑人:謙稱自己;

%   卑下:謙稱自己地位、品格等比對方低。  
  
% **  
  
% 【竊】字一族:  
% 有私下、私自之意,有冒失、唐突嘅含義在內。  
% 如,竊念:私下想念(表示個人意見嘅謙辭);  
%   竊惟:私下思惟;  
%   竊比:私自比擬;  
%   竊言:私下談論;  
%   竊庇:私下包庇;  
%   竊竊:暗中,偷偷地;  
%   竊議:私下議論,私自評論。  
  
% **  

  
% 【臣】字一族:  
% 謙稱自己不如對方嘅身份地位高。

% 如,臣僕:僕人嘅自稱;  
%   臣妾:妻妾嘅自稱;  
%   臣子:君主時代嘅官吏。  
  
% **  

  
% 【僕】字一族:  
% 舊謙稱自己[我]嘅用詞。

% 如,僕:謙稱自己係對方嘅僕人,願意受差遣。而[為奴為僕]含有為對方效勞嘅意思。

  
% **

  
% 【敢】字一族:  
% 表示冒昧請求人哋做事嘅用詞。

% 如,敢問:用喺詢問對方嘅問題;  
%   敢請:用喺請求對方做某事;  
%   敢煩:用喺麻煩對方做某事。  
  
% **  
  
% 【拙】字一族:  
% 用喺對人哋稱呼自己事物嘅用詞。

% 如,拙筆:謙稱自己嘅[書畫];  
%   拙着:謙稱自己嘅[文章];  
%   拙作:同〔拙着〕;  
%   拙見:謙稱自己嘅[見解];  
%   拙荊:謙稱自己嘅[妻子],重有[賤內]、[內人]等。  
  
% **  
  
% 【小】字一族:  
% 謙稱自己或同自己有關嘅人或事物嘅用詞。

% 如,小弟:男性喺朋友或熟人之間嘅謙稱自己;  
%   小兒:謙稱自己嘅[兒子];  
%   小女:謙稱自己嘅[女兒];  
%   小人:地位低嘅人[自稱];  
%   小子:子弟晚輩對父兄尊長嘅自稱;  
%   小可:多見喺早期嘅白話,係有一定身份嘅人嘅自謙,  
%      意思係自己好平凡、不足掛齒;  
%   小店:謙稱自己嘅[鋪頭]。  
%   小生:讀書人自謙詞,重有[晚生]、[晚學]等,表示自己係新學後輩。  
  
% **  
  
% 【家】字一族:  
% 古人稱呼自己一方親屬朋友嘅常用謙詞。

% 家,係對人哋稱呼自己輩份高或年紀大嘅親屬時用嘅謙詞。  
% 如,家父:稱呼自己[父親];  
%   家尊:同[家父];  
%   家嚴:同[家父];  
%   家君:同[家父];  
%   家母:稱呼自己[母親];  
%   家慈:同[家母];  
%   家兄:稱呼自己[兄長];  
%   家姐:稱呼自己[姐姐];  
%   家叔:稱呼自己[叔叔]。  
  
% **  
  
% 【舍】字一族:  
% 用來謙稱自己家或自己卑幼親屬嘅用詞。

% 如,寒舍:謙稱自己嘅家;  
%   敝舍:客氣對朋友稱自己嘅家。  
% 如,舍弟:稱自己嘅[弟弟];  
%   舍妹:稱自己嘅[妹妹];  
%   舍侄:稱自己嘅[侄子];  
%   舍親:稱自己嘅[親戚]。  
  
% **  
  
% 【老】字一族:  
% 老人家自謙時用來謙稱自己或同自己有關事物嘅用詞。

% 如,老朽:謙稱自己做[老邁衰朽];  
%   老夫:指年齡超過七十歲嘅男人謙稱自己;  
%   老漢:年老嘅男人謙稱自己;  
%   老拙:老年嘅男人謙稱自己,亦稱做〔老粗〕;  
%   老粗:謙稱自己冇咩文化;  
%   老臉:年老人指自己嘅面子;  
%   老身:老年婦女謙稱自己;  
%   老衲:老和尚謙稱自己;  
%   老臣:老官員謙稱自己。  
  
% **  
  
% 【貧】字一族:  
% 僧、道、尼姑自謙稱呼嘅用詞。

% 如,貧僧:和尚對自己謙稱;  
%   貧道:道士對自己謙稱;  
%   貧尼:尼姑對自己謙稱。  
  
% **

% \newpage

% \section{穿崩閗湊}

% 穿崩閗臭?

% \newpage

% \section{粵典——𢬿}

% #文  #老粵語
% 又到咗古典粵語嘅時間。大家有冇聽過「[[𢬿]]」呢個用法呢?

% (kaai4 同「鞋」近音,不過聲母要變做 k)

% 【械/𢬿 kaai4 / kaai3 / gaai5 (prep.) 】 [http://beta.words.hk/zidin/𢬿](https://l.facebook.com/l.php?u=http%3A%2F%2Fbeta.words.hk%2Fzidin%2F%25F0%25A2%25AC%25BF%3Ffbclid%3DIwAR27MOU2cN-YtCSBCwb1h4IN83YncyqiA8QBW0g7yVVDrAP6wQYk4rRCJ8w&h=AT23Gp_xzhRntcCNhZLcH_y3gKjRkpI_lrsOZ2M8wFwhrUyWdH2R9ffVn7-DcN9xZNxqr6dYIBriab8UhpEQAoc5BxxgozK37a2Wnx6GM52r1HlVGcZTMd3rBa-WcYZYACyGvK0&__tn__=-UK-R&c[0]=AT0z70hYIGdXQc3MqSw4dwhohaCgZCnG6XLb3s1ssfCcMeCY2cjwqw43hMZt3mTiH9AFYm5eUNhJKllIMj9ka2Idx1bYI9JixDyEgX-qSeYeGVJw7MKpXe147_aXHczHe-jyAFzHG1KHGJ2GBGSru4ixa9WbgM-09R10g2I)

% 引出一個動作嘅受事對象或者工具;而家非常之少用 used to introduce the patient or the tool before a verb; rarely used nowadays

% 例句:

% 唐人有𢬿牛乳共糖嚟攪茶有冇呢?(《漢語讀本之廣東方言 A Chinese Chrestomathy in the Canton Dialect》1841)

% Do the Chinese use milk and sugar with their tea?

% 擰個毛掃嚟𢬿衣服嘅沙塵掃乾淨佢。(《散語四十章》1877)

% 拎個毛掃嚟將啲衫嘅沙塵掃乾淨佢。

% Take a brush to brush dust off the clothes.

% 嗰個女人嘅指甲長,𢬿佢手臂搲爛嘵。(《散語四十章》1877)

% 嗰個女人指甲長,將佢手臂搲損咗。

% That woman has long fingernails, scratching and thus getting her arm hurt.

% 可以械錢或械力幫助人嘅。(《教話指南 Beginning Cantonese (Rewritten)》1927)

% 可以出錢或者出力幫人嘅。

% (One) can help others with money or effort.

% 械條布帶同佢包住個傷口喇。(《教話指南 Beginning Cantonese (Rewritten)》1927)

% 攞條布帶同佢包住個傷口啦。

% Take a cloth band to wrap the wound.

% 你械嗰啲嘢擺品字樣喇。(《教話指南 Beginning Cantonese (Rewritten)》1927)

% 你將嗰啲嘢擺做品字形啦。

% You arrange those things in a triangular pattern like the character 品.

% 你械條命嚟教飛咩?(《教話指南 Beginning Cantonese (Rewritten)》1927)

% 你攞條命嚟教飛咩?

% Aren't you putting your life at risk?

% 街上擺賣嘅食物,都要𢬿罩罩住。(《分類通行廣州話》)

% 街上擺賣嘅食物,都要搵罩冚住。

% The food sold in the street has to be covered with a cover.

% 佢械條毛布嚟做抹抬布。(張洪年,2007:《香港粵語語法的研究(增訂版)》頁409) (keoi5 kaai4 tiu4 mou4 bou3 lai4 zou6 maat3 toi2 bou3.)

% 佢用條毛布嚟做抹抬布。

% He used a coarse cotton cloth as a table cloth.

% 你𢬿嗰部書擰俾我啊,聽見冇?

% (You) take that book to me. Can you hear me?

% 借錢過人,係𢬿自己嘅錢俾過人使。

% To lend money to someone, is to have your own money given to the person for him to spend.

% 你械個空樽裝滿佢。

% Take the empty bottle and have it filled.

% 你點可以𢬿人家嘅嘢夾硬擰去得𠺝?

% How could you take away someone else's belonging without their agreement?

% 佢械箭射完我,又埋嚟械條大棍扑我,都唔明解佢做咩嘢事幹。

% First he shot me with an arrow, then he came over and hit me with a stick. I have no idea what he's doing.

% 隻雀械對爪抓實枝樹枝。

% The bird used its claws to hold on to the branch.

% 械啲濕咗嘅衫曬乾佢。

% Take the soaked clothes and have them dried.

% 【械/𢬿 kaai4 / kaai3 (v.) 】 [http://beta.words.hk/zidin/𢬿](https://l.facebook.com/l.php?u=http%3A%2F%2Fbeta.words.hk%2Fzidin%2F%25F0%25A2%25AC%25BF%3Ffbclid%3DIwAR03j7j_MGCvySsjQB1uTs5q3lYtyrwaN3ARtAkpcklecnYUPtTkmLhd7vQ&h=AT23Gp_xzhRntcCNhZLcH_y3gKjRkpI_lrsOZ2M8wFwhrUyWdH2R9ffVn7-DcN9xZNxqr6dYIBriab8UhpEQAoc5BxxgozK37a2Wnx6GM52r1HlVGcZTMd3rBa-WcYZYACyGvK0&__tn__=-UK-R&c[0]=AT0z70hYIGdXQc3MqSw4dwhohaCgZCnG6XLb3s1ssfCcMeCY2cjwqw43hMZt3mTiH9AFYm5eUNhJKllIMj9ka2Idx1bYI9JixDyEgX-qSeYeGVJw7MKpXe147_aXHczHe-jyAFzHG1KHGJ2GBGSru4ixa9WbgM-09R10g2I)

% 用;攞;要;愛。老一輩先會講呢個字。 to use; to take something; to use a thing as a tool to do something; only spoken by the elder generation

% 例句:

% 蠶吐絲人械嚟整衣服着。(《教話指南 Beginning Cantonese (Rewritten)》1927)

% 天蠶嘅絲係好噤人械嚟整樂器。(《教話指南 Beginning Cantonese (Rewritten)》1927)

% 竹劍係械嚟頑嘅呮。(《教話指南 Beginning Cantonese (Rewritten)》1927)

% 竹劍係攞嚟玩嘅唧。

% The bamboo sword is only used for playing.

% 呢部嘢係械嚟影相嘅。 (ni1 bou6 je5 hai6 kaai3 lei4 jing2 soeng2 ge3.)

% This machine is used for taking photos.

% \newpage

% \section{粵語成語}

% #文 
% 「粵語成語」類中嘅版

% 呢類有下面嘅51版,總共有51版。

% **D**

% - 大安指意
% - 大安旨意
% - 大聲夾惡

% **M**

% - 乜乜物物

% **N**

% - 臥薪嘗膽

% **S**

% - 誓神劈願
% - 实牙实齿
% - 實牙實齒
% - 声大夹恶
% - 聲大夾惡

% **T**

% - 天花龍鳳
% - 天花龙凤
% - 天花亂墜
% - 挑通眼眉

% **W**

% - 乌哩单刀
% - 烏利單刀
% - 烏哩單刀
% - 烏哩馬叉

% **一**

% - 一仆一碌

% **三**

% - 三口六面

% **冇**

% - 冇雷公咁遠

% **反**

% - 反口覆舌

% **同**

% - 同撈同煲

% **唔**

% - 唔多唔少
% - 唔湯唔水

% **奀**

% - [[奀嫋鬼命]]

% **家**

% - 家嘈屋閉

% **心**

% - 心大心細

% **捩**

% - [[捩橫折曲]]

% **擺**

% - 擺明車馬

% **整**

% - 整色整水
% - 整蠱做怪

% **有**

% - 有紋有路

% **望**

% - 望天打卦

% **死**

% - 死蛇爛鱔

% **照**

% - 照辦煮碗

% **狗**

% - 狗咬狗骨

% **生**

% - 生安白造

% **疊**

% - 疊埋心水

% **眈**

% - 眈天望地

% **知**

% - 知慳識儉

% **神**

% - 神神化化

% **見**

% - 見步行步

% **身**

% - 身光頸靚
% - 身水身汗

% **醒**

% - 醒醒定定

% **阿**

% - 阿支阿左

% **鬼**

% - 鬼五馬六
% - 鬼鬼鼠鼠

% **黑**

% - 黑口黑面
% - 黑古勒特

% \newpage

% \section{粵語用辭}

% #文 

% 華夏南北講嘢發音用辭都唔同。家下國語以北方話做底[1],主流文化寫粵語嘅傳統斷咗好耐,以致廿世紀末後生一代嘅廣府話用辭同上一代開始唔同,漸離古漢語同南話,趨近現代北方話。呢頁有粵京兩話用辭嘅對照表;有意見,歡迎去討論版傾。

% 維基文《粵語》入面都有張粵京用辭對照表,可以參考;其中有啲非傳統嘅詞。呢版同Wikipedia:粵語本字就儘量用傳統粵語同正字。如果用辭有考據資料或者俗寫,可以寫響「註」度。


  
% 名詞

% |   |   |   |
% |---|---|---|
% |**粵**|**京**|**註**|
% |[[咿唈:]]|動靜||
% |輘輷|蹊蹺||
% |冚唪唥|全部||
% |色水|色澤||
% |日頭|太陽||
% |宵夜|夜宵||
% |人工、糧|薪水、薪金||
% |火水|煤油||
% |樽|瓶||
% |雪櫃|冰箱||
% |窗(口/門)|窗(戶)||
% |夾萬|保險箱||
% |櫃桶|抽屜|「櫃桶」亦可以指水平敞開、無「抽」出結構嘅容器。|
% |屋企|家|正寫係「屋下」[未記出處或冇根據]|
% |荷包、銀包|錢包||
% |皮篋|皮箱||
% |林瀋嘢、家爛豆|勞什子||
% |老襯|冤大頭(無對應詞,被愚弄的人)|唔同「老親」(親家)|
% |番梘、番鹼|肥皂||
% |掣|(按)鈕|讀zai3|
% |挨憑|靠背|音aai1 peng1。「挨」係挨年近晚,挨晚嘅挨;「憑」原本讀 bang6音,例:憑喺幅牆度;俗讀 peng1。[2]|
% |屋、鬥(竇)|房子||
% |膥、春|卵|音ceon1|
% |衫|衣(服)||
% |鑊|(底較平的)鍋||
% |煲|(壁較陡峭的)鍋||
% |鐺|鐺(少用)、平底鍋||
% |埞、埞方|地方||
% |銀仔|硬幣|「銀」音ngan2|
% |蚊、文|塊、圓、元|貨幣單位|
% |毫、粒神|毛、角(正式)|貨幣單位|
% |檯、枱|桌子||
% |匙羹、匙|勺子||
% |地氈|地毯||
% |豉油|醬油||
% |窿|孔、洞||
% |罅|裂縫||
% |炮仗|鞭炮、爆竹||
% |遮|雨傘||
% |坑渠|下水道||
% |煙花|煙火、煙花||
% |騸雞|閹雞||
% |瓦罉|一種煲湯用的煲||
% |冷氣(機)|空調(機)||
% |貨櫃|集裝箱||
% |雪條、雪批|冰棒、冰棍||
% |雪糕|冰琪林||
% |飯盒|盒飯||
% |靚仔、令仔|帥哥|音leng3 zai2|
% |鉸剪、較剪|剪刀||
% |螺絲批|螺絲刀、改锥||
% |餸|(隨同飯的)菜||
% |鎖匙|鑰匙||
% |古仔|故事||
% |公仔|娃娃、玩具人物||
% |箭嘴|箭頭||
% |頸鏈、頸鍊|項鏈、項鍊||
% |韆鞦|鞦韆||
% |輘輷|秘密、玄機|讀「ging2 gwang2」,古漢語原意係「群車行走的隆隆巨響」|
% |花灑|蓮蓬頭||
% |膠擦、擦膠、擦紙膠|橡皮、橡皮擦|喺香港叫「擦膠」嘅比較多,而廣東就係「膠擦」比較常見。|
% |鹹蛋黃|日落||
% |射甩膽、士得膽|啟輝器|粵語乃英文「starter」嘅音譯。|
% |凳|椅子||
% |碟|碟子||
% |光碟、CD、DVD|光盤/鐳射光碟||
% |錔|鎖|讀「daap3」或「taap3」,廣州地區多數習慣讀成「taap3」。|
% |燙斗|熨斗||
% |椗|把兒、梗兒||
% |煮飯仔|过家家|兒童遊戲|
% |埞|位置||
% |猜呈尋、包剪揼|剪刀石頭布|猜拳遊戲|

  

% **人**[編輯]

% |   |   |   |
% |---|---|---|
% |**粵**|**京**|**註**|
% |人客|客人||
% |老世、事頭、波士|老闆||
% |先生、阿Sir、Miss|老師|「阿Sir」音「aa3 se4」|
% |後生|後輩、年輕人||
% |老豆(老竇)、阿爸、爹哋|爸、爹||
% |老母、阿媽、媽咪、老媽子|媽、娘|喺「老母」前面或後面加字通常會變成粗口。|
% |老爺、家公|公公||
% |奶奶、家婆、阿婆|婆婆||
% |外父(佬)|岳父||
% |外母(乸)|岳母||
% |爺爺、阿爺、阿公|爺爺、公公|「阿爺」喺粵語重有一種意思係「政府、公務」,譬如將「做公務員」稱為「打阿爺工」|
% |嫲嫲、阿嫲、阿婆|奶奶||
% |婆婆、阿婆|姥姥||
% |公公、阿公|姥爺||
% |細孥、細孥仔、細孥哥、細蚊仔|小孩||
% |𡃁仔、𡃁妹|少年、少女||
% |BB、BB仔、臊孲、臊孲子(蘇蝦仔)、孲𤘅、孲𤘅子|嬰兒|「孲」音aa1。「赤子」義。[3]|
% |冚家|全家|「冚」音ham4,本字係「咸(haam4)」,粵語亦有用「全家」。|
% |妗母|舅媽|「妗」為「舅母」之合音,讀kam5[4],南方諸語常用。|
% |老公、老婆|丈夫/妻子/愛人(對外用)||
% |大嫂|大媽||
% |姑丈|姑夫||
% |姑姐|姑姑||
% |伯爺|伯父||
% |伯娘|伯母||
% |舅仔|小舅子、妻弟||
% |師姑|尼姑||
% |嬸嬸、嬸|姨姨/姨|對一啲中年女人嘅叫法,有時對一啲女工人都會噉叫。|
% |鬼佬|白人/老外||
% |嚤囉叉、摩囉差、阿叉|阿三|貶低嘅稱呼|
% |乞兒|乞丐|音「乞衣」|
% |塞豆窿|熊孩子|好拽嘅細路稱呼|
% |花靚倞|小鮮肉|好後生啲男仔,出自2009年臺灣綜藝節目《康熙來了》——徐熙娣嘉賓嘅口頭禪。|
% |仔、亞仔|兒子|男仔細路|
% |女、亞女|女兒|女仔細路|
% |影帝|戏精|加戲搏出位啲人。2017年網絡流行詞。|
% |鬼佬|歪果仁|形容無國籍或非華人身份啲人。2016年中國大陸網上流行詞,出自歐美啲鬼佬講普通話嘅「外國人」一詞。|
% |本地人、當地人|當地民眾、當地群眾||

  

% **動植物**[編輯]

% |   |   |   |
% |---|---|---|
% |**粵**|**京**|**註**|
% |老鼠|老鼠、耗子||
% |馬騮|猴子||
% |曱甴、小強|蟑螂|「曱甴」係粵語最傳統嘅用法;而「小強」一詞出喺周星馳嘅電影《唐伯虎點秋香》中。|
% |雀、雀仔|鳥、小鳥||
% |生果|水果|喺廣東嘅粵語區重有人叫「水果」|
% |菜|蔬菜||
% |烏蠅|蒼蠅||
% |黃犬、黃螾|蚯蚓|可以寫作「黃䘆」|
% |蠄蟧|蜘蛛||
% |崩紗|蝴蝶||
% |塘尾|蜻蜓||
% |粟米|玉米||
% |薯仔|土豆、馬鈴薯||
% |蕃茄|西紅柿、蕃茄||
% |蕹菜、通菜|通心菜||
% |蛤乸、蛤𧊅、田雞|蛙||
% |蠄蟝|蟾蜍||
% |芫茜|芫荽、香菜||
% |勝瓜|絲瓜||
% |涼瓜|苦瓜||

  

% **身體部位**[編輯]

% |   |   |   |   |   |   |   |   |   |   |   |
% |---|---|---|---|---|---|---|---|---|---|---|
% |**粵**|**京**|**註**||**粵**|**京**|**註**||**粵**|**京**|**註**|
% |眼|眼睛|||鼻哥、鼻|鼻子|||頭皮|頭屑、頭皮屑|亦解做頭部皮膚,譬如「頭皮好痕」|
% |耳仔|耳朵|||(一塊)面|(一張)臉|||樣|樣子、模樣|粵音陰上聲(joeng2)|
% |膊頭|肩|||膶|肝(食用)|||翼|翅膀||
% |脷|舌(頭)|||橫脷|胰(食用)|亦有講「脌貼」嘅||髀|腿||
% |𢆡、姩、波|乳房|||手指公|大拇指|||手指屘、屘指|小指||
% |膝頭哥|膝蓋|||啫啫、朘仔、賓舟|雞雞、陰莖、陽具|||頸|脖子||
% |鬍鬚|鬍子|||胳肋底|腋窩、胳肢窩|||腰|腎(食用)||
% |腎|胗(食用)|禽鳥嘅胃||背脊|背部|||屎窟|屁股||
% |腳踭|腳跟|||手踭|手肘||||||

  

% **外來譯音詞**[編輯]

% 下便呢啲係外語音譯詞,唔包括人名同地名(因為太多數唔嗮)。由於音譯詞大多數都係喺香港形成嘅,有部分喺內地唔一定會用。

% |   |   |   |   |
% |---|---|---|---|
% |**粵**|**京**|**英**|**註**|
% |巴士|公共汽車|bus|廣東近年都有人講「公交(車)」|
% |的士|出租車、計程車|taxi|台灣叫「計程車」;星加坡叫「德士」|
% |芝士|奶酪|cheese|台灣叫「起司」|
% |多士|吐司|toast||
% |安士、盎士|盎司|ounce||
% |啫哩|果凍、凝膠|jelly|「啫哩」音ze1lei2|
% |朱古力|巧克力|chocolate||
% |士巴拿|扳手|spanner(美:wrench)||
% |士多啤梨(港、粵)/草莓(粵)|草莓|strawberry|廣東多數人講「草莓」|
% |布冧|李子|plum||
% |𨋢()(港)/升降機、電梯(粵)|升降機、電梯|lift(美:elevator)|廣東主要講「升降機」同「電梯」,受香港影響亦都有講「𨋢」嘅|
% |士擔、郵票|郵票|(postage) stamp||
% |燕梳、保險|保險|insurance||
% |車厘子|櫻桃|cherry|粵語亦有人講「櫻桃」|
% |泊車、停車|停車|park||
% |三文魚|鮭魚|salmon||
% |菲林|膠卷|film||
% |布菲|自助餐|buffet|粵語亦有人講「自助餐」|
% |呔|輪胎|tyre/tire||
% |呔|領帶|(neck)tie||
% |梳化|沙發|sofa||
% |沙律|沙拉|salad||
% |三文治|三明治|sandwich||
% |士碌架|斯諾克|snooker||
% |笨豬跳|蹦極|bungee jumping||
% |乒乓波|乒乓球|pingpong|台灣叫「台球」|
% |摩打|馬達|motor||
% |啤令|軸承|bearing|台灣叫「培林」|
% |咪高峰、咪頭、咪|麥克風、麥|microphone|「咪」讀mai1|
% |XX啤梨|X莓|*-berry|廣東比較少用。|
% |結他|吉他|guitar||
% |低音結他|貝斯、貝司|bass||
% |士多|商店|store|通常指鋪頭比較細嗰種。|
% |(發)毛|(發)霉|mold|「毛」讀 mou1|
% |忌廉|奶油、乳脂|cream|「廉」讀 lim1|
% |冷|毛線|法語:laine|讀 laang1。例:冷衫(毛衣)|

  

% 代詞同代詞詞尾[編輯]

% |   |   |   |   |   |   |   |   |   |   |   |
% |---|---|---|---|---|---|---|---|---|---|---|
% |**粵**|**京**|**註**||**粵**|**京**|**註**||**粵**|**京**|**註**|
% |佢|他、她、牠、祂(簡體字「牠、祂」統一為「它」)|「佢」,或作「渠」,本字「其」。「他人」義[5]。||邊|(疑問詞詞頭,相當於「哪」)|本字「焉」||點解|為什麼|「點」嘅本字可能係「怎」|
% |(我、你、佢)哋|(我、你、他/她/它)們|粵語無「朋友哋」(「朋友們」)呢種用法。本字「等」||邊個、乜誰|誰,誰人|或作「物誰」||噉|這樣、那樣|「噉」讀 gam2,「咁」讀 gam3。本字「恁」|
% |呢、爾|這|||邊道、邊處|哪兒,哪裡|||咁|這麼、那麼|
% |嗰、箇|那|「嗰」喺其他地方嘅中國話相當於「嘅」||點、點樣|怎麼、如何|||你|你、您||
% |人哋|人家、別人、他人||||||||||

  

% 動詞[編輯]

% |   |   |   |   |   |   |   |   |   |   |   |
% |---|---|---|---|---|---|---|---|---|---|---|
% |**粵**|**京**|**註**||**粵**|**京**|**註**||**粵**|**京**|**註**|
% |講明、講清楚|說清楚|||來往|往來、交往|||幫襯|光顧,惠顧|「幫襯」喺其他方言入便有「幫手」同「幫補」嘅意思|
% |呷醋|吃醋|||趁墟|趕集|||應承|答應||
% |鍾意|喜歡|||隻揪|單挑||||||
% |得閒|有空|||蕩失(路)|迷路|||講畀……聽、話畀……聽|告訴……||
% |屙尿|小便、排尿|||屙屎|大便、拉屎|||埋單|結賬、買單||
% |拍拖|約會、談戀愛|||打乞嗤、打乞嚏|打噴嚏|||傾、傾偈|聊、聊天、談、談話||
% |沖涼|洗澡|||抆屎|擦屁股|||講大話|說謊、撒謊||
% |發吽豆、發怐瞀|發呆、發愣|「怐瞀」讀 ngau6 dau6||出世|出生|||起身|起来、起床||
% |發夢|做夢|||打交|打架|||嘈交、嗌交|吵架||
% |𠺘口/口|漱口|「𠺘」讀 long2||㿺㿭|(皮膚)破裂|||影相|照相||
% |打尖/尖隊/櫼隊|插隊||||||||||

  

% **單字動詞**[編輯]

% |   |   |   |   |   |   |   |   |   |   |   |
% |---|---|---|---|---|---|---|---|---|---|---|
% |**粵**|**京**|**註**||**粵**|**京**|**註**||**粵**|**京**|**註**|
% |係|是|普通話喺非常正式嘅場合會用「係」字代替「是」。||掟|擲,扔|||食,喫|吃,喫|喫音jaak3[6]|
% |唱|傳播、散佈(消息)|例如「唱衰」||明|明白|||知|知道||
% |行|走|音 haang4||企|站|本字可能係「徛」||撳(掣)|按(鈕)||
% |𥄫|瞧|||睇|看|正視用「睼」斜視、隨意看眼用「睇」||拎|拿||
% |攞、挪|取|本字可能係「挪」||搵、捃|找(東西)、掙(錢)|本字可能係「捃」||揈|甩|本字唔確定,睇fing(有人借寫作「捹」)|
% |落|下|||𡁻|嚼|亦寫作「噍」||嘥|浪費|本字「㩄」|
% |鎅|切割|||捽|搓擦|||𢫏、冚|蓋|本字可能係「衾」|
% |厾|戳|有時寫成「篤」||揚|抖動|音 jeong2||孭|揹||
% |燂|灼烤|||轆」或者「碌|滾動|音luk1||撚|玩弄,挑弄||
% |蝕|虧(損)|「蝕」音 sit6。||耖[7]、摷|翻找|音caau3。可能寫成「抄」「找」||畀|給||
% |踩|踏|||㷫|加熱(引申義:憤怒)|||嬲,發嬲|氣憤||
% |坐、搭(車)|乘(車)|||嗌|喊|||嗑/噏|說|「說」嘅唔同意思分別對應「嗑」、「講」同「話」|
% |飲|喝,飲(少用)|||撠|卡|||講|說,講|
% |掹|拉扯、拔|||喊|哭|||話|說|
% |諗、惗|想|本字「念」||𧨾/氹|哄騙|||估|猜,估計|「估」重可以解作「以為」:「你估我唔知咩?」|
% |揸|拿|粵語有復合詞「揸拿」,另外「揸」重可以解做「駕駛」||着|穿,穿上(衣物等)|||除|脫,脫下(衣物等)||
% |掀、揭|翻(書等)|||敨/唞|歇、休息|||瞓|睡|本字「睏」|
% |剒|(無對應詞,突然用力拉扯)|||撞|碰(運氣)|||鬧|罵||
% |斬,斲|砍,斬(少用)|斲音doek3[8]||撩|挑惹,撩(少用)|||呃、𧦠、gwan2、𧥺|欺騙|gwan2有時寫成「滾」,「𧥺」有時寫成「昆」或者「坤」|
% |郁、㤢、逳|動|俗寫(郁).心㤢㤢, 㤢:《玉篇》心動也。逳:《玉篇》轉也,行也。《廣韻》步也。||閂|關|||噬|狠咬||
% |侲|壞|||添|加|添飯||搽|塗、抹、搽(少用)|音 caa4|
% |扠[9]|塗污|音 caa⁵,劃花噉解||搲||用手指捉住,音 we²||嘈|嚷,喧鬧,喧嘩|「嘈」可以做形容詞,同普通話意思一樣|
% |拗(詏)|爭論,爭吵|「拗」音aau3||拗|彎折,折斷|「拗」音aau2||掂|觸碰,摸||
% |撐|挺|解做支持、擁護||嚟、蒞|來|||撈|拌、混|「撈」音lou1|
% |入|進|||掗|占|||孭|揹||
% |惜|吻、疼愛|音「錫」||𢯎(痕)|搔(癢)|||慳|節省、省||
% |黐|粘、黏|||驚|怕|||厭|膩||
% |冧|塌|「冧」音lam3||擳|擠、胳肢|||返|回|(返去:回去;返嚟:回來)|
% |匿(埋)|藏(起來)|||蝦|欺負|||詐|(假)裝||
% |踎|蹲|||䟴|抖|䟴腳、腳䟴䟴||似|像||
% |捉(棋)|下(棋)|||識|會、認識|||㓟|削||
% |瀡|滑、溜|「瀡」音seoi5||揀|選|||搣|捏、撕||
% |𢳂|舀|||扚|拉開、手掐、拔擢|扚起心肝、扚住件衫、扚佢起身[10]||褪|後退|「褪」音tan3,例如「褪後」|
% |㧬|推擠|||湊(細路仔)|帶(小孩)|||瓊|凝結、凝固||
% |濁|嗆|||抌|扔|||攬|(擁)抱、摟||

  

% **助動詞**[編輯]

% |   |   |   |   |   |   |   |   |   |   |   |
% |---|---|---|---|---|---|---|---|---|---|---|
% |**粵**|**京**|**註**||**粵**|**京**|**註**||**粵**|**京**|**註**|
% |識、曉、能夠、可以|會、能、能夠、可以|||[[䎺]]、肯|願意、肯|||夠膽、敢|敢||
% |得、掂(𠶧)|行、可以、成功|用喺動詞後邊。「掂」、「𠶧」有分工:唔好掂樖樹;搞𠶧晒。||唔使|不用、不必、不需要、不需|問問題會問「使唔使」=普通話「用不用」||抵|值得||

  

% 形容詞

% |   |   |   |   |   |   |   |   |   |   |   |
% |---|---|---|---|---|---|---|---|---|---|---|
% |**粵**|**京**|**註**||**粵**|**京**|**註**||**粵**|**京**|**註**|
% |[[虢礫𡃈嘞]]|林林總總|||燶|焦|||[[狼戾]]|固執無理||
% |零星、濕碎|零星、零碎|||求祈、[[俹簁]]|馬虎|「俹簁」音laa² sai¹||斯文|斯文(書面語)、文雅||
% |求祈|問天|錯寫:求其||攰|累|||[[姣]]|放蕩||
% |事旦|明天才算|錯寫:是但、是旦、事但||冇所為|不為甚麽|錯寫:冇所謂||隨便|隨意及方便||
% |戇居|呆|||後生|年輕|「後生」亦係名詞,表示「年輕人」。||熱鬧、墟撼|熱鬧||
% |鶻突[11]、核突|惡心|形容啲嘢好難睇,或者形容一個人個樣好得人憎||腍|軟|||蠢|笨||
% |肉酸、朒朘|難看|令人不舒服||骨痺|肉麻|||肉赤|心疼||
% |凍|冷|||細|小|||耐|久||
% |啱|對、合適、巧、正確|||犀利|厲害|||崖广、[[牙煙]]|危險|「崖广」同「牙煙」同音。「广」係一個康熙部首,唔係簡體嘅「廣」字。|
% |鮓|差|||肥|胖|形容人,物,有時狀況。||晏|晚||
% |曳、跳皮|調皮、頑皮、淘氣|||論盡|笨拙|||痕|癢||
% |叻|聰明能幹、出色、有本事、很棒|||靚|好看、漂亮、美、帥、好|「靚」 音leng3||平|便宜|「平」音peng4[12]|
% |||||勤力、努力|努力|||邋遢、污糟|髒、骯髒、齷齪||
% |抵、著數/着數|合算、划算|||竄|拽|||燶|焦、糊||
% |牙擦|自負|||惡|兇|||眼瞓|睏|「瞓」本字「睏」|
% |唔啦更|不相關、不相干|||得意|可愛|||嬲|生氣||
% |(好)似|(好)像|||麻麻地|馬馬虎虎、一般|||怕醜|害羞||
% |[[菂薂]]|細小、小巧|||黃黚黚|淺黃帶黑|||[[腌臢]]|挑剔||
% |蹺|湊巧|「蹺」音kiu2||孤寒|吝嗇|||[[酸醙]]|酸臭|「[[醙]]」音suk1|
% |郇[13]、筍|正(貨)、好(東西)|例如「郇貨」、「郇盤」||盲舂舂|亂闖亂撞|||光掁掁、掁眼|明亮刺眼|   |
% |好彩|幸運|||穿煲|露陷|||[[孬攪、撈絞]]、擇使|難處理||
% |光|亮|||滐|稠|||水過鴨背|善忘、健忘||
% |穩陣|靠譜|||闊|寬|||[[蠱惑]]|滑頭||
% |[[娿哿]]|優柔寡斷|||山旯旮|偏僻|||得戚|得意||
% |[[㪐㩿]]|斷斷續續|||標青|出眾|||[[捩咁𠾍]]|一塌糊塗||
% |贔屭|一籌莫展||||||||||
% |水皮|差勁|||||||硬頸|固執||
% |眼冤|心煩|||肉緊|緊張|||心郁郁|心動||

  

% 副詞[編輯]

% |              |             |                                        |     |             |                        |                         |     |           |          |                            |
% | ------------ | ----------- | -------------------------------------- | --- | ----------- | ---------------------- | ----------------------- | --- | --------- | -------- | -------------------------- |
% | **粵**        | **京**       | **註**                                  |     | **粵**       | **京**                  | **註**                   |     | **粵**     | **京**    | **註**                      |
% | 明明、分明、擺明、明框  | 明明、分明       |                                        |     | 真箇、真個       | 確實                     | 「確實」做形容詞嘅時候,唔用「真箇」、「真個」 |     | 唔         | 不        |                            |
% | 無、冇          | 沒、沒有、無(書面語) | 「無」係「冇」嘅本字,「沒(有)」重可以做動詞,粵語同樣係「冇」、「無」   |     | 仲、重         | 還                      | 「重」係「仲」嘅本字              |     | 又         | 又、也      | 「也」唔同嘅意思分別對應「又」、「亦」、「都」三個字 |
% | 倒            | 約           | 位置唔同,例如普通話「約三斤重」對應粵語「三斤重倒」             |     | 約莫,大概,上下,左右 | 大約,大概,左右,約莫(少用),上下(少用) |                         |     | 都         | 都、也      |                            |
% | 咪、唔好(合音mou2) | 別、不要        | 咪讀mai3,可能係「勿」嘅變音,另外咪重可以讀mai6,「唔係」嘅合音字。 |     | [話晒](話曬),點講 | 好歹(不管怎樣,無論如何)          |                         |     | 亦、亦都      | 也、亦(書面語) |                            |
% | 幾……          | 多……         | 喺形容詞前便                                 |     | 幾多          | 多少                     |                         |     | 幾時        | 甚麼時候、幾時  |                            |
% | 未            | 還沒有、尚未      |                                        |     | 先           | 才                      | 「先」成日擺喺句尾,用法比「才」多變      |     | 特登        | 故意       |                            |
% | 梗係/更係        | 當然          |                                        |     | 好           | 很                      | 喺形容詞前便                  |     | 搏命        | 拚命       |                            |
% | 千祈           | 千萬          | 一般話「千祈唔好……」,普通話係「千萬不要……」               |     | 夾硬 、監硬      | 強逼、強制                  |                         |     | 禁(用、著)    | 耐(用、穿)   | 「禁」音kam1                   |
% | ……過龍、……過籠    | ……過頭        |                                        |     | 一齊          | 一起                     |                         |     | 第二啲、第啲、遞啲 | 其他       | 睇埋#時間入面「遞時」註解              |
% | 實            | 一定          |                                        |     | 不留          | 從來、一向                  | 「留」讀lau1                |     | 成日        | 整天、總     | 「成」讀seng4                  |
% | 好彩           | 好在、幸好、幸虧    |                                        |     | 淨係、剩係       | 只是、只有、單是、光是            |                         |     |           |          |                            |

  

% 量詞[編輯]

% |   |   |   |   |   |   |   |   |   |   |   |
% |---|---|---|---|---|---|---|---|---|---|---|
% |**粵**|**京**|**註**||**粵**|**京**|**註**||**粵**|**京**|**註**|
% |樖|棵|棵字廣東話讀fo2||嚿|塊/件|||啲|些|本字或為「尐」|
% |坺|堆|||些小、丁咁多,丁doe1、一啲|少許、一點(兒)、些小(書面語)|||橛|段|「橛」原來指「小段」[14],不過喺而家嘅粵語裏面泛指「段」。|
% |種、亭、隻|種|此情形亭字讀ting2或寫作「挺」||碌|條|||𠹻|股|味道|
% |啖|口|||埲、幅|堵、道|牆||對|雙|筷子、鞋|
% |羹|匙|鹽||兜|碗|糖水||隻|只|蛋|
% |笪|處|污漬||棚|排|牙||浸|層|油|
% |餐|頓|飯||篤|坨|糞便||拃|把|米|

  

% 介詞[編輯]

% |   |   |   |   |   |   |   |   |   |   |   |
% |---|---|---|---|---|---|---|---|---|---|---|
% |**粵**|**京**|**註**||**粵**|**京**|**註**||**粵**|**京**|**註**|
% |由|從、自、由|||向|往、朝|||喺、向(音「響」)|在||
% |將|把|||令、令到、使、使到|使、使得、讓|||同|和、同(少用)、跟、給|「給」:我同你做嘢(我給你辦事)|
% |俾、畀|被/讓/用||||||||||

  

% 連詞

% |                 |        |         |     |          |       |       |     |                              |             |       |
% | --------------- | ------ | ------- | --- | -------- | ----- | ----- | --- | ---------------------------- | ----------- | ----- |
% | **粵**           | **京**  | **註**   |     | **粵**    | **京** | **註** |     | **粵**                        | **京**       | **註** |
% | 同、同埋、連埋、以及、與及、與 | 和、及    | 「與」比較正式 |     | 或、或者     | 或     |       |     | 但係、之不過、不過                    | 但是、不過       |       |
% | [就算](就算)、即使(正式) | 即使、即便  |         |     | 於是乎,於是   | 於是    |       |     | 卒之                           | 終於          |       |
% | 即係              | 就是、意思是 |         |     | [唔通](唔通) | 難道    |       |     | 如果(有人讀成jyu4 gu2)、若果、若然、倘若、假如 | 如果、假如、倘若、假若 |       |
% | 尤其係             | 尤是     |         |     | 費事       | 免得    |       |     | 定係                           | 還是          |       |
% | 一係              | 要麼     |         |     |          |       |       |     |                              |             |       |

  

% 助詞[編輯]

% |   |   |   |   |   |   |   |   |   |   |   |
% |---|---|---|---|---|---|---|---|---|---|---|
% |**粵**|**京**|**註**||**粵**|**京**|**註**||**粵**|**京**|**註**|
% |嘅|的|||噉|地|||到|得||
% |咗|了|本字「徂」||緊|着、正在 (用於動詞前邊)|「緊」同「住」都對應普通話嘅「着」,但係意思唔同||親(嚫)|(無對應詞,表示動作已經完成且對現在有影響)、到|例:整親、跌親|
% |零[15]|來、把(用在數詞後表示約數)|「零」讀leng4||住|着|||||

  

% **語氣助詞**[編輯]

% |   |   |   |   |   |   |   |   |   |   |   |
% |---|---|---|---|---|---|---|---|---|---|---|
% |**粵**|**京**|**註**||**粵**|**京**|**註**||**粵**|**京**|**註**|
% |咩|嗎|表示懷疑或疑問||啩|吧(表猜測)|||喎|啊(表強調或不滿)|音wo3|
% |嘞/喇|吧(表催促)、了(表完成)|「嘞」讀laa3, 「喇」讀laa1||啝|類似「的啊」(表輕微訝異)|「啝」讀wo5||噃|(表提醒或勸告,無對應詞)|「噃」讀bo3|
% |囉|唄(表勉強)、嘛(表明顯)|「囉」讀lo1||啫|而已、罷了|||㖭/𠻹|(無對應詞)|表遞進,普通話通常講「還……」|

  

% 方向[編輯]

% |   |   |   |   |   |   |   |   |   |   |   |
% |---|---|---|---|---|---|---|---|---|---|---|
% |**粵**|**京**|**註**||**粵**|**京**|**註**||**粵**|**京**|**註**|
% |前頭、前邊、前面、前便[16]|前面、前邊|||下頭、下邊、下面、下便|下面、下邊|||左手邊、左邊、左便|左邊、左面||
% |右手邊、右邊、右便|右邊、右面|||出便、出面、外便、外面|外面|||入便、入面、裏面、裏便、裡便、裡面|裡面||
% |隔籬|隔壁、旁邊|||側邊|旁邊||||||

  

% 時間[編輯]

% |   |   |   |
% |---|---|---|
% |**粵**|**京**|**註**|
% |家下、而家、依家、家陣|現在、眼下||
% |一陣、陣間、一陣間|一會兒||
% |一排、一牌|一陣子(通常多過一日)||
% |傝正、搭正、踏正|整點||
% |一個字|5分鐘|指鐘面嘅數目字相距嘅時間。|
% |一個骨|一刻||
% |十二點踏(搭/傝)四、十二點四個字、十二點四|12時20分|「踏四」即係踏正之後過咗四個字(廿分鐘)。呢個情況踏、搭、傝都讀低入聲(daap9)。|
% |頭先、先頭、正話、啱啱、啱先|剛才||
% |嗰陣|的時候||
% |收尾、後尾|後來||
% |今日|今天||
% |聽日|明天|「聽」音ting1。|
% |琴日、禽日、尋日|昨天||
% |琴晚、禽晚、尋晚|昨晚||
% |舊年|去年||
% |出年|明年||
% |今朝、今朝早|今早|「朝」音ziu1。|
% |聽朝、聽朝早|明早|「聽」音ting1;「朝」音ziu1。|
% |遞時、第時、第日、第史、第二時、第二史|換個時間/第二天/改天|本字係「遞」(《廣州語本字》卷二)。「史」字由「歷史」呢個詞借用過嚟,攞佢個時間意義。|

  

% 時份[編輯]

% |   |   |   |   |   |   |   |   |   |   |   |
% |---|---|---|---|---|---|---|---|---|---|---|
% |**粵**|**京**|**註**||**粵**|**京**|**註**||**粵**|**京**|**註**|
% |挨晚|傍晚|||晏晝、下晝|下午|||上晝|上午||
% |晚黑、晚頭黑、夜晚|晚上、夜晚|||挨晚|黃昏|||日頭|白天||
% |朝(早)、朝頭早|早(上)|「朝」音ziu1|||||||||

  

% 日常用語[編輯]

% |   |   |   |
% |---|---|---|
% |**粵**|**京**|**註**|
% |好話、過獎、言重、見笑|謝謝、誇獎了|受人稱讚|
% |唔該|謝謝|受行為之恩|
% |多謝|謝謝|受物件之恩|
% |早晨|早上好、早安||
% |早敨、早唞|晚安|音zou2 tau2,有時第二個字打唔出,就打成「早透」。|
% |再會、請請、後會有期、拜拜|再見||
% |慢慢行|慢走||
% |唔送咯喎|不送了||
% |對唔住、唔好意思|對不起、不好意思||
% |唔緊要、冇所謂、冇相干|沒關係||
% |唔使客氣、乜說話呀、唔使唔該、舉手之勞啫|不用客氣||
% |借過、借歪、借借|讓一下|歪音me2[17]。通常喺前邊加個「唔該」。|

  

% 數詞[編輯]

% |   |   |   |
% |---|---|---|
% |**粵**|**京**|**註**|
% |廿-|二十-|音jaa6。「二十幾」都有人用,但係嚴格嚟講,最正宗嘅係「廿幾」[18]。|
% |卅-|三十-|音saa1,連讀時音調可能會有變化。「三十幾」都有人用,但係嚴格嚟講,最正宗嘅係「卅幾」[18]。|
% |卌-|四十-|音sei3,連讀時可能會變成類似「se-a」嘅發音。「四十幾」都有人用,但係嚴格嚟講,最正宗嘅係「卌幾」[18]。|

  

% 網絡用詞[編輯]

% 從2008年開始,由於中國北京同臺灣高雄先後舉辦過奧運同世運,令網絡資訊越嚟越發達,加上一堆詞都出自北話嘅媒體,令未翻譯嘅新詞入侵粵語,導致相當一部分2000年代同2010年代出世啲細路唔知點用粵語表達。以下係內地同臺灣用嘅部分網上用詞啲粵語表達方式[19]:

% |   |   |   |
% |---|---|---|
% |**粵**|**普通話同國語嘅熱門網絡詞**|**註**|
% |咦咦唈唈|打情罵俏||
% |着草啦,契弟!|奔跑吧,兄弟!|出自浙江衛視《奔跑吧兄弟》同《奔跑吧》系列節目|
% |搵笨/坤水|中國大陸:坑爹<br><br>臺灣:耍花樣||
% |好㜺<br><br>zaan2<br><br>鬼|萌萌噠<br><br>美美哒||
% |有冇|有木有|2011年中國大陸網絡流行詞|
% |暈得一陣陣/頭都大埋|也是醉了|   |
% |O嘴/O嗮嘴|驚呆了|2012年中國大陸網絡流行詞|
% |玩嘢/五條煙|(中國北方)忽悠<br><br>(臺灣)耍花樣||
% |唔衰攞嚟衰|不作死就不会死<br><br>no zuo no die||
% |細路仔唔識世界|图样图森破|「too young」同「too simple」嘅音譯,係江澤民鬧香港記者時講嘅|
% |無啦啦,多噠瘌。|躺着也中枪||
% |大花洒/耕濫嗮|有钱(就是)任性<br><br>有钱就是大爷|2014年4月初喺浙江寧波發生嘅45萬人民幣嘅銀行轉賬詐騙案|
% |搏嗮(老)命|也是蛮拼的||
% |心悒/贔屭|心塞|2014年中國大陸熱門網絡詞|
% |擁躉、Fans、番薯|粉丝|出自2005年湖南衛視嘅《超級女聲》節目|
% |鍾意/中意|么么哒|2014年中國大陸熱門網絡詞|
% |定檔|稳稳的|2015年中國大陸熱門網絡詞|
% |撐場/奸爸爹|打call|日本啲演唱會嘅直播文化嘅「コール」口號|
% |裝彈弓/裝彈氹|套路|競技網遊術語,2016年中國大陸熱門網絡詞。|
% |心照|心里没点逼数|山東話膠東方言片|
% |玩嗮/玩澌、噉都得|还有这种操作<br><br>还带这种操作|2017年中國大陸熱門網絡詞|
% |老屎忽|老司机|競技網遊術語,源自雲南民歌《老司機帶帶我》嗰首。|
% |擔櫈(仔)霸頭位/七加一|吃瓜群众|2016年網絡流行詞,中國大陸網絡或線上論壇覆帖常用語之一。|
% |個心揦<br><br>laa2<br><br>住揦住|蓝瘦香菇|臭青南普方言,出自「難受想哭」嘅諧音。2016年10月9號由廣西南寧人——韋勇喺失戀之後發佈嘅直播短片而流行[20]。|
% |舐腳趾、舐鞋底、舐春袋、舐袋底|跪舔||
% |眼冤/冇眼睇|辣眼睛|2016年4月23號劉梓晨喺美拍上傳嘅短片入面。|
% |苦過Dee Dee|苦逼||
% |廣州:廢柴<br><br>香港:毒撚/毒L/宅男|屌丝|2011年10月由雷霆三巨頭吧啲人整出嚟,源自百度貼吧嘅李毅吧同雷霆三巨頭吧鬧人戰鬥,後嚟喺2012年初逐漸流行成中國大陸熱門網絡詞。|
% |動L|duang/DUANG|出自bilibili網站網友——绯色toy喺2015年2月20號發佈嘅「【成龙】我的洗发液」惡搞短片[21],後嚟俾同站網友「泪腺战士」喺新浪微博轉發之後就猛咁喺網站流行。|
% |好犀利|猴賽雷|出自《2016年央視春晚》廣州演區嘅部分北佬節目表演團隊啲細路口音|
% |批個頭落嚟畀你當凳坐|你们尽管XX了算我输||
% |閒人:你好衰㗎<br><br>大頭:打鑊金|小拳(拳)锤你胸口||
% |重有冇良心㗎/冇心肝|你的良心不会痛吗||
% |串/窒|怼/diss|2017年網絡流行詞,出自英文單詞disrespect嘅翻譯。|
% |果然有錢就大嗮<br><br>有錢想點就點<br><br>有錢真係為所欲為[22]|贫穷限制了我的想象力|2017年網絡流行語句|
% |雞同鴨講|尬聊|出自2016年5月嘅中國大陸兒童題材劇集——《舞法天女》,後嚟通過網絡傳播而大幅度流行[23]。|
% |(形容好崖广)走鬼啊<br><br>(形容好嘢)犀利/冇得彈|666666...|2017年中國大陸網絡流行詞。出自騰訊屬下網遊——《英雄聯盟》嘅網友對話——普通話「溜」嘅諧音字。|
% |No. 1/第一|中國大陸:吃鸡<br><br>臺灣:吃烤雞|源自2017年電影《鬥智21點》嘅台詞——「WINNER WINNER, CHICKEN DINNER!」嘅中文翻譯[24][25]。後嚟喺遊戲《食雞》入面出現咗「WINNER WINNER, CHICKEN DINNER!」嘅詞而開始猛咁流行。|
% |犀飛利|厉害了|2018年3月嘅CCTV-2電影作品《厉害了,我的国》。「利」好多時發音「啤李」個「李」。|
% |笑餐懵/笑餐飽/笑爆嘴/笑爆肚/笑到鬼|23333...|中國大陸網絡流行詞。出自貓撲網論壇表情包嘅第233張相。|
% |叼/仆(街)|次奥|中國大陸嘅北話粗口,喺中國大陸嘅網上常用。|
% |(形容愛情)錫啖<br><br>(形容其它嘢)——|么么哒/摸摸大/摸摸打|2010年中國大陸網絡流行詞。源自2010年湖南衛視嘅《快樂男聲》節目季軍歌手——武藝嘅其一口頭禪。|
% |懵咗|懵圈(儿)|北京話嘅口頭禪|
% |密實姑娘假正經[26]|綠茶婊|中國大陸網語,泛指外貌清純,實質生活糜爛,思想拜金,扮到楚楚可憐,但善於心計,靠出賣肉體上位嘅妙齡女子。該詞出自2013年海南三亞舉辦「海天盛筵」,大批o靚模參加,陪睡3日可得60萬元人民幣報酬,網友因此發明「綠茶婊」一詞加以譏諷,其使用範圍亦已不局限於海天盛筵事件。|
% |衰咗|凉凉/凉了凉了|2017年11月中國大陸新出網絡流行詞。最初出現喺直播間,後嚟好快喺貼吧、朋友圈同微博入面猛咁出現。|
% |熱天落冚大雨|下开水|2017年夏天中國大陸新出網絡流行詞。最先喺廣東省氣象廳發佈嘅高溫同暴雨警報信號。|
% |搏嗮命/瞓嗮身/盡嗮力[27]|洪荒之力|2016年夏季奧運會期間,傅園慧第一次接受央視採訪時出現嘅口頭禪。|
% |玩嗮啦[27]|无FUCK(可)说|源自「無話可說」的方言諧音,個詞出自新浪微博博主——十一月末君喺2017年2月13號喺新浪微博發佈嘅所謂「自帶英文餸名」[28]。|
% |攬埋一齊死[27]/大家一齊死|来啊,互相伤害啊!|2017年秋天新出嘅網絡流行词。最先啲網友用做黃玲喺2007年發佈嘅《痒》呢首歌嘅「來啊 快活啊 反正有大把時間」嘅歌詞嚟猛咁惡搞,改成「來啊 XXX啊」呢啲類似語句品種[29]。|
% |冇嗮符[27]|我能怎么办,我也很绝望啊。<br><br>我能怎么样,我真的很绝望。|源自TVB8播出嘅劇集——《》嘅台詞。|
% |哨牙佬/哨牙仔|龅牙哥|2011年3月喺網絡拍客出現嘅相,後嚟經過PS處理之後出現好多惡搞版。|
% |花靚倞|小奶狗|形容啲好靚仔兼好細個、好得意嘅男仔。個詞出自新浪微博博主——莫里___(福建泉州晉江人)喺2017年12月17號發佈嘅微博文章。|
% |好嘢/好波|中國大陸:skr<br><br>臺灣:絲哥兒|中國大陸2018年網絡流行詞,出自2018年7月16號《中國新說唱》節目入面,吳亦凡嘅口頭禪[30][31]。|
% |淆底|怂|2018年嘅網上流行詞|
% |毒撚|钢铁直男||
% |好緊要,所以要講n次。|重要的事情说三遍||
% |係愛呀哈利|真愛啊/好有愛|出自2018年嘅電影小說——《哈利波特魔法石》入面,喺哈利波特打死怪獸之後出現嘅對話。|
% |好自私/唔生性|巨婴||
% |埋嚟睇、埋嚟揀|這邊有東西出售,請過來看,挑選合心意的。|舊史街邊攤檔,小販、檔主喊叫呢個詞令到路人注意佢嘅商品,體現第一代粵式「應用市場學」,較為硬銷,發展到而家,越嚟越少街邊喊叫推廣,視覺視效軟性手法成熟亦較為大眾所接受。|

% \newpage

% \section{翳我獨無}

% [翳](翳)我獨無

% \newpage

% \section{肉緊}

% 肉緊

% #情緒

% \newpage

% \section{花腰}

% 花腰  [[警察]];

% \newpage

% \section{要}

% [愛](愛)

% \newpage

% \section{話晒}

% [for](for)

% \newpage

% \section{話曬}

% #介詞連詞

% 話曬
% [for](for)

% \newpage

% \section{諗}

% [斟](斟):
% - 你睇睇有冇得點斟?呢個問題都幾難斟。
% - 你 睇々 有冇  斟?󰳞󱟡 問題 󰧶 幾難 斟。

% \newpage

% \section{諗法}

% [諗](諗)法
% [計仔](計仔)
% [idea](idea)
% [睇法](睇法)
% [意見](意見)
% [諗見](諗見)
% [斟](斟)

% [心腸](心腸)
% [心諗](心諗)
% [心機](心機)

% [心斟](心斟)
% [心思](心思)
% [心度](心度) (心踱)
% [心事](心事)

% [心掛](心掛)

% \newpage

% \section{講唔通}

% #metaphysics #logic
% [

% [[邏輯上點都講唔通]]


% [唔通](唔通)

% [理由](理由)

% \newpage

% \section{通}

% #logic 通 : valid

% \newpage

% \section{逞一時之快}

% (Empty file)

% \newpage

% \section{遇時}

% **3.4** [[遇時]]

% 此詞今粵語未見,大槪意思同「經常」:同「[成日](成日)」

% (12)  二成生得兩個仔,臧姑遇時自己贊好命。(39-40)

% (13)  我前日買定一張單刀,放在床頭,遇時預備要用佢,若真來尋打,就先下手為強,免至受虧一著。(370-371)

% \newpage

% \section{邏輯上點都講唔通}

% #logic 

% 邏輯上點都講唔[[通]]

% \newpage

% \section{酸醙}

% 󱔆󱓆
% [[醙]]

% \newpage

% \section{醙}

% 󱓆味

% \newpage

% \section{野}

% #metaphysics #logic
% [野食](野食)
% [野飲](野飲)
% [野做](野做)
% [野講](野講)

% \newpage

% \section{野做}

% #metaphysics

% \newpage

% \section{雷}

% 又叫[義],所謂嘅[雷氣]等於[義氣]  
%      以前嘅差佬重有[兩柴],所以又叫[雷柴]

% \newpage

% \section{飛屎}

% #英源粵詞 #
% 飛屎  [[面子]]  

% \newpage

% \section{食尿甕雞}

% (Empty file)

% \newpage

% \section{香港九約竹枝詞}

% #geography 
% ​許永慶、羅文祥
% **前言:**《香港九約竹枝詞》又稱《圍名歌》,或《新界竹枝詞》。據文獻記載,由清代晚期兩名秀才(私塾老師),居於沙田石古壟的許永慶和火炭九肚村的羅文祥共同創作。書中收集一百五十五首四行七言絕句,描述香港當時景物、名勝、風俗及民情,並且嵌入了新界各村圍的地名,猶如一個世紀前的香港地理誌。二○一四年,該書列入「首份香港非物質文化遺產清單」。

% 鄙人亦好遊山玩水,登高遠足。多年來,造訪了香港許多名勝古跡。有意仿前人用竹枝詞,把瀏覽這些地方的所見所聞所想抒寫出來,傳承下去。以饗讀者。


% [瀝源](https://zh.wikipedia.org/wiki/%E7%80%9D%E6%BA%90 "w:瀝源")洞裡樂從融,人住同安化日中,閒道約中多好景,  
% 拾年一[醮](https://zh.wikipedia.org/wiki/%E5%A4%AA%E5%B9%B3%E6%B8%85%E9%86%AE "w:太平清醮")舞金龍。[大圍](https://zh.wikipedia.org/wiki/%E5%A4%A7%E5%9C%8D%E6%9D%91 "w:大圍村")風景實如何,村裡人居雜姓多,  
% 耕著[田心](https://zh.wikipedia.org/wiki/%E7%94%B0%E5%BF%83%E5%9C%8D "w:田心圍")逢稔歲,社前醉唱太平歌。勝蹟堪傳是[隔田](https://zh.wikipedia.org/wiki/%E9%9A%94%E7%94%B0%E6%9D%91 "w:隔田村"),  
% 峰前[背子石](https://zh.wikipedia.org/wiki/%E6%9C%9B%E5%A4%AB%E7%9F%B3 "w:望夫石")相連,望郎歷盡艱辛苦,志節如同鐵石堅。

% [車公廟](https://zh.wikipedia.org/wiki/%E6%B2%99%E7%94%B0%E8%BB%8A%E5%85%AC%E5%BB%9F "w:沙田車公廟")畔水洋洋,[徑口](https://zh.wikipedia.org/wiki/%E5%BE%91%E5%8F%A3%E6%9D%91 "w:徑口村")林深草木荒,共說神靈多庇佑,  
% 五更鐘聲即焚香。[沙田頭](https://zh.wikipedia.org/wiki/%E6%B2%99%E7%94%B0%E9%A0%AD%E6%9D%91 "w:沙田頭村")又值年豐,[作壆坑](https://zh.wikipedia.org/wiki/%E4%BD%9C%E5%A3%86%E5%9D%91%E6%9D%91 "w:作壆坑村")源水蔭涌,  
% 直對[沙田](https://zh.wikipedia.org/wiki/%E6%B2%99%E7%94%B0_(%E9%A6%99%E6%B8%AF) "w:沙田 (香港)")禾麥熟,家家相慶賦千中。風光鬧熱是[圓洲](https://zh.wikipedia.org/wiki/%E5%9C%93%E6%B4%B2%E8%A7%92%E6%9D%91 "w:圓洲角村"),  
% 客似雲來載滿舟,遙望[插桅杆](https://zh.wikipedia.org/wiki/%E6%8F%92%E6%A1%85%E6%9D%86%E6%9D%91 "w:插桅杆村")插處,小灘[多石](https://zh.wikipedia.org/wiki/%E5%A4%9A%E7%9F%B3%E6%9D%91 "w:多石村")亦無憂。

% [小瀝源](https://zh.wikipedia.org/wiki/%E5%B0%8F%E7%80%9D%E6%BA%90%E6%9D%91 "w:小瀝源村")中水曲灣,龍騰浪下起雲瀾,[牛皮沙](https://zh.wikipedia.org/wiki/%E7%89%9B%E7%9A%AE%E6%B2%99%E6%9D%91 "w:牛皮沙村")住貪風月,  
% [峴咀](https://zh.wikipedia.org/wiki/%E5%B3%B4%E5%92%80%E6%9D%91 "w:峴咀村")徘徊望客還。[南山](https://zh.wikipedia.org/wiki/%E5%8D%97%E5%B1%B1%E6%9D%91_(%E6%B2%99%E7%94%B0) "w:南山村 (沙田)")修竹引蕉風,拂動岩前[石古壟](https://zh.wikipedia.org/wiki/%E7%9F%B3%E5%8F%A4%E5%A3%9F%E6%9D%91 "w:石古壟村"),  
% 修整[大南寮](https://zh.wikipedia.org/wiki/%E5%A4%A7%E8%97%8D%E5%AF%AE%E6%9D%91 "w:大藍寮村")廣閣,[黃宜頭](https://zh.wikipedia.org/wiki/%E9%BB%83%E6%B3%A5%E9%A0%AD%E6%9D%91 "w:黃泥頭村")宿暫停踪。[大輋](https://zh.wikipedia.org/wiki/%E5%A4%A7%E8%BC%8B%E6%9D%91 "w:大輋村")禾麥實婆娑,  
% [崗板](https://zh.wikipedia.org/wiki/%E5%B4%97%E8%83%8C%E6%9D%91 "w:崗背村")田連割最多,頻向[花深坑](https://zh.wikipedia.org/wiki/%E8%8A%B1%E5%BF%83%E5%9D%91%E6%9D%91 "w:花心坑村")處落,牧童[牛凹](https://zh.wikipedia.org/wiki/%E7%89%9B%E5%87%B9 "w:牛凹")唱山歌。

% [觀音山](https://zh.wikipedia.org/wiki/%E8%A7%80%E9%9F%B3%E5%B1%B1%E6%9D%91_(%E6%B2%99%E7%94%B0) "w:觀音山村 (沙田)")上景悠悠,歲歲豐隆[老鼠田](https://zh.wikipedia.org/wiki/%E8%80%81%E9%BC%A0%E7%94%B0%E6%9D%91 "w:老鼠田村"),[茂草岩](https://zh.wikipedia.org/wiki/%E8%8C%82%E8%8D%89%E5%B2%A9%E6%9D%91 "w:茂草岩村")前堪畜牧,  
% [芙蓉別](https://zh.wikipedia.org/wiki/%E8%8A%99%E8%93%89%E5%88%A5%E6%9D%91 "w:芙蓉別村")愛種花園。[白鶴墩](https://zh.wikipedia.org/wiki/%E7%99%BD%E9%B6%B4%E6%B1%80%E6%9D%91 "w:白鶴汀村")前晒羽毛,飛來[橫壆](https://zh.wikipedia.org/wiki/%E6%A9%AB%E5%A3%86 "w:橫壆")看流潮,  
% [銅鑼](https://zh.wikipedia.org/wiki/%E9%8A%85%E9%91%BC%E7%81%A3_(%E6%B2%99%E7%94%B0) "w:銅鑼灣 (沙田)")海外流氓客,[山下](https://zh.wikipedia.org/wiki/%E6%9B%BE%E5%A4%A7%E5%B1%8B "w:曾大屋")[禾輋](https://zh.wikipedia.org/wiki/%E7%A6%BE%E8%BC%8B "w:禾輋")夜氣涼。[麵房](https://zh.wikipedia.org/wiki/%E9%BA%B5%E6%88%BF%E6%9D%91 "w:麵房村")[火炭](https://zh.wikipedia.org/wiki/%E7%81%AB%E7%82%AD%E6%9D%91 "w:火炭村")自相連,  
% [龜地](https://zh.wikipedia.org/wiki/%E6%A1%82%E5%9C%B0%E6%96%B0%E6%9D%91 "w:桂地新村")形成極自然,尚有船枚[落路下](https://zh.wikipedia.org/wiki/%E8%90%BD%E8%B7%AF%E4%B8%8B%E6%9D%91 "w:落路下村"),擔魚男女不停肩。

% 河東[河瀝背](https://zh.wikipedia.org/wiki/%E6%B2%B3%E7%80%9D%E8%83%8C%E6%9D%91 "w:河瀝背村")相通,雲出寒山石徑風,[黃竹揚](https://zh.wikipedia.org/wiki/%E9%BB%83%E7%AB%B9%E6%B4%8B%E6%9D%91 "w:黃竹洋村")風[山尾](https://zh.wikipedia.org/wiki/%E5%B1%B1%E5%B0%BE%E6%9D%91 "w:山尾村")過,  
% [石寮](https://zh.wikipedia.org/wiki/%E7%9F%B3%E6%A6%B4%E6%B4%9E%E6%9D%91 "w:石榴洞村")[凹背](https://zh.wikipedia.org/wiki/%E5%9D%B3%E8%83%8C%E7%81%A3%E6%9D%91 "w:坳背灣村")路西東。[龍窩](https://zh.wikipedia.org/wiki/%E4%B9%9D%E8%82%9A%E6%9D%91 "w:九肚村")水曲更週全,戶外煙霞似錦田,  
% [樟樹灘](https://zh.wikipedia.org/wiki/%E6%A8%9F%E6%A8%B9%E7%81%98%E6%9D%91 "w:樟樹灘村")前魚艇滿,生涯渡活樂餘年。[大陂尾](https://zh.wikipedia.org/wiki/%E5%A4%A7%E5%9F%94%E5%B0%BE%E6%9D%91 "w:大埔尾村")居富足全,  
% [赤泥坪](https://zh.wikipedia.org/wiki/%E8%B5%A4%E6%B3%A5%E5%9D%AA%E6%9D%91 "w:赤泥坪村")住亦安然,[長瀝尾](https://zh.wikipedia.org/wiki/%E9%95%B7%E7%80%9D%E5%B0%BE%E6%9D%91 "w:長瀝尾村")人[馬料水](https://zh.wikipedia.org/wiki/%E9%A6%AC%E6%96%99%E6%B0%B4 "w:馬料水"),望窮山外有人煙。

% [梅子林](https://zh.wikipedia.org/wiki/%E6%A2%85%E5%AD%90%E6%9E%97%E6%9D%91 "w:梅子林村")間再一行,遊遊不覺到[茅坪](https://zh.wikipedia.org/wiki/%E8%8C%85%E5%9D%AA "w:茅坪"),迴環聳翠多奇景,  
% [黃竹山](https://zh.wikipedia.org/wiki/%E9%BB%83%E7%AB%B9%E5%B1%B1 "w:黃竹山")高不必驚。[石壟仔](https://zh.wikipedia.org/wiki/%E7%9F%B3%E5%A3%9F%E4%BB%94 "w:石壟仔")是定行踪,直上[昂坪](https://zh.wikipedia.org/wiki/%E6%98%82%E5%9D%AA_(%E9%A6%AC%E9%9E%8D%E5%B1%B1) "w:昂坪 (馬鞍山)")捉地龍,  
% 回首[馬鞍山](https://zh.wikipedia.org/wiki/%E9%A6%AC%E9%9E%8D%E5%B1%B1_(%E9%A6%99%E6%B8%AF) "w:馬鞍山 (香港)")頂望,居然人在[廣寒宮](https://zh.wikipedia.org/wiki/%E5%BB%A3%E5%AF%92%E5%AE%AE "w:廣寒宮")。[烏溪沙](https://zh.wikipedia.org/wiki/%E7%83%8F%E6%BA%AA%E6%B2%99 "w:烏溪沙")上景盤桓,  
% [大水坑](https://zh.wikipedia.org/wiki/%E5%A4%A7%E6%B0%B4%E5%9D%91%E6%9D%91 "w:大水坑村")源水一灣,[輋下](https://zh.wikipedia.org/wiki/%E8%BC%8B%E4%B8%8B "w:輋下")[泥涌](https://zh.wikipedia.org/wiki/%E6%B3%A5%E6%B6%8C "w:泥涌")通路過,春來[樟木](https://zh.wikipedia.org/wiki/%E6%A8%9F%E6%9C%A8%E9%A0%AD_(%E6%B2%99%E7%94%B0) "w:樟木頭 (沙田)")茂如山。

% 落葉人居[大洞](https://zh.wikipedia.org/wiki/%E5%A4%A7%E6%B4%9E "w:大洞")邊,耕田討海亦皆能,遨遊偶向[企嶺下](https://zh.wikipedia.org/wiki/%E4%BC%81%E5%B6%BA%E4%B8%8B "w:企嶺下"),  
% [浪徑](https://zh.wikipedia.org/wiki/%E6%B5%AA%E5%BE%91 "w:浪徑")[深涌](https://zh.wikipedia.org/wiki/%E6%B7%B1%E6%B6%8C "w:深涌")一帶連。[山寮](https://zh.wikipedia.org/wiki/%E5%B1%B1%E5%AF%AE "w:山寮")聯絡[大灣](https://zh.wikipedia.org/wiki/%E5%A4%A7%E7%92%B0_(%E8%A5%BF%E8%B2%A2) "w:大環 (西貢)")村,[大網](https://zh.wikipedia.org/wiki/%E5%A4%A7%E7%B6%B2%E4%BB%94 "w:大網仔")[坪墩](https://zh.wikipedia.org/wiki/%E5%9D%AA%E5%A2%A9 "w:坪墩")幾路分,  
% [沙角尾](https://zh.wikipedia.org/wiki/%E6%B2%99%E8%A7%92%E5%B0%BE "w:沙角尾")居人物眾,[南山](https://zh.wikipedia.org/wiki/%E5%8D%97%E5%B1%B1_(%E8%A5%BF%E8%B2%A2) "w:南山 (西貢)")昂我路分分。春遊不覺[荔枝莊](https://zh.wikipedia.org/wiki/%E8%8D%94%E6%9E%9D%E8%8E%8A "w:荔枝莊"),  
% 並及[井頭](https://zh.wikipedia.org/wiki/%E4%BA%95%E9%A0%AD "w:井頭")各小鄉,海面生涯[榕樹澳](https://zh.wikipedia.org/wiki/%E6%A6%95%E6%A8%B9%E6%BE%B3 "w:榕樹澳"),[白沙灣](https://zh.wikipedia.org/wiki/%E7%99%BD%E6%B2%99%E7%81%A3_(%E8%A5%BF%E8%B2%A2) "w:白沙灣 (西貢)")是打魚郎。

% [北廣](https://zh.wikipedia.org/wiki/%E5%8C%97%E6%B8%AF "w:北港")[澳投](https://zh.wikipedia.org/wiki/%E6%BE%B3%E9%A0%AD "w:澳頭")[西貢](https://zh.wikipedia.org/wiki/%E8%A5%BF%E8%B2%A2%E5%B8%82 "w:西貢市")街,生成[大埗](https://zh.wikipedia.org/wiki/%E5%A4%A7%E5%9F%97 "w:大埗")子無猜,[鹽田仔](https://zh.wikipedia.org/wiki/%E9%B9%BD%E7%94%B0%E4%BB%94 "w:鹽田仔")細堪遊玩,  
% 到處雄雞一樣啼,停駿[窩尾](https://zh.wikipedia.org/wiki/%E7%AA%A9%E5%B0%BE "w:窩尾")抵[蠔涌](https://zh.wikipedia.org/wiki/%E8%A0%94%E6%B6%8C "w:蠔涌"),大佬神前廟[響鐘](https://zh.wikipedia.org/wiki/%E9%9F%BF%E9%90%98 "w:響鐘"),  
% 北廣[北圍](https://zh.wikipedia.org/wiki/%E5%8C%97%E5%9C%8D "w:北圍")聯絡盡,船枚西貢樂春分。[斬竹灣](https://zh.wikipedia.org/wiki/%E6%96%AC%E7%AB%B9%E7%81%A3 "w:斬竹灣")灣作箭弓,  
% [黃毛鷹](https://zh.wikipedia.org/wiki/%E9%BB%83%E6%AF%9B%E9%B7%B9 "w:黃毛鷹")子走無蹤,遙指[石坑](https://zh.wikipedia.org/wiki/%E7%9F%B3%E5%9D%91 "w:石坑")和[赤徑](https://zh.wikipedia.org/wiki/%E8%B5%A4%E5%BE%91 "w:赤徑"),[疍家灣](https://zh.wikipedia.org/wiki/%E7%96%8D%E5%AE%B6%E7%81%A3 "w:疍家灣")是攞魚公。

% [滘西](https://zh.wikipedia.org/wiki/%E6%BB%98%E8%A5%BF%E6%B4%B2 "w:滘西洲")海隔[打蠔墩](https://zh.wikipedia.org/wiki/%E6%89%93%E8%A0%94%E5%A2%A9 "w:打蠔墩"),[黃竹洋](https://zh.wikipedia.org/wiki/%E9%BB%83%E7%AB%B9%E6%B4%8B_(%E8%A5%BF%E8%B2%A2) "w:黃竹洋 (西貢)")來日已昏,借問行人何處宿,  
% [糧船灣](https://zh.wikipedia.org/wiki/%E7%B3%A7%E8%88%B9%E7%81%A3%E6%B4%B2 "w:糧船灣洲")泊月如銀。春遊忽到[蘇茅坪](https://zh.wikipedia.org/wiki/%E7%A7%80%E8%8C%82%E5%9D%AA "w:秀茂坪"),看見[牛頭角](https://zh.wikipedia.org/wiki/%E7%89%9B%E9%A0%AD%E8%A7%92 "w:牛頭角")又生,  
% [茜草灣](https://zh.wikipedia.org/wiki/%E8%8C%9C%E8%8D%89%E7%81%A3 "w:茜草灣")前多石匠,山歌嘹亮幾重山。[魷魚灣](https://zh.wikipedia.org/wiki/%E9%AD%B7%E9%AD%9A%E7%81%A3 "w:魷魚灣")起波中錦,  
% 海面生涯莫[浪下](https://zh.wikipedia.org/wiki/%E6%B5%AA%E4%B8%8B "w:浪下"),[大埗仔](https://zh.wikipedia.org/wiki/%E5%A4%A7%E5%9F%94%E4%BB%94 "w:大埔仔")是定行藏,嬌女牽情飾冶粧。

% [竹角](https://zh.wikipedia.org/wiki/%E7%AB%B9%E8%A7%92 "w:竹角")[南圍](https://zh.wikipedia.org/wiki/%E5%8D%97%E5%9C%8D "w:南圍")人壯勇,圓墩多是打魚郎,[坑口](https://zh.wikipedia.org/wiki/%E5%9D%91%E5%8F%A3 "w:坑口")郎灣一帶連,  
% 疍家歌唱夕陽天。何人來接[孟公屋](https://zh.wikipedia.org/wiki/%E5%AD%9F%E5%85%AC%E5%B1%8B "w:孟公屋"),風物人倫亦偉然,  
% 上陽行過下陽來,偶過釣魚公上台。借問大灣頭裡過,  
% 大坑口亦可徘徊,[白沙澳](https://zh.wikipedia.org/wiki/%E7%99%BD%E6%B2%99%E6%BE%B3 "w:白沙澳")向海下來,即見[高塘](https://zh.wikipedia.org/wiki/%E9%AB%98%E5%A1%98 "w:高塘")大浪吹。

% 唯看[大灘](https://zh.wikipedia.org/wiki/%E5%A4%A7%E7%81%98 "w:大灘")須[障上](https://zh.wikipedia.org/wiki/%E5%B6%82%E4%B8%8A "w:嶂上"),木頭舟渡[北潭](https://zh.wikipedia.org/wiki/%E5%8C%97%E6%BD%AD%E6%B6%8C "w:北潭涌")開,去年曾到[鰂魚湖](https://zh.wikipedia.org/wiki/%E9%B0%82%E9%AD%9A%E6%B9%96 "w:鰂魚湖"),  
% 上下窰統共一途,行向[爛泥灣](https://zh.wikipedia.org/wiki/%E7%88%9B%E6%B3%A5%E7%81%A3 "w:爛泥灣")裡過,[黃麖](https://zh.wikipedia.org/wiki/%E9%BB%83%E9%BA%96%E5%9C%B0 "w:黃麖地")子應走喪狐,  
% [井欄樹](https://zh.wikipedia.org/wiki/%E4%BA%95%E6%AC%84%E6%A8%B9 "w:井欄樹")茂發奇香,祝屋平分上下鄉。遙望[澳頭](https://zh.wikipedia.org/wiki/%E6%BE%B3%E9%A0%AD "w:澳頭")通路旁,  
% 山行時過[馬油塘](https://zh.wikipedia.org/wiki/%E9%A6%AC%E6%B2%B9%E5%A1%98 "w:馬油塘"),耕種人居塘面家,田寮下至大芒輋。

% 湖洋灣出皆新屋,大小庵山向日斜,去年既至[相思灣](https://zh.wikipedia.org/wiki/%E7%9B%B8%E6%80%9D%E7%81%A3 "w:相思灣"),  
% [大澳](https://zh.wikipedia.org/wiki/%E5%A4%A7%E6%BE%B3 "w:大澳")茅埗屋數間。田下蕉窩逢客問,更遊[布斗澳](https://zh.wikipedia.org/wiki/%E5%B8%83%E8%A2%8B%E6%BE%B3 "w:布袋澳")方還,  
% [茶菓嶺](https://zh.wikipedia.org/wiki/%E8%8C%B6%E6%9E%9C%E5%B6%BA "w:茶果嶺")中石匠留,順風時過[昂船洲](https://zh.wikipedia.org/wiki/%E6%98%82%E8%88%B9%E6%B4%B2 "w:昂船洲")。文明欽羡[九華徑](https://zh.wikipedia.org/wiki/%E4%B9%9D%E8%8F%AF%E5%BE%91 "w:九華徑"),  
% 借問[葵涌](https://zh.wikipedia.org/wiki/%E8%91%B5%E6%B6%8C "w:葵涌")歷幾秋,[荃灣](https://zh.wikipedia.org/wiki/%E8%8D%83%E7%81%A3 "w:荃灣")菓木出菠蘿,遙望[青衣](https://zh.wikipedia.org/wiki/%E9%9D%92%E8%A1%A3%E5%B3%B6 "w:青衣島")隔海河。

% [三棟屋](https://zh.wikipedia.org/wiki/%E4%B8%89%E6%A3%9F%E5%B1%8B%E6%9D%91 "w:三棟屋村")前風自古,[老圍](https://zh.wikipedia.org/wiki/%E8%80%81%E5%9C%8D_(%E8%8D%83%E7%81%A3) "w:老圍 (荃灣)")人係姓尤多,直上城門聳一峰,  
% 陂頭肚亦在其中。唯有藍房相隔遠,[八鄉](https://zh.wikipedia.org/wiki/%E5%85%AB%E9%84%89 "w:八鄉")路界竟然同,  
% [大埔](https://zh.wikipedia.org/wiki/%E5%A4%A7%E5%9F%94_(%E9%A6%99%E6%B8%AF) "w:大埔 (香港)")南坑並下坑,碗窰燒料出江城。山豬遊到圓墩下,  
% 寄語年頭莫放生,白牛山上視眈眈,水窩留連到大庵。

% 大坑流水注新塘,行下埗前掮客忙,唯有社山和寨乪,  
% 黃蜂寨塞更平常。黃泥澳過路匆匆,子燕岩邊掠泮涌,  
% 揭起[紗羅洞](https://zh.wikipedia.org/wiki/%E6%B2%99%E7%BE%85%E6%B4%9E "w:沙羅洞")裡帳,[汀角](https://zh.wikipedia.org/wiki/%E6%B1%80%E8%A7%92 "w:汀角")[船灣](https://zh.wikipedia.org/wiki/%E8%88%B9%E7%81%A3 "w:船灣")一帶通。[打鐵印](https://zh.wikipedia.org/wiki/%E6%89%93%E9%90%B5%E5%B1%BB "w:打鐵屻")知[蓮澳](https://zh.wikipedia.org/wiki/%E8%93%AE%E6%BE%B3 "w:蓮澳")近,  
% [石湖墟](https://zh.wikipedia.org/wiki/%E7%9F%B3%E6%B9%96%E5%A2%9F "w:石湖墟")上樂融融,為道三聲籮穀貴,[烏蛟騰](https://zh.wikipedia.org/wiki/%E7%83%8F%E8%9B%9F%E9%A8%B0 "w:烏蛟騰")處起雲風。

% 一帆風送到蔭涌,橫嶺頭高小徑通,聞名金竹排古廟,  
% 涌尾停舟瀝晚風。[長沙灣](https://zh.wikipedia.org/wiki/%E9%95%B7%E6%B2%99%E7%81%A3 "w:長沙灣")出[九龍塘](https://zh.wikipedia.org/wiki/%E4%B9%9D%E9%BE%8D%E5%A1%98 "w:九龍塘"),[深水埗](https://zh.wikipedia.org/wiki/%E6%B7%B1%E6%B0%B4%E5%9F%97 "w:深水埗")前過客商,  
% 新摘荔枝[蔴地](https://zh.wikipedia.org/wiki/%E6%B2%B9%E9%BA%BB%E5%9C%B0 "w:油麻地")賣,回頭[旺角](https://zh.wikipedia.org/wiki/%E6%97%BA%E8%A7%92 "w:旺角")一銀莊。[尖沙嘴](https://zh.wikipedia.org/wiki/%E5%B0%96%E6%B2%99%E5%98%B4 "w:尖沙嘴")盡屯胡兵,  
% 輪渡往向日不停,[紅磡](https://zh.wikipedia.org/wiki/%E7%B4%85%E7%A3%A1 "w:紅磡")[土瓜環](https://zh.wikipedia.org/wiki/%E5%9C%9F%E7%93%9C%E7%81%A3 "w:土瓜灣")在望,船塢動用幾多人。

% 譚公廟近九龍街,官地衙前不必猜,三十七年光緒主。  
% 紅毛轇輵掛門牌。[馬頭涌](https://zh.wikipedia.org/wiki/%E9%A6%AC%E9%A0%AD%E6%B6%8C "w:馬頭涌")過[宋皇臺](https://zh.wikipedia.org/wiki/%E5%AE%8B%E7%8E%8B%E8%87%BA "w:宋王臺"),鶴佬村前玩一回,  
% 行向沙埗醫院過,疑魂[打鼓嶺](https://zh.wikipedia.org/wiki/%E6%89%93%E9%BC%93%E5%B6%BA "w:打鼓嶺")中催。[牛池灣](https://zh.wikipedia.org/wiki/%E7%89%9B%E6%B1%A0%E7%81%A3 "w:牛池灣")聽牧童歌,  
% [沙地園](https://zh.wikipedia.org/wiki/%E6%B2%99%E5%9C%B0%E5%9C%92%E9%84%89 "w:沙地園鄉")堪種菜蔬。愛吃沙梨圓嶺進,[蒲崗](https://zh.wikipedia.org/wiki/%E8%92%B2%E5%B4%97 "w:蒲崗")荔枝實婆娑。

% 爭傳香港景繁華,巨艦通洋各埠家,[上](https://zh.wikipedia.org/wiki/%E4%B8%8A%E7%92%B0 "w:上環")[下](https://zh.wikipedia.org/wiki/%E7%81%A3%E4%BB%94 "w:灣仔")[中環](https://zh.wikipedia.org/wiki/%E4%B8%AD%E7%92%B0 "w:中環")多貨值,  
% 寄情式式錦添花。[太平山](https://zh.wikipedia.org/wiki/%E5%A4%AA%E5%B9%B3%E5%B1%B1_(%E9%A6%99%E6%B8%AF) "w:太平山 (香港)")透[西營盤](https://zh.wikipedia.org/wiki/%E8%A5%BF%E7%87%9F%E7%9B%A4 "w:西營盤"),妓女青樓數百間,  
% 紅袖添香[文武廟](https://zh.wikipedia.org/wiki/%E6%9D%B1%E8%8F%AF%E4%B8%89%E9%99%A2%E6%96%87%E6%AD%A6%E5%BB%9F "w:東華三院文武廟"),梳粧將樣巧雲鬢。打鐘樓出大王街,  
% 多少工人歇杉排,[跑馬場](https://zh.wikipedia.org/wiki/%E8%B7%91%E9%A6%AC%E5%9C%B0 "w:跑馬地")中堪一玩,回頭[灣仔](https://zh.wikipedia.org/wiki/%E7%81%A3%E4%BB%94 "w:灣仔")敘幽懷。

% [燈籠洲](https://zh.wikipedia.org/wiki/%E5%A5%87%E5%8A%9B%E5%B3%B6 "w:奇力島")越[鰂魚涌](https://zh.wikipedia.org/wiki/%E9%B0%82%E9%AD%9A%E6%B6%8C "w:鰂魚涌"),[七姊妹](https://zh.wikipedia.org/wiki/%E4%B8%83%E5%A7%8A%E5%A6%B9_(%E9%A6%99%E6%B8%AF) "w:七姊妹 (香港)")來謁廟中,丹桂紅香爐內插,  
% 鞠躬禮拜一雙雙。[筲箕灣](https://zh.wikipedia.org/wiki/%E7%AD%B2%E7%AE%95%E7%81%A3 "w:筲箕灣")裡實如何,艇女生涯夜半歌,  
% 稅廠斧頭洲外設,[鯉魚門](https://zh.wikipedia.org/wiki/%E9%AF%89%E9%AD%9A%E9%96%80 "w:鯉魚門")口過船多。揚帆直指[大嶼山](https://zh.wikipedia.org/wiki/%E5%A4%A7%E5%B6%BC%E5%B1%B1 "w:大嶼山"),  
% 便至[長洲](https://zh.wikipedia.org/wiki/%E9%95%B7%E6%B4%B2_(%E9%A6%99%E6%B8%AF) "w:長洲 (香港)")訪玉顏,遙望[薄扶林](https://zh.wikipedia.org/wiki/%E8%96%84%E6%89%B6%E6%9E%97 "w:薄扶林")突屹,頻頻女女更為難。

% \newpage

% \section{香港江湖術語}

% #文 
% [[之]]  一;又叫[豬],所謂嘅[豬粒]即係[一粒花]  
%      以前話食咗人隻[豬],即系攞咗人哋第一次,有[破處]嘅意思  
% [[雷]]  二;又叫[義],所謂嘅[雷氣]等於[義氣]  
%      以前嘅差佬重有[兩柴],所以又叫[雷柴]  
% 汪  三;又叫[戇]  
%      以前嘅三柴,又叫做[戇柴]或者[戇鳩柴]  
% 扇  四  
% 乍  五  
% 龍  六  
% 吉  七  
% 絲  八  
% 色  九  
  
% **———————————————————————————————  
  
% 女  原意係解〔毒品〕特別係茄粉,依家同溝女意思一樣  
%    條女;女朋友  
% 仔  條仔;男朋友  
% 皮  皮或被,讀[丕],意思係[衫]  
% 朵  信件;  
% 灰  毒品[海洛英],又叫[黑米]  
% 扯  吸;以前話〔扯翻口〕,即係[吸翻啖]  
% 罕  藥;以前話[落罕],即係話[落藥]  
% 車  警探;以前叫[車頭],即係[探頭]或者[探長]  
% 抹  判案;  
% 狗  槍;  
% 茄  氯胺酮;俗稱:小姐、K仔、K粉、茄粉  
% 青  兩解:1)刀  
%       2)指其他女人嘅丈夫  
% 砂  米/飯;耕砂,即係[食飯]  
% 砌  兩解:1)普通用嚟解[毆打]   
%       2)情欲用嚟解[做愛]  
% 草  大麻;又叫[火麻],咁[火麻仁]即係......  
% 骨  門;骨場  
% 格  屋;  
% [[眧]]  讀[超];睇;眼鏡;以前好興講一句[眼眧眧,唔順眧]  
% 粉  白粉[海洛英]  
% 針  線人;穿針引線嘅人  
% 馬  手下〔人〕;又叫[馬仔]  
% [[流]]  係[[[假]]]嘅意思  
% [[堅]]  係[[[真]]]嘅意思  
% 盔  帽;  
% 蛋  手錶  
% 雀  香煙  
% 撇  離開,走人〔逃走〕;其他重有:閃、鬆、散......  
% 漒  煙;煙仔  
% 輪  電話;以前話[撥個輪],即係[打電話]  
% 閹  已成為社團成員  
% 鴨  男妓;又叫[舞男]  
% 糖  丸仔;一般指苯二氮類精神藥物,包括:巴比通、甲哇酮  
% 櫃  肛門;以前電影話啲監犯入倉前要先[通櫃],即係......  
% 雞  妓女;又叫[舞女]  
% 瓣  社團;以前嘅人問[你邊瓣呀〕,即係問你屬於邊一個幫會  
% 飄  船;  
% 疊  多;疊水,疊友,疊馬[好多手下]  
% 曬  瞓覺[睡覺]  
% 𡃁  手下;佢嘅量詞係[條]  
  
% **———————————————————————————————  
  
% BM  原指Body Message,依家同[起機]一樣意思  
% NC  Night Club,一般指色情夜總會  
% PR  公關小姐〔夜總會〕,舊時叫做[舞女]  
% 一碌  一年;  
% 一簡  犯案一次;而[第一簡]係指第一次入入冊  
% 入冊  坐監〔入獄〕  
% 出冊  放監[出獄]  
% 上馬  開香堂收門生  
  
% 大瓦  被;  
% 大圈  廣州;以前來自廣州洪門嘅人,就叫做[大圈〕  
% 大菜  牛;  
% 大圍  阿公;大圍事,即係阿公嘅事,又即係大家嘅事  
% 大爺  被騙嘅對象[老千門專用〕  
% 大檔  非法賭場  
  
% 小姐  夜總會舞小姐,後來又叫[女公關]或[公關小姐]  
% 公海  以前係指[油尖旺+銅鑼灣]  
% 天牌  老竇〔父親〕  
% 地牌  老母〔母親〕  
% 地頭  地盤  
% 孔明  燈;指路  
% 文雀  掱手[扒手],俗稱[三隻手]  
% 文膽  白紙扇;  
% 毛瓜  豬;  
% 毛詩  利是;  
% 火柴  金枝;  
% 北姑  大陸來港嘅夜總會舞小姐〔妓女〕  
% 企街  企喺街度招攬客仔嘅小姐〔妓女〕  
% 全套  按摩+做愛[按摩場用語]  
% 吉佬  女人;  
% [[吉屎]]  勇氣〔Guts.英文翻譯〕  
% 收山  上岸〔退休〕  
% 拜山  探監;  
% 受靶  坐監;  
% 羊牯  非社團人物,又同[老襯]  
% 老芝  芝麻灣監獄  
% 老旺  MK〔旺角〕  
% 老朋  普通非江湖上嘅朋友  
% 老表  同門手足〔江湖上嘅結拜兄弟〕  
% 老狀  律師;  
% 老記  記者;  
% 老域  域多利收押所  
% 老強  強姦;  
% 老笠  打劫〔搶劫〕  
% 老童  道友;  
% 老道  吸毒者  
% 老爆  爆格〔爆竊〕  
% 老襯  非社團人物  
% 列印  佔有該女子  
% 扯粉  吸毒;  
% 扯漒  吸煙;  
% 杜女  用毒品或者藥物來迷魂啲女人,方便同佢去開房  
% 走粉  運送白粉,依守泛指運送所有毒品  
% 走數  兩解:1)貴利〔高利貸〕指債仔欠債唔還   
%        2)社團用嚟解[欠社團錢]  
% 阿公  又解社團,但多數喺同門吹水用,例如〔幫阿公做事〕  
% 阿頂  龍頭大佬  
% 陀地  原指保護費,依家係指本土出身嘅悪霸〔本地〕  
% 冼超  四眼仔  
% 叔父  黑社會大佬輩人士  
% 拆家  分銷毒品嘅人  
% 放蛇  兩解:1)警察扮客仔截查同捉人  
%        2)夜晚將蛇放入房或者客廳  
% 放數  放貴利〔高利貸〕  
% 狗咬  槍傷;畀狗咬,即係受槍傷  
% 肥妹  肥佬嘅花名  
% 架生  原指武器,依家亦指搵食工具  
% 架步  地方  
% 架兩  和事佬;話人[做架兩],即係話人[多事]  
% [[皇氣]]  有[[警察]]  
% 祠堂  赤柱監獄  
% [[飛屎]]  [[面子]]  
% 骨妹  從事〔邪骨〕嘅按摩女郎  
% 骨場  按摩桑拿浴室,依家主要係指色情按摩桑拿浴室  
% 正骨  正式按摩  
% 邪骨  按摩+提供性服務  
% 起機  按摩後提供打飛機〔手淫〕,大陸叫〔推油〕  
% [[花腰]]  [[警察]];  
% [[白鮓]]  交通[[警察]]  
% 黑腳  軍裝警察  
% 柳記  獄警〔懲教署〕  
% 柳條  鎖匙  
% 孖葉  手銬;  
% [[魁斗]]  即系[[[鬼佬]]]或[[[鬼頭]]]  
% 格屎  社團單位,量詞係〔瓣〕  
% 海底  會員名冊  
% 班蓮  飲茶  
% 粉檔  非法販賣同吸食白粉嘅地方  
% 耕罕  食藥  
% 耕砂  食飯;又人寫[間沙]  
% 臭格  警署拘留所  
% 莫財  無錢  
% 莫薑  無膽  
% 隻抽  一個打一個〔單挑〕  
% 桂枝  香港  
% 馬交  澳門  
% 馬伕  負責照顧啲妓女同接生意嘅男人  
% 馬檻  色情架步〔妓寨〕  
% 高買  竊取店鋪貨物  
% 索茄  吸氯胺酮  
% 啤灰  食白粉  
% 啪丸  食丸仔  
% 啪針  注射毒品  
% 帶家  運送毒品嘅人  
% 斜牌  出賣色相嘅女人  
% 著草  走路〔犯罪後走去第二處〕  
% 過江  渡海  
% 隊㨆  指[㨆友]嘅行為  
% 麻希  少;  
% 淋油  兩解:1)夜晚喺門前淋電油;  
%        2)喺門前淋油漆  
% 單拖  單人匹馬  
% 圍抽  圍毆  
% 報串  報案  
% 揩草  食大麻  
% 揩粉  食白粉  
% 散水  逃走  
% 無雷  無義氣  
% 睇水  視察有冇警察巡邏  
% 跑山  欠人錢或等錢洗,四處籌錢  
% 跑街  接客〔妓女〕  
% 跑鐘  接客〔應召妓女〕  
% 開房  到時鐘酒店〔做愛〕  
% 開片  劈友〔械鬥/打架〕  
% 開拖  打交〔打起來〕  
% 還拖  還手〔打鬥〕  
% 溝女  追女仔  
% 搣灰  戒白粉  
% 㨆友  殺人;[㨆,東漢《方言》:殺也。今關西人呼打為㨆。]  
% 較腳  逃走,走人  
% 雷氣  義氣;因為雷掛第二,二義同音  
% 飯堂  非法販賣同吸食白粉嘅地方  
% 摣水  痾尿〔小便〕  
% 摣扒  握手,社團講數儀式一種,依家已經好少用  
% 摣數  管數人,通常都係由文膽兼任  
% 墨七  小偷[鼠摸]  
% 墨漆  衣盜  
% 金蛋  金錶  
% 黃指  金戒指  
% 黃圈  金手鈪〔金鐲〕  
% 幹張  紙  
% 線超  眼鏡  
% 蓮花  碗  
% 耍花  筷子  
% 錨花  匙羹  
% 踩街  鞋  
% 橫角  褲  
% 底橫  內褲  
% 拖水  毛巾  

% 靚姑  花枝招展嘅老年醜婦

% 數簿  社團帳簿  
% 輪古  賭輸錢  
% 薄頭  浮頭〔再行露面〕     
% 賴嘢  失手  
% 青蓮  茶葉  
% 擺尾  魚  
% 鵝毛  扇  
% 環頭  警區  
% 講數  談判  
% 點相  認人  
% 擺杯  擺杯陣;社團講數儀式一種,依家已經好少用  
% 擺柳  痾尿〔小便〕,又叫[摣水]  
% 擺堆  痾屎〔大便〕  
% 擺橫  食鴉片,舊時叫[福壽膏]  
% 擺錫  丁雨;落雨  
% 覆綽  返轉頭  
% 爆房  有時等於開房,有時等於〔召妓〕  
% 爆勇  報串  
% 爆缸  流血  
% 爆骨  開門  
% 瓣數  社團  
% 響朵  講出自己所屬嘅社團  
  
% **———————————————————————————————  
  
% 一蚊雞  一元〔一蚊〕  
% 一草嘢  十元〔青蟹〕  
% 一斤嘢  一百元〔一舊水〕,又叫〔紅衫魚〕  
% 一猜嘢  一千元〔一叉水〕,又叫〔一撇水〕或〔一戙水〕  
% 一餅嘢  一萬元〔一盤水〕,又叫〔一皮嘢〕或〔一雞嘢〕  
% 一球嘢  一百萬〔一百個〕  
% 二五仔  社團內奸,又叫〔二頭蛇〕或[兩頭蛇]  
% 四一五  白紙扇〔十底〕負責文職,講數,通常亦負責社團財務管理  
% 四二六  紅棍〔十二底〕金牌打手,最出衆嘅就叫〔雙花紅棍〕  
% 四三二  草鞋〔九底〕對內外事務聯系  
% 四三八  二路元帥〔十五底〕  
% 四八九  龍頭,坐館〔爲社團最高領導人〕龍頭大佬  
% 四九仔  普通社團會員  
% 九二友  好少人就進行械鬥  

% 老四九  入咗社團三年升唔到做[大底]嘅普通社團會員;

% 大底:即係紅棍、草鞋或者白紙扇  
% 打八爪  打指模〔蓋手指模〕  
% 安家費  撫恤金  

% 摣𢞵人  話事人〔社團實行區頭制嘅制度〕

% 天文臺  負責睇水嘅人  
% 大口環  形容一啲貌似智商偏低嘅人  
% 大圈仔  大陸人;以前嚟自廣州洪門嘅,就叫做[大圈〕  
% 大祠堂  赤柱監獄;依家泛指所有監獄  
% 大耳窿  放貴利〔高利貸〕嘅人   
%      傳說話喺開埠初年,大耳窿主要係向啲碼頭咕哩[苦力]放數;  
%      為咗方便辨認,佢哋就將個銀仔塞喺耳窿度,暗示有錢借;  
%      久而久之,就將個耳窿都塞大咗  
% 牛屎飛  新界鄉村出身嘅社團人士  
% 蠱惑仔  黑社會人士  
% 出嚟行  撈偏,又叫行蠱惑〔行走江湖〕  
% 玩波仔  食紅丸  
% 金手指  警方線人;又叫[二五仔];專[椓背脊]  
% 起飛腳  背叛兄弟或者社團  
% 起雙飛  一皇雙后〔同時叫兩隻雞嚟搞嘢〕  
% 桂支仔  香港人  
% 馬交仔  澳門人  
% 新界仔  新界新市鎮出身嘅社團人士  
% 咬老軟  靠女人食飯;食軟飯  
% 姑爺仔  靠呃女人感情,同要佢哋出去應召維生嘅渣男  
% 炒千張  抄戲飛或者船飛  
% 流千張  假銀紙〔偽鈔〕  
% 熬老襯  暫時做緊正當職業  

% 海鮮檔  街邊賭檔,所謂嘅[魚蝦蟹]賭檔

% 魚蛋檔  係指喺啲只可以摸〔愛撫〕同為客人打飛機〔手淫〕;  
%      不過唔可以直接搞嘢〔做愛〕嘅色情架步  
% 唧魚蛋  喺魚蛋檔入面愛撫魚蛋妹胸部嘅行為  
% 跑私鐘  應召;同流鶯、妓院唔一樣,私鐘係唔公開接客  
% 私鐘妹  應召妓女嘅一種;一般都會有個所謂嘅馬伕[仲介]跟住嘅  
% 媽媽生  帶夜總會舞小姐接客嘅女人[日式稱呼],男嘅就叫[爸爸生]  
% 糶流罕  讀[跳流罕],即係[賣假藥]  
% 輪大米  輪姦[一班仆街咸家剷做嘅嘢]  
% 擘口仔  戲子;即係俗稱食開口飯[靠把口搵食]嘅後生,有負面意思  
%      傳說嘅二十年代,粵劇小武周少保曾做東設宴共和酒樓宴請十班台柱,  
%      被酒家樓面綽號豆粉水嘅人譏為[擘口仔]而被人亂打。  
%      豆粉水召集行家全武行包圍酒家,周少保就班齊打武班支援,  
%      創下咗當年香港塘西嘅[開片]紀錄。  
% 點錯相  認錯人  
% 爆冷格  入無人屋行竊  
% 爆熱格  入有人屋行竊  
% 爆馬欄  開房  
  
% **———————————————————————————————  
  
% Keeper  賭檔、外圍、馬檻嘅管理員  
% 四四六六 三口六面  
% 灌湯灌水 吃喝玩樂  
  
% **——————————

% \newpage

% \section{鬼頭}

% [[鬼佬]]

% \newpage

% \section{魁斗}

% 即系[[[鬼佬]]]或[[[鬼頭]]]  
% #香港城邦

% \newpage

% \section{魚口疳疔}

% [疳疔](疳疔)

% \newpage

% \section{}

% :[想必](想必) - [梗係](梗係)

% \newpage

% \section{:咿唈}

% (Empty file)

% \newpage

% \section{𢆡}

% **𢆡**(粵拼:nin1)(https://zh-yue.wikipedia.org/wiki/%F0%A2%86%A1#cite_note-1)[[2]](https://zh-yue.wikipedia.org/wiki/%F0%A2%86%A1#cite_note-2),卽「奶」(https://zh-yue.wikipedia.org/wiki/%F0%A2%86%A1#cite_note-4),又或者寫做**姩**(https://zh-yue.wikipedia.org/wiki/%F0%A2%86%A1#cite_note-5),係[哺乳類動物](https://zh-yue.wikipedia.org/wiki/%E5%93%BA%E4%B9%B3%E9%A1%9E%E5%8B%95%E7%89%A9 "哺乳類動物")嘅[乸](https://zh-yue.wikipedia.org/wiki/%E4%B9%B8 "乸")用來餵[奶](https://zh-yue.wikipedia.org/wiki/%E5%A5%B6 "奶")畀[蘇蝦](https://zh-yue.wikipedia.org/wiki/%E8%98%87%E8%9D%A6 "蘇蝦")嘅[器官](https://zh-yue.wikipedia.org/wiki/%E5%99%A8%E5%AE%98 "器官"),亦即係醫學講嘅**乳房**。𢆡響[𢆡頭](https://zh-yue.wikipedia.org/wiki/%F0%A2%86%A1%E9%A0%AD "𢆡頭")下面有[乳腺](https://zh-yue.wikipedia.org/w/index.php?title=%E4%B9%B3%E8%85%BA&action=edit&redlink=1 "乳腺 (無呢版)"),會響生咗蘇蝦之後,整奶畀蘇蝦飲。雖然[公](https://zh-yue.wikipedia.org/wiki/%E5%85%AC "公")都有𢆡,結構都一樣,但係未發育,所以呢度集中講乸嘅𢆡。

% #百越底層詞

% \newpage

% \section{𢬿}

% #文 #老粵語 
% 普通話的處置句一般用“把”,而現代粵語則不用“把”而用“將”來表達同樣的意

% 思。在19世紀,歐美傳教士留下了很多粵語資料。分析這些資料之後,我們發現當時也有處置句,但“將”用得較少,與此相反,現代主流粵語中幾乎已消失的“𢬿”用得較多。  

  

% “𢬿”有以下四種主要用法︰(1)用作動詞,意思是“拿”;(2)引介工具的介詞,相當於現代漢語的“用”;(3)與格標誌,相當於“給”;(4)像“把”那樣引介受事賓語的處置標記。漢語的“把”字句,本來是“拿”義的動詞,後來經過語法化的過程才具有處置功能。本文認為“𢬿”也經過了同樣的語法化過程。

  
  


  

  

% - 《康熙字典》:第433頁,第19字
% - 《漢語大字典》:第3卷,第1877頁,第13字
% ### 發音


% - 廣州音一(重構):[粵拼](https://yue.wiktionary.org/wiki/%E7%B2%B5%E6%8B%BC "zh-yue:粵拼"):**[kaai⁵](https://yue.wiktionary.org/wiki/%E5%88%86%E9%A1%9E:%E7%B2%B5%E6%8B%BC%E7%B4%A2%E5%BC%95/kaai "分類:粵拼索引/kaai")**
% - 廣州音二(重構)(https://yue.wiktionary.org/wiki/%F0%A2%AC%BF#cite_note-%E7%89%87%E5%B2%A1%E6%96%B0-1):

% [粵拼](https://yue.wiktionary.org/wiki/%E7%B2%B5%E6%8B%BC "zh-yue:粵拼"):**[gaai⁵](https://yue.wiktionary.org/wiki/%E5%88%86%E9%A1%9E:%E7%B2%B5%E6%8B%BC%E7%B4%A2%E5%BC%95/gaai "分類:粵拼索引/gaai")**


%   1. 【舊】用手移動、攞。

% 【例】揩[SĪC]把刀仔嚟。(攞把刀仔來。)
% 2. 【舊】用。

% 【例】唐人有𢬿牛乳共糖嚟攪茶有冇呢?(唐人有無用牛奶同糖來媾茶呢?)
% 3. 【舊】與格標誌。

% 【例】我寫貨單𢶷你。(我寫貨單[畀](畀)/[過](過)你。)
% 4. 【舊】賓格標誌,通常導致唔同語序出現。

% 【例1.1】等我哋𢬿呢處出奇嘅物件帶你去睇吓。(等我等攞呢一帶出奇嘅物件帶你去睇下。)
% 【例1.2】嗰個醜樣嘅就嬲起嚟𢬿隻茶杯嚟掟爛嘵。(嗰個醜樣嘅嬲起身就將隻茶杯擲爛咗。)
% 【例2】等我𢬿一段古講過你聽。(等我將一段古講畀/過你聽。)
% 【例3】你𢬿個[SĪC]樖樹種落地去喇。(你將嗰棵/樖樹種落地啦。)
% 【例4】主𢬿地方嘅陰翳變為光明。(主將呢個地方嘅陰翳變成光明。)
% 【例5】有個人燒槍𢬿佢嘅細蚊仔打得好傷。(有個人開槍將佢個細氓仔打到好傷。)

% \newpage

% \section{𨂽}

% #百越底層詞 

% 𨂽

% \newpage

% \section{𨅝}

% #百越底層詞

