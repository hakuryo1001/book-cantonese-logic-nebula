「共」呢個廣東話嘅連詞,係一個好老餅嘅時文。咁我就梗係諗緊辦法點樣畀佢可以復活啦。

但姐係話呢,要恢復呢個時文,畀佢重操某種嘅故業,應該唔係就咁調晒啲字就得咁簡單。唔係可以就咁話個「同」字過主返個位出嚟就得。點解?講到尾就係因為根本啲人唔會慣。講嗰個難慣,聽過嗰個仲難慣。咁鬼󱱹耳,點搞啊?

我諗最好嘅計仔呢就係將「共」呢個時文變成一種喺要提高某中古典風味嘅時候先至攞嚟使嘅嘢。歌都有得你唱啦——「俗塵渺渺,天意茫茫,將你共我分開」,「背棄了理想,誰人都可以,那會怕有一天只你共我」。照計呢就會有薄弱嘅靚嘢濫用效應畀我地梗知定會諸他不易嘅畸士下使用數量會慢慢遞增嘅效果。

但未完架啵,仲有得玩架啵。你睇下,呢兩句歌詞都係「你共我」——「你共我」喺呢樹呢刻就可以即刻變成一個定左嘅,變成一個自成個體嘅東東。驟眼睇嚟呢,我地可以將「你」同「我」換做其他嘢,仲要係有微妙挐褦嘅嘢,咁個「共」字嘅特殊語感呢就會最佢哋兩舊野黐埋一齊,變成一舊野喺度。譬如「家共國」「正共邪」「法共義」「刀共菊」「槍共錢」「㕦共吾直」。

既然係咁嘅話,個「共」字生出嚟嘅嘢,喺啱嘅肥沃嘅語境度呢譬如法律啊、哲學啊、數學啊,仲可以變成 。效果可能可以同「\ruby{}{cum}」「\ruby{}{et}」有得揮。

% \chapter{估與估話:粵語中的邏輯思維}

% 估,一個看似簡單嘅字,其實蘊含住豐富嘅邏輯內涵。喺粵語入面,估唔單止係「猜測」咁簡單,而係一種思維方式,一種對未知事物嘅理性推斷。

% 估話,更加係粵語獨有嘅表達方式。佢唔係純粹嘅假設,而係一種帶住可能性嘅陳述。當我哋講「估話」嘅時候,我哋其實係喺度進行緊一種邏輯推理——我哋唔係肯定,但係我哋有理由相信。

% 呢種表達方式同西方嘅邏輯學有住微妙嘅差異。西方嘅邏輯學強調確定性,要麼真,要麼假。但係粵語嘅「估話」就容許住一種中間狀態——一種基於有限資訊嘅合理推測。

% 估話嘅邏輯結構其實好複雜。當我哋講「估話佢會嚟」嘅時候,我哋唔係喺度做一個純粹嘅預測,而係喺度表達一種基於已知事實嘅合理期望。呢種期望有住佢嘅邏輯基礎,雖然唔係絕對確定,但係有住一定嘅可信度。

% 粵語嘅估話系統其實反映住一種更加靈活嘅思維模式。佢唔會因為缺乏絕對確定性就停止思考,而係會喺不確定性入面尋找合理嘅可能性。呢種思維方式其實更加接近現實生活嘅複雜性。

% 估話嘅潛力在於佢能夠處理反事實嘅情況。當我哋講「估話如果當時我咁做嘅話」嘅時候,我哋其實係喺度進行緊一種反事實推理。呢種推理唔係純粹嘅幻想,而係一種基於邏輯嘅思維實驗。

% 粵語嘅估話系統其實係一種更加成熟嘅邏輯工具。佢能夠處理不確定性,能夠進行反事實推理,能夠喺有限嘅資訊基礎上做出合理嘅判斷。呢種能力其實係人類智慧嘅重要組成部分。

% 估話唔單止係語言現象,更加係思維現象。佢反映住粵語使用者嘅邏輯思維特點,一種更加靈活、更加實用嘅理性思維方式。

% 喺呢個意義上,估話其實係粵語對邏輯學嘅獨特貢獻。佢提供咗一種處理不確定性嘅新方法,一種更加貼近現實嘅邏輯思維模式。

% 估話嘅邏輯價值在於佢能夠喺不確定性入面保持理性,能夠喺有限資訊嘅基礎上做出合理嘅推斷。呢種能力其實係人類智慧嘅核心,亦都係粵語文化嘅重要特徵。

% 估話,唔單止係語言,更加係思維,係邏輯,係智慧。


% \chapter{}


% 理 
% 黹
% 李

