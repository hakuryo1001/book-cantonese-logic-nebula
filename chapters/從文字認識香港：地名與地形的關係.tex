#百越底層詞 #geography
## 坑

**解釋:** 坑指的是一個小而深的窪地或溝渠,通常由自然力量如雨水沖刷而成。例如:蕉坑。

  

## 井

**解釋:** 井通常指的是挖掘在地面上的深洞,用於取水或其他用途。例如:井欄樹。

  

## 埗

**解釋:** 埗指的是一個小型的港口或碼頭,用於船隻停泊和貨物裝卸。例如:深水埗。  

## 灣

**解釋:** 灣指的是一個彎曲的海岸線,通常形成一個天然的避風港。例如:銅鑼灣。  

## 口

**解釋:** 口指的是進入某地的入口或出口,通常是河流或海灣的入海口。例如:大涌口。  

## 河

**解釋:** 河指的是流動的水體,通常是河流或溪流。例如:沙田河。  

## 潭

**解釋:** 潭指的是一個深水池,通常位於河流或瀑布的底部。例如:新娘潭。  

## 笏

**解釋:** 笏指的是一種平坦的石板或石塊,通常用於地名中表示該地區的石質地形。例如:笏石。  

## 濠

**解釋:** 濠指的是一個人工或自然形成的護城河或水道。例如:濠江。  

## 港

**解釋:** 港指的是一個天然或人工的港口,用於船隻停泊和貨物裝卸。例如:香港。  

## 灘

**解釋:** 灘指的是一個平坦的沙灘或泥灘,通常位於海岸線附近。例如:紅磡灘。  

## 洲

**解釋:** 洲指的是一片由沉積物形成的陸地,通常位於河流或海洋中。例如:長洲。  

## 谷

**解釋:** 谷指的是兩山之間的低地或河谷,通常形成一個狹長的地形。例如:大埔谷。  

## 地

**解釋:** 地指的是廣泛的土地或地區,通常用於表示某地的地形特徵。例如:濕地。  

## 坪

**解釋:** 坪指的是一片平坦的土地,通常用於耕地或建築用地。例如:秀茂坪。  

## 屻

**解釋:** 屻指的是一個小山丘或高地,通常用於描述該地區的地形特徵。例如:大刀屻。  

## 門

**解釋:** 門指的是一個進入某地的入口,通常用於表示該地區的地理位置。例如:鯉魚門。  

## 窩

**解釋:** 窩指的是一個小而隱蔽的地方,通常位於山谷或山坡上。例如:水浪窩。

  

## 頭

**解釋:** 頭指的是一個地區的最高點或最突出的部分。例如:青龍頭。

  

## 埔

**解釋:** 埔指的是一片平坦的土地,通常用於耕地或建築用地。例如:大埔。

  

## 尖

**解釋:** 尖指的是一個尖銳的地形特徵,通常是山峰或突出部分。例如:蚺蛇尖。

  

## 滘

**解釋:** 滘指的是一個狹窄的水道或河流,通常位於兩岸之間。例如:大埔滘。

  

## 㘭

**解釋:** 㘭指的是一個山坡或斜坡,通常用於描述該地區的地形特徵。例如:大風㘭。

  

## 寮

**解釋:** 寮指的是一個簡陋的房屋或棚屋,通常用於描述農村或偏遠地區。例如:雞寮。

  

## 壩

**解釋:** 壩指的是用於攔截水流的堤壩或水壩,通常用於灌溉或防洪。例如:船灣淡水湖壩。

  

## 圍

**解釋:** 圍指的是一個用圍牆或柵欄包圍的地區,通常是村莊或農田。例如:屏山圍。

  

## 岩

**解釋:** 岩指的是堅硬的岩石或山岩,通常用於描述地區的岩石地形。例如:老虎岩。


涌
- 鰂魚涌
- 東涌

澳
- 將軍澳 
- 大澳
- 澳門
- 

這些地名字不僅是地形特徵的描述,也反映了香港豐富的自然景觀和多樣的文化歷史。理解這些地名和地形的關係,有助於更深入地認識香港的地理環境和歷史文化。