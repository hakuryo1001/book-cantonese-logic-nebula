**3.11** 想像

今「[想]()像」只作動詞,但在《俗話傾談》可作名詞,即「計劃」、「打算」之意:

(39)  你估我用個的錢文真正冇想象麼?狗醜主人羞,唔打办吓光輝,人話齊思賢老婆,衣衫襤褸,失禮到你呀。所以遇時拜神拜佛,無非見自己命鄙,歸到你門兩年,未有所出,都係想菩薩庇佑,早日生個花仔,得到三十七八時,娶個新婦。(學翻你咁好)你做家公,我做家婆,有仔有孫,慢慢享福(不可先折福),人家同話你好命咯。唔通等到五六十歲生仔,扒向棺材頭麼?你做男人曉得發財,唔慌有个的想像吓咯。(253-254)

(40)  有子有孫,亦人生之想像也。(176)