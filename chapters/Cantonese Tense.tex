https://www.reddit.com/r/CantoneseScriptReform/comments/18p1j7i/cantonese_linguistics_re_aspect_and_the_lack_of/ 
Tense

Cantonese has no tense system. Verbs have no inflection, that means they do not conjugate.

[Tense and Time are two different concepts](http://www.beniculturali.unipd.it/extra/dispense%26materiale/dispense_materialepubblico/Lingua%20inglese/2013-14%20-%20Lezione%203%20ottobre%202013%20-%20Time%2C%20tense%2C%20and%20aspect.pdf). Time is basically our real-world perception of _when something happens_, but tense is the grammaticalized expression of location in time. Compare the following sentences:

> Yesterday I **ate** cartnoodles.  
> 尋日我**食**車仔麵。  
> _Cam4jat6 ngo5_ _**sik6**_ _ce1zai2min6._

> Today I **eat** rice.  
> 今日我**食**飯。  
> _Gam1jat6 ngo5_ _**sik6**_ _faan6._

"Yesterday”, “today”, “尋日”, “今日” are _adverbs of time_; they are **lexical**. They show _time_.

“Eat” and “ate” are different **grammatical** forms of the verb “to eat”. They show _tense_.

“食” in both sentences is not inflected.

Note the example sentences referring to the past do not require the use of aspect markers that has past meaning e.g. _zo2_ and _gwo3_.

Aspect

Aspect is the speaker's notion that enables the same situation to be viewed in different ways, i.e. _how does the speaker look at a situation?_ In English, verbs have three aspects: _simple_, _progressive_ ("-ing") and _perfect_ (=_retrospective_). Aspect should not be confused with _tense_.

Although Cantonese has no tense system, it does have an aspect system, which puts affixes around the verb. Cantonese has eight aspects.

**NB** Aspect markers are not to be confused with verbal particles. Verbal particles indicate ideas such as result (effect of an object) and phase of action (beginning, continuing or ending). In some websites, some verb particles are wrongly listed as aspect markers, e.g. 晒 _saai3_ (quantifying particle "all, completely").

---

Perfective 咗 zo2

(Implies past time reference) Indicates a completed event. It is used to report an event, seen as a whole or as completed.  
There are three typical uses of the perfective _zo2_, each of them comparable to three different English tenses:

(1) The resultative meaning. Translates to the English perfect, where the event has a result.

```
佢      炒咗        部     車。
Keoi5   caau2zo2    bou6   ce1  
He      crash-PFV  CL     car
"He's crashed the car." (The car is a wreck now.)
```

(2) Reporting past events without any such result. Translates to the English simple past:

```
公司      舊年       賺咗       唔少         錢。
Gung1si1  gau6nin4  zaan6-zo2  m4siu2       cin2
company   last-year earn-PFV   not-little   money
"The company made a good deal of money last year."
(A statement on the company's performance)
```

(3) Expressing a period of time up to and including the present. Translates to the English perfect progressive:

```
我    部    車    揸咗       兩      年    幾。
Ngo5  bou6  ce1  zaa1-zo2   loeng5  nin4  gei2
I     CL    car  drive-PFV  two     year
"I've been driving the car for over two years."
```

Replacing _zo2_ with _gwo3_ (experiential aspect) implies that the state of affairs no longer holds.

The perfective can be combined with adverbs of the past, or adverbs of the recent past such as 啱啱 _ngaam1-ngaam1_ and 頭先 _tau4sin1_.

```
佢      頭先        炒咗        部     車。
Keoi5   tau4sin1   caau2-zo2   bou6   ce1. 
He      just-now   crash-PFV   CL     car
"Just now he's crashed the car."
```

**NB** The perfective aspect is not to be confused with the "perfect" aspect (nonexistent in Cantonese), which denotes an event with relevance to the point of speech.

**NB** _Zo2_ can be used in imperative sentences (an order or command) or complement clauses referring to the present or future, and should not be treated as a past tense marker.

```
食咗      佢     先。
Sik6-zo2  keoi5 sin1
Eat-PFV   it    first
Eat it up. (Imperative)

我   想      賣咗         部    車。
Ngo5 soeng2  maai6-zo2   bou6  ce1
I    wish    sell-PFV    CL    car
I want to sell the car. (Complement clauses referring to the present or future)
```

**NB** In rapid speech, _zo2_ may be realised as a tone change.

**NB** _zo2_ does not occur in negative sentences. The negative existential 冇 _mou5_ is used before the verb.

```
佢      冇炒         部     車。
Keoi5   mou5-caau2   bou6   ce1  
He      not-crash    CL     car
"He didn't crash the car."
```

---

Experiential 過 gwo3

(Implies past time reference) Similar to present perfect in English. It implies that the situation took place prior to the point of speech or reference. It suggests experience, or something having occurred "at least once before".

```
我   學過        紮鐵。  
Ngo5 hok6-gwo3   zaat3tit3.  
I    learn-EXP  steel fixing.
"I have learnt steel fixing before."  
```

Another use of the experiential aspect is the "indefinite past", which is common for non-human subjects. Similar to the English perfect.

```
部    電腦      壞過         幾   次
Bou6  din6nou5  waai6-gwo3  gei2  ci3
CL    computer  break-EXP  few  time
"The computer has crashed a few times before."
```

Another use of _gwo_ is "inferential perfect", where the speaker infers from the evidence available that something has happened:

```
好似      落過       雨    喎。
Hou2ci5  lok6-gwo3  jyu5  wo3.
"It seems to have been raining."
```

The experiential can be combined with adverbs of the distant past such as 以前 _ji5cin4_ and 曾經 _cang4ging1_.

```
我   曾經         學過       紮鐵。  
Ngo5 cang4ging1  hok6-gwo3  zaat3tit3.  
I    previously  learn-EXP  steel fixing.
"I have learnt steel fixing before."  
```

**NB** The difference between _zo2_ and _gwo3_ is usually whether a result of the event holds at the time of speaking (perfective) or not (experiential).

**NB** _Gwo3_ also has other uses than an aspect marker.

---

Progressive 緊 gan2 / 喺度 hai2dou6

For dynamic, ongoing actions **only**; it implies change over time, i.e. the action is not timeless. By default, _gan2_ applies to the present unless indicated otherwise. Similar to Progressive "-ing" in English.

```
啲  學生        上緊網。  
Di1 hok6saang1  soeng5-gan2-mong5  
CL  student     get on-PROG-Internet
"The students are surfing the web."  

黃     生      嘆緊         杯    鴛鴦。  
Wong4  saang1  taan3-gan2   bui1  jin1joeng1
Wong   Mr      enjoy-PROV   CL    milk coffee
"Mr Wong is enjoying a cup of milk coffee."  

噖晚        黃      生      嘆緊        杯    鴛鴦。  
Kam4-maan5  Wong4  saang1  taan3-gan2  bui1  jin1joeng1.
Last-night  Wong   Mr      enjoy-PROV  CL    milk coffee
"Last night Mr Wong was enjoying a cup of milk coffee."  
```

_Hai2dou6_ (lit. _be here_) has the same function, but it precedes the verb:

```
啲  學生        喺度       上網。  
Di1 hok6saang1  hai2dou6  soeng5-mong5 
CL  student     be-here   get on Internet
"The students are surfing the web."  
```

_Hai2dou6_ can be used together with _gan2_ to reinforce the progressive meaning:

```
啲  學生        喺度       上緊網。  
Di1 hok6saang1  hai2dou6  soeng5-gan2-mong5  
CL  student     be-here   get on-PROG-Internet
"The students are surfing the web."  
```

---

Continuous 住 zyu6

Denotes a continuous activity or state without change, typically present or timeless. No strict equivalent in English.

```
啲   雲     遮住         個   太陽 
Di1  wan4   ze1-zyu6     go3  taai3joeng4.
CL   cloud  block-CONT   CL  sunlight.  
"The clouds are blocking out the sunlight."  
```

**NB** Some verbs are frequently used with _zyu6_, such as 對住 _deoi3-zyu6_ ("face"), 趕住 _gon2-zyu6_ ("in a hurry"), 掛住 _gwaa3-zyu6_ ("miss"), 趕住 _gon2-zyu6_ ("in a hurry"), 揸住 _zaa1-zyu6_ ("keep hold of"), 阻住 _zo2-zyu6_ ("block, obstruct"), 望住 _mong6-zyu6_ ("stare at").

**NB** In a few cases, some [verb _zyu6_] combination has slightly different meaning from the simple verb, e.g. 諗 _lam2_ ("think") vs 諗住 _lam2-zyu6_ ("intend").

**NB** _zyu6_ is also used in serial verb constructions to denote simultaneous activities. _zyu6_ can be used as a particle at the end of a clause to mean "yet". It can also be used together with 先 _sìn_ to mean "for the time being".

---

Delimitative 吓 haa5

Has the meaning "do ... for a while" or "have a ...". It is typically used with verbs denoting activities, with or without an object.

```
行吓 
hang4-haa5
walk-DEL  
"take a walk"  

飲吓茶 
jam2-haa2 caa4 
drink-DEL-tea
"have some tea"  
```

Another equivalent construction is "_... jat1 ..._", which is often contracted to become a tone change. A similar construction "_... leong5 ..._" has no contracted form.

```
行一行 
hang4-jat1-hang4

行行 
hang4→2-hang4

行兩行 
hang4-leong5-hang4
"take a walk"  
```

Reduplication of verb followed by _haa5_ indicates that the action is prolonged or repeated. It translates to the present perfect progressive in English.

```
我   諗諗吓,          都   係   買     多     打   好。
ngo5 lam2-lam2-haa5   dou1 hai6 maai5  do1   daa1  hou2
I    think-think-DEL  also is   buy    extra dozen good.
"I've been thinking, it's best to buy a dozen more."
```

In serial verb constructions that express two simultaneous actions, the reduplicated verb denotes an interrupted action.

**NB** An idiomatic use of the delimitative can be found in the [verb _haa5_ ... _sìn_] construction, which means "to do something temporarily", especially for killing time.

```
你   望吓        食乜      先,  轉頭       同    你   落單。
nei5 mong6-haa5 sik6-mat1 sin1  zyun2tau4 tung4 nei5 lok6-daan1
You  look-DEL   eat-what  first turn-head with  you  place-order
"Have a look at what to eat first, (I'll) be back soon to place the order for you." 
```

**NB** The idiom 眨吓眼 _zaam2-haa6-ngaan5_ is an adverbial phrase meaning "in the blink of an eyelid".

---

Inchoative 起上嚟 hei2-soeng5-lai4

_Seong5-lai4_ is a directional component meaning "come-up", but as an aspect marker, _hei2-seong5-lai4_ means "begin". There is no equivalent aspect in English that has the same meaning.

```
棟    樓           突然       燒   起上嚟。  
dung6 lau4        dat6jin4  siu1 hei2-soeng5-lai4  
CL    building    sudden    burn rise-up-come
"The building suddenly went up in flames."  
```

_Seong5_ is sometimes omitted.

---

Continuative 落去 lok6-heoi3

It is a directional complement that means "go down", but by extension it is an aspect marker meaning "continue".

```
咁    樣     燒    落去,      棟    樓         遲早       冧。  
gam3  joeng6 siu1 lok6-heoi3  dung6 lau4      ci4zou2    lam3  
such  way    burn continue    CL    building  late-early collapse
"Sooner or later, the building is going to collapse if it keeps on burning."  
```

---

Habitual 開 hoi1 / 慣 gwaan3

Denotes habitual, customary activity. Unlike the English _used to_ construction, it is not restricted to past time. In fact, it is typically used of the present. It can be used together with an adverb of time.

```
佢    睇開       中       醫      嘅  
keoi5 tai2-hoi1 zung1    ji1     ge3
S/he  see-HAB   Chinese  doctor  SFP
"S/he usually goes to a Chinese doctor."  

佢    以前      睇開       中       醫      嘅  
keoi5 ji5cin4  tai2-hoi1 zung1    ji1     ge3
S/he  before   see-HAB   Chinese  doctor  SFP
"S/he used to go to a Chinese doctor."
```

_Gwaan3_ can also be used to indicate habitual aspect, with the more specific meaning "accustomed to":

```
佢    食  慣        貴          嘢
keoi5 sik6 gwaan3   gwai3      je5
S/he  eat used      expensive  thing 
"S/he is used to eating expensive food."
```

---

Source: Matthews, S., & Yip, V. (2011). _Cantonese : A comprehensive grammar_ (2nd ed., Routledge comprehensive grammars). London: Routledge.

---

For learners, I recommend the following book to learn more about Cantonese grammar:

Yip, V., & Matthews, S. (2017). _Basic Cantonese: A Grammar and Workbook_: Taylor & Francis.

Edit: formatting Edit 2:. Fixed jyutping errors