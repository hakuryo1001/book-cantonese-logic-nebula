#文 


It is generally accepted that Cantonese is a Sinitic language with heavy influences from ancient Baiyue languages in terms of phonology, grammatical structure and vocabulary. Even though our Baak-jyut/Baiyue ancestors were colonised and eventually Sinicised, they had nevertheless left in our language an obvious Baiyue substratum, including substrate words which are mostly of Tai-Kradai, Austroasiatic and Hmong-Mien origins. These substrate words, though relatively small in number compared to Sinitic words, are among the commonest and most frequently used words in daily Cantonese conversations.

Here is a list of substrate words I can think of, with possible cognates inside the brackets.

Disclaimer: I am neither a scholar nor an expert in relevant fields. I've made this list purely based on my memories, a few essays of “dialect” (🙄) studies in Chinese and wiktionary. Regarding some words I've listed below, whether they are substrate words or not is still unconfirmed or disputed.

_verbs_

[[搣]] mit1: to tear up  
掹 mang1: to pull  
篤 duk1: to poke  
[[冧]] lam3: to fall [Thai ล้ม (lom), Zhuang "laemx"]  
[[冚]] kam2: to cover [Thai ห่ม (hom)]  
諗 nam2: to think  
[[𨂽]] dam6: to stamp one's foot [Zhuang "daemh"]  
撳 gam6: to press down, to click [Thai ข่ม (khom), Zhuang "gaemh"]  
嘥 saai1: to waste [Zhuang "sai", Thai เสีย (sia)]  
踎 mau1: to squat  
郁 juk1: to move  
躝 laan1: to crawl, “get out!” [Thai คลาน (khlaan)]  
𨂾/𨈇 naam3/laam3: to cross [Thai ข้าม (khaam)]  
[[𨅝]] jaang3: to tread on [Thai ย่าง (yaang)]  
[[呃]] ngak1/ngaak1: to lie to, to deceive [Zhuang "ngaek"]  
[[囈]] ngai1: to beg  
扱 kap1: to lid, to cover  
孭 me1: to carry on the back  
棟/戙 dung6: to erect, to stand straight [Thai ตั้ง (dtang)]  
𠺘 long2: to rinse [Thai ล้าง (laang)]  
hap1 or 蝦 haa1: to bully, to pick on.  
甩 lat1: to slip off, to drop [Vietnamese "lột"]  
批 pai1: to peel  
鎅/𠝹 gaai3: to cut  
擁 ung2: to push  
焫 naat3: to sear [Zhuang "ndat", Thai เดือด (dueat)]  
淥 luk6: (liquid) to scald [Thai ลวก (luak)]  
㓤/拮 gat1: to pierce, to prick  
揈 fing6: to swing, to sway [Bouyei "veengh"]  
𢯎 ngaau1: to scratch [Thai เกา (gao)]

_adjectives_

[[痕]] han4: itchy [Zhuang "haenz"]  
lak1 kak1 (couldn't find characters): road bumping or person stuttering  
孖 maa1: twin, double [Thai **ฝา**แฝด (**faa**faet)]  
乸 naa5: female, original meaning "female who fills the role of mother" as in 後底乸 [Thai น้า (naa)]  
腍 nam4: soft, tender [Thai นุ่ม (num)]  
啱 ngaam1: correct, suitable [Thai งาม (ngaam), Vietnamese "ngám"]  
凹 nap1: concave  
嬲 nau1: angry [Zhuang "naeuq"]  
奄尖 jim1 zim1: picky  
曳 jai5/jai4: ill-behaved, naughty, of poor quality [Zhuang "yaez"]  
𢛴憎 mang2 zang2: irritable  
蚊 man1 as in 细蚊仔, 蚊皮 ("naughty" in my dialect): naughty [Zhuang "manz"]  
杰 git6: viscous [Zhuang "gwd", Bouyei "geg"]  
孻 laai1 as in 孻仔, 拉尾/孻尾: last [Zhuang "byai", Thai ปลาย (bplaai)]

_nouns_

甩 lat1 as in 麻甩: sparrow [Zhuang "laej"]  
㯷 buk6: pomelo [Zhuang "mak**bug**" ("mak" meaning "fruit")]

> The fruit is called 碌柚 in Canton/Hong Kong Cantonese, but 孤㯷 is actually the common name for pomelo in my dialect. "Buk" is the original Tai-Kradai word for pomelo. Many other Cantonese dialects, such as Sanwui dialect or Taishanese also call it 布碌/波碌/etc. meaning "little pomelo" (buk/bu/bo=pomelo, luk=child/classifier for fruit)

蠄蟧 kam4 lou2: spider  
馬騮 maa5 lau1: monkey  
蛤 gap3/gaap3 as in 蛤乸: frog [Thai กบ (kop)]  
項 hong2 as in 雞項: young hen  
[[拏褦]] naa1 nang3: connection, link, relation [Zhuang "nanaengq"]  
𧕴 naan3: inflammation on the skin resulting from an insect bite or an illness [Zhuang "nwnj"]  
蝻 naam4 as in 蝻蛇 or 大蝻蛇: python [Zhuang "nuem", Thai เหลือม (lueam)]  
陸 luk6 as in 豬陸: pen, sty [Thai คอก (khaok)]  
馬蹄 maa5 tai4: water chestnut [Zhuang "makdaez"]  
椗 ding3 as in 慈姑椗: stem of plants  
佬 lou2: (colloquial) man [related to the historical Raew/Rau people (僚人) and Laos (寮國/老撾)]

> 廣州謂平人曰佬,亦曰獠,賤稱也。— 《廣東新語》  
> People in Canton call common people “lau” or “leu”, both derogatory. -- _Guangdong Xinyu_, 1678

碌 luk1 as in 碌柚: child [Thai ลูก (luuk)]  
细路仔 sai3 lou6 zai2: child, boy [路 is the same word with 碌 above. 路仔 is obviously a cognate with Thai ลูกชาย (luukchaai) or Zhuang "lwgsai" meaning "son"]

_others_

呢 nei1/ni1: this [Thai นี้ (ni), Zhuang "neix", Khmer នេះ (nih), Vietnamese "nầy", Malay "ini"]  
咪 mai5: don't [possibly related to Thai มิ (mi) or ไม่ (mai)]  
啲 di1/dit1: a bit, some [Zhuang "di", uncertain]

_words of unknown origin_

嘢 je5: thing

