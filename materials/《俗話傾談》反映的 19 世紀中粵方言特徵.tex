#老粵語 #文

本文件是呈送編輯的定稿。全文已刊於:戴忠沛(2009)「《俗話傾談》反映的 19 世紀中粵方言特徵」,載錢志安、郭必之、李寶倫、鄒嘉彥編《粵語跨學科研究:第十三屆國際粵方言研討會論文集》,香港:香港城市大學語言資訊科學研究中心,第 245-259 頁。

# 《俗話傾談》反映的 **19** 世紀中粵方言特徵

戴忠沛

香港大學

《俗話傾談》是目前可考最早的「三及第」小說。學界很早就注意到《俗話傾談》,但一般認為「三及第」著作裏粵語與文言相混,難以用來研究清代粵語。其實《俗話傾談》的性質與後來的「三及第」略有分別,它本來是講生的話本,其中旁白敘述部份雖然文白相混,但人物對話卻幾乎全屬口語,例如以下對話:

「做乜叫人刮痧刮得咁淒涼呀?」

「刮刮刮刮你個條命。分明係被籐鞭所打,重話我刮痧。」(60)

因此,只要將書中的人物對話抽出,即可作為研究當時粵語的材料。這些對話與同時期翻譯得來的粵語材料或作為教材的粵語文句相比,更加通俗生動;而與「粵謳」一類的粵語曲詞相比,則更接近實際口語。本文將整理《俗話傾談》裏的人物對話,研究其中反映的字音、詞彙、語用及語法特徵,期望為清代粵語的研究提供一份新材料。

**1.**     背景

《俗話傾談》輯選者邵彬儒,字紀棠,廣東四會人,生卒生不詳,以說書為業,活躍於同治、光緒年間,是當時廣州、佛山、南海一帶著名的講生,擅長以通俗手法宣講人倫禮教(魯金 1990、黃仲鳴 2002:53、耿淑艷 2007),而《俗話傾談》很可能是在他的話本基礎上輯選而成的。該書共兩集四卷,其中初集收 11 則故事,分別是「橫紋柴」、「七畝肥田」、「邱瓊山」、「種福兒郎」、「閃山風」、「九魔托世」、「饑荒詩」、「瓜棚遇鬼」、「鬼怕孝心人」、「張閻王」、「修整爛命」,二集收七則故事,分別是「骨肉試真情」、「潑婦」、「生魂遊地獄」、「借火食烟」、「好秀才」、「砒霜砵」、「茅寮訓子」。這些故事部份可考以民間傳說或其他小說為藍本,如「橫紋柴」故事原型為《聊齋誌異》的「珊瑚」;「積福兒郎」故事初見於《德育古鑒》等。《俗話傾談》曾出現多種版本。目前所見最早的版本,初集有中國國家圖書館藏同治九年(1870)秋鐫「粵東省城十七甫五經樓」本,及北京大學圖書館藏同治九年(1870)「粵東立經樓」本,而二集最早的是同治九年(1870)春鐫「羊城十七甫五經樓」本。按說初集應於二集前出版,因此同治九年秋出版的初集很可能不是最早的版本。葉春生(1996:318)指《俗話傾談》初版於同治三年(1864)(#_ftn2),我們也不妨推測該書初版於 1860 年代中至末期出版。

本文所據的版本,初集為同治九年(1870)秋「粵東省城十七甫五經樓」本,二集為同治九年(1870)春「羊城十七甫五經樓」本,皆已由上海古籍出版社(1994)影印出版。二集「羊城十七甫五經樓」本缺最後一頁,以出版地點不明、同治十年(1871)春鐫版的《俗話傾談二集》配補(中華書局 1990),這版本除了封面和少數內文的旁注直音略有分別,其他皆與「羊城十七甫五經樓」本完全相同。

**2.**     語音

**2.1** 直音

《俗話傾談》在兩方面提供了當時粵語的語音訊息。首先,該書作為講生的話本,為了應用方便,在部份漢字旁邊注有直音。以下所列各組,前一次為原字,後一字為直音:

![](file:////Users/hongjan/Library/Group%20Containers/UBF8T346G9.Office/TemporaryItems/msohtmlclip/clip_image001.jpg)![](file:////Users/hongjan/Library/Group%20Containers/UBF8T346G9.Office/TemporaryItems/msohtmlclip/clip_image001.jpg)填-田(2187/9)、鹵-老(38)、撞-狀(38)、折-節(46)、![](file:////Users/hongjan/Library/Group%20Containers/UBF8T346G9.Office/TemporaryItems/msohtmlclip/clip_image002.jpg)-篤(50)、筶-教(69)、懲-呈(126)、趁-親(225)、螺-羅(251)、褸-慮(254)、倘-湯上聲(268)、遞-弟(270)、瓶-平(277)、馗-葵(281)、羲-希(282)、斥-勅(282)、脊-炙(288)、唆-踈(289)、腔-康(290)、概-蓋(291)、烘-紅(296)、樁-莊(359)、段-斷(359)、畀-彼(370)、撼-坎(383)、鑄-註(388)、鋼-降(388)、膠-交(401)、嘯-笑(401)、桅-為(402)、贏-形(411)、扯-者上聲(412)、筋-根(413)、拼-并(414)、窄-責(415)、蹈-道(418)、駁-博(422)、衙-牙(440)、炊-吹(444)、膥-春(334、

445)

另有三例直音與原字音明顯不符:

攪-叔(2191/13)(#_ftn5)、箸-目(37)、隊-在(280)

以下各字,旁邊雖注有直音,但在影印本內直音字模糊不清,姑且存目備考:

畀(2218/40、99,從上同例估計直音為「彼」)、釉(37)、顱(44)、擅(52)、膥(84,從上同例估計直音為「春」)、濬(109)、臬(159)、嗽(175)、僱(179)、護(181)、儼(187)、拼(193,從上同例估計直音為「并」)、碓(292)、春

(292)、嘶(295)、壓(320)

**2.2** 同音別字

其次,透過書中的別字,也可以得知當時部份漢字的讀音。以下各組前一字為原本的別字,後一字為正字,括號內附別字所出的詞語句子及相應頁數:

引-癮(烟引 4、酒引 278)、順-遜(出言不順 38)、式-適(看過合式 58)、事-肆(大事辦過衣裝 63)、準-准(太爺不準 142)、便-邊/面(城之東便大起瘟疫 179)、證-症(時證大行 179)、到-道(知到 182、202、230、238、 306、363、417-418)、止-只(你止曉得日日醉 224)、什-拾(收什 239、411)、資-貲(積此資財 232)、办-扮(打办 254、410)、剝-薄(刻剝百姓 285)、氏-侍(妾氏 304、325、328)、稟-品(稟性難易 309、醜稟 431)、桩-裝(桩定身勢 378)、捫-毛(腰下束一條捫巾 313)、跌-鐵(跌鍊鎖住頸上 417)

            以上直音和別字,雖然未足以讓我們分析當時的粵語音系,但也足見若干字音的演變。例如「螺」今讀 lo2,但《粵音韻彙》讀音只有 lo4,與《俗話傾談》直音為「羅」相合。「贏」據《粵音韻彙》有 jeng4 及 jing4 二讀,今多讀 jeng4,從《俗話傾談》直音為「形」可知當時讀作 jing4。

**3.** 詞匯

在研究清末粵語的材料當中,絕大部份是典籍翻譯,或例句教材。這些材料的用詞,一般都比較文雅。相比之下,《俗話傾談》裏依粵語口語寫成的人物對話,用詞則比較通俗,其中包括了不少當今粵語[[6]](#_ftn6)已不常見或意思已經改變的詞匯,能補其他同期粵語材料之不足,以下略舉數例:

**3.1** [[色水]]

現今一般指貴價金屬或玉器的色澤。在《俗話傾談》裏「色水」似乎有兩個意思,較常用的意思是指女性的外表樣貌,例如:

(1)    我當初做新婦時,重好色水過你十倍,唔估今日老得個樣醜態,減去三分。(2)

(2)    唔通六七十歲老大婆重整作咁好色水麼?(410)

另一個意思指贓害別人的把戲,或類近今日粵語「畀啲顏色佢睇」的「顏色」,今粵語唯一相關的用法只有「整色整水」:

(3)    遲數日兩個去探過佢,若係恭恭敬敬,有的禮貌便了,若仍然冷淡,要整佢色水開井水過人食。(236)

**3.2** [[周致/周至]]

此詞現今粵語未見,大槪意思是「仔細」、「妥當」:

(4)    你估同我地後生,慢慢梳光頭、搽了粉、戴好花,又要扎周致雙脚麼?(21)

(5)    誰知蚊帳、被裖,樣樣虔潔光鮮,方知珊瑚每日整理周至。(83)

(6)    父親臨病之時,見我服事得佢周至,話我孝心,父在牀頭,親筆寫云,七畝餘田,交與亞定永遠耕管。(94)

**3.3** [[抽身抽勢]]

現今形容人坐立不安。在《俗話傾談》大槪指人意氣風發、大搖大擺:

(7)    臧姑凸起眼精曰,我就咒你,你點樣惡法呀?我唔怕惡,共你打清。然後食飯都唔做得,話完即捲起衫袖、扎緊包頭帶,抽身抽勢,裝模作樣,好似猛虎下山想人肉食。(21)

(8)    二成抽身抽勢,向兜肚內擒出一渣袋,約一百之多。(58)

(9)    馮氏曰:打都唔怕你。話完即抽身抽勢,扎緊隻髻。(382)

與「抽身抽勢」相近的還有「整定身勢」、「桩(裝)定身勢」:

(10)  各人整定身勢,今日去攞人命呀(東莞叫做食腊鴨飯)。(333)

(11)  誰不知繼業桩定身勢,扎起髻氏的,繼功亦抽高褲脚,捲實衫袖。(378-379)

**3.4** [遇時](遇時)

此詞今粵語未見,大槪意思同「經常」:

(12)  二成生得兩個仔,臧姑遇時自己贊好命。(39-40)

(13)  我前日買定一張單刀,放在床頭,遇時預備要用佢,若真來尋打,就先下手為強,免至受虧一著。(370-371)

**3.5** [[兇性]]/[[兇勢]]/[[勢兇]]/[[兇橫]]

「兇」在今日粵語多獨用作形容詞,在《俗話傾談》經常與其他字組合成詞,意思除了本意「兇惡」外,還可指人「狠心」,略同今粵語詞「[[狼死]]」:

(14)  適值旁邊有一個婦人,見他如此兇性,即用力擒住他手,盡勢推開,大喝一聲,乜你咁勢兇呀?(10)

(15)  各人見他咁兇勢,咁撒賴,難以用手相爭。(265)

(16)  有咁樣惡法,我个新婦既死,巳經傷心不了,重來毁我房屋,散我家私,將我老婆咁樣凌辱,有咁太過兇橫。佢恃拳頭在近,官府在遠麼?(341)

**3.6** 真正

即「真的」、「真是」,同今粵語「真係」,此詞在《俗話傾談》裏出現頻率頗高:

(17)  暇暇,你個橫紋柴,真正好笑咯,你個仔既寫分書,就如路人,那一個重係你新婦呀?(12)

(18)  難得咯、難得咯,真正第一好新婦咯。(34)

(19)  亞哥你真正冇本心,盡將銅銀分過我,你自已要了好銀,我被人捉住搽黑面辦做烏龜,毒打一身,真正唔抵咯。(62)

(20)  姚氏聽到此話,知係真情,个陣口軟聲低,細聲問曰:亞叔真正嗎?(306)

「真正」在今粵語已罕用,但在 60 年代粵語電影還可聽到,可知此詞消失是近年的事。在《俗話傾談》裏「真正」不時與「係」連用,似乎是「真正」過渡至今日「真係」的中間階段:

(21)  唔通都係銅銀?伯爺真正係唔好人咯,佢听用之銀,聞得俱是好的。我所用係假的,分明欺你愚蠢。(60-61)

(22)  話起亦有理,今晚我飲酒,食了一砵仔鹹蘿蔔,唔通真正係心躁發夢?(75)

(23)  睇你唔出做如人咁佮俐呢,你個把嘴真正係審死官咯。(76)

**3.7** 撞板/梗板/硬板/個一板豆腐

在《俗話傾談》裏有一組與「板」有關的詞彙。其中「撞板」雖然在今日粵語依然常用,但意思略有不同。今日「撞板」指人犯錯。在《俗話傾談》裏「撞板」除了「犯錯」的意思,還能形容一件事很糟糕、倒霉:

(24)  無奈咁撞板,想孝心,老母就死,天不從人願,整定要該衰咯。(90)

(25)  點算呀?撞板咯,嚇死我兩個仔咯。(100)

(26)  女呀,你唔好去,個的唔係別樣病,係叫做冇牙老虎。你偏回去,若撞板起來,連你都死乾淨咯。(180)

除了「撞板」,在《俗話傾談》還有「硬板」和「梗板」,這兩個詞語在今粵語已甚少出現:

(27)  父母家財,亦唔係定局,佢話要多的,我作父母剩少的,假如生多幾個兄弟,唔通硬板要翻咁多麼?(37)

(28)  暇暇,數日之間,又是一場變卦,方信閻王簿上有添有改,都無梗板寫法。(175)

另外,《俗話傾談》還有熟語「個一板豆腐」,大槪意同「不是好東西」:

(29)  內心大驚,[料必](料必)又係個一板豆腐咯。(68)

(30)  你兩個真正好舉荐,好發財門路,製个板豆腐,打得我死過翻生,真唔抵咯。(247)

**3.8** 光棍

今粵語「光棍」多指單身漢,但在《俗話傾談》卻指騙子,詞意發生了變化:

(31)  唔係做賊,人家話我做光棍,用假銀買真貨,白白受打一場。(60)

(32)  向以世上好多周身八寶,計多過米,曉做光棍,曉謀害人,曰撈曰縮,到底依然貧困也。(165)

**3.9** 心事

在《俗話傾談》裏出現與「心事」有關的詞語包括「假心事」、「有心事」、「好心事」,今日粵語「心事」只能用作名詞,一般指負面的想法,但在《俗話傾談》可組成形容詞短語,意思也不一定負面:

(33)  你勿整成個的假心事來戲弄我。(假心事都勝過有心事)。我知你底子,不是個樣人,不知你聽誰人所教。(26)

(34)  真正好心事,唔話得咯,算第一個婦人。(61)

(35)  睇佢心事,好似思疑你做亞哥,瞞騙于佢。(220)

(36)  到是真咯,唔請你唔嫁,就係你死,我都唔娶。不憂無老婆,難得你唔好心事呀。(250)

**3.10** 光輝/周身輝

《俗話傾談》裏以「輝」形容一個人衣著打扮得體,如「光輝」、「周身輝」,這種用法今日未見:

(37)  細佬哥,个陣拋了個隻砵,買的好衣裳,裝得周身輝,去歸買屋,娶老婆做財主,都係哩条門路咯。(239)

(38)  你估我用個的錢文真正冇想象麼?狗醜主人羞,唔打办吓光輝,人話齊思賢老婆,衣衫襤褸,失禮到你呀。(253)

**3.11** [想像](想像)

今「想像」只作動詞,但在《俗話傾談》可作名詞,即「計劃」、「打算」之意:

(39)  你估我用個的錢文真正冇想象麼?狗醜主人羞,唔打办吓光輝,人話齊思賢老婆,衣衫襤褸,失禮到你呀。所以遇時拜神拜佛,無非見自己命鄙,歸到你門兩年,未有所出,都係想菩薩庇佑,早日生個花仔,得到三十七八時,娶個新婦。(學翻你咁好)你做家公,我做家婆,有仔有孫,慢慢享福(不可先折福),人家同話你好命咯。唔通等到五六十歲生仔,扒向棺材頭麼?你做男人曉得發財,唔慌有个的想像吓咯。(253-254)

(40)  有子有孫,亦人生之想像也。(176)

以下詞語,意思尚待考証:

**3.12** [[穿崩閗湊]]

此詞在《俗話傾談》只有一例,從文句看此詞意思似乎與「粉飾浮跨」類同:

(41)  板障花窗,可以粉飾浮誇,穿崩閗湊。獨至四條大柱,須用堅石,須用實木,自頭到脚,都要咁堅,都要咁實。(17)

 在書中近似的詞語有「穿崩爛破」或「穿崩破爛」,然而這兩個詞語似乎與「穿崩閗湊」正好相反:

(42)  好似扯得穿崩爛破。(68)

(43)  有的去打爛水缸,有的去打穿米塔,有的去打崩飯鑊,有的拈斧頭砍破大門,有的執竹篙掃屋瓦,打得穿崩破爛,好處無存。(331)

**3.13** 阿瓜/亞瓜

此詞在《俗話傾談》中有二例,只知是一種人物性的負面比喻,具體來源和含義不明:

(44)  二成夫妻暗偷歡喜,可以無拘無束,自作自為。置一張鬼子枱,油了金漆,兩張竹椅,可以伸腰。象牙快箸,磁器碗碟,白釉茶壺,描花局盅等頂頂件件俱全,鮮明雅潔,居然要鬧做阿瓜,老婆好似十萬身家都冇咁鬧罵。餐餐要飲有色酒。(37-38)

(45)  朋友相交,未嘗不設飲食,亦唔係專以飲食為題,當飲食時,講得了不得咁知心,唔通冇飲食就水咁淡,觀佢形容,整聲色,講惡氣,如敗水亞瓜,新出匪類。(231)

**3.14** [[泒]]/沠

此字從語料看似乎有三種意思,第一種意思類同「批評」、「抱怨」:

(46)  兩公婆只怨老母不仁、沠老母不是。(39)

(47)  若只曉得泒翁姑不是,叔伯不是,做男子就唔著聽咯。(432)

       而第二種意思類同系詞,但只用於排名:

(48)  論起層次,長子亞孝泒第一,亞忠泒第三,亞信泒第四,此三個仔,俱係正妻所生,亞悌泒第二,亞仁泒第五,亞義泒第六,此三個仔俱係妾氏所生。(325)

(49)  長子繼業泒第一,繼德泒第三,此兩个係結髮所生。繼功泒第二,繼績泒第四,此兩个係妾所生。繼祖泒第五,此一个係婢所生。(368)

       而最後一個意思,類同「分送」:

(50)  誰不知你行前人指後,話你等豬兄狗弟,實在都唔係人,今鬧起官司要將我大仔沠與乞兒,問你於心何忍?(103)

            據《異體字字典》(教育部國語推行委員會 2002),此字可以是「派」或「流」的異體,但不論視作「派」還是「流」,都不能圓滿解釋以上所有意思。如果視作「派」字,則能符合最後一個意思,也可在第二個意思裏理解為「排」的近音別字,卻解釋不了第一種意思。如果將該字視作「流」,在第一種意思裏也許能視作「鬧」的近音別字,但未能解釋第二和第三種意思。

**4.** 熟語、詈語

**4.1** 熟語

《俗話傾談》作為講生的故事話本,為了吸引聽眾,行文用句十分通俗,使用了不少熟語或比喻手法,部份在今日粵語依然能夠找到:

(51)[[傷風夾膩]]:誰知肚內尚有風痰,未能疏發得透,食了豬肉,謂之傷風夾膩。(7)

(52)[[心頭跌落脚筋踭]]:你肯聽我教,我就心頭跌落脚筋踭咯。(20)

(53)[[稟神咁樣稟]]:橫紋柴有時落得水多、落得水少,其飯煮得太軟太硬,臧姑就沈吟密咒,好似稟神咁樣稟(21)

(54)圩咁嘈、蝦咁跳:去到死者之家,如雀鳥歸巢,鵝鴨到埠,圩咁嘈,蝦咁跳,話逼死佢个女,逼死佢个妹,逼死佢亞姨,詐哭得嗚嗚。(333)

(55)好喉頸、日子長、好尾運:勸雞頸與珊瑚曰,你一生好喉頸。勸雞腸與珊瑚曰,你後來日子長。勸雞尾與珊瑚曰,你將來好尾運。又勸珊瑚飲雞酒,話後生飲過好兆頭。(33-34)

(56)龍肉:你不用問我,我與你分開食,你唔管得我。個的就是龍肉,與你無干。(38)

(57)亞崩養狗:兩公婆,十年唔叫一句老母,十年唔叫一句家婆,為何今早如此恭敬,好似亞崩養狗,轉了性,都唔定咯。(82)

(58)實過鐵釘:我講沙虫變蚊仔,人人共見。道理至愛真實,最忌虛浮,我句對文,重實過鐵釘,落水都唔浸得爛,重話唔好過佢麼?(116-117)

(59)冇牙老虎:女呀,你唔好去,個的唔係別樣病,係叫做冇牙老虎。你偏回去,若撞板起來,連你都死乾淨咯。(180)

(60)大花筒:捉狗仔、切魚生、彈琵琶,吹鴉片,嫖賭飲蕩,練得周身引,好似大花筒。(223)

(61)捉虱上頭壳養:本心之講,事關人命,連累非輕,非比同狗肉魚生就帮吓手,都贃得的食呀,個死佬見過都衰,有乜咁蠢才,捉虱上頭壳養呢?(235)

(62)鹹豆都唔食得一粒:我亦為此之故,所以即刻推辞,佢尚唔知利害,實在佢有條人命案,在我兩個手來。我兩個若容忍他,佢便有碗安樂飯食,若係唔顧舊相與,我要佢鹹豆都唔食得一粒。(235-236)

(63)風吹臘鴨:其妻使人走往女處,誰知吊在親家門上,好似風吹臘鴨,搖搖擺擺咯。(269)

(64)易過執豆:个件事重易過執豆,執豆尚要顧低頭。(278)

(65)一隻牛唔好攪壞一欄:亞孝自高自傲,以亞悌亞仁亞義,係庶母所生,不以骨肉相待,作佢為低一格,而卑賤之。結埋亞忠亞信,作為一党,話我三兄弟係大婆仔,佢三個係妾氏仔,就欺佢打佢,都唔奈得我乜何。(果然好亞哥好帶头好倡率所謂一隻牛唔好攪壞一欄)。(328)

(66)橫吞欖核:細仔呀,我知你屈氣咯,个的龜蛋唔中用,我來教佢,佢一句頂住我喉嚨,好似橫吞欖核(生鵞喉都唔定),話佢唔聽,打佢唔贏,鬱抑憂愁,何處可寬懷一二。(376)

(67)吊燒豬:你唔願打,要用吊法。二成未曾見人吊過,以為吊好過打。二成曰,我願吊罷咯。巡丁將他吊起,名為吊燒豬。(66)

(68)食腊鴨飯:各人整定身勢,今日去攞人命呀(東莞叫做食腊鴨飯)。(333)

**4.2** 詈語

在《俗話傾談》的故事裏,經常會出現一些負面的人物角色,或描寫家人爭執不和,藉此突出不修善行、不重人和的惡果,最終達至宣揚人倫的目的。因此,《俗話傾談》內不難找到一些當時的罵人話,這些詞語,在其他的清末粵語材料裏少有記錄:眼中釘(11)、心頭火(12)、壞鬼女(15)、惡家婆(15)、昏婆(15)、冇天裝(20)、[[霸巷雞乸]](20,指經常發惡的婦女)、老龜婆(21,用來罵家姑)、老狗乸(21,用來罵家姑)、衰家狗(21,用來罵媳婦)、乞食骨(23,用來罵兒子)、[[盲虫頭]](23,用來罵兒子)、發戇(35)、賤婦人(43)、蠢婆(102)、蠢才(122)、廢物(122)、負心人(149)、掃把星(157)、敗家精(157)、丟駕(217)、龜蛋(220)、潑婦(268)、[[砧板蟻]](330,指人廢物)、[[溝渠鴨]](330,指人廢物)、臘豬頭(330,指人廢物)、烏龍尾(330,指人廢物)、攪屎棍(333,指人無事生非)、風炉扇(333,指人無事生非)、[[食尿甕雞]](404,指人言語討厭)、食死顛狗(404,指人言語討厭)、[[砒霜砵]](407,指狠毒的婦女)、蠢才

(430)、陰陰濕濕(430)、老婆奴(430)

**5.** 句法

            語法句式的歷時演變,一向是清末粵語材料的研究重點。《俗話傾談》裏粵語句子的句式,也顯露出當時語法的一般特點,以下略舉數例,並與 1893 年[[7]](#_ftn7)寫成的《麥仕治廣州俗話書經解義》(余靄芹 2000)作比較:

**5.1** 正反問句

《俗話傾談》裏的正反問句,主要以「VP-neg-V」形式出現,共有 7 例:

從我唔從(239)、應在天堂唔應(280)、應落地獄唔應(281)、服佢唔服(365)、甘心唔甘(386)、係我欺負你唔係(411)、食著唔食(425)

另有 2 例是「VP-neg-VP」形式:

記得唔記得(25)、得做唔得做(306)

「V-neg-V」形式一例:

你起唔起(418)

「VP-neg」型,只見 4 例,其中 3 例同屬一個句子:

做得唔呢(278、303、423)、你話妙到極唔呢(113)

至於今日最常見的「V-neg-VP」結構,則沒有任何例子。

余藹芹(Yue-Hashimoto 1993)認為「VP-neg」型的正反問句是粵語裏最基本的句式,「V-neg-V」及「VP-neg-V」後來才產生。在《書經解義》裏,最常見的是「VP-neg」類型,但《俗話傾談》裏反而是「VP-neg-V」比較常見。似乎《俗話傾談》成書雖在《書經解義》之先,但卻呈現了較後期的語法狀態,這情況大抵說明粵語正反問句的演變,在各地方並不同步。

**5.2** 比較句

       在《俗話傾談》裏,所有的比較句都以「過」作標記:

重好色水過你十倍(2)、你要好過佢為是(20)、新婦惡過家婆(21)、假心事勝過冇心事(26)、為何你重要多過亞哥呢(36)、雞肉重高過鼻哥(83)、佢好得過我個比(112)、重唔勝過佢(116)、重實過鐵釘(117)、重話唔好過佢麼(117)、哭得眼胞腫起,大過雞膥(156)、計多過米(165)、但有荔枝,勝過冇荔枝(199)、有時話朋友好過兄弟(220)、大約勝過他人(230)、我兩個重打得多過你(247)、我亦話夫妻親過父母(251)、你丈夫親過我(259)、个件事重易過執豆(278)、重關係過做賊(285)、斷唔輸得過佢(329)、佢重醜過我十分(364)、大約我重先做過佢(365)、你慌駛輸過佢麼(370)、徒弟惡過師傅咯(372)、飽死好過餓死(416)、你到來惡得過我(419)、重關係過砒霜砵。(423)、天眼明過鏡(429)

 《俗話傾談》裏沒有找到以「比」作標記的比較句,情況與同時期的其他粵語材料相同。有趣的是,《俗話傾談》作為三及第小說,文言、白話與粵語夾雜,但全書即使是非粵語句式,都沒有發現以「比」作標記的比較句。這一點與其他清末的粵語材料情況一致。

**5.3** 雙賓結構

這裏所指的雙賓結構包括同時有直接和間接賓語的句式及它的轉換形式。余藹芹(2000)為了比較間接賓語前用「過」或「俾」的歷時變化,在討論《書經解義》雙賓結構時將狹義的雙賓句(如「賜佢個官銜」)、謂補結構(如「封賜福樂俾人」)和處置式(「將呢一件事先稟告過上帝知」)都列入討論範圍,本文也採用相同的標準。《俗話傾談》裏的雙賓結構在間接賓語之前絕大多數都要加上標誌「過」、「與」,只有少數例子在間接賓語前沒有標誌。

5.3.1       沒有標誌的句子

       在沒有標誌的句子裏,有 4 例是處置式將直接賓語提前,例如:

(69)   誰肯將女嫁佢個仔呢?(19)

(70)   我今發你還陽,將此事轉傳於人。(421)

       而非處置式的有 3 例,其中兩例屬「V+Oi+V+Od」結構:

(71)   兩公婆……四時八節唔叫老母食一餐飯,唔請亞哥飲一杯酒。(39)

(72)   你三兄弟唔請我食一餐,留一宿。(358)

屬最典型雙賓句的「V+Od+Oi」只有一例:

(73)   著咯、著咯,唔駛畀情面佢,佢叫我做亞哥,都唔好應佢。(370)

5.3.2       間接賓語前以「過」作標誌

            在以「過」為標誌的句子裏,大約有一半(5 例)是直接賓語與間接賓語兼備的「V+Od+ 過+Oi+(V)」結構,例如:

(74)   就係繡条大紅裙,聯件花衫袖過你著,你都無愧咯。(230-231)

(75)   你慌我冇飯過你食,冇屋過你住麼?因你父唔知,於理不合。(374)

至於另外一半的例子(7 例)是只有間接賓語的「V+過+Oi+(V)」結構,省略掉的直接賓語可據上文下理推敲出來,例如:

(76)   我窮然後賣女,賣過你使喚,唔係賣過你打死呀。(43)

(77)   太爺一一解過我知咯。(99)

(78)   乜你先時唔話過我知呀?(100)

            還有兩例以處置式「將+Od+V+過+Oi」結構出現,由於處置式能將直接賓語提前,因此直接賓語不能省略:

(79)   亞哥你真正冇本心,盡將銅銀分過我。(62)

(80)   將我個仔來分過乞食佬。(99)

5.3.3       間接賓語前以「與」作標誌

 以「與」作標誌的句子,多數(8 例)是同時有直接和間接賓語的「V+Od+與+Oi+(V)」結構,例如:

(81)   勸雞頸與珊瑚曰,你一生好喉頸。勸雞腸與珊瑚曰,你後來日子長。勸雞尾與珊瑚曰,你將來好尾運。(33-34)

(82)   所以天有眼,賜福賜祿與佢。(57)

(83)   公公,我煲粥與你共大眾食。(183)

(84)   亞孝等唔肯分田地與我。(361)

有三例是省略直接賓語或直接賓語在句子前段已有出現的「V+與+Oi+(V)」結構:

(85)   你要換就換與你。(62)

(86)   世上有一等人,買魚買肉多讓與仔食。(77)

(87)   擇好雞肉,勸與老母。(83)

另外有三例屬處置式的「將+Od+V+與+Oi+(V)」結構:

(88)   今鬧起官司要將我大仔沠與乞兒。(103)

(89)   命差役將亞明次子亞定長子,押去養濟院,交與乞食頭做親男。(97)

(90)   你將臘鴨送與亞姨,送與契女。(409) 《俗話傾談》雙賓結構的一大特點,就是間接賓語前經常以「與」或「過」作為標誌,這在《書經解義》及其他清末粵語材料似乎不多見。相反,在其他材料常見的「俾」和「俾過」字句,於《俗話傾談》只能找到一例。余藹芹(2000)曾指出「過」是早期用法,後來逐漸被「俾」字取代。《俗話傾談》的情況比較特殊,最常見的不是「俾」或「過」,而是「與」。我們很難判定這裏的「與」在當時口語是否真的存在,還是受文言文影響的結果。不過即使將所有以「與」作標誌的例子排除,《俗話傾談》裏以「過」

作標誌的例子還是遠比以「俾」作標誌的多,反映了比較早期的面貌。

                                            **6.**    《俗話傾談》的共時差異

在考察《俗話傾談》裏的粵語時,除了要注意與今日粵語的歷時差異,還應該注意方音之間的共時差異。《俗話傾談》作者邵彬儒是四會人,四會話在《中國語言地圖集》

(中國社會科學院、澳大利亞人文科學院 1987)歸入勾漏片,李新魁(1994:26-27)劃為羅廣片,詹伯慧(2002:168)則將這一帶的粵語次方言稱為「西江流域粵語」。不論如何劃分命名,各家都認同四會話與廣州話分屬不同的次方言。在《俗話傾談》裏,我們也從個別詞匯和語音窺見四會話的痕跡。

**6.1** 詞匯

以下三個詞匯很可能反映了四會話的影響:

(91)   狗牯:呢个乞食仔,你話失了亞叔,个隻狗牯,就係你亞叔呢?(245)

粵語裏稱「公狗」為「狗牯」於廣州未見,但於粵西、粵北都比較普遍(詹伯慧、張日昇 1994:520;1998:524),其中包括了《俗話傾談》作者邵彬儒家鄉的四會話。

(92)   快箸:象牙快箸,磁器碗碟,白釉茶壺,描花局盅等頂件件俱全。(37-38)

粵西四會、廣寧一帶皆稱「筷子」為「箸」(詹伯慧、張日昇 1998:550),此說法於廣州話未見。

(93)   個處:由是摩頭摩耳,眼望天、脚拍地,磨吓墨,又拈吓筆,走去小便個處,企住想一回,行理(按:應為「埋」之誤)書位,坐住椅,扭完手指,伏低枱頭,都唔想得出。(112-113)

今日粵語裏可用「處」或「度」指代一個地方或地點,其中廣州主要使用「度」,如「呢度」、「嗰度」,而四會主要使用「處」,如「呢處」、「嗰處」(詹伯慧、張日昇 1998:664)。在《俗話傾談》裏,指代地方或地點的用詞只見「處」而沒有「度」,更接近四會話的用法。

不過,在四會、廣寧一帶,人稱代詞眾數標記可說成「呢」,如「我呢」、「你呢」(詹伯慧、張日昇 1998:661),但《俗話傾談》裏卻只有廣州一帶通用的「我地」、「你地」。似乎作者遇到四會話與廣州話差異較明顯的地方,都儘量改用廣州話的說法,但在不為人注意的地方,則保留了四會話的痕跡。

**6.2**    語音

正如 2.2 節所述,《俗話傾談》裏有部份漢字注有直音,其中「扯-者上聲」、「跌-鐵」這兩組例子,似乎都涉及送氣和不送氣聲母的交替。而在《俗話傾談》的作者家鄉的四會話裏,古全濁塞擦音聲母平聲字幾乎都不送氣(詹伯慧、張日昇 1998:46),這是四會話與廣州話語音的主要區別。「扯-者上聲」、「跌-鐵」這兩組例子雖然都不涉及古全濁塞擦音的平聲字,但作者卻依然將送氣與不送氣的字音互注。似乎他也意識到自己家鄉話裏很多不送氣的字音在廣州、佛山一帶都唸成送氣,使他矯枉過正將一些本應不送氣的字也誤為送氣,導致誤讀。

其實,如果檢視一下書中注有直音的例子,可發現部份被注的都不是甚麼難字罕用字,如「瓶」、「折」、「拼」等,使人不禁懷疑作者注直音的動機,並非因為他不懂那些字,而是用來提醒自己這些字在廣州音的讀法。

**6.3**    總結

《俗話傾談》作為講生話本,性質與後來的「三及第」不同,其中的人物對話部份,白話程度甚高,是研究 19 世紀中葉粵語的好材料。《俗話傾談》與同期的粵語材料相比,十分通俗入世,記錄了大量當時的俗語、熟語,以至詈語,正好補同期資料的不足。尤其值得注意的是,《俗話傾談》作者邵彬儒本身是四會人,他長期在廣州佛山說書,要學會廣州一帶的口語並不困難,但他原有語言習慣在《俗話傾談》裏依然會留下痕跡。除了上述詞匯和語音的線索外,第 5 節所述的部份《俗話傾談》語法特徵與《書經解義》等同期材料不盡一致,很可能同樣是地域間的共時差異,未來值得進一步探究。

引用文獻

[清]博陵紀棠氏評輯. 1990.《俗話傾談》、《俗話傾談二集》。上海:上海古籍出版社《古本小說集成》影印《俗話傾談》同治九年(1870)粵東省城十七甫五經樓本、《俗話傾談二集》同治九年(1870)羊城十七甫五經樓本。

[清]博陵紀棠氏評輯. 1994.《俗話傾談》、《俗話傾談二集》。北京:中華書局《古本小說叢刊》影印《俗話傾談》年份不詳省城學院前華玉堂本、《俗話傾談二集》同治十年(1871)出版地不詳本。

教育部國語推行委員會. 2002.《異體字字典》。台北:教育部。(光碟版)李新魁. 1994.《廣東的方言》。廣州:廣東人民出版社。

魯金. 1990.〈用「三及第」文體寫成的俗話傾談〉。香港:《明報》1990 年 8 月 3 日。

耿淑艷. 2007.〈嶺南晚清小說家邵彬儒生平著作考〉,《文獻》2007.3:93-98。

孔仲南. 1992.《廣東俗語考》。上海:上海文藝出版社影印 1933 年南方扶輪社版本。

黃仲鳴. 2002.《香港三及第文體流變史》。香港:香港作家協會。葉春生. 1996.《嶺南俗文學簡史》。廣州:廣東高等教育出版社。

余藹芹. 2000.〈粵語方言的歷史研究──讀《麥仕治廣州俗話〈書經〉解義》〉,《中國語文》

2000.6:497-507。

詹伯慧、張日昇主編. 1994.《粵北十縣市粵方言調查報告》。廣州:暨南大學出版社。詹伯慧、張日昇主編. 1998.《粵西十縣市粵方言調查報告》。廣州:暨南大學出版社。

詹伯慧主編. 2002.《廣東粵方言槪要》。廣州:暨南大學出版社。

中國社會科學院、澳大利亞人文科學院編. 1987.《中國語言地圖集》。香港:朗文出版社。

Yue-Hashimoto, Anna. 1993. “The lexicon in syntactic change: lexical diffusion in Chinese syntax”. _Journal of Chinese Linguistics_ 21.2:213-254.

  

---

(#_ftnref1) 括號內數字表示該例子所在的頁碼,除非另外說明,所有頁碼均為上海古籍出版社影印本(1994)的新編頁碼。

(#_ftnref2) 葉春生未有提供資料出處,也許他於廣東省的圖書館內見過該書初版。

(#_ftnref3) 此字只在上海古籍出版社影印本(1994)注有直音,括號內前一號為上海古籍出版社影印本(1994)的新編頁碼,斜線後是對應中華書局影印本(1990)的新編頁碼。下同。

(#_ftnref4) 原句為「你去巷亞美叔借一張熟鐵鋤頭。」(50)。「」據《廣東俗語考》上卷(孔仲南 1992[1933]:40)音篤,「尾後」之意,此處當指「末端、盡頭」。承黃得森賜教,特此鳴謝。

(#_ftnref5) 原句為「為何不回母家而在此攪擾姨婆。」估計誤將「攪擾」作「騷擾」再以「叔」為「騷」直音。

(#_ftnref6) 本文以香港粵語作為當代粵語的參照。

(#_ftnref7) 《麥仕治廣州俗話書經解義》據余藹芹(2000)乃癸巳年出版,合 1833 年或 1893 年。承郭必之指出該書載有麥仕治照片,而攝影技術於 19 世紀中晚期才傳入中國,故《書經解義》出版年期當以 1893 年為合。