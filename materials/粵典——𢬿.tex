 #文  #老粵語
又到咗古典粵語嘅時間。大家有冇聽過「[[𢬿]]」呢個用法呢?

(kaai4 同「鞋」近音,不過聲母要變做 k)

【械/𢬿 kaai4 / kaai3 / gaai5 (prep.) 】 [http://beta.words.hk/zidin/𢬿](https://l.facebook.com/l.php?u=http%3A%2F%2Fbeta.words.hk%2Fzidin%2F%25F0%25A2%25AC%25BF%3Ffbclid%3DIwAR27MOU2cN-YtCSBCwb1h4IN83YncyqiA8QBW0g7yVVDrAP6wQYk4rRCJ8w&h=AT23Gp_xzhRntcCNhZLcH_y3gKjRkpI_lrsOZ2M8wFwhrUyWdH2R9ffVn7-DcN9xZNxqr6dYIBriab8UhpEQAoc5BxxgozK37a2Wnx6GM52r1HlVGcZTMd3rBa-WcYZYACyGvK0&__tn__=-UK-R&c[0]=AT0z70hYIGdXQc3MqSw4dwhohaCgZCnG6XLb3s1ssfCcMeCY2cjwqw43hMZt3mTiH9AFYm5eUNhJKllIMj9ka2Idx1bYI9JixDyEgX-qSeYeGVJw7MKpXe147_aXHczHe-jyAFzHG1KHGJ2GBGSru4ixa9WbgM-09R10g2I)

引出一個動作嘅受事對象或者工具;而家非常之少用 used to introduce the patient or the tool before a verb; rarely used nowadays

例句:

唐人有𢬿牛乳共糖嚟攪茶有冇呢?(《漢語讀本之廣東方言 A Chinese Chrestomathy in the Canton Dialect》1841)

Do the Chinese use milk and sugar with their tea?

擰個毛掃嚟𢬿衣服嘅沙塵掃乾淨佢。(《散語四十章》1877)

拎個毛掃嚟將啲衫嘅沙塵掃乾淨佢。

Take a brush to brush dust off the clothes.

嗰個女人嘅指甲長,𢬿佢手臂搲爛嘵。(《散語四十章》1877)

嗰個女人指甲長,將佢手臂搲損咗。

That woman has long fingernails, scratching and thus getting her arm hurt.

可以械錢或械力幫助人嘅。(《教話指南 Beginning Cantonese (Rewritten)》1927)

可以出錢或者出力幫人嘅。

(One) can help others with money or effort.

械條布帶同佢包住個傷口喇。(《教話指南 Beginning Cantonese (Rewritten)》1927)

攞條布帶同佢包住個傷口啦。

Take a cloth band to wrap the wound.

你械嗰啲嘢擺品字樣喇。(《教話指南 Beginning Cantonese (Rewritten)》1927)

你將嗰啲嘢擺做品字形啦。

You arrange those things in a triangular pattern like the character 品.

你械條命嚟教飛咩?(《教話指南 Beginning Cantonese (Rewritten)》1927)

你攞條命嚟教飛咩?

Aren't you putting your life at risk?

街上擺賣嘅食物,都要𢬿罩罩住。(《分類通行廣州話》)

街上擺賣嘅食物,都要搵罩冚住。

The food sold in the street has to be covered with a cover.

佢械條毛布嚟做抹抬布。(張洪年,2007:《香港粵語語法的研究(增訂版)》頁409) (keoi5 kaai4 tiu4 mou4 bou3 lai4 zou6 maat3 toi2 bou3.)

佢用條毛布嚟做抹抬布。

He used a coarse cotton cloth as a table cloth.

你𢬿嗰部書擰俾我啊,聽見冇?

(You) take that book to me. Can you hear me?

借錢過人,係𢬿自己嘅錢俾過人使。

To lend money to someone, is to have your own money given to the person for him to spend.

你械個空樽裝滿佢。

Take the empty bottle and have it filled.

你點可以𢬿人家嘅嘢夾硬擰去得𠺝?

How could you take away someone else's belonging without their agreement?

佢械箭射完我,又埋嚟械條大棍扑我,都唔明解佢做咩嘢事幹。

First he shot me with an arrow, then he came over and hit me with a stick. I have no idea what he's doing.

隻雀械對爪抓實枝樹枝。

The bird used its claws to hold on to the branch.

械啲濕咗嘅衫曬乾佢。

Take the soaked clothes and have them dried.

【械/𢬿 kaai4 / kaai3 (v.) 】 [http://beta.words.hk/zidin/𢬿](https://l.facebook.com/l.php?u=http%3A%2F%2Fbeta.words.hk%2Fzidin%2F%25F0%25A2%25AC%25BF%3Ffbclid%3DIwAR03j7j_MGCvySsjQB1uTs5q3lYtyrwaN3ARtAkpcklecnYUPtTkmLhd7vQ&h=AT23Gp_xzhRntcCNhZLcH_y3gKjRkpI_lrsOZ2M8wFwhrUyWdH2R9ffVn7-DcN9xZNxqr6dYIBriab8UhpEQAoc5BxxgozK37a2Wnx6GM52r1HlVGcZTMd3rBa-WcYZYACyGvK0&__tn__=-UK-R&c[0]=AT0z70hYIGdXQc3MqSw4dwhohaCgZCnG6XLb3s1ssfCcMeCY2cjwqw43hMZt3mTiH9AFYm5eUNhJKllIMj9ka2Idx1bYI9JixDyEgX-qSeYeGVJw7MKpXe147_aXHczHe-jyAFzHG1KHGJ2GBGSru4ixa9WbgM-09R10g2I)

用;攞;要;愛。老一輩先會講呢個字。 to use; to take something; to use a thing as a tool to do something; only spoken by the elder generation

例句:

蠶吐絲人械嚟整衣服着。(《教話指南 Beginning Cantonese (Rewritten)》1927)

天蠶嘅絲係好噤人械嚟整樂器。(《教話指南 Beginning Cantonese (Rewritten)》1927)

竹劍係械嚟頑嘅呮。(《教話指南 Beginning Cantonese (Rewritten)》1927)

竹劍係攞嚟玩嘅唧。

The bamboo sword is only used for playing.

呢部嘢係械嚟影相嘅。 (ni1 bou6 je5 hai6 kaai3 lei4 jing2 soeng2 ge3.)

This machine is used for taking photos.