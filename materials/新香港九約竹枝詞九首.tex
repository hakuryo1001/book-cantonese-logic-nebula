# 新香港九約竹枝詞九首



**劉祖榮**

**前言:**《香港九約竹枝詞》又稱《圍名歌》,或《新界竹枝詞》。據文獻記載,由清代晚期兩名秀才(私塾老師),居於沙田石古壟的許永慶和火炭九肚村的羅文祥共同創作。書中收集一百五十五首四行七言絕句,描述香港當時景物、名勝、風俗及民情,並且嵌入了新界各村圍的地名,猶如一個世紀前的香港地理誌。二○一四年,該書列入「首份香港非物質文化遺產清單」。

鄙人亦好遊山玩水,登高遠足。多年來,造訪了香港許多名勝古跡。有意仿前人用竹枝詞,把瀏覽這些地方的所見所聞所想抒寫出來,傳承下去。以饗讀者。  
[![](https://cdn.hkwriters.ph4day.com/wp-content/uploads/2024/06/2024062517353860.jpg)](https://cdn.hkwriters.ph4day.com/wp-content/uploads/2024/06/2024062517353860.jpg)

**一,大帽山**

百山環繞勢軒昂  
大霧籠頭似帽妝  
疊翠雲間通訊塔  
仿如佇立一金剛  
[![](https://cdn.hkwriters.ph4day.com/wp-content/uploads/2024/06/2024062517354976.jpg)](https://cdn.hkwriters.ph4day.com/wp-content/uploads/2024/06/2024062517354976.jpg)**二,重遊鹽田梓**

簇翠山頭天主堂  
朝輝共與普祥光  
紅林掩映滄桑史  
玉帶橫穿碧海央

**三,米埔**

米埔河灘匯海潮  
魚蝦跳躍嬉波濤  
時聞百鳥盤空至  
逐水爭撈競翅翺

**四,荃灣西方寺**

九層高塔倚雄峰  
八角銜雲紫氣籠  
聳峙西天迎萬佛  
釋迦彌勒共融融

**五,汀九泳灘**

三塔橋橫汲水門  
巍然屹立聳瑤軒  
長灘泳會如鱗次  
碧海灣前舴艋喧

**六,昂坪棧道**

彌勒山中木棧旋  
凌空壑谷響溪泉  
時聞笑語纜車過  
俯拾祥雲上寶蓮

[![](https://cdn.hkwriters.ph4day.com/wp-content/uploads/2024/06/2024062517353942.jpg)](https://cdn.hkwriters.ph4day.com/wp-content/uploads/2024/06/2024062517353942.jpg)  
**七,城門水塘**

一湖碧水盈春色  
八面青山拱日輝  
守得心門風月韻  
城池內外盡芳菲  
[![](https://cdn.hkwriters.ph4day.com/wp-content/uploads/2024/06/2024062517354427.jpg)](https://cdn.hkwriters.ph4day.com/wp-content/uploads/2024/06/2024062517354427.jpg)

**八,山頂盧吉道**

金鐘踏步覽秋歡  
環繞爐峰賞壯觀  
廣廈高樓逶兩岸  
白雲滄海翠屏巒

**九,清水灣泳灘**

波光翡翠逐斑斕  
足底銀沙趾甲彈  
左右群峰迎綠島  
輕遊潛泳戲魚歡

(本文圖片由作者提供)

**劉祖榮簡介:**香港人,祖籍福建南安市。作品曾獲得二○二一年首屆全球「藝術與和平」詩詞大賽的三等獎,二○二○年第二屆義烏駱賓王國際兒童詩歌大賽的提名獎,二○二一年第三屆中國徐霞客散文獎佳作獎,二○一八年全球華文「曼麗雙輝」填詞大獎賽優異獎,二○二一年首屆中國「華潯杯.我愛我家」詩歌優秀獎,二○二一年第九屆「禾澤都林杯」城市、建築與文化散文大賽優秀獎,二○二二年第三屆猴王杯詩歌大賽佳作獎等。