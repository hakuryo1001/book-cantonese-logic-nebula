泒/沠

此字從語料看似乎有三種意思,第一種意思類同「批評」、「抱怨」:

(46)  兩公婆只怨老母不仁、沠老母不是。(39)

(47)  若只曉得泒翁姑不是,叔伯不是,做男子就唔著聽咯。(432)

       而第二種意思類同系詞,但只用於排名:

(48)  論起層次,長子亞孝泒第一,亞忠泒第三,亞信泒第四,此三個仔,俱係正妻所生,亞悌泒第二,亞仁泒第五,亞義泒第六,此三個仔俱係妾氏所生。(325)

(49)  長子繼業泒第一,繼德泒第三,此兩个係結髮所生。繼功泒第二,繼績泒第四,此兩个係妾所生。繼祖泒第五,此一个係婢所生。(368)

       而最後一個意思,類同「分送」:

(50)  誰不知你行前人指後,話你等豬兄狗弟,實在都唔係人,今鬧起官司要將我大仔沠與乞兒,問你於心何忍?(103)

            據《異體字字典》(教育部國語推行委員會 2002),此字可以是「派」或「流」的異體,但不論視作「派」還是「流」,都不能圓滿解釋以上所有意思。如果視作「派」字,則能符合最後一個意思,也可在第二個意思裏理解為「排」的近音別字,卻解釋不了第一種意思。如果將該字視作「流」,在第一種意思裏也許能視作「鬧」的近音別字,但未能解釋第二和第三種意思。
