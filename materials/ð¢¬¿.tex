#文 #老粵語 
普通話的處置句一般用“把”,而現代粵語則不用“把”而用“將”來表達同樣的意

思。在19世紀,歐美傳教士留下了很多粵語資料。分析這些資料之後,我們發現當時也有處置句,但“將”用得較少,與此相反,現代主流粵語中幾乎已消失的“𢬿”用得較多。  

  

“𢬿”有以下四種主要用法︰(1)用作動詞,意思是“拿”;(2)引介工具的介詞,相當於現代漢語的“用”;(3)與格標誌,相當於“給”;(4)像“把”那樣引介受事賓語的處置標記。漢語的“把”字句,本來是“拿”義的動詞,後來經過語法化的過程才具有處置功能。本文認為“𢬿”也經過了同樣的語法化過程。

  
  


  

  

- 《康熙字典》:第433頁,第19字
- 《漢語大字典》:第3卷,第1877頁,第13字
### 發音


- 廣州音一(重構):[粵拼](https://yue.wiktionary.org/wiki/%E7%B2%B5%E6%8B%BC "zh-yue:粵拼"):**[kaai⁵](https://yue.wiktionary.org/wiki/%E5%88%86%E9%A1%9E:%E7%B2%B5%E6%8B%BC%E7%B4%A2%E5%BC%95/kaai "分類:粵拼索引/kaai")**
- 廣州音二(重構)(https://yue.wiktionary.org/wiki/%F0%A2%AC%BF#cite_note-%E7%89%87%E5%B2%A1%E6%96%B0-1):

[粵拼](https://yue.wiktionary.org/wiki/%E7%B2%B5%E6%8B%BC "zh-yue:粵拼"):**[gaai⁵](https://yue.wiktionary.org/wiki/%E5%88%86%E9%A1%9E:%E7%B2%B5%E6%8B%BC%E7%B4%A2%E5%BC%95/gaai "分類:粵拼索引/gaai")**


  1. 【舊】用手移動、攞。

【例】揩[SĪC]把刀仔嚟。(攞把刀仔來。)
2. 【舊】用。

【例】唐人有𢬿牛乳共糖嚟攪茶有冇呢?(唐人有無用牛奶同糖來媾茶呢?)
3. 【舊】與格標誌。

【例】我寫貨單𢶷你。(我寫貨單[畀](畀)/[過](過)你。)
4. 【舊】賓格標誌,通常導致唔同語序出現。

【例1.1】等我哋𢬿呢處出奇嘅物件帶你去睇吓。(等我等攞呢一帶出奇嘅物件帶你去睇下。)
【例1.2】嗰個醜樣嘅就嬲起嚟𢬿隻茶杯嚟掟爛嘵。(嗰個醜樣嘅嬲起身就將隻茶杯擲爛咗。)
【例2】等我𢬿一段古講過你聽。(等我將一段古講畀/過你聽。)
【例3】你𢬿個[SĪC]樖樹種落地去喇。(你將嗰棵/樖樹種落地啦。)
【例4】主𢬿地方嘅陰翳變為光明。(主將呢個地方嘅陰翳變成光明。)
【例5】有個人燒槍𢬿佢嘅細蚊仔打得好傷。(有個人開槍將佢個細氓仔打到好傷。)